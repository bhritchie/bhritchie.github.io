% Dropbox/Dissertation/Kumarila.tex

% !TEX TS-program = xelatex
% !TEX encoding = UTF-8

%COMPILATION COMMANDS:
% xelatex -interaction=nonstopmode -file-line-error-style -synctex=1  -output-directory="Dropbox/Dissertation" "Dropbox/Dissertation/Kumarila.tex"
% bibtex "Dropbox/Dissertation/Kumarila"

\documentclass[11pt,letterpaper,oneside]{amsart}

%\usepackage{Documents/Dissertation/customxetex}
%\usepackage{customxetex}

\usepackage{geometry}

%\usepackage{xltxtra}
\usepackage{fontspec}

%\usepackage[math]{mathspec}
\setmainfont[Mapping=tex-text]{Gentium Plus}
\setsansfont[Mapping=tex-text]{Linux Biolinum O}


\usepackage{polyglossia}
%\makeatother
\setmainlanguage{english}
\setotherlanguage[variant=poly]{greek}

\usepackage{verbatim}

\usepackage{natbib}
\bibpunct{(}{)}{;}{a}{,}{,}
%\usepackage{biblatex}


%\usepackage[usenames,dvipsnames]{color}
%\usepackage[pdfborder={0 0 0}, colorlinks=true, hyperfootnotes=false, urlcolor=RoyalBlue, filecolor=RoyalBlue, citecolor=RoyalBlue, anchorcolor=RoyalBlue, linkcolor=RoyalBlue]{hyperref}

\usepackage{bigfoot}
\DeclareNewFootnote[para]{default}
\AtBeginDocument{\RestyleFootnote{default}{para}}


\renewcommand{\thesection}{\Roman{section}}
\renewcommand{\thesubsection}{{\bf\Roman{section}.\roman{subsection}}}

\newcommand{\tab}{\hspace*{4em}}
\newcommand{\sdots}{$\ldotp\ldotp\ldotp\ldotp$\ }
%\newcommand{\g}{\emph}
\newcommand{\g}{\textgreek}
\newcommand{\e}{\emph}

%\titlelabel{\thesubsection}.\quad}
%\titleformat*{\subsection}{\itshape}

%\renewcommand\refname{Bibliography}

\newenvironment{squote}{\begin{quote}\sf\small}{\rm\end{quote}} %SMALL AND SANS
%\newenvironment{squote}{\begin{quote}\small}{\end{quote}}
%\newenvironment{squote}{\begin{quote}\sf}{\rm\end{quote}} %SANS, NORMAL SIZE
\newenvironment{lquote}{\begin{quote}\sf}{\rm\end{quote}}
%\newenvironment{ssquote}{\begin{quote}\sf}{\rm\end{quote}}
%\newenvironment{squote}{\begin{quote}}{\end{quote}} %SIMPLE QUOTE ENVIRONMENT


\defcitealias{sharma2004charaka}{Sharma}

\newcommand{\ips}{\e{ipso facto}}
\newcommand{\ps}{P.S.\ } %% Post Scriptum      - an addition to the main text
\newcommand{\ie}{i.e.\ } %% id est             - that is
\newcommand{\eg}{e.g.\ } %% exampli gratia     - for example
\newcommand{\nb}{N.B.\ } %% Nota Bene          - note well
\newcommand{\etc}{etc.\ }%% et cetera          - and so on/forth
\newcommand{\ca}{ca.\ }  %% circa              - approximately
\newcommand{\cf}{cf.\ }  %% confer             - compare
\newcommand{\etal}{\emph{et al}.\ } %% et alii or et alia - and others
\newcommand{\opcit}{op.~cit.\ } %% opere citato - in the work previously quoted
\newcommand{\ibid}{ibid.\/} %% ibidem          - in the same place as the previous reference
\newcommand{\typ}{typ.\ }   %%                 - typically

	
%KUMARILA PAPER
\newcommand{\kum}{Kum\={a}rila}
\newcommand{\sabara}{\'S\=abara}

\newcommand{\mim}{M\={\i}m\={a}\d{m}s\={a}}
\newcommand{\mimamsakas}{M\={\i}m\=a\d ms\=akas}
\newcommand{\mimamsaka}{M\={\i}m\=a\d ms\=aka}
\newcommand{\mimsut}{\e{M\={\i}m\={a}\d{m}s\={a} S\={u}tra}}

\newcommand{\cod}{\e{Codan\={a}s\={u}tra}}
\newcommand{\slk}{\'{S}lokav\={a}rtika}

\newcommand{\svapra}{\e{svata\d{h} pr\={a}m\={a}\d{n}ya}}
\newcommand{\sva}{\e{svata\d{h}}}
\newcommand{\pra}{\e{pr\={a}m\={a}\d{n}ya}}



%TITLE AND AUTHOR
\author{Brendan Ritchie}
\title{Svata\d{h} Pr\={a}m\={a}\d{n}ya in Kum\=arila Bha\d t\d ta}





%NEED TO FIX SUBSECTIONS
%FIX HYPERLINKS AND COLOR
%FIX GREEK TEXT


\begin{document}

\maketitle

\thispagestyle{empty}

%\input{MAINBODY.tex}

% !TeX root = Kumarila.tex

%\={x}, \={\i}, \d{x}, \.x, \~x, \'x, \b{x}

%Kathleen Norris --- Akrasia ---accidie

%check with Taber: stuff on difficulties of I.i.5; something that gives a good general sense of Kumarila's importance

%got to discuss Patanjali and Plato o eternality of language - see Taber eternality article

%Will need to talk to Parimal about translations:
  %first: just check to make sure my adjustments are ok
  %second, problematic verses: 53 (Taber); 52--4 in general re my interpretations; 80, 52 and TS 2904 re Taber's passage of time suggestion
  %regardning my note on v.38 in the appendix, have to check with Parimal whether we can actually tell whose voice stuff is in, or whether anything indicates objections etc.


%-----------------------------------------------------------------------------------------------
%-------------------------------------1. INTRODUCTION-------------------------------------------
%-----------------------------------------------------------------------------------------------


\section{Introduction: The M\={\i}m\=a\d ms\=a Inquiry into Dharma} \label{intro}

The P\=urva M\={\i}m\=a\d ms\=a was first of all a school of exegesis. The great bulk of their founding text---the \emph{M\={\i}m\=a\d ms\=a S\=utra}, set down in the 1st or 2nd century AD and credited to one Jaimin\={\i}---treats interpretative difficulties in the Vedas, discusses their study, and details rules for settling procedural questions regarding sacrificial ritual. But by the first century, the heyday of ritual was already long past, and there was much skepticism as to whether M\={\i}m\=a\d ms\=a's constitutive aim---according to \emph{MS} I.i.1 the inquiry into \emph{dharma}---was best pursued in the M\={\i}m\=a\d ms\=a fashion. There were rival religious traditions: Buddhists and Jains, as well as the plainly nihilistic C\=arv\=akas. And there were rival schools of Vedic exegesis, and many Hindus who, though acknowledging the authority of the Vedas, attributed them to gods---unlike the atheistic M\={\i}m\=a\d ms\=akas, who claimed that the scriptures had simply always existed. 


% (i.e.\ pursuit of the highest good)

So it would have seemed appropriate to establish some philosophical foundation for the school. And indeed several epistemological theses are set out in the first chapter of the \emph{M\={\i}m\=a\d ms\=a S\=utra}:\begin{squote}\e{MS} I.i.4.\ Perception is knowledge which one has by the senses coming in contact with the soul. It is not the cause of duty [\emph{dharma}] by reason of acquiring knowledge of the thing existing.\footnote{Trans.\ \citet{jaimini1923mimamsa}. For style, clarity, or to remove or add glosses, I've lightly altered most of the older translations I've quoted. Meaning has not (I hope) been affected.}\end{squote}Indian texts are characteristically cryptic. Pithily rendered verses are meant to be committed to memory and elucidated orally by teachers, or studied with the aid of commentaries. And the \emph{M\={\i}m\=a\d ms\=a S\=utra} is difficult even by these standards. The verse is obviously invoking some psychological theory---though written around the 1st century, the \emph{MS} codifies ideas of much great antiquity.\footnote{For a survey of the school see e.g.\ \citet{keith1921karma}.} But we must be satisfied---as M\={\i}m\=a\d ms\=akas of later generations themselves had to be---with the text itself and the only slightly less difficult 5th-century commentary of \sabara.

Nevertheless the import of the verse is reasonably clear (and this will follow \sabara). For M\={\i}m\=a\d ms\=akas, \emph{dharma} concerns a causal system relating actions and future outcomes, so that to perceive \emph{dharma} would be to perceive the future. And this verse points out that sense perception makes contact only with what exists in the present. So one cannot learn about \emph{dharma} through perception (as the Buddha was purported to have done).

But, happily, \emph{dharma} is attested to by the Vedas:\begin{squote}\emph{MS} I.i.5. Certainly there is eternal connection between the word and its meaning; its knowledge is injunction; it is never erroneous in matters invisible; it is authoritative in the opinion of B\=adar\=aya\d na\footnote{Credited with the \emph{Brahma} (or \emph{Ved\=anta}) \emph{S\=utras}.} by reason of its not depending on others.\end{squote}This is yet more puzzling than the previous verse, and \sabara\ offers little more than a paraphrase.\footnote{\emph{\'S\=abarabh\=a\d sya} \emph{ad loc.} Though he notes that M\={\i}m\=a\d ms\=akas must still rule out other potential ways of knowing \emph{dharma}, for example by inference or analogy.} We seem to be presented with the claim that testimony (i.e.\ Vedic testimony) is infallible with regard to ``matters invisible'' (specifically \emph{dharma}).\footnote{The restriction to matters invisible was taken seriously by M\={\i}m\=a\d ms\=a: what the Vedas said about anything other than \emph{dharma} was reinterpreted or disregarded.\label{invisiblenote}} The reason for this is, apparently, that Vedic testimony about \emph{dharma} does not depend upon ``others'': it does not ultimately rely on any of the other legitimate sources of knowledge catalogued by \small\'S\normalsize \=abara.\footnote{I leave aside the part about the eternal connection between words and meanings, which would take us too far afield. Suffice it to say that the M\={\i}m\=a\d ms\=akas spilt a good deal of ink developing an ontology and a philosophy of language to make room for the idea of testimony that is authoritative but authorless. See \citet{tabereternality}.} In other words, testimony, which in the ordinary case---and M\={\i}m\=a\d ms\=a joins every other Indian school on this point---can only convey knowledge which was originally acquired in some other way, is, in the special case of \emph{dharmic} knowledge, an infallible authority, precisely \emph{because} no other means of acquiring knowledge will do the job. And thus:\begin{squote}\e{MS} I.i.2.\ Duty is a purpose having injunction for its sole authority.\footnote{Trans.\ \citet{jha1907skb}.}\end{squote}

Now we don't have much idea how these ideas might have been fleshed out in the earlier centuries of the M\={\i}m\=a\d ms\=a school. We ought also to allow for the distance between ourselves and these medieval atheistic scriptural exegetes who take the authorless but eternal and transcendent texts of the Hindu Vedas as the only possible authority for their ritualistic morality. (Though atheists do still walk among us.) We note in particular an astonishingly `atemporal' Indian attitude towards literature in general. Thus \begin{squote}we can read thousands of pages of Sanskrit on any imaginable subject and not encounter a single passing reference to a historical person, place, or event---or at least to any that, historically speaking, matters.\footnote{Thus \citet[p.\ 606]{pollock1989mima}, who argues that M\={\i}m\=a\d ms\=a itself encouraged the rejection of history. But there was already fertile soil, e.g.\ in the cyclical conception of the universe.}\end{squote}The idea that language has a natural or eternal status seems indeed to have been anticipated in earlier grammatical texts.\footnote{\citet{tabereternality}.} The \emph{M\={\i}m\=a\d ms\=a S\=utra}'s bold epistemology of \emph{dharma} is somewhat more intelligible given such facts and the attendant conceptions of the cosmos and our place therein.


%the names of the authors of many Indian texts are little more than conventions

%that the names of the authors of many Indian texts are little more than conventions, and that India worked hard not to record any history:


Evidently this doesn't vindicate M\={\i}m\=a\d ms\=a epistemology. Nor were the philosophers of other Indian schools much impressed. There can still be little wonder, then, if even one of the most serious and sympathetic recent scholars of Indian philosophy has found in this aspect of M\={\i}m\=a\d ms\=a ``a kind of  fundamentalism.''\footnote{\citet[p.\ 32]{matilal1986perception}; see \citet[pp.\ 589--91]{arnold2001ivr} on the modern reception of the doctrine of \emph{svata\d h pr\=am\=a\d nya}.}

\

Enter Kum\=arila Bha\d t\d ta, a 7th century M\={\i}m\=a\d ms\=aka. His \emph{\small\'S\normalsize lokav\=artika}, comprising well over three thousand verses, corresponds to just the first thirty-two verses of the \emph{M\={\i}m\=a\d ms\=a S\=utra}---i.e.\ those verses which had set out the philosophical basis of M\={\i}m\=a\d ms\=a. In this paper we are primarily concerned with the second chapter of the \emph{\small\'S\normalsize lokav\=artika}---the \emph{Codan\=as\=utra}, or chapter on injunction---which is framed as a commentary on \emph{MS} I.i.2.\footnote{Testimony (\emph{\'sabda}) is sometimes treated under the heading of `injunction' (\emph{codan\=a}) because the M\={\i}m\=a\d ms\=a school is concerned with Vedic testimony insofar as it offers prescriptions for living, whether in a grammatical form or not (see also n.\ \ref{invisiblenote} above). In our passages, Kum\=arila is explicitly treating \emph{\'sabda} in general (vv.\ 7--8).} Kum\=arila defends the authority of the Vedas on matters of \emph{dharma} by way of a quite general epistemological thesis, the `\emph{svata\d h pr\=am\=a\d nya},' or `intrinsic validity' thesis:\begin{squote}\e{CS} 47--8.\ The validity of all valid cognitions is to be understood as intrinsic, since a potency not existing intrinsically cannot be brought about by something else.\footnote{Trans.\ \citet[p.\ 207]{taber1992dkb}.}\end{squote} Kum\=arila is claiming that ``cognition'' (`\emph{j\~n\=ana}' approximates to `occurrent judgment') is (in some way) epistemically legitimate by default, or immediately. This will ultimately support the authority of the Vedas, since the ideas conveyed by Vedic testimony will also have immediate legitimacy.\footnote{Thus v.\ 68: ``Then, too, in the case of the Veda, the assertion of \emph{freedom from reproach} is very easy to put forward, because there is no speaker in this case; and for this reason the unauthoritativeness of the Veda can never even be imagined.''} And there is no danger that anything will undermine this legitimacy, since in principle no other mode of knowledge acquisition (no other \emph{pram\=a\d na}, in the Sanskrit terminology) provides knowledge of \emph{dharma}.\footnote{vv.\ 13--8.}

%The interpretation of this thesis occupies the bulk of this paper. But it is uncontroversial that

%\footnote{I have appended Jh\=a's translation of vv.\ 1--88 of the \emph{Codan\=as\=utra} to the end of this essay.}


Given its role in \kum's defense of Vedic authority, the doctrine of \svapra\ has understandably incurred suspicion. But, as John Taber has pointed out, the thesis doesn't in its own right entail the authoritativeness of the Veda.\footnote{\citet[p.\ X]{taber1992dkb}.} That requires accepting the Vedas' authorlessness, which demands a good deal of theorizing about language and the ontology of reference, the nature of the other \emph{pram\=a\d nas}, and so on; this is also Kum\=arila's own understanding of the dialectical situation. So one could accept the M\={\i}m\=a\d ms\=a school's most general epistemological theses while rejecting Vedic authority. And I believe Taber is also right in arguing that, whatever its provenance, the \emph{svata\d h pr\=am\=a\d nya} thesis has merit.

%But I'm also inclined to go further. If we may take \emph{dharma} as corresponding roughly to \emph{morality}, or to practical concerns of the highest order,\footnote{The difficulties here will be roughly analogous to the difficulties in comparing modern morality theory with ancient (Greek) ethical theory.} then most elements of the M\={\i}m\=a\d ms\=a outloook, \emph{including} the relationship between epistemology and \emph{dharmic} (or moral) authority, are not terribly foreign. The M\={\i}m\=a\d ms\=a see \emph{dharma} as a casual system which, properly understood, allows one to realize one's ultimate good. That idea that moral or ethical life allows one to realize one's good has had an enormous influence in Western thought; the suggestion that morality is a system of broadly instrumental rules is not foreign even to the contemporary discussion.\footnote{I think especially of Philippa Foot's ``Morality as a System of Hypothetical Imperatives'' \citep{foot1972morality}---though she has repudiated that view, for example in \citet{foot2003natural}.} A general worry about how moral epistemology is possible, manifest in the M\={\i}m\=a\d ms\=a attacks on other philosophical schools as well as in their own recourse to scripture, is a pervasive theme in contemporary metaethics.\footnote{The classic statement being \citet{mackie1977ethics}, though he is much motivated by the purportedly non-instrumental nature of moral demands.} The view that moral properties are in some way non-natural, and would thus require some special variety of epistemological insight, likewise retains enormous influence.\footnote{See again \citet{mackie1977ethics} for the skeptical view. For the positive view, \citet{moore1903principia} is the \emph{locus classicus}; cf.\ more recently \citet{huemer2005ethical}. But the terms of metaethical debate of the last century \emph{in general} were set by Moore.} The idea that moral knowledge, or some moral knowledge, has ultimately a divine provenance also has its contemporary defenders. None of this is to say that these positions must therefore be plausible (I myself see nothing at all persuasive in some of these views), and there is also the (hardly trivial) matter of combining these various ideas into a coherent theory. The point is rather this: once we realize that the only element of the M\={\i}m\=a\d ms\=a view that will not be familiar from even a relatively limited exposure to moral philosophy in a contemporary philosophy department is the specific means by which moral knowledge is acquired (namely the Vedas conceived of as authorless---and this could be thought of as an accident of history) then we can understand M\={\i}m\=a\d ms\=a epistemology as, in part, constituting an attempt to reconcile a general epistemological theory with a moral epistemology. The M\={\i}m\=a\d ms\=akas will by no means have been the first to espouse a conception of knowledge that answers to practical as well as theoretical considerations; the thought that what constitutes reasonable belief may well depend in part on the demands laid upon us as practical creatures is again a familiar one, and this time it's surely a reasonable one as well.\footnote{See further \citet{clooney1987vhn} and \citet{tabereternality}.}


I believe, though, that a satisfactory interpretation of \kum's thesis has yet to be offered. In the next section, \S\ref{interpreting}, I'll explain why some prominent interpretations are inadequate, and identify some constraints for an adequate interpretation. In S\ref{validity} I'll develop my own interpretation, and in \S\ref{textualinterpretation} I do the exegetical work. Now as Taber says, the \emph{svata\d h pr\=am\=a\d nya} thesis \emph{can} be considered independently of the matter of moral knowledge. II'll do just that in \S\S\ref{validity}--\ref{textualinterpretation}. But to appreciate the theory's merits and motives, it is necessary to discuss Kum\=arila's epistemological motives more broadly, which I'll do in \S\ref{conception}. In particular, we can understand \kum\ as attempting to reconcile a general epistemological theory with a moral epistemology, and as espousing a conception of knowledge that answers to practical as well as theoretical considerations. 


%-----------------------------------------------------------------------------------------------
%---------------------------2. INTERPRETING KUMARILA‘S EPISTEMOLOGY-----------------------------
%-----------------------------------------------------------------------------------------------

\section{Interpretative Approaches and Contraints}
\label{interpreting}

Let us return to Kum\=arila's thesis: \begin{squote}\e{CS} 47--8.\ The validity of all valid cognitions is to be understood as intrinsic, since a potency not existing intrinsically cannot be brought about by something else. And [in general] things depend on [other] causes in arising, but once they exist they exercise their functions by themselves.\footnote{vv.\ 47--8; cf.\ e.g.\ vv.\ 6, 80.}\end{squote} This tells us less than we might suppose. The commentarial nature of the Indian philosophical tradition means that conspicuous novelty is avoided. As such, all of the members of a given school will defend nominally the same answer to what is nominally a single question of inter-school interest. In fact different schools, and different writers within a school, may be answering different questions.\footnote{Cf.\ \citet[p.\ 1]{mohanty1989gangesa}.} Likewise nominally identical theses may not at all mean the same thing to different writers. The \emph{svata\d h pr\=am\=a\d nya} thesis is a case in point.





%\subsection{Interpretative Issues} \label{interpretativeissues}


%Some difficulties in reconstructing Kum\=arila's epistemology, including the extraordinary concision of the texts, are characteristic of Indian philosophy. That concision allowed plenty of scope for commentaries on Jaimin\={\i} and \small\'S\normalsize \=abara to rival Kum\=arila's; later commentators on Kum\=arila were likewise able to differ greatly in interpreting \emph{him}.



Every school had a position on whether \emph{pr\=am\=a\d nya} was \emph{svata\d h} or \emph{parata\d h} to cognition. In English this is often glossed as a question about whether the `truth' or `validity' (\emph{pr\=am\=a\d nya}) of a cognition is intrinsic (\emph{svata\d h}) or rather extrinsic (\emph{parata\d h}) to that cognition. A scholar will sometimes begin a survey by saying that the Ny\=aya school holds that truth is extrinsic to cognitions---where the Ny\=aya view is roughly that there is never a feature of a cognition which establishes that cognition's accuracy---and then go on to explain how the Buddhists think that both truth \emph{and} falsity are extrinsic to cognitions, even though that would be an absurdity on the Ny\=aya understanding of `truth.' Buddhists can say that \emph{pr\=am\=a\d nya} and \emph{apr\=am\=a\d nya} (`invalidity' or `falsity') are both \emph{parata\d h} because they have a quite different understanding of \emph{pr\=am\=a\d nya} in the first place.

Buddhists and Naiy\=ayikas are different enough that no-one is much confused about this, but M\={\i}m\=a\d ms\=a is closer to Ny\=aya and less understood, and confusion about the meanings of theses is a real danger. That there is in fact little value in saying merely that the M\={\i}m\=a\d ms\=a believe that \emph{pr\=am\=a\d nya} is \emph{svata\d h} is brought out by the observation that there were plainly different understandings of the \svapra\ thesis in the different branches of M\={\i}m\=a\d ms\=a, and even among Kum\=arila's own later followers.\footnote{So for example Prabh\=akara before Kum\=arila, and Umbeka, P\=arthas\=arathi, and Sucarita after him---see \citet{chatterjea2003svatah} and \citet{taber1992dkb} for brief accounts.} Still less can the standard English gloss for the thesis---`intrinsic validity'---be supposed to convey much about \kum's position. (But this gloss for ``the'' M\={\i}m\=a\d ms\=a thesis has a long history, and is well-represented in recent discussions of \kum, so it would be silly to coin a new phrase now.\footnote{The usage goes back at least to \citet{keith1921karma} and \citet{radhakrishnan1927indian}, and is employed in the recent discussions of \kum\ offered by \citet{taber1992dkb} and \citet{arnold2001ivr}.})

%\subsection{Rival Interpretations} \label{rivalinterpretations}


Nevertheless \kum's reasoning in vv.\ 47--8 allows us to eliminate at least some possible readings of `validity.' Validity, which is intrinsic, contrasts with invalidity, which is extrinsic. In the following verses, Kum\=arila supports these claims with a regress argument: if cognitions were not \emph{intrinsically} valid, they would have to derive validity from further cognitions. Those other cognitions must themselves be valid if they are to help, but then, by the hypothesized extrinsic nature of validity, there would have to be yet further cognitions validating \emph{them}, and so on \emph{ad infinitem.}\footnote{E.g.\ vv.\ 49--51.} Thus validity must be intrinsic. But while other cognitions cannot make a cognition valid, they \emph{can} make it \emph{invalid}, and hence invalidity is extrinsic.\footnote{E.g.\ vv.\ 52--3, 56, 86--7.} Now for Buddhist philosophers, the world is made up of fleeting particulars, and the association or categorization of these particulars which is found in thought is \emph{ipso facto} a falsification. The notion of `validity' on the Buddhist view, then, if it is to have utility, cannot be understood in terms of a correspondence between mind and world, but will rather be made out in pragmatic terms: a cognition is valid if the expectations it generates are not later frustrated. To say that validity is \sva---in some way internal to the cognition itself---rules out at least that view.\footnote{Even with Kum\=arila it's not trivial to rule out a pragmatic understanding. Karl \citet{potter1984die, potter1992presuppositions} has argued that `knowledge' has \emph{always} more a pragmatic than a theoretical import in Indian philosophy---but see \citet{mohanty1984pramanya}. A pragmatic understanding of validity was read into the \emph{SV} itself by Sucarita Mi\'sra, a later M\={\i}m\=a\d ms\=a commentator---see \citet[p.\ 52--3]{chatterjea2003svatah}. My own interpretation gives a role to pragmatic considerations; cf.\ also \citet[pp.\ XX]{bhatt1962epistemology}.} But that doesn't narrow the options enormously.

\subsubsection*{Validity as truth}

A popular interpretation of Kumarila's thesis has been this: cognitions are true unless shown otherwise.\footnote{Assuming a correspondence understanding of truth. But scholars of Indian philosophy also often use `truth' as a label of convenience, and thus may describe Kum\=arila's thesis in these terms without subscribing to the interpretation now under consideration.}  (Alternatively, validity may only \emph{entail} truth.) K. K. Dixit, who adopts this interpretation, responds as one might then expect:\begin{squote}the difficulty with Kum\=arila's argumentation is that a piece of cognition not proved to be invalid is not necessarily valid, it might be valid but it might as well be otherwise.\footnote{\citet[p.\ 5]{dixit1983slokavarttika}; for validity as truth see also e.g.\ \citet{chatterjea2003svatah}, and \citet{hiriyanna1932outlines}; among traditional authors see e.g.\ Ga\d nge\'sa, \emph{PV} [XX]. \citet{bhatt1962epistemology} interprets validity as truth-entailing, but understands intrinsicality in a way that would avoid Dixit's complaint; classically, cf.\ Umbeka \emph{SVVT} [XX].} \end{squote} That would indeed seem to settle the matter. But this is an uncharitable interpretation, and is anyway at odds with the text: Kum\=arila several times allows that a false cognition may be valid.\footnote{Thus \e{CS} v.\ 6 (trans.\ Jh\=a): ``Even the unauthoritative Means would, by itself, lead to the conception of its object, and its function could not cease unless its falsity were ascertained by other means.'' Cf.\ e.g.\ vv.\ 85--6.}

\subsubsection*{Validity as Commitment}

John Taber has defended another interpretation, the basic lines of which he attributes to P\=arthas\=arathimi\'sra, another M\={\i}m\=a\d ms\=a commentator.\footnote{\citet{taber1992dkb}.} According to Taber, validity is not truth but rather as it were \emph{commitment:} ``every cognition is accompanied by an initial sense of conviction; one is initially inclined to believe of every cognition that it is true.''\footnote{p.\ 216; cf.\ \citet[p.\ 203]{taber1987review}, \citet[p.\ 172]{taber2002mo}, and \citet[p.\ 2]{taber2005hindu}. \cite{arnold2001ivr, arnold2001ivs} is broadly supportive of the Taber-P\=arthas\=arathi interpretation. For other suggestions in terms of presumptive or \emph{prima facie} truth see \citet[p.\ 18]{keith1921karma}, \citet[pp. 372--5]{dasgupta1957history}, who offers a coherentist spin, and \citet[pp.\ 131ff.]{bhatt1962epistemology}. \citet[p.\ 52]{chatterjea2003svatah} and \citet[p.\ 8]{mohanty1989gangesa} also suggest that---at least in P\=arthas\=arathi's case---the claim is a psychological one. On P\=arthas\=arathi's interpretation cf.\ \citet[pp.\ 50--5]{chatterjea2003svatah} and \citet[pp.\ 129--45]{bhatt1962epistemology}.} But Taber rightly allows that Kum\=arila is concerned not only with psychology but also with what is \emph{reasonable}. Verse 60 tells us that when a cognition is not undermined, then ``there is no reasonable ground for doubt.''\footnote{Cf.\ \emph{TS} 2872--4, noted by Taber.} So we are supposed to be \emph{right} in accepting our beliefs as a guide to how things actually are. Taber incorporates this point into his interpretation as follows. \begin{squote}One continues to be so inclined as long as the cognition is not called into question. Almost certainly, however, if the cognition is never overturned, it \emph{is} true. Falsehood cannot conceal itself forever. If, over the long run, the cognition is not shown to be false, then on the basis of its initial, intrinsic validity one is certainly \emph{justified} in believing that it is not false, that it is really true.\footnote{p.\ 216, his emphases.}\end{squote} This, for Taber, shows Kum\=arila adopting a kind of hard-headed empiricism: one may believe only that for which one has evidence; but, in addition, to have \emph{no} evidence for something (to have no evidence that your view is wrong, for example) is to be warranted in thinking that no such evidence exists.\footnote{pp.\ 205, 216.} To ask for more is in any case futile, since the demand only leads to regress.

%So we are supposed to be \emph{right} in accepting our beliefs as a guide to how things actually are. ``In the end,'' says Taber (to give a fuller version of the quote just above), \begin{squote}it appears that Kum\=arila adopts a common sense position. Every cognition is accompanied by an initial sense of conviction; one is initially inclined to believe of every cognition that it is true. One continues to be so inclined as long as the cognition is not called into question. Almost certainly, however, if the cognition is never overturned, it \emph{is} true. Falsehood cannot conceal itself forever. If, over the long run, the cognition is not shown to be false, then on the basis of its initial, intrinsic validity one is certainly \emph{justified} in believing that it is not false, that it is really true.\footnote{p.\ 216, his emphases.}\end{squote}

Taber's is one of the most sympathetic and careful interpretations of Kum\=arila on intrinsic validity, and there is much to like about it. But how can we take seriously the idea that a cognition never undermined is ``almost certainly'' true, a view that Taber says is ``common sense''? Dixit's comment is again apt: it might be true and then again it might not. Suppose, for instance, that, from a distance, I mistake for a woman a man whom I am unlikely to encounter again, and would not recognize if I did. The passage of time is unlikely to expose my error. This will often be a problem in practical life as well: it is often hard to know whether one has acted well. I could not possibly have a general expectation that advice---to buy fair-trade coffee, say---will, if bad, always, or almost always, ultimately be recognized as such. Taber himself actually raises roughly the same worry about Kum\=rila's defense of the Vedas, namely that they seem to come out unscathed only by default.\footnote{p.\ 217.} But he seems to think that the worry is largely \e{restricted} to the Vedic case, which, as my examples show, is not the case.% Moreover, even if the \svapra\ thesis is separable from the doctrine of Vedic authority, it seems strange that an interpretation of \kum's theory should result in Vedic knowledge faring so badly, when that is his chief concern.\footnote{A complicating factor here is the M\={\i}m\=a\d ms\=a adherence to the view (very contentious in the Indian setting) that `absence' (\emph{abh\=ava}) is a sound means of acquiring knowledge. That is, one may rightly infer from the absence of any indication of something that it doesn't exist. But this principle makes no appeal to the passage of time, and Kum\=arila is liberal and reasonable about what counts as an indication of a real or potential difficulty about evidence. Taber does not rely on an appeal to \emph{abh\=ava} in his defense of \emph{svata\d h pr\=am\=a\d nya}.}



%Even if there are independent arguments---from the threat of regress, most obviously---it does not seem to me that we do Kum\=arila's view much of a service if the realm of knowledge of most interest to him fares particularly poorly on his own epistemological theory.






%[[A POINT FROM CHATTERJEA 51: it;s the idea of absence as a valid source of cognition that allows K to say what he does --- but note that clearly we'll have to treat validity in this case in terms of warrant or validity and not truth; because obviously sometimes something \emph{does} come up later.]]

The difficulty here is particularly acute for Taber because, although in vv.\ 47--8 \kum\ says that the validity of \e{valid} cognitions is intrinsic, Taber's view that \emph{all} cognitions are intrinsically valid, at least initially. This creates the impression that \emph{every judgement} has \emph{prima facie} justification or warrant. There is no good reason, on this view, why just any judgment should not be warranted, even if it's formed on the basis of a dream or wishful thinking. Taber may say that Kum\=arila's defense of the Vedas is unsuccessful because it makes Vedic injunctions valid just by default, but really that's tantamount to an objection to the \emph{svata\d h pr\=am\=a\d nya} thesis itself, on Taber's interpretation thereof. The whole \emph{point} would be to make cognitions justified by default, despite the fact that many types and instances of judgements---including quite empirical judgements---cannot be supposed likely to be corrected if faulty.\footnote{Taber (p.\ 215, n.\ 56) cites v.\ 80 in support of the idea that the passage of time improves confidence, but while the verse shows that Kum\=arila is trying to make more than a psychological point, I cannot see in it any indication that he is thinking about the passage of time, except in the trivial sense that a counter-indicating cognition would come after the initial cognition. Taber also adduces \emph{TS} 2904, but again I see in that verse only the claim that cognitions are valid if they're not upset by other cognitions. Kum\=arila refers in this verse to cognitions occurring at ``other times and places,'' but that doesn't mean that length of time (or travel experience, for that matter) makes a difference to warrant except insofar as counter-indication cognitions would, by their nature, have to occur at other times.\label{tabertextual}}

Taber's interpretation draws attention to crucial but often overlooked features of Kum\=arila's epistemology, including the lack of coincidence between truth and validity and the importance of warrant or justification. But Taber does not explain how warrant enters the picture; the interpretation falls into the same trap that P\=arthas\=arathi's does, and which other scholars have noted in that context: the theory reduces to a theory of judgement.\footnote{Thus \citet[p.\ 8]{mohanty1989gangesa} and \citet[p.\ 52]{chatterjea2003svatah}.} Taber has provided an argument for his claim that a cognition ``compels our assent''\footnote{p.\ 216.} only if that means merely that judgment \emph{entails} assent. We are offered no reason to suppose that it would be unreasonable to stop believing something upon realizing that one does in fact believe it, if one cannot then adduce some evidence for it. But Kum\=arila explicitly draws the conclusion that this \e{is} unreasonable.\footnote{v.\ 60; cf.\ vv.\ [XX], \emph{TS} 2872--4. Taber himself acknowledges the importance of this point.\label{reasonableness}}


%, Taber's attempted defense of Kum\=arila's epistemology must be judged a failure.

Moreover, while Taber is right to say that the \svapra\ thesis is separable from the doctrine of Vedic authority, the thesis is plainly motivated in great part by a desire to defend the possibility of Vedic (or `moral' or `practical') knowledge. Even if there are independent arguments---from the threat of regress, most obviously---it does not seem to me that we do Kum\=arila's view much of a service if the realm of knowledge of most interest to him fares particularly poorly on his own epistemological theory.\footnote{A complicating factor here is the M\={\i}m\=a\d ms\=a adherence to the view (very contentious in the Indian setting) that `absence' (\emph{abh\=ava}) is a sound means of acquiring knowledge. That is, one may rightly infer from the absence of any indication of something that it doesn't exist. But this principle makes no appeal to the passage of time, and Kum\=arila is liberal and reasonable about what counts as an indication of a real or potential difficulty about evidence. Taber does not rely on an appeal to \emph{abh\=ava} in his defense of \emph{svata\d h pr\=am\=a\d nya}.} Besides which, it is hard to imagine that Kum\=arila would rest his thesis on the idea that the passage of time makes a judgement more likely to be correct when this line of thought so clearly could not, \emph{in principle}, apply to the Vedic case, and which, indeed, would make validity \emph{extrinsic} in an obvious sense, since it would accrue with time. But Taber's interpretation relies on this claim, for which the textual evidence is ambiguous at best.\footnote{See n.\ \ref{tabertextual}.}



%Nevertheless, Taber \emph{does} draw attention to important and neglected points in Kum\=arila's thought. By drawing out yet more critical features of the view, I believe it is possible to show that Kum\=arila's view is genuinely compelling.

%\subsection{Interpretative Constraints} \label{constraints}

\

Two further points, besides those to which Taber draws attention, will help in constructing a better interpretation. The first stems from a bit of a puzzle. As noted, it seems that, on Taber's interpretation, any judgement or belief whatever is supposed to be possessed of validity. He insists that \emph{all} cognitions are valid, unless sublated or otherwise brought into doubt by some further cognition---at which point they are not really cognitions anymore at all, because they have been abandoned psychologically. And that is a common reading of Kum\=arila. But the M\=im\=a\d ms\=as, like other schools, provide a list of the sources of knowledge or valid cognition---for the M\=im\=a\d ms\=as the list includes, among other things, perception, inference, and testimony.\footnote{The six \e{pram\={a}\d{n}as} of (Bh\={a}\d{t}\d{t}a) \mim\ are perception (\e{pratyak\d{s}a}), inference (\e{anum\={a}na}), comparison (\e{upam\={a}na}) testimony (\e{\'{s}abda}), postulation (\e{arth\={a}patti}), and absence (\e{abh\=ava}).\label{thesix}} But what would be the need for such a list if every cognition is automatically valid?

%\footnote{p.\ [XX].}

In fact Kum\=arila does \emph{not} think that every cognition is valid---mere fancies or wishful thoughts, for example, are not valid.\footnote{vv.\ 94--5; cf.\ also vv.\ 85 and 128.} Kum\=arila also notably excludes memorial judgements from the realm of validity---only cognitions or judgments with novel contents are apt for validity.\footnote{\emph{SV} II.22--3, V.11.} And although memories are, on Kum\=arila's view, generally recognizable as such, it isn't \e{always} obvious whether a judgment is memorial, since on the \mim\ view mistaken judgements---formed, say, in a dream---\e{are} effectively memorial events or collocations of memorial elements. So we cannot say that every judgement has intrinsic validity, or that any commitment we have is warranted barring some counter-indication. In which case we'll need to find a way to understand how cognitions can be \emph{intrinsically} valid without severing the connection between \emph{valid sources of cognitions} and \emph{valid cognitions}.


%\footnote{[SL CITE]; [SECONDARY CITE].}

%RE MEMORY BEING IDENTIFYIABLE AS SUCH\footnote{[CITE - CHECK BHATT 90--3]}

The second point is this. The M\={\i}m\=a\d ms\=as are strict about distinguishing the mental state which provides some awareness of the world (I will call this a first-order cognition) from the mental state which provides awareness of that first-order cognition (I'll call this a second-order cognition).\footnote{\emph{SV}, \emph{\'S\=unyav\=ada} v.\ 64ff.; for the school see ~\citet[pp.\ 20--1]{keith1921karma} and citations there.} Some Indian schools are in accord with Descartes: a cognition is itself present to consciousness as it reveals its object. Not so M\={\i}m\=a\d ms\=a.\footnote{See \citet[pref., chs.\ 4, 10]{sen1984concept} for discussion.} But the implications of this point for the \emph{svata\d h pr\=am\=a\d nya} thesis have not been properly appreciated. In effect, we have \emph{two} theses.\footnote{cf.\ \citet[p.\ 41]{chatterjea2003svatah}.} One thesis concerns the status of the first-order cognition, and the second concerns the apprehension of that status by a second-order cognition. It may then be, for example, that only some cognitions \e{are} valid, but that every cognition, when apprehended, is \e{taken to be} valid. 

In my view the failure to appreciate this second point is partly responsible for the failure to appreciate the first. Tara Chatterjea, for example, though observing that the \svapra\ thesis is really two theses, says that Kum\=arila makes ``contradictory suggestions'' concerning the scope of the \svapra\ thesis, and identifies v.\ 53 as apparently claiming that \emph{all} cognitions are intrinsically valid, in contrast with e.g.\ v.47 which says only that \emph{valid} cognitions are intrinsically valid.\footnote{\citet[p.\ 51]{chatterjea2003svatah}.} (Taber also relies on v.\ 53 in claiming that all cognitions are valid.\footnote{\citet[p.\ 171, n.\ 27]{taber2002mo}, \citet[p.\ 212]{taber1992dkb}.}) But if v.\ 47 concerns the status of a cognition while v.\ 53 concerns the \emph{assessement of} the status of a cognition, then that is too quick. We may thus hope that care regarding these two points will make an adequate interpretation possible.


%[PUT THE MANIFESTATION VS VALIDITY THING HERE]


%\

%Let us briefly review the interpretative contraints identified so far. A successful interpretation of \kum's \sva\ \pra\ thesis will: allow that a valid cognition may be false or inaccurate; explain why not just any judgment is validity-apt, and explain the relationship between valid cognitions and valid cognition-producing mechanisms; make sense of \kum's normative as well as pscyhological claims; distinguish the roles of first- and second-order cognitions; not conflict with \kum's particular interest in moral or Vedic knowledge. If \kum's theory is incoherent, or the text inconsistent, then it may not be possible to satisfy all of these contraints. But I believe that these contraints can all be satisfied simultaneously, and that no interpretation has so far done so only because these contraints have not thus far been fully accounted for. 








%-----------------------------------------------------------------------------------------------
%--------------------------------3. KUMARILA ON INTRINSIC VALIDITY------------------------------
%-----------------------------------------------------------------------------------------------


\section{A New Interpretation of Kum\=arila's Intrinsic Validity}
\label{validity} \label{interpoutline}


Since the \cod\ is dense and difficult, I begin by describing my interpretation of \kum's \sva\ \pra\ thesis in general terms. Then, interpretation in hand, I'll take up exegesis in \S\ref{textualinterpretation}.

\

`Validity' has, for Kum\=arila, as for most Indian philosophers, both a causal (or psychological) aspect and also a normative aspect; in India, as in the West, different understandings of perception were taken to have implications for what we can know, and epistemological commitments motivated views about mind and perception.

Beginning with the psychological side, consider the idea, common in the history of philosophy, that experience confronts us with certain \emph{appearances} or \emph{ideas} or \emph{impressions} (though these appearances need not always be pictorial or visual in nature). One may hope that these appearances at least sometimes resemble or represent the world in some way. Locke, for example, thought that they sometimes did, and that we could sometimes be confident of this; the Stoics had a comparable view. Certain ancient skeptics thought that indeed our impressions \emph{might} resemble or represent the world, but they argued that we can never know whether---and should not believe that---they do. Hume and Kant, in different ways, decided that the whole question was nonsense. Parallel positions existed in India concerning what we have called `cognitions.' As in the early modern case, skepticism was more threat than reality, but Locke, Hume and Kant have their counterparts among Naiy\=ayikas, Buddhists, and idealists. And, as a start, we can think of Indian cognitions as being roughly equivalent to the West's appearances or impressions.

%[EMPHASIZE: K HAS A THEORY OF PERCPETION AND I SHOULD TRY TO STAY COMPATIBLE WITH IT, BUT I'ME NOT TALKING ABOUT PERCEPTION MYSLEF BUT A BROADER THING.]

One might think of cognitions or appearances as offering themselves for either acceptance or rejection. This is the view of some of Kum\=arila's opponents---hence an objection considered by Kum\=arila to the effect that cognitions do not present themselves as authoritative.\footnote{v.\ 85.} For Kum\=arila himself, on the other hand, a cognition does not simply present the cognizer with a view or thought which he might, but equally well might not, accept as his own. For \kum, to form a cognition is, as Taber says, already to view the world in the way portrayed by the cognition.\footnote{``The intrinsic validity of cognitions is vested for the M\={\i}m\=a\d msaka in the sense of conviction with which they typically arise.'' \citet[p.\ 172]{taber2002mo}.}

In \kum's case, the language of `appearance' may in fact easily mislead. Taber employs such language too uncritically when he sees in Kum\=arila ``a kind of rigid philosophical realism which insists strictly on the reality of appearances,'' and when he says that the ``emphasis on the inability to distinguish true from false cognitions expresses what seems an almost skeptical attitude on the part of \kum, viz., that we can never establish with absolute certainty that a cognition corresponds with reality.''\footnote{\citet{taber1992dkb}, pp.\ 206 \& 215; cf.\ pp.\ 207, 216, 218, 221.} This suggests that we stand back from, or in judgment of, cognitions, which \kum does not think is the normal case. It's not so much that skeptical questions can't be settled as that they don't arise, precisely because, for \kum---and this much is uncontroversial---our cognitions just constitute the way we see things without any prior need for the passing of evidential judgments.

Consider on the one hand a (representational) painting. One may appreciate it without supposing that what is represented there ever existed elsewhere. For most of us, on the other hand, a photograph or recording invites us to suppose that we are getting a bit of a glimpse of another time and place. Only when something seems off about the photograph do we feel the need to inspect it more carefully---when what is displayed seems very improbable, say, or when a face on the cover of a magazine is suspiciously free of wrinkles. But Kum\=arila's cognitions are not like paintings \emph{or} photographs. A photograph, like a painting, is something which stands apart from own way of seeing things, and Kum\=arila does not believe cognitions are like that. A better analogy for Kum\=arila's cognitions would be a (very clean) window. One may realize that there is a window at the end of a hall because one has already seen through it to the trees beyond. If one realizes that there is a window, that is only because one has already looked through it. Likwise, the existence of a cognition is inferred from what is shown us by it, once we have already adopted it as part of our thinking.\footnote{\emph{SV} [XX]. Compare the familiar idea that, in order to figure out what one believes, one (often) looks not inward but outward, towards the world.}



%\subsubsection*{Intrinsic validity as a psychological thesis.} 

It was noted earlier that the `intrinsic validity' thesis is really \emph{two} theses, one about a first-order cognition and one about a second-order cognition. There are likewise two aspects of the thesis still considered as a psychological thesis, and we are now in a position to articulate these.\begin{description}
\item[\g{Ψα}] Cognitions are transparent.
\item[\g{Ψβ}] The existence of a cognition is inferred.\end{description} First, as regards the first-order cognition, the claim (\g{Ψα}) is that it is the nature of cognitions to provide a window on the world, rather than a picture of it---to have a cognition just is, at least in the basic case, to have a view of the world.  Regarding the second-order case, the point (\g{Ψβ}) is that one reads the existence of the first-order cognition off of the portion of the world thus revealed (you see the trees through the window, and because you can see trees you may infer that there is a window). The first-order cognition does not become an object of thought directly; it can be contemplated, indirectly as it were, without stepping back from the cognition or regarding it as something now in question, or as something which might now be rejected, since it is the very having of the view that makes it possible to recognize the cognition.\footnote{We must leave aside the details of Kum\=arila's theory of perception, which would also take us into his metaphysics and philosophy of language, but a translation of the relevant chapter of the \emph{SL} (the \emph{Pratyak\d sapariccheda}), along with extensive notes and discussion, has recently been provided by \citet{taber2005hindu}.}

Kum\=arila also claims that \e{invalidity} is \e{extrinsic} (\e{parata\d{h}}). If cognitions manifest the world rather than representing it, Kum\=arila nevertheless thinks that cognitions may err in various ways. We might again compare a window, which, even if clean and without impurities---so that we cannot inspect it directly---may yet be warped or tinted so that although we see the world through it and not in it, we see the world through it \emph{wrongly}. The thesis that invalidity is extrinsic emphasizes the point that even when a cognition portrays the world wrongly, we cannot consider it directly and note features of that cognition that are misleading, the way you could point out parts of a photo that may mislead. What you can do is form new cognitions, which may reveal that earlier cognitions were defective. Kum\=arila identifies two cases. First there is the case where you directly apprehend something as being different than you had thought: this is like looking out of a different window, seeing that it's bright outside, and realizing that the first window had been tinted. Second there is the case where you realize that the first cognition was formed in some defective way: this is like being told that the windows are tinted and realizing that it must then be brighter out than you thought.\footnote{\citet{kataoka2002validity} outlines the structure of invalidity; cf.\ \citet[p.\ 132]{bhatt1962epistemology}.}

The extrinsic invalidity thesis, regarded as a psychological thesis, may thus again be divided into two parts, corresponding to the first- and second-order cases respectively.\begin{description}
\item[\g{Ψα-}] Cognitions do not contain signs of their own defeciencies.
\item[\g{Ψβ-}] One can only appreciate the deficiencies of a cognition through a further cognition.\end{description} The first thesis is that there is nothing in a cognition itself which could indicate a problem with it---a problem can only be indicated by another cognition, or by something turned up by another cognition (either because it reveals the world differently or because it reveals that the first cognition was formed in a problematic way). And second, for the very same reason, one can only \emph{recognize} a problem with a cognition on the basis of what is turned up by some further cognition. The difference between the two cases is that, in the first case, one might never even notice that one had possessed a cognition that was problematic: one might adopt a new view without ever noticing an evolution in one's thinking, or one might wind up holding inconsistent views (i.e.\ one might retain the original view). In the second case, by contrast, the original cognition is \e{seen} to be defective. To see the old cognition as defective (as invalid) is already to have set it aside, and the new cognition which revealed the defectiveness of the first has thus become your new way of seeing the world. 

%\


%Again, this interpretation of \emph{svata\d h pr\=am\=a\d nya} differs from the interpretation according to which it is the thesis that a cognition or judgement is characteristically attended by conviction, or, perhaps better, according to which a cognition presents the world a certain way?\footnote{The second gloss is preferable, I think, since it does not suggest that cognitions are appearances or mental images.}

%So again, on my interpretation of \emph{svata\d h pr\=am\=a\d nya} 









%[NEED TO SORT THIS ALL OUT] This explanation of the \svapra\ thesis allows us to articulate the differences between \e{falsity}, \e{invalidity}, and \e{lack of capacity to manifest.} A cognition can be false or inaccurate without failing to manifest the word in the same way that a window can be warped without completely obstructing one's view. 

%[] that we distinguish between the validity of a cognition and the capacity of a cognition to manifest the world: a cognition can manifest  (whether accurately or not) [REITERATE EXPLANATION]. They are not the same, nor are validity and manifestation always found together. 


%[PUT SOME OF THIS IN PREV SECTION?] Now

Kum\=arila sometimes appeals to the fact that a cognition is present and manifesting the world in a certain way as a reason for saying that cognitions are intrinsically valid.\footnote{See e.g.\ vv.\ 6, 47, 85.} But Kum\=arila also says that even invalid cognitions can manifest the world, at least until the error is noticed.\footnote{v.\ 85, cf.\ vv.\ 25, 94--5.} This point seems to have been overlooked (i.e.\ by interpreters who think that \emph{all} cognitions are---at least initially---valid) primarily on account of a failure to distinguish the first- and second-order cases.\footnote{The other factor is perhaps the failure to see the connection between the validity of means of knowledge and the validity of cognitions.} Kum\=arila also assumes that once an error has been acknowledged, the erroneous cognition is set aside; thus a cognition will manifest the world in a certain way until it is recognized to be invalid. Thus again it has been supposed that validity and manifestation go hand in hand. But again this conflates the first- and second-order theses. A cognition could continue to manifest the world a certain way even after its invalidity has been revealed, if that invalidity has not been noticed (in the same way that you can learn something new without realizing that it undermines something else you believe). And there is another way that manifestation may come apart from validity. It is only cognitions derived from certain sources that can be valid: in particular valid cognitions derive from valid modes of knowledge. But you could see the world some way without having a view that is even apt for validity. (Memorial beliefs, for example, are not validity-apt.\footnote{\emph{SV} [XX].}) A cognition is intrinsically valid, then, neither in virtue of being accurate nor in virtue of manifesting (or pretending to manifest) the world. A cognition is intrinsically valid when it constitutes a genuine (though perhaps distorted) window on the world, in virtue of having a psychological or causal basis which allows it to be such a window. Cognitions arising from another source, however much they present or seem to present the world, and however accurately they do so, are not and cannot be valid, intrinsically or otherwise.


\


%\footnote{It may be that Kum\=arila actually has this case primarily in mind at v.\ 85.}

%\subsubsection*{Intrinsic validity as a normative thesis.} 

Although Kum\=arila generally describes cognitions and validity in psychological or causal terms, he takes his account to have (what we would consider to be) normative implications, and we can consider Kum\=arila's thesis to have a normative or evaluative aspect.\footnote{For citations see n.\ \ref{reasonableness} above.}

On the normative side, we may understand validity as the status a cognition has when one is rationally entitled to it: I'll call it `warrant' or `entitlement.' Once again there are two theses here, corresponding to the first- and second-order cases.\begin{description}\item[\g{Φα}] A cognition's warrant does not depend on other cognitions.
\item[\g{Φβ}] Seeing a cognition as warranted does not require seeing it as supported by other cognitions.\end{description} For the first-order case, the claim (\g{Φα}) is that the warrant of a cognition does not depend on its being buttressed by further cognitions. (Thus, for example, if you read in the \emph{Times} that Hirut is beautiful, and come to believe that Hirut is beautiful, this belief does not require, for its warrant, that you have any other supporting belief---that the \emph{Times} is a reliable source, say.) Second (\g{Φβ}), one may realize that one possesses no evidence (or no further supporting beliefs) for some belief without being rationally required to suspend or abandon that belief. Indeed, barring undermining evidence, it is irrational to reject your beliefs.  (So, for example, suppose you realize that you believe Hirut is beautiful, but can't remember how it is that you came by that belief. If you're reasonable, and barring some more definite source of doubt, you will stand by this belief regardless.)

To say that invalidity is extrinsic again means two things.\begin{description}
\item[\g{Φα-}] a warranted cognition loses that warrant when some further cognition undermines it.\end{description} So, for example, the belief that Hirut is beautiful will lose its warrant if you read that the reporter has been fabricating stories.\begin{description}
\item[\g{Φβ-}] A cognition is properly judged unwarranted when it is seen to be undermined.\end{description} For example, if you realize that the reporter discovered to have been fabricating stories is the reporter who wrote about the beautiful Hirut, then you will properly see the belief you had about the beautiful Hirut as unwarranted.

A cognition is not simply warranted by default. The warrant of a \emph{warranted} cognition is intrinsic in the sense described. But to have that status in the first place depends on the causal or psychological origin of the cognition. Hence the list of `valid' \emph{sources} of cognition, or of warrant-grounding sources of cognition. Thus a cognition arrived at through testimony, inference, perception, or other sound basis of knowledge is warranted---though it may not be \emph{correct}, since not every belief derived from perception (say) is in fact veridical. A cognition originating merely in a dream or wishful thinking was never even warranted in the first place, though it may very well masquerade as a valid cognition, and as a cognition from a good source.

\

How are the psychological and the normative aspects of \emph{svata\d h pr\=am\=a\d nya} related? In \kum's view, the various \emph{pram\=a\d nas} are ways of opening up windows (in the form of cognitions) on the world. They don't represent it but rather reveal it. This is what distinguishes the \emph{pram\=a\d nas} from other sources of cognition such as wishful thinking or dreams---and even, in Kum\=arila's view, from memory, which is parasitic on genuine \emph{pram\=a\d nas}. And if this is how \emph{pram\=a\d nas} work, it would be a mistake to distance yourself from your cognitions, as if they merely represented the world, or as if in looking around your surroundings you were only looking at something in your own head. A confused philosophical perspective on these matters leads to skepticism---it did so, or was thought to, in India as in the West---and a clear view ought to ward off that skepticism. Now of course sometimes we're wrong about things---Kum\=arila doesn't pretend otherwise. One often learns this in the ordinary course of things, as when one gets a closer view or a view in better light. But it is because cognitions open up the world to us the way they do that we can correct ourselves in that manner.\footnote{It is particularly characteristic of the Indian tradition to (as we would see it) transform normative or evaluative questions into epistemological or metaphysical questions, so far as possible. Most notoriously, there seems to have been virtually no direct philosophical inquiry into final ends---no ethics or moral philosophy in the Western sense. The introduction has already given an illustration of the kind of thing that happened instead: debates about final ends were fought by proxy, in terms of epistemology, ontology, perception, language, and so on, with each school developing an increasingly sophisticated philosophical apparatus to buttress their approach to duty or liberation. See further \citet{sen1984concept}.}

%And of course this sort of thing is familiar in the West as well.


%And in our present case, it's quite unsurprising that there should be a normative or evaluative side to the intrinsic validity thesis, since, as a psychological theses, it offers resistance to the idea that some objects should stand between ourselves and the world, an idea that M\={\i}m\=a\d ms\=akas plainly felt led to skepticism.


%And that is Kum\=arila's strategy.



\kum's claim that cognitions are grasped as valid---the second-order intrinsic validity thesis---may give the impression that Kum\=arila is trying to help himself to evidence. But we can now see that grasping a cognition as being valid corresponds more closely to \emph{reasonably believing something in a self-aware and occurrent manner} than it does to \emph{believing that a belief is reasonable.} The latter---believing that a belief is reasonable---looks like it ought generally to be a response to some kind of evidence one can marshall regarding the belief in question (or its content). But that kind of reflective attention involved in \emph{evaluating} a belief is different from the reflective attention involved in simply \emph{exploring} one's beliefs. Consider the familiar idea that, in order to figure out what one believes, one looks not inward but rather \emph{outward} towards the world---or the world as one sees it. Suppose I've seen Hirut, and I'm asked at some later point whether she is beautiful (or whether I believe that she is). I'll think back to what she looked like, or try to frame her image in my mind, and then say ``yes'' or ``no'' depending on what I see there. There need be no question in this case of assessing evidence in any ordinary sense. I'm not going to review the visual conditions in which I first saw her, or anything like that. This is the case Kum\=arila is concerned with (and which he thinks is the normal case). And Kum\=arila's claim then seems fair. This kind of reflection just assumes the reasonableness of the original belief, and it's hard to see how else it could be. In Kum\=arila's view there is no need to be concerned with the further question about the basis of one's beliefs, so far as no particular cause for concern arises; indeed, if there were such cause for concern, you could never answer this simple question about whether Hirut is beautiful: the moment you thought back on Hirut your earlier cognition of her would have to be in doubt. Compare J.~L.~Austin on demanding how one knows:\begin{squote}If you say `That's not enough', then you must have in mind some more or less definite lack. `To be a goldfinch, besides having a red head it must also have the characteristic eye-markings': or `How do you know it isn't a woodpecker? Woodpeckers have red heads too'. If there is no definite lack, which you are at least prepared to specify on being pressed, then it's silly (outrageous) just to go on saying `That's not enough'.\footnote{\citet[p.\ 84]{austin1979other}.}\end{squote}

That also helps us to make sense of the idea that \emph{all} cognitions may be taken as valid just in virtue of being cognitions. Sure enough, not every cognition \emph{is} valid. But accepting it as valid doesn't \emph{make} it valid, either: \kum\ does not think that reflection renders a cognition valid any more than it renders cognition invalid, and neither does taking a cognition to be valid in any way establish that it is valid, or even imply that any evidence favors that cognition (unless such evidence as is manifested in that cognition already). It does not mean that warrant or evidence has \emph{accrued}. To put the point differently, we might say that Kum\=arila is interested rather in \emph{being} justified than in \emph{having} a justification.


%; not clear in a way that K ever deals with the other thing, not that he thinks there's anything wrong with it. Just doesn't affect whetehr you have justification.]]

Again, Kum\=arila's view is that reasonable belief does not \emph{require} inquiry into evidence; it is no part of his view that there is anything \emph{wrong} with such inquiry. We saw above that Taber attributed to \kum\ the idea that if no contrary evidence comes up then there must not be any. But whether or not that is \kum's view, it is not part of the present doctrine. There is thus no willful blindness in this approach, though the M\={\i}m\=a\d ms\=as' larger aim of defending Vedic authority may create that impression. \kum's view is not that testimony (for instance) ought always to be accepted unquestioningly. There are lots of reasons why one might distrust any given testimony, and likewise for other cognition-forming mechanisms. It is only that \kum\ gives no ground to skepticism. The fact is that quite often we know or can easily know plenty about the various ways we form our beliefs and about when further investigation is required. %\kum\ is not confronting the skeptical worry that we can never get outside our own beliefs to see how well they fit the world by saying baldly that we just \emph{can} accept our beliefs as generally sound. He is, rather, refusing to allow the skeptical worries to set the agenda at all.


% tells us only that, absent problematic evidence, one is not rationally required to suspend or reject belief.



%\


%\kum's view, then, is that there are certain causal-cum-psychological mechanisms which can present one with a genuine view of the world (and not just an apparent view, as in a dream). A judgment, or cognition, to which such a mechanism gives rise will constitute a way in which one sees the world. It is reasonable to see the world in that way, though you may in fact have a defective view of things. When you reflect on your way of seeing things, you still see things in that way. Moreover it remains reasonable to see things in that way. However, further cognitions (new ways of seeing things), may upset this situation.

% [[HERE WE CONFRONT AN ISSUE ABOUT `ABSENCE']].




%The position is quite intelligible. It may be reasonable to believe based on testimony, and unreasonable to believe only from wishful thinking. And it may be that a reasonable person will shed a belief once he realizes that it reflects only wishful thinking. But neither of those facts entail that it is unreasonable to accept those beliefs at face value, since after all I may have no reason to suspect that the beliefs \emph{are} rooted in wishful thinking.

%\begin{quote}It is in the case of objection (2) that you would be more inclined to say right out 'Then you don't know'. Because it doesn't prove it, it's not enough to prove it. Several important points come out here: 

%(a) If you say 'That's not enough', then you must have in mind some more or less definite lack. 'To be a goldfinch, besides having a red head it must also have the characteristic eye-markings': or 'How do you know it isn't a woodpecker? Woodpeckers have red heads too'. If there is no definite lack, which you are at least prepared to specify on being pressed, then it 's silly (outrageous) just to go on saying 'That's not enough'. \end{quote}






\section{Exegesis} \label{textualinterpretation} %Having outlined my inbterpretation of \kum's \svapra\ thesis, I now work through a substantial portion of the \emph{Codan\=as\=utra} in detail.

%Having developed an interpretation of \kum's \svapra thesis, I will now work through a portion of the \e{CS} to show that the interpretation accords with the text.

%\

As noted, the \e{CS} is a commentary on \emph{MS} I.i.2, which says that ``duty is a purpose having injunction for its sole authority.'' A number of the initial verses of the \emph{CS} are concerned with minor clarifications of that thesis, but vv.\ 5--6 are close to a statement of the chapter's main theses:\begin{squote}5.\ Inasmuch as authoritative character [regarding duty] is possible only to the Word, [\footnotesize\'S\small \=abara] has also pointed out the incapacity, with regard to such objects [necessary for knowledge of duty], of Sense-Perception and the rest, which is to be described hereafter.

\noindent 6.\ Even with regard to purely non-existing objects, the Word brings about some conception. And consequently, in the absence of any discrepancy, authoritative character must be accepted to belong to it by its very nature.\footnote{Jh\=a refers to `cognitions' as `conceptions'; `validity' is `authoritativeness'; `intrinsic is `inherent'; Jh\=a also uses `means of right notion' to refer to a valid cognition.}\end{squote} The later parts of the \emph{CS} are concerned more with the role and value of testimony as opposed to the other sources of knowledge alluded to here in v.\ 5, and the earlier portions (up to around v.\ 102) focus on the general epistemological theses (of interest to us) indicated in v.\ 6.

The latter verse picks up on a common M\={\i}m\=a\d ms\=a theme, namely the independence of testimony as a source of knowledge:\footnote{See \S\ref{intro} above.} testimony produces cognitions, including of things non-existent, hence which could not have been produced by sense perception or any other \e{pram\=a\d na}---and thus is a \emph{primitive} producer of cognitions, i.e.\ a \emph{pram\=a\d na} in its own right. Now this seems rather problematic. On the one hand Kum\=arila will not want to count just anything as a \emph{pram\=a\d na} merely because it leads us to think things. But if the point is rather (as I think it is) that testimony can (at least in the case of Vedic testimony regarding duty) provide us with a window on things that other \emph{pram\=a\d nas} cannot, then the claim will be question-begging. But this is only a preliminary statement of the view (and fortunately we are not concerned with the case of testimony specifically), so we may simply note that a primitive cognition-producing mechanism is \emph{eo ipso} a producer of \emph{valid} cognitions.%[WHAT ABOUT MEMORY, WICHFUL THINKING ETC?]

The starting-point for the discussion of \emph{svata\d h pr\=am\=a\d nya} is an objection to the M\={\i}m\=a\d ms\=as doctrine that an authorless text is our authority on \emph{dharma}, the nub of which objection Kum\=arila presents himself:\begin{squote}22.\ It is always an object perceived by other means of knowledge that is got at by the Word [testimony]; and, like memory, no authority can belong to it by itself.\end{squote} When one accepts testimony, it is because ``it is an assertion of a person who is believed to have perceived the object.''\footnote{v.\ 23.} In other words, testimony is not a primitive source of knowledge like perception or inference are. The objector adds that we sometimes disregard testimony when we \emph{don't} judge the source reliable---and how could that be, if a reliable source isn't in fact essential to testimony as a source of knowledge?\footnote{v.\ 30.} The opponent also makes the objection I worried about a moment ago:\begin{squote}25.\ ``Thus then, as even when producing a conception, Fancy, \&c., are no authorities by themselves, so we may apply the same rule to the case of Veda also.''\end{squote} So, again, we do not want to say that just any psychological mechanism which produces some judgment is therefore to be accorded authority.

Kum\=arila responds by broadening the inquiry, turning away from the specific case of testimony: \begin{squote}33.\ With regard to all conceptions, you must consider the following question: ``Is the authoritativeness or unauthoritativeness due to itself or to something else?''\end{squote} There are three possible positions besides Kum\=arila's own:\begin{squote}34.\ Because those that are by themselves false cannot by any means be proved to be true. Some people attribute both (authoritativeness and its contrary) to (the conception) itself. Others attribute them to the proved excellences or discrepancies of its origin.\end{squote} In the first sentence Kum\=arila briefly hints at his reason for rejecting the view which will prove to be the principle rival to its own, namely that invalidity is intrinsic and validity extrinsic (he will return to it at v.\ 38ff, and we return to it on p.\ \pageref{rival} below). The remaining possibilities are, first, that both validity and invalidity might be intrinsic to cognitions, and, second, that both might be extrinsic to cognitions. Verses 35--7 are devoted to rejecting these two possibilities. Regarding the first, Kum\=arila says that \begin{squote}35.\ Both cannot be due to itself, because the two are mutually contradictory\sdots

\noindent 36.\ How can it be possible that any one thing, independent of all extraneous agency, should have contradictory characters? \ldots

\noindent 37.\ If ``non-contradictoriness'' were possible with regard to different conceptions---even then, if nothing else is taken into consideration, it cannot be ascertained which is which, and where.\end{squote} We might, uncharitably, see the argument of vv.\ 35--6 as turning on a confusion of type and token---after all, why shouldn't some cognitions have the one character and other cognitions the other? Verse 37 would then acknowledge the confusion and offer a better argument. But that would be an unhelpful procedure, and v.\ 36 offers some textual basis for rejecting this reading. Verse 36 asks not just how cognitions could have contradictory characters, but specifically how anything could have contradictory characters \emph{independently of all extraneous agency.} If we were talking about a single cognition it would not matter whether or not extraneous agencies were implicated---a single thing could not be valid and invalid at the same time regardless. That suggests that even in vv.\ 35--6 Kum\=arila has in mind cognitions in general rather than particular cognitions---some cognitions will be valid and some invalid, but the difference has to be explained by further, external factors.

But why should we have to appeal to extraneous factors to explain the difference? And what then is the real objection in vv.\ 35--6 to the possibility that both validity and invalidity might be intrinsic? A reasonable possibility is that Kum\=arila thinks that either validity or invalidity should be in some sense \emph{natural} to conceptions. But then only one or the other could be natural to cognitions in general, not both. Thus if validity were natural or intrinsic to cognitions then the invalidity of a given cognition would require some further explanation appealing to extrinsic factors. This interpretation would fit the language employed in the string of objections that are shortly posed. Thus 38: ``for those that hold the authoritativeness of conceptions to be natural\ldots''; v.\ 40:  ``if authoritativeness were inherent or natural\ldots''; v.\ 44: ``while unauthoritativeness, being natural\ldots .'' So it seems reasonable to say, on the basis of vv. 35--6, that, for Kum\=arila, the question of the intrinsic or extrinsic nature of validity or invalidity is a question about the natural state of cognitions in general, the point then being that there could be (at most) only one such natural state. Remember that part of what of what \kum\ is considering is whether cognitions are presentative or rather \emph{re}presentative;\footnote{See \S\ref{interpoutline} above.} here \kum\ is saying that at any rate they aren't \e{both}.

In v.\ 37 Kum\=arila sets aside the inquiry into cognitions in general. Now he considers the possibility that cognitions should be considered individually. Might some then be intrinsically valid and others invalid, just as there are both natural redheads and natural blondes? Kum\=arila now objects that ``it cannot be ascertained which is which, and where.'' The idea, I think, is this. The suggestion at hand is that some cognitions are presentative by nature and others representative.\footnote{At least that's how Kum\=arila, a realist, would look at the matter. As indicated in \S\ref{interpreting} above, other Indian philosophers would have seen the matter differently. It's not clear that Kum\=arila is really trying to seriously respond to such views as opposed to just considering all of the possibilities in a realistic spirit.} But in \kum's view, there remains an open question about the accuracy of a representation which does not arise in the case of a presentation. So the suggestion being considered in v.\ 37 is effectively a suggestion that sometimes cognitions arrive warranted, as it were, while other times cognitions are warranted only when supported or certified by another cognition. The ``which is which and where'' objection seems very much to the point: \kum's thought would be that the suggestion just mooted undermines the possibility of intrinsic validity altogether---for how could you be warranted in accepting the intrinsically warranted cognitions if you can't tell them apart from the ones that need to be certified by appeal to another cognition? A cognition that requires either buttressing or discarding must be discarded if you can't buttress it; but if you can't tell the cognitions to which that requirement applies apart from the cognitions to which it doesn't, how can you be rationally required to stand by the latter, given that for all you know they must either be buttressed or discounted? True, there could still \emph{be} intrinsically warranted beliefs---i.e. cognitions to which you are entitled even if they are not supported by other cognitions. But that entitlement would seem to disappear upon reflection.\footnote{At present we are asking a question about the warrant of the first-order cognition, and not a question about how it can be \emph{determined} whether a first-order cognition is warranted. But one part of the suggestion in the contemplated scenario is that some cognitions may be warranted only \emph{if} assessed in a certain way.} Much of the epistemological benefit of understanding cognitions as presentative would thus disappear if it is only in the nature of \emph{some} cognitions or types of cognitions to be presentative.

%\footnote{At least that's how Kum\=arila, a realist, would look at the matter. Buddhists, who might actually hold the view that both validity and invalidity are extrinsic, will generally think about the matter in very different terms. It's not clear that Kum\=arila is really trying to seriously respond to such views as opposed to just considering all of the possibilities in a realistic spirit. Kum\=arila's principal target is evidently the view that validity is extrinsic and invalidity intrinsic}

In the latter halves of vv.\ 35--6, \kum\ argues that validity and invalidity cannot both be \emph{extrinsic} to cognitions either.\begin{squote} 35.\ \ldots\ Nor can both be due to something else, because in this latter case, there would be no definiteness in the conception.

\noindent 36.\ \ldots\ And when devoid of both these characters, of what form could the conception be?\end{squote}Here Kum\=arila restricts his attention to individual cognitions, so if `being devoid of both characters' would mean that a cognition is neither presentative nor representative, neither warranted nor unwarranted, then it's unsurprising that Kum\=arila would dismiss this possibility. This seems to be confirmed when we look at what Kum\=arila means by saying that these cognitions would lack `form' or `definiteness.' Consider v.\ 168, which comes in the course of a discussion of sentence meaning and the importance of reliable speakers:\begin{squote}168.\ Though the meaning [of a sentence] may have been comprehended beforehand, yet it depends for its definiteness upon the fact of its originating directly from the [trustworthy] speaker's cognition; hence such a fact can only be comprehensible through the comprehension of the meaning. But in the matter of authenticity, it takes the first place.\end{squote} This seems to say that while \emph{understanding} a sentence is simply a matter of sentence meaning, definiteness and authenticity (ie.\ validity) are a matter of the source of the belief in question. In any case the ``definiteness'' of a cognition seems to depend somehow on the causal origin of that cognition. What would that mean? It seems plausible to suppose that Kum\=arila is gesturing at the difference between merely entertaining a thought and having a thought that is delivered by some belief-producing mechanism which lends it weight (the latter corresponding to \kum's cognition-producing mechanisms).\footnote{Cf.\ \citet[ch.\ 1]{sen1984concept} and eg.\ \emph{SV} IV.v.78.} For example, no one considering my hypothetical example of the beautiful Hirut is going to see something to actually believe in that---the example doesn't assert that some Hirut is a beauty. But if one \emph{read} that some Hirut is beautiful, the text would effectively present Hirut to you as beautiful. In the latter case the thought would be ``definite.''

%\footnote{BUT SPECIFICALLY THE WINDOW THING: MORE ABOUT DETERMINATENESS OF THE OBJECT OF COGNITION?---Cf.\ \citet[ch.\ 1]{sen1984concept}. [[notice regarding eg detemrinateness worry, prob has to do with the idrect realism --- if you thought that what we had were appearances you wouldn't worry the same way. Anyway for example the Nyaya consider both extrinsic]] --- IV.v.78 seems to be relevant to the function question: basically to "specify" an object (as Jha translates)]]}

Considered from the perspective of warrant, the proposal in v.\ 168 may be urged against the idea that a cognition is warranted only if you can buttress it with some other cognition and that a cognition is also unwarranted only if it is undermined by some other cognition. For suppose that were true. Now you are reading a newspaper, and you read that a certain Hirut is beautiful. And suppose you have no particular reason for doubting this or for distrusting the newspaper. But you can only rationally adopt the belief that Hirut is beautiful if you can ground it in some other cognition. But also you can only reject it if you can oppose it with some other cognition. But then in what sense could this report be offering itself to you for acceptance, if you could accept it only by checking somewhere else for support first, but at the same time can't rationally reject it either? Of course we could say that the newspaper is just offering you some things for consideration, but this would be exactly Kum\=arila's point: you may as well have spent your morning daydreaming, if that's all reading the newspaper gets you.\footnote{[TESTIMONY ISSUES]}% So the view would would make a mockery of the very notion of reporting: it would make reporting like daydreaming.

\label{rival} The big question is whether validity is intrinsic and invalidity extrinsic, as Kum\=arila thinks, or the other way round. Verses 38--46 set out the position of Kum\=arila's opponent, and some of the opponent' reasons for saying that invalidity is intrinsic and validity extrinsic. One objection to Kum\=arila's view is offered in v.\ 40:\begin{squote}40.\ ``If authoritativeness were inherent or natural and its absence artificial, then dream-cognitions would be authoritative, self-supported; for what is there to refute this?''\end{squote} This is nearly a restatement of the objection of v.\ 25 within the new dialectical context: attention has shifted from modes of knowledge to cognitions. (There Kum\=arila had claimed that testimony---and thus Vedic testimony---is a source of knowledge since it produces cognitions, and the opponent had objected that cognitions are also produced by plainly problematic mechanisms.)% Here Kum\=arila is adopting the view that cognitions are intrinsically valid, and the opponent is observing that this will apply to cognitions from problematic sources.

Certainly from the perspective of warrant, the opponent's view is intuitive: the thought is that, given all the defective ways cognitions can be formed, one can be warranted in thinking something only if one can recognize that one came by that thought in some sound way. My confidence in what I think should be affected, for example, by whether I read it in the \emph{National Enquirer} or on Wikipedia or in the \emph{New York Review of Books}, and that requires that I should have some appreciation of how my views originated.

But Kum\=arila's response is also intuitive:\begin{squote}49--51. If even on the birth of conception, the object thereof be not comprehended until the purity of its cause has been ascertained by other means, then in all cases we should have to wait for the production of another conception from a new source; for until its purity has been ascertained, the conception would be equal to nothing. And this second conception too, would be true only on the ascertainment of the purity of its cause; and so on and on, there would be no limit.\end{squote} The opponent thinks that the soundness of your view should turn on your appreciation of the soundness of the origins of your view---but to this sort of verification there is no end, for how can you trust that you have indeed properly assessed the origins of your original view? Your assessment itself might be wrong, and indeed according the opponent the assessment itself should be in doubt until verified. At that rate we will never be entitled to any beliefs.\footnote{cf.\ v.\ 56, and \citet{alston1980level} on `level-confusions.'}

Though warrant is important here, Kum\=arila is more directly making a psychological point, as is especially evident from the two preceding verses.\begin{squote}47.\ You must understand that authoritativeness is inherent in all Means of Right Notion. For a faculty, by itself non-existing, cannot possibly be brought into existence by any other agency;

\noindent 48.\ since it is only for the sake of its birth that a positive entity requires a cause. And when it has once been born, it's application to its various effects proceeds naturally out of itself.\end{squote} A valid cognition must of course have some causal basis, but once produced it provides comprehension of its object---that is what I take the reference to `faculty' and `application' to mean. \kum\ thinks that if the validity of a cognition---that, say, Hirut is beautiful---depends on some other cognition, then we would not after all have a comprehension of Hirut being beautiful until the further cognition were available. I understand the point here as follows. Suppose you pick up a copy of the \emph{New York Times.} In it you read a story about Ethiopia. The story is about a singer named Hirut and it reports that she is beautiful. Now suppose you know essentially nothing about the \emph{Times}, but also have no particular reason to doubt it or to doubt its reliability in this area. (Compare asking someone on the street for the time.) And in fact you come to believe, as a result of reading the story, that there is a beautiful Ethiopian singer named Hirut. Then assuming that the reporter is an honest and careful one, with sound aesthetic sensibilities, hasn't the \emph{Times} opened up a bit of the world for you? Doesn't your new belief constitute a window on a little part of Ethiopia? This is what \kum\ claims is the case: a valid cognition is a window on the world, and nothing more than what has been described is required to open up that window.\footnote{Compare \citet[pp.\ 66--7]{kripke1972naming} on Jonah.}

Why does \kum\ say that once it has originated, the cognition thereby acts in this way (as this window), and that nothing could make it act this way if it couldn't do so of it's own accord? If we considered the matter in terms of \emph{thoughts} or \emph{beliefs}, this would seem to be a mistake. Suppose another man reads the story about Hirut, but only in a detached fashion, forming no beliefs, because he dismisses the \emph{Times} as a part of the reactionary old-media elite. The report that Hirut is beautiful opens no windows for him. But then he reads that the reporter in question supports some bold progressive cause---open borders, perhaps---and thus decides the reporter must be a trustworthy sort after all. He now remembers how this reporter said that a certain Hirut was beautiful, and now believes it. Doesn't the initial thought about Hirut now serve as a window for him, though one opened by further thoughts?

Here we must remember that Kumarila's cognitions are not beliefs as we understand them. A cognition is anchored to its genetic origin---to the particular episode of perceiving or of receiving testimony that gave rise to the cognition---in a way that a belief is not. Or again, validity, unlike justification, cannot be provided later for a cognition that lacked it. A cognition either captures a view of the world or it doesn't, and if it doesn't then nothing can change that later. You could later come by a new cognition that gives you what the first one might have, but that will just be a different cognition.

So, if the validity of a cognition depends on other cognitions, then we'll have a regress, while Kum\=arila's view avoids that difficulty:\begin{squote}52.\ In case, however, authoritativeness be accepted to be due to (the conception) itself, nothing else is wanted (for its cognition). Because in the absence of any cognition of discrepancies, falsity (unauthoritativeness) becomes precluded by itself (i.e. without the help of any extraneous Means).\footnote{Cf.\ v.\ 56: ``The fact of mere Unauthoritativeness being due to discrepancies does not lead to any \emph{regressus ad infinitum}---as is found to be the case with the theory of the cognition of excellences (being the cause of authoritativeness)---for us who hold the theory of ``self-evidence'' [\emph{svata\d h pr\=am\=a\d nya}].''}\end{squote} (Note that Jh\=a---rightly---glosses `falsity' here as `unauthoritativeness'---i.e.\ as invalidity. We return to `falsity' in connection with v.\ 54 below.)

Verse 47 above said that validity is intrinsic to all \emph{valid} cognitions. But v.\ 53 has been cited by Taber to support the reading according to which \emph{all} cognitions are intrinsically valid, and this pair of verses has also been adduced by Tara Chatterjea to show that \kum's work contains ``contradictory suggestions'' concerning the scope of the \emph{svata\d h pr\=am\=a\d nya} thesis.\footnote{See \S2.3 above.}\begin{squote}53. Therefore the authoritative character of a conception, cognised through the mere fact of its having the character of ``cognition,'' can be set aside only by the contrary nature of its object, or by the recognition of discrepancies in its cause.\end{squote} But while the discussion (both mine and Kum\=arila's) has thus far primarily concerned the first-order cognition, verses 52--3 connect up the first- and second-order elements of the intrinsic validity thesis. We remember that, for Kum\=arila, a cognition is not itself an object \e{in} consciousness, though it manifests some other object \e{to} consciousness. If you are aware of a cognition as opposed to its object, that is because this cognition has itself become the object of a \e{further} cognition. The \emph{svata\d h pr\=am\=a\d nya} thesis in this connection concerns the relationship between cognizing a cognition and cognizing that cognition as being (i.e.\ cognizing it as though it were) valid.\footnote{\citet[p.\ 3]{mohanty1989gangesa} offers a useful note on the difference between (and different positions on) grasping `\emph{prak\=a\'sa}' and grasping `\emph{pr\=am\=a\d nya},' and see \citet{sen1984concept} for a more extensive treatment.} Verse 53 is asserting that the two are the same thing.

Why are cognitions apprehended as valid? Because to reflect upon a cognition just is to reflect upon the world which that cognition reveals. Such reflection presupposes that the initial cognition constitutes your view of the world. For suppose instead that, upon recognizing and attending to the belief that Hirut is beautiful, you were not to apprehend it as valid---whether actually as invalid or somehow as neither valid nor invalid. Then that very second-order cognition would undermine the first-order cognition. Even granting that the initial cognition were valid up to that point, there would now be a new cognition to the effect that the first cognition is invalid or false or doubtful or whatever the case may be. Of course this new cognition might itself be invalid, but that will not always be so, even when it is false. So reflection will be a way of undermining your cognitions, except to the extent that you are able to identify some support for them. But, thinks \kum, reflection should not undermine our beliefs in that way. So the two forms of the \emph{svata\d h pr\=am\=a\d nya} thesis mesh nicely, and both are supported by the appeal to the threat of infinite regress.% If validity were not intrinsic, then cognitions could never be valid, because each cognition would (\emph{ex hypothesi}) have to confirmed or supported by a further cognition. And if in reflection one could not see one's cognitions as valid, one would again have to identify an unending chain of supporting cognitions in order to see any cognition as valid.

%It would be possible, of course, to say that reflection does throw open the matter---that to reflect on a belief is to step back from it, and to now require some new basis for accepting it, if one is to continue to accept it. 

In v.\ 54 Kumarila identifies several varieties of invalidity:\begin{squote}54.\ Unauthoritativeness is three-fold: as being due to falsity, non-perception, and doubt. From among these, two (falsity and doubt) being positive entities, are brought about by discrepancies in the cause.\end{squote} With the reference to `positive entities,' Kum\=arila responds to ontological qualms of a pan-Indian variety concerning explanation; that need not detain us.\footnote{These were mooted by the opponent at vv.\ 38--9.} The reference to `discrepancies' is more important for our purposes: the proponent of intrinsic invalidity will claim that a cognition is rendered valid when the mechanisms which produced it are seen to have been sound, whereas Kum\=arila claims that a cognition is shown invalid when the mechanism is shown to have been defective, i.e.\ to have had `discrepancies.' In v.~54 Kum\=arila is thus cataloguing the ways in which a cognition can be rendered invalid and saying that two of these ways are indeed reducible to the the identification of discrepancies.\footnote{Notice it's not enough for there to \emph{be} discrepancies---they must be turned up somehow. Otherwise those cognitions would not have been valid in the first place, and Kum\=arila is claiming that they were.} (The third, non-perception, will be a special case.)

By `falsity,' Kumarila means not the actual falsity or inaccuracy of the initial cognition; rather he is addressing the case in which some such inaccuracy is turned up, or in which some defect in the mechanism behind the cognition is turned up, so that the initial cognition is altogether overturned. The case of doubt requires not so much an overturning as an unsettling; thus in a stock Indian example you might be unsure whether you're looking at (or saw) a piece of silver or rather a bit of shell. Kum\=arila is considering these two cases in vv.\ 57--61:\begin{squote}57.\ Unauthoritativeness (falsity) is got at directly through the ``Cognition (of its contradictory).'' For, so long as the former is not set aside, the subsequent cognition (of its contradictory) cannot be produced.

\noindent 58.\ Though the cognition of the discrepancy of the cause is known to refer to a different object (\emph{i.e.}, not the object which is the effect of tlie cause), yet we have co-objectivity (of the two cognitions) as being implied thereby; and hence we have the preclusion of the former,---as in the case of the ``milking-pot.''\footnote{The `milking-pot:' Kum\=arila adduces an example from ritual (Jh\=a \emph{ad. loc.}).}

\noindent 59.\ But this rule applies only to those cases in which (with regard to the second conception) there is neither cognition of any discrepancy, nor any contradictory conception. In those cases, however, in which we have any of these two factors, the second conception becoming false, the first comes to be true. 

\noindent 60.\ But in that case too, the authoritativeness is due to the conception itself, in the absence of any cognition of discrepancies. And in a case where there is no such cognition of discrepancies, there is no reasonable ground for doubt. 

\noindent 61.\ Thus (in this manner) we do not stand in need of postulating more than three or four conceptions. And it is for this reason that we hold to the doctrine of ``Self-evidence'' [i.e.~\emph{svata\d h pr\=am\=a\d nya}].\end{squote} The `falsity' case well illustrates one form of the extrinsic nature of invalidity. It is in having the new cognition that the earlier one is set aside and rendered invalid. And should the new cognition itself prove faulty, the original cognition is valid again not because it has gained new support but because the objections have been lifted---thus the validity of a valid cognition is intrinsic even if it loses and regains that status.\footnote{vv.\ 59--60.}

This is not just a psychological point. It may be that, as a psychological matter, it's impossible to credit both the old idea that Hirut is beautiful and the new idea that she's plain, but as a psychological matter one could reject the new information rather than the old. But, for Kumarila, the new cognition is itself intrinsically valid, barring some further problem with \emph{it}. The process of discarding the old way of looking at things is, moreover, an inferential process and not merely a causal one. This is clearest in the case where a cognition is not just straightforwardly at odds with the earlier cognition in terms of its content, but where the new cognition reveals something problematic about the causal basis of the original cognition, which is the case Kumarila considers in v.\ 58. For an example, imagine that rather than seeing Hirut and finding her plain, you learn that the \emph{Times} reporter in question has abnormal aesthetic sensibilities. And notice, finally, that v.\ 60 above is explicit that there is a question of rationality here: it would be a \emph{mistake} to give up your view of Hirut's beauty barring some specific reason for doing so.

%:\begin{squote}58. Though the cognition of the discrepancy of the cause is known to refer to a different object (i.e. not the object which is the effect of the cause), yet we have co-objectivity (of the two cognitions) as being implied thereby; and hence we have the preclusion of the former,---as in the case of the ``milking-pot.''\footnote{With the reference to the milking-pot \kum\ adduces an example from ritual to illustrate his point: Jh\=a \emph{ad.\ loc.}}\end{squote}


When in the case of `falsity' a cognition is set aside, it is, in a sense, not really even apt for validity any more, since it no longer constitutes one's view of the world anyway. Likewise with the case of doubt; the view has been suspended in a way that would make it inappropriate to ask, for example, whether the subject is justified in holding it.\footnote{A doubt carries no commitment in any direction in the first place---this point fits with my earlier discussion where I supposed that validity presupposses \emph{definiteness} as I interpreted that term; cf.\ \citet[pp.\ 93--5]{bhatt1962epistemology}.} The same point holds doubly for the case of non-perception, the case in which you were simply not in contact with the world at all, as might be the case in dreams, or the aforementioned case of `fancy,' or wishful thinking:\begin{squote}55. In the case of non-perception, however, we do not admit the action of such discrepancies. Because for us all non-perception is due to the \emph{absence} of cause,---just as you have asserted.\end{squote} This case is important because it illustrates a claim I've made a couple of times, namely that cognition must have a certain causal origin if it was ever to have validity in the first place. `Fancy' and wishful thinking, for example, are not sources of valid cognition,\footnote{vv.\ 94--5; cf.\ v.\ 128.} but only special cases of non-perception. We remember that v.\ 48 had said that it was only for its birth that a positive entity required a cause---here there is no (sound) cause in the first place, and no positive entity (i.e.\ no valid cognition). This implicitly offers a response to the dream objection of v.\ 40.

% (at the same time it helps address the earlier objections about dream cognitions and so on)

%---cognitions without a source in a valid source of knowledge aren't viable candidates for validity in the first place.

%That means that Kum\=arila thinks that you are justified in believing something so long as you believe it on the basis of some approved means of knowledge such as testimony, or perception, \emph{whether or not} the specific instance of testimony or perception is sound (unless of course you can actually see that it isn't sound). 

%Other psychological basis for cognitions are merely 

%[[MAYBE us this P to connect above P with the whole issue of independence etc.]] 

The non-perception case also allows us to appreciate more clearly that validity cannot be anything like \emph{truth.} If I daydream the beautiful Hirut into existence, and later forget it was a daydream and come actually to believe that there was a beautiful Ethiopian singer named Hirut, the belief may or may not in fact be \emph{true}; Ethiopia has its singers, Hirut is a reasonably common name there, and it is a land of beautiful women; some woman may very well be all of these things. But there is no question or possibility of \emph{validity} here. Validity requires a causal basis (such as testimony or perception) that can (though may not always) mediate some kind of contact between you and the word. If there's no actual chain between the world and your thoughts about it, you can't have valid cognitions about it.%, however accurate your speculations might be.% (Remember too that a cognition is anchored to it's origin; new causal bases can be formed between a \emph{belief} and the world, but not so in the case of cognitions. Hence Kumarila's claim---noted earlier---that a cognition cannot be rendered valid by some other cognition, but must rather have that potential in it's own right, if it is ever to be valid at all.) 

Kum\=arila applies his conclusions to the case of testimony:\begin{squote}64.\ In (truthful) human (speech) we find two (factors)---\emph{absence of discrepancies,} and (\emph{presence of}) \emph{excellence}; and we have already explained that authoritativeness cannot be due to \emph{excellence}.

\noindent 65--6.\ Therefore excellences must be held to help only in the removal of discrepancies; and from the absence of these latter (discrepancies), proceeds the absence of the two kinds of unauthoritativeness; and thus the fact of (authoritativeness) being inherent in Words remains untouched. And inasmuch as the word gives rise to a conception, its authoritativeness is secured.\end{squote} When Kum\=arila says it has already been explained that validity cannot be due to the excellences of testimony, he must mean that a cognition is not valid in virtue of \emph{recognizing} that the source of knowledge---in this case testimony---was free of fault. He exploits this point to respond to the objection that his own view does face a regress despite his earlier protestations:\begin{squote}66. ``If the absence of discrepancies be held to result from excellences, then there is the same \emph{regressus ad infinitum} (that you urged against us.''

\noindent 67.\ (Not so): because at that time (ie.\ at the time of the conception of the absence of discrepancies), we do not admit of any active functioning of the excellences, though they continue to be recognised all the same;---because in the conception of the absence of discrepancies they help by their mere presence.\end{squote} As we have seen before, one need not actually have an appreciation of the soundness of the basis of a conception for it to be valid.

There is also a slight puzzle here, though, because v.\ 67 seems to suggest that in fact a valid cognition \emph{must} have a \emph{sound} basis---i.e.\ a source without discrepancies. But that cannot be right, because it would mean that testimonial cognitions are not after all valid just as such, and, as we can see even here (from the the last sentence of vv.\ 65--6), validity \emph{does not} depend on the actual faultlessness of the testimony in question: testimony just as such \emph{is} a source of valid cognitions. I therefore take v.\ 67 to mean that the \emph{presumed} excellences of the source of a cognition suffice (so far as the source is of an approved variety: testimony, perception, etc.). This is further confirmed by the next verses, as we turn to the case of the Vedas, where Kum\=arila claims that the very testimonial nature of Vedic-derived cognitions imbues them with validity:\begin{squote}70.\ Hence the mere fact of the Veda not having been composed by an authoritative author, ceases to be a discrepancy\sdots 

\noindent 71.\ It is only human speech that depends for its authority upon another Mean of Right Knowlede; and hence in the absence of the latter, the former becomes faulty; but the other (ie.\ Vedic sentences) can never be so (on that ground).\end{squote} We have returned here to the second-order case: testimonial cognitions may be presumed sound just on the basis of being testimonial cognitions, though in the typical or human case, where testimonial knowledge is parasitic on other sources of knowledge, that presumption would be overruled by the observation that the testifier had no way of coming by his supposed knowledge. In the case of the Vedas, no such difficulty can arrive, and the presumption stands.

Verse 72 returns us to the matter of independence:\begin{squote}72. Thus then, the very fact of the incompatibility of the Veda with other Means of Right Notion, constitutes its authoritativeness; for if it were not so incompatible, it would only be subsidiary (to other such means).\end{squote} The point here is that since Vedic testimony produces cognitions which could never be called into question by any other source of knowledge, the Vedas are an authority: they produce cognitions which must be trusted. This may seem to reopen our question about dreams and fancies. Perhaps you have dreams about the future, and come to have beliefs about the future in that way, and these beliefs cannot be undermined by perception or inference. Does that make dreaming authoritative concerning the future? Kum\=arila does not think so, but not, I think, for deep reasons of epistemological theory proper. It is just that dreaming and Vedic testimony are different cases. On Kum\=arila's view, dreams and fancies just take the matter of perception and mix it up in arbitrary ways; they are not new sources of information. One could, of course, say the same about the Vedas---but then you wouldn't be a \mimamsaka. \kum\ thinks that language and meaning are deep and primitive aspects of the cosmos, that the cosmos is eternal, that it's reasonable to suppose that the Vedas were never written, and that they tap into eternal meanings somehow. Those are matters separate from epistemology, nor does Kum\=arila suppose himself to have defended any of those claims at this stage.\footnote{Verse 70 in its entirety runs thus: ``Hence the mere fact of the Veda not having been composed by an authoritative author, ceases to be a discrepancy. Of the syllogistic arguments urged against us, we shall lay down counter-arguments hereafter.''}% EMPHASIZE THIS MORE?

We pass over vv.\ 73--81, which continue discussing the independence of Vedas, and arrive at an objection in v.\ 82:\begin{squote}82.\ (Objection): ``But Sense-Perception and the rest are not comprehended as that `these are authoritative'; nor is it possible to carry on any business by means of such perceptions, when they are not comprehended as such.''\end{squote} The opponent objects that since in forming cognitions we do not yet have any confidence in those cognitions, they cannot be a basis for action or further inference. In responding, Kum\=arila provides a helpful recapitulation of his view.\begin{squote} 83. (Reply): Even prior to comprehension, the Means of Right Notion had an independent existence of their own and they come to be comprehended subsequently (as such), through other cognitions.\end{squote} A cognition is valid and serves as a basis for action and inference even without any recognition of its existence; when its existence is recognized, this is in a new cognition.\begin{squote} 84.\ Therefore the fact of its being comprehended as such, does not in any way help the authoritativeness (of the Means of Right Notion); because the idea of the object is got at through the former alone.\end{squote} The second-order recognition of a cognition as valid is not necessary for the validity of that first-prder cognition, nor can it confer validity upon it.\begin{squote}85. Even the unauthoritative Means would, by itself, lead to the conception of its object; and its function would not cease unless its falsity were ascertained by other means.\end{squote} Even an invalid cognition manifests the world, though it may do so wrongly (it may for example `reveal' a real shell to be a piece of silver\footnote{Jh\=a \emph{ad loc.}}), but until it is revealed as misleading it is accepted, and forms a basis for further inference. Notice here that a cognition can be invalid but still manifest the world.%; hence validity cannot be manifestation.

We should also see here a connection to action: so far as v.\ 85 itself goes, we might think that the `function' of a cognitions was restricted to manifestation, and that may indeed be the primary meaning, but in the context of the objection of the objection at v.\ 82, Kum\=arila evidently means that, in virtue of the exercising its function, a cognition (even an invalid one) functions as a basis for action.

Finally we end with a reiteration of the intrinsicality of validity and the extrinsicality of invalidity:\begin{squote}86.\ The falsity of an object is not, like its truthfulness, perceived by its very first conception. For the recognition of unauthoritativeness, the only cause is one's consciousness of the falsity of its subject itself, or of he faultiness of the cause thereof.

\noindent 87.\ Thereby alone is falsity (of a conception) established; and by no other means. And the truthfulness (or authoritativeness of a conception) is proved to \emph{belong to the state of its birth} (\emph{i.e.}, is natural or inherent in it).

\noindent 88.\ Therefore even in cases where falsity is proved by other means, these two (causes of falsity) should be noted, and not only certain points of similarity (with another false idea).\end{squote}

\


We have clarified a number of puzzles about the text that remained in other interpretations, and found confirmation for the approach I laid out in the previous section. We have reaped these rewards primarily by simply attending carefully to the difference between the first- and second-order aspects of the \svapra\ thesis.


%[SUMMARY; HOW MINE IS SUPERIOR]



%But the point is also, I think, that this belief is a basis for action, and this is because the original cognition gives you a view of the world on which to act, without any further assessment of that view. 
%  - connect with reasonableness of action

%Consider the next verse:

%`Truth' and `falsity' cannot, presumably, mean actual truth or falsity in a correspondence sense---the point seems rather to be the one that Taber had urged: a belief gives you an outlook on the world, and you don't need some further view \emph{about} that view before this becomes true. This gives us the needed response to the objection in v. 82, because a belief just is a basis for action.





%RELEVANT QUOTES FROM SLIV

%32. Just as ordinary Fancy, independently of Sense-perception and the other (Means of Right Notion), is not able to give rise to any definite idea, so also would be the Imagination (or Intuition) of the Yogi. 

%34. Specially, because, Duty is not perceptible, prior to its performance ; and even when it has been performed, it is not perceptible, in the character of the means of accomplishing particular results, 

%[[DOES THIS SUGGEST THAT DEFECTIVE CIGS JUST DON'T COUNT?]] 38-39. The word " >Sam " is used in the sense of " proper (or right) " ; and it serves to preclude all faulty ' prayoga.' And by " prayoga " is here meant the "functioning" of the senses with reference to their objects. In the case of the perception of silver in mother-o-peavl, the functioning of the Sense-organ i� faulty ; and hence such perceptions become piecluded (by the prefix ' Sam ' ). 

%53-54. By " cognition-production " is meant that " cognition becomes authoritative as soon as it is produced." In the case of all causes, we find that their operation is something apart from their birth (or manifestation). In order to preclude such character from the Means of Right Notion (Cognition), the word " production " has been added. 

%55. Not even for a moment does the cognition continue to exist ; nor is it ever produced as doubtful (or incorrect) ; and as such, it can never subsequently operate towards the apprehension of objects, like the Senses, \&c. 

%56. Therefore the only operation of Cognition, with regard to the objects, consists in its being produced ; that alone is Right Notion (Prama) ; and tbe cognition itself as accompanied by this Right Notion is tlie Means (of Right Notion: Pramana).

%60. The Means of Right Notion may be (1) either the sense, or (2) the contact of the sense with the object, or (3) the contact of the mind with the senses, or (4) the connection of the mind with the Soul, or (5) all these, collectively. 

%61. In all cases, cognition alone would be the Result ; and the character of the Means would belong to the foregoing, on account of their operating (towards cognition) ; for when there is no operation of these, then the Result, in the shape of cognition is not brought about. 

%71-72. When the object of cognition is the qualification itself, then the abstract (or undefined) perception subsequently gives rise to a definite cognition; and in this case the character of Pramana belongs to the undefined Perception, and that of the Result, to the subsequent definite (or concrete) cognition. 

%**** 72. When, however, there is no definite cognition, then the char- acter of Pramana could not belong (to the foregoing undefined perception); because of its not bringing about any definite idea with regard to any object. ****

%***** 78. The result being the specification of the object, the character of Pramana belongs, according to us, to that which immediately precedes it ; and so, if the cognition be said to be the Pramana, then the Result must be held to be something else. ****  FROM 71--2 I TAKE THE RESULT TO BE A COGNITION. THE FIRST HALF THEN MEANS THAT A A PRAMANA AND A COGITION ARE ABOUT SPECIFYING AN ACTUAL THING.



%-----------------------------------------------------------------------------------------------
%-----------------------------4. KUMARILA'S CONCEPTION OF KNOWLEDGE-----------------------------
%-----------------------------------------------------------------------------------------------

\section{Kum\=arila's Conception of Knowledge}
\label{conception}

%[[WILL MAKE COMPARISONS WITH CONTEMPORARY PHIL FREQUENTLY --- BEG NOT TO ASSUME MORE THAN WHAT I NOTE, AND DON'T START POSING THOSE QUESTIONS. CONTEMPORARY EPIST IS WONDERFULLY SOPHISTICATED BUT NARROW IN HISTORICAL TERMS ---LIJKEWISE MENTIONS OF HISTORICAL STUFF SHOULD NOT BE TAKEN TOO FAR; AIM IS TO USE THE FAMILIAR AS AN AID TO UNDERSTANDING THE FOREIN, AND CERTANLY NOT AS A STANDARD OF VALUE]]\footnote{Thus Robert Audi, in a recent introduction to epistemology \citep[p.\ 1]{huemer2002ecr}, can say flatly that ``knowledge is constituted by belief (of a certain kind).''}

%[[INTERESTING RE QUESTION OF CERTAINTY THAT KUMARILA TRADES ONE VARIETY OF CONFIDENCE FOR ANOTHER---CAN'T HAVE PROOF SO GOES WITH INHERENT WARRANT --- THAT HELPS US RESPOND TO A CONCERN STEMMING ROM EG CHATTERJEA'S CLAIM THAT CERTAINTY IS A PART OF THE PICTURE --- ALSO AGAIN A COMPARISON WITH PHILO THERE.]]

%[[BUT I'M STILL INTERESTED IN UNDERSTANDING AND NOT SELLING THE VIEW]] - if I were then: Bonjour's comment about the radical departure epistemology makes... - though of course it arguably goes even further back, perhaps eg to Philo]] 


%\subsection{PHILO COMPARISON}

%Taber compares the intrinsic validity thesis with Descartes' idea that some ideas are ; 


Since we wish to sympathize with \kum's theory rather than only to set it out, it will be helpful to conclude by looking briefly at a couple of historical comparisons.

For analogies with \kum's intrinsically valid cognitions, Taber adduces Descartes' ``clear and distinct ideas,'' the truth of which are self-evident, and, again, the Stoic's notion of a `\g{φαντασία καταληπτική},' an impression or appearance which contains some mark guaranteeing its own truth or accuracy.\footnote{p.\ 217.} Taber grants that, in contrast with \e{kataleptic} impressions or clear and distinct ideas, intrinsically valid cognitions are common; on Taber's interpretation all cognitions are at least initially intrinsically valid. But the more important disanalogy, which Taber must himself insist on, but nevertheless seems really to \emph{spoil} the analogy, is that intrinsically valid cognitions needn't be true, whereas the whole point of the Stoic and Cartesian notions is to isolate a special class of impressions or ideas which are sure to be such. J.~N.~Mohanty is, I think, right in saying that the whole spirit of Kum\=arila's view is to \e{reject} the notion of self-evidence: \begin{squote}it is precisely the contention of the \emph{svata\d hpr\=am\=a\d nya} theory\ldots that \emph{there is no criterion of truth, though there are criteria of error.}\footnote{\citet[p.\ 22]{mohanty1989gangesa}, his emphasis.}\end{squote}

%`\g{fantas'ia katalhptik\'h},' 

%Taber's comparison with Reid's commmon-sense philosophy is probably more apt;\footnote{pp.\ 217--18.} and I think that 


A nearer Hellenistic counterpart for \kum\ is Philo. According to what seems to be the dominant reconstruction, Philo, who led the Academy for some time, was initially a radical skeptic in the manner of Clitomachus, then later adopted the mitigated skepticism of Metrodorus, and finally in his (unfortunately not extant) `Roman Books' advocated a sort of fallibilism.\footnote{\citet{frede1987sceptic}, \citet{barnes1989antiochus}, \citet{striker1997academics}, \citet{brittain2001philo}; see \citet{sep-philo-larissa} for a survey.} The Academics seem to have generally accepted (if only for the purpose of launching counter-arguments) the Stoic understanding of knowledge, which required the afore-mentioned \e{kataleptic} impressions: knowledge requires an accurate impression appropriately caused by what it represents, and which contains some mark which shows that it meets those two conditions. The Academics argued that there were no such impressions, since any true impression could be mimicked by a false one. The skeptics, like the Stoics, thought that one ought not to assent to what one does not know; thus a radical skeptic would ideally refuse ever to assent to anything. A mitigated skepticism, on the other hand, would allow the reasonableness of forming some beliefs even without allowing the existence of a criterion of truth; some appearances might at least constitute good, if never conclusive, evidence. But ultimately Philo reinterpreted the notion of a \e{kataleptic} impression, accepting the requirements of truth and of a suitable causal basis, but removing the requirement that such an impression guarantee its own truth, which he still considered to constitute an impossible standard. This meant that it might not only be reasonable to form beliefs but that \e{kataleptic} impressions were after all possible, and one might rationally accept these without qualificaton. But since the requirement for a criterion of truth had been abandoned, one might very well be mistaken about whether one actually possessed a \e{kataleptic} impression.

Philo's innovation provoked a good deal of hostility. He had, in effect, given up the idea that knowledge must be certain (or infallible)---something assumed by dogmatists and skeptics alike. Kum\=arila's validity does not require truth, but, like Philo, he decided that the reigning epistemic standard was impossible for any cognition to attain, and thus redefined it so that it could be claimed as a matter of course, as it properly should have been. As with Philo's view, this view was met by complaints from outside \kum's own school and with disbelief from inside it. Thus a later commentator, Umbeka, found a way interpret Kum\=arila as making validity entail truth after all. In India, as in the Hellenistic world, knowledge was supposed to be infallible, and Kum\=arila's innovation, like Philo's, lies precisely in giving up that requirement as in any case impossible.

In both cases, the reinterpretation of knowledge is at least in part a response to the demands of practical activity. A major criticism of the radically skeptical position was that it made action and virtue impossible, since both require belief. Efforts were made to reconcile action with the skeptics' traditional refusal to assent, but the mitigated skeptical position amounted to a way of allowing assent without admitting the possibility of knowledge or certainty. Since the mitigated position winds up buying into dogmatic conception of knowledge in a way that radical skeptics could claim to avoid (the latter could more easily claim to `accept' it only to attack it), the Philonian position can be understood as an attempt to just find a new understanding of knowledge that was a better fit with the practical considerations driving the mitigated skeptics' concessions in the first place.

\kum\ likewise reinvents the notion of knowledge to allow for the demands of practice. Moral or \emph{dharmic} knowledge will be impossible without the authority of Vedic texts---the Buddhists, for example, are wrong to suppose that perception will suffice. But Vedic testimony cannot meet the standards imposed by the epistemological theories of other schools, which would require either a kind of self-evidence of which no cognition is capable, or verification which is in this case impossible. How then would one pursue \emph{dharma}? Human beings must act, and they must have moral guidance. It cannot be unreasonable to act as best you can, relying on the best resources available, even if the soundness of these resources cannot be established. That is part of what it is to be reasonable, and the standards of theoretical reason are misconceived if they seem to say otherwise.













\input{abbreviations.tex}

%\input{Documents/Dissertation/MAINBODY}

%\input{Documents/Dissertation/abbreviations}

%\input{csshell} %(Codanasutra 1-88) 


\nocite{jha1937tattvasangraha}
\nocite{sepkumarila}


\bibliography{bibliography}{}
\bibliographystyle{apalike}

%\bibliography{Documents/Dissertation/bibliography}

\end{document}
