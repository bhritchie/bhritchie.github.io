% Dropbox/Dissertation/ReasonableAkrasia.tex

% !TEX TS-program = xelatex
% !TEX encoding = UTF-8

%COMPILATION COMMANDS:
% xelatex -interaction=nonstopmode -file-line-error-style -synctex=1  -output-directory="Dropbox/Dissertation" "Dropbox/Dissertation/ReasonableAkrasia.tex"
% bibtex "Dropbox/Dissertation/ReasonableAkrasia"


\documentclass[11pt,letterpaper,oneside]{amsart}

%\usepackage{custom}

\usepackage{Dropbox/Dissertation/customxetex}

\author{Brendan Ritchie}
\title{Reasonable Akrasia}

\begin{document}

\maketitle





\begin{abstract}Reason prohibits \emph{akrasia}. This entails that the \emph{akrates} is sometimes irrational. Philosophers have almost universally supposed that the \emph{akrates} is \emph{always} irrational, but this confuses failure and culpable failure. The two should be distinguished in the case of \emph{akrasia} as anywhere else; I demonstrate this on the basis of general considerations and with examples.\end{abstract}


%\section{Failure vs.\ Culpability}

%[[GENDER]]

%The principle of \emph{Enkrateia} (i.e.\ `continence' or `self-control') is the most general 

%is it right to say ``most general'' what does that mean?


 %The claim that the \emph{akrates} acts against reason is liable to seem a trivial implication, or even a mere rephrasing, of the claim that he does not act as the principle of \emph{enkrateia} requires. But at a \emph{minimum} there is a conceptual difference. % To put the point another way, it is not in general the case that the failure to conform to some procedure or principle betrays some deficiency or culpability on the part of an agent. Thus the doctor's art and oath call upon him to do no harm; if he does harm, he falls short of what that art and oath require. But we would not say that in every case a doctor who does harm thereby manifests some negligence or incompetence.% And this is a quite general conceptual point about procedures.


%It might be that the procedures simply cannot guarantee success: the failure of an organ transplant does not necessarily impugn a surgeon's skill. Or success might require cooperation: a legislator might do everything he can to help revive the economy, but be foiled by the intransigence of his colleagues. There are an enormous variety of possibilities, which vary from procedure to procedure. [[LAST COUPLE SENTENCES MAKE IT SEEM LESS GENERAL---DO I HAVE SOME LARGER THEORETICAL CONSIDERATION HERE?]]



%We begin with general distinction between what I will call \emph{procedural failures} and \emph{agential failures}.\footnote{There are further distinctions to be drawn here; I will indicate some but pass over many others.}




\section{Enkrateia, Akrasia, and Irrationality}

A principle directs a person to a certain course of action when appropriate circumstances arise. Examples are the soccer player's ``never touch the ball with your hands,'' the Hippocratic doctor's ``do no harm,'' and the dinner-guest's ``don't stay too long after the other guests have left.'' Thus:\begin{description}\item[\g{A}'s principle \g{P}] \ \g{F} when \g{S}.\end{description} `When' indicates that \g{S} triggers the injunction ``\g{F}.'' That is, if \g{A} employs or is otherwise subject to \g{P}, then, when \g{S} occurs (or when \g{A} supposes that \g{S} occurs), \g{P} tells \g{A}: ``\g{F}.'' In the limiting case, \g{S} is all circumstances, or all circumstances of the relevant practice, as in the case of ``do no harm'' and ``don't touch the ball with your hands.''





%The `when' does not indicate that this is a conditional.

%(We will understand \g{F} broadly so as to include proscriptions.)

%[examples and clarifications [[GOALS MORE BROADLY? PRACTICES, PURSUITS SKILL?]]]following (participating in, pursuing, employing, called upon by) some procedure (practice, pursuit, skill, principle), and that there is some goal which is the (or a) proper object of this procedure, and attainable in or through it. For example: a politician, politics, and fixing the economy; a player, a soccer game, and victory; a doctor, the medical art, and a patient's health.GUIDES. NOT NEC CONSC. GIVEN BY SKILL OR SOME PRACTICE, GOALS ETC.] [[somewhere note super broad.]]

Sometimes \g{S} occurs, but \g{F} doesn't get done. Hence:\begin{description}\item[Procedural failure for \g{P} in \g{S}] \g{A} does not \g{F}.\end{description} But as Aristotle says, ``it is possible to fail in many ways,'' and only \emph{some} procedural failures are really the \emph{agent's} failure.\footnote{\emph{NE} II.6 (trans.\ \citetalias{ross1980nicomachean}).} If you turn without putting on your turn signals, then you have not satisfied a traffic requirement, but if the reason you did not put the signal on is only that the signal broke unexpectedly as you flipped the switch, then the failure is not \emph{yours}. Such things are always possible. So let us also define:\begin{description}\item[Agential failure for \g{P} in \g{S}] \ \g{A} does not hew to \g{P}.\end{description} Notice that agential failure is compatible with procedural success. For example, you might forget to put on your turn signal, but have it come on anyway just at the right time because of some electrical fault.




%[[ROUGHLY EXCUSES VS MAYBE ONLY JUSTIFICATION. MAYBE QUOTE AUSTIN HERE.]]

Agential failures themselves come in a variety of forms. Some agential failures manifest incompetence or negligence or another fault of that sort. Other agential failures represent deliberate abuse: As Socrates pointed out, and as Harold Shipman notoriously demonstrated, a capacity does not dictate its own use.\footnote{\emph{Hippias Minor}.} The deaths of Shipman's patients were certainly attributable to Shipman himself, but they reflected Shipman's malpractice rather than any incompetence or negligence or any other properly medical error on his part. Still other agential failures are justifiable, even in light of the principle in question (or in light of the spirit of it, or of the broader practice in which it is embedded). The Hippocratic doctor's art and oath call upon him to do no harm, and when he knowingly does harm to a patient, the failure to satisfy that requirement of his oath is an agential one, but if he does so in order to prevent some much greater harm in accordance with the further Hippocratic commitment to healing the sick whomever they may be, then his failure to avoid harm is justified, and he has not demonstrated any shortcoming as a doctor. Finally it may be noted that it is sometimes possible to beg off certain principles or applications thereof altogether, say if you are charged with being a poor parent but the child in question is not yours. But properly speaking this would mean that the principle does not apply to you in the first place (either on this occasion or \e{tout court}), and there could not in this case even be a procedural failure.

Some principles cannot be begged off. One of these is \enk. \begin{description}\item[Enkrateia] \g{F} when you ought to \g{F.} \end{description} Whenever anybody at all sees (or supposes) that he ought to \g{F}, and no matter \g{F} is, \enk\ tells him: ``\g{F.}''  Anyone who can reflect on the merits of his actions, and shape his behaviour accordingly, cannot but acknowledge the authority of \emph{Enkrateia}. One cannot even ask by what authority \emph{Enkrateia} may instruct you to \g{F}. If you think ``I ought to ``\g{F},'' and \emph{Enkrateia} then tells you: ``\g{F},'' then the basis for this instruction must be whatever basis you had in the first place for thinking that you ought to \g{F}.



%Nor, by the same token, does it make sense to ask \emph{whether} \emph{Enkrateia} has authority to instruct you to \g{F}. If `ought' talk has sense at all, and you are capable of understanding it, then \emph{Enkrateia} has application to, and authority over, you.

%After all, the principle requires nothing more than your own admission that there is something you ought to do.

%; it's just \emph{Enkrateia} never lets you off the hook

%And \emph{Enkrateia} is not shy about giving orders. But if \enk asks a lot, it also has the best credentials.








%One principle whose demands are always legitimate is \enk. When someone thinks he ought to do something, the principle of \emph{enkrateia} directs him to do that thing. That is, when an agent thinks ``I ought to \g{F},'' the principle of \enk tells him ``\g{F}.'' Hence:\begin{description}\item[Enkrateia] \g{F} when you ought to \g{F}.\end{description} 




%(A suitable variation for \emph{not} \g{F}-ing may be understood as given throughout.)







%Henceforth the capitalized `\emph{Enkrateia}' will refer to this formulation, and the uncapitalized \emph{enkrateia} will refer to the principle that forbids akrasia, however it \emph{should} be formulated.


%[[MOST ABSTRACT POSSIBLE PRINCIPLE---WOULD BE MEAINGESS EXCEPT THAT WE DO ACTUALLY SOMETIMES REFUSE TO DO IT.]]




%There is some potential for confusion here, because one might suppose that if \emph{Enkrateia} is a principle of reason, and requires you to \g{Φ} on some occasion, then \emph{reason} requires you to \g{Φ}. But this inference is fallacious. The sense in which \emph{reason} requires anything here is just that rational agents must, \emph{qua} rational agents, accept and employ \emph{Enkrateia}; but it is \emph{Enkrateia} that requires particular actions.

% be trivial at best (``if you ought to \g{F}, then you ought to \g{F}'').







%And if she were inclined to say ``if you ought to \g{F}, then you ought to \g{F},'' I would not have married her in the first place.


% This says nothing at all.



%[[point about what the requirement is]]

%[[EQYIVALENCE?]] The conditional form is also inappropriate to a principle. [[FOR ALREADY INDICATED REASONSABOUT INCORPORATING CIRC---OF COURSE THERE MIGHT BE PRINCIPLE THAT SPECIFY CERTAIN CONDITIONAL ACTIONS, BUT THAT'S DIFF]]

%That principle cannot be captured by this conditional:\begin{squote}If you see a beggar, then give him a dollar.\end{squote} Suppose you see a beggar. The principle then says: ``give him a dollar.'' And you will stand condemned by your principle if you then shuffle past awkwardly, avoiding looking in the beggar's direction. The conditional, however, has nothing to say about this, since it is true as long as you aren't looking at the beggar. [[ACTUALLY NOTICE THAT THE NARROW-SCOPE-PROCESS-REQ LOOKS PROBLEMATIC HERE TOO.]]


%[[BEGGAR EXAMPLE?]]

%[[SAY A BIT ABOUT SENSE IN WHCIH PRINCIPLES *ARE* CONDITIONAL]]





%[[FOR SIMILAR REAOSNS?]] Such formulas do not capture the requirement imposed by \emph{Enkrateia}. Now \emph{is} a direct correspondence between principles and requirements, but of a different sort. Being a principle, \emph{Enkrateia} is inherently general, but the requirements it imposes are particular---\emph{you are to \g{F}}. For illustration, take another possible principle:\begin{squote}When you see a beggar, give him a dollar.\end{squote} If I accept this principle (or if it otherwise applies to me), then when I see a beggar, the principle instructs me to give him a dollar. \emph{That} is the requirement imposed upon by the principle: that I give the beggar a dollar. Being particular in this way, the requirement also provides a criterion for whether or not the principle is satisfied in a given case, namely whether or not the beggar gets a dollar.


%[[FIRST ONE]] is a trivial truth. [[]]

%\emph{Enkrateia} does not entail these formulas, since as a principle is cannot entail anything. That an agent accepts the principle also would not entail [[truth of these formulas]], since that would allow an agent to create new obligations and requirements for himself merely by accepting new principles. [[HMM---THAT'S NOT SO IMPLAUSBLE]]]]



%Nor would the fact that \emph{Enkrateia} must be the principle of any rational agent entail that [[second one]] is true, if it means that\begin{squote}You are required by reason to \g{F }when you ought to \g{F}.\end{squote} That is because while \emph{Enkrateia} is obviously a sound principle, you might on some occasion, and quite reasonably, not realize that you ought to \g{F}, or, again, you might not be able to \g{F}.



% [[if ought doesn't make sense or you don't think this then it just doesn't]]% Happily this means that our discussion may remain neutral r HOLD ON---I'VE RELATIVIZED TO JUDGMENT WHICH I DON'T THINK I WANT...








Being a principle, \emph{Enkrateia} provides instruction or requires action. If what it instructs is not carried out, that is a procedural failure of \enk. But sometimes such a failure is due only to the arising of some insurmountable obstacle. For instance you might think you ought to sail to Crete, and attempt it, but be carried elsewhere by the wind; it is not right then to say that \e{you} sailed elsewhere. In such cases we will have a \emph{merely} procedural failure of \enk.


%Hence failures of \enk are not always instances of akrasia, and neither then will a failure of \enk always be irrational. The failure to make this distinction has caused confusion, but making it does not settle the question of primary interest here, which is not whether failures of \emph{enkrateia} are necessarily irrational, but rather whether akrasia is necessarily irrational.

Sometimes the agent \emph{departs} from \emph{Enkrateia}. `Akrasia' is the name for these agential failures of \emph{Enkrateia}, reflecting the thought that akratic action is in fact \emph{action}. Typically akrasia means that \g{F} does not get done, but, as before, this does not strictly matter---if \g{F} gets done only by chance, or because the agent waffles back \e{and} forth, the success is not due to the agent \e{abiding} by \enk.

Since \enk\ is a principle of \e{reason}---\ie a principle of all rational agents just as such---the \ak\ is irrational. \e{Unless} there are varieties of agential failure of \enk---\ie varieties of akrasia---in the same way that there are a variety of agential failure with respect to ``do no harm.'' Might it be that the \ak\ is not always irrational, in the way that a doctor who violates ``do no harm'' does not always thereby manifest incompetence? It stands to reason that we should sometimes be able to excuse or justify violations of \enk\ (from a rational perspective); the most general principles are often the most frequently violated, and violations of the most general principles are often the most readily excused (we must not confuse legitimacy with inflexibility). Besides which, \enk's rule is as capricious as its title is indisputable. It doesn't care whether your judgment is mistaken or fleeting or ill-considered or foolish or evil;\footnote{See e.g.\ \cite{bennett1974conscience}.} it doesn't care whether you have any reason \emph{whatever} to do as you think you ought;\footnote{See eg.\ \citet{broome1999normative}.} And things are more easily said than done---since \enk\ will say \e{anything}, you might expect it to be generous with failure. For instance, since a rush of enthusiasm is often all that \emph{Enkrateia} requires in order to assert its authority, might not a loss of enthusiasm often be enough to excuse me? I have often tried to so excused myself: ``I ought to go for a jog! Of course then I would have to get up. I really rather just sit here.'' I might then be lazy---but irrational? The charge would confuse me, suggesting as it does a kind of \e{blindness} to reasons, whereas I am rather defiant of them.% (A man who flies into a violent rage and strikes his toddler for dropping the remote control in the toilet acts irrationally; a man who knows that his toddler cannot be blamed for such things but still finds a vindictive satisfaction in striking the toddler is merely a brute.)





%; generality trades off with stringency. And \enk\ is nothing if not highly general; expect, perhaps, as I said, \e{capricious}: 

% If it were levelled outside

%there will be excuses and justifications in this case as well 

% She rewards the indulgent and persecutes the scrupulous. She teaches nothing and offers no reward for obedience; she doesn't care  she pays no regard to triviality or monument of your plans are, or to judgment is fleeting or deliberate. In whatever way, and to whatever extent, one can judge that one ought to \g{F}, in that way and to that extent will \emph{Enkrateia} instruct one.



%\footnote{Hence it is unnecessary to decide, for the purposes of interpreting \emph{Enkrateia}, what counts as a relevant sort of reason for the purposes of the formula's `ought' (i.e.\ moral, prudential, etc.), or what kind of judgment is required in order for \emph{Enkrateia} to instruct you on a given occasion. However, not every failure to satisfy \emph{Enkrateia} is of equal \e{interest}. If someone thinks, ``If I knew what was good for me I would \g{F}, but I sort of feel obligated to \g{Y} instead,'' and goes on to \g{Y}, thereby violating a prudential instruction of \emph{Enkrateia} but satisfying a moral instruction thereof, that is not particularly puzzling or problematic. So I assume throughout that the agents under consideration have more or less reflective views about the state of the relevant considerations; that they have a fairly definite and unconflicted view about what they ought to do (though they may not \emph{want} to do it); that these views about what ought to be done are not seriously confused or otherwise disordered (though they may be \emph{wrong}); that the agents are not themselves mentally ill or otherwise cognitively deficient; nor, finally, are they abnormally weak of will (i.e.\ abnormally prone to akratic behavior).}







%It seems to me that there are in fact many ways of excusing akrasia---many cases in which akrasia is not irrational. For example, 

%Since 

%Consider these two cases.\begin{squote}Brendan thinks he ought to get up early tomorrow to go for a jog. Being a lazy man, he doesn't \e{want} to go for a jog. But \e{tomorrow} is the day to turn over a new leaf! When his alarm goes off, he knows he ought to get off, but hits the snooze button a dozen times instead, each time fully intending to get up after this last bit of rest. Having finally gotten up, he finds himself surfing the internet. [[]]\end{squote} [[genuinely irrational]]

%The crucial question for us will be whether there are any such extenuating circumstances. I will argue that there are, and that akrasia, like other agential failures, comes in a variety of forms: it is not [SHARPEN?]

%One might suppose that if \emph{Enkrateia} demands so much and so capriciously, she ought at least to be generous with failure. 

%I once thought that I should learn \e{Ge'ez}, and even got so far as mastering the syllabary. But languages are hard and I lost enthusiasm. Was there anything \emph{more} than a loss of enthusiasm? No. I never imagined that the circumstances had changed; nor was I \emph{weak}---perhaps a bit flighty. But it was only a passing fancy. 







% It does not befit a queen to fret about a single one of her subjects: more general a principle, the more readily disregarded.



Ancient philosophers tended to take a less casual view of \emph{Enkrateia}. But perhaps that is to be expected. Consider that when Socrates lived it was not unusual to suppose that the true expert would always accomplish his aim, and that every failed attempt showed incompetence. It is a thought expressed by Glaucon in the \emph{Republic}: ``a first-rate captain or doctor\ldots knows the difference between what his craft can and can't do. He attempts the first but lets the second go by, and if he happens to slip, he can put things right.''\footnote{\emph{Republic} II, 360e--361a (trans.\ \citetalias{grube1992plato}). This is still a common view: see e.g.\ \citet{holden1977needs} on the case of medicine.} In the \emph{Euthydemus}, Socrates doesn't even bother with the qualification that there might be things a craft \emph{can't} do: ``wisdom makes men fortunate in every case [of \g{t'eqnh,} or expertise], since I don't suppose that she would ever make any sort of mistake but must necessarily do right and be lucky---otherwise she would no longer be wisdom.''\footnote{280a (trans.\ \citetalias{sprague1993euthydemus}). Adducing a variety of \g{t'eqnai}, Socrates appeals to the idea that \g{t'eqnh} is a guard against bad \g{t'uqh} (279d ff.; see below), finally going so far as to say that \g{sof'ia} (here synonymous with \g{t'eqnh}) \emph{is} (good) \g{t'uqh} (279d), or at any rate provides it (282a).} This is the basis of the Socratic thesis that Wisdom (not of a particular variety this time, but wisdom \emph{punkt}) is sufficient for happiness: that thesis expressed his hope that there could be some infallible science of living in general. Our ``salvation,'' he says, will be a ``craft [\g{t'eqnh}] of measurement, which, by ``making clear the truth'' will ``save our life.''\footnote{\emph{Protagoras} 356d--e (trans.\ \citetalias{allen1996dialogues}).} But in that case akrasia had \emph{better} be impossible, because otherwise knowledge could never \emph{assure} success.\footnote{In the \emph{Protagoras} (255d--256a), Socrates present practical deliberation as nothing more than figuring out where the balance of pleasure lies, pleasures all being both comparable and quantifiable. This is plainly something of a toy model of deliberation tailored precisely to preclude akrasia (Socrates isn't actually a hedonist, for example). I discuss this topic further in \citet{ritchiesocrates}.} And Plato's solution was to say that although akrasia does occur, it is a disease of the soul, and in principle curable. Happily, the cure is precisely that wisdom which Socrates sought, namely \g{filosof'ia}.\footnote{\emph{Republic} IV, 439c ff.\ for akrasia; the \emph{Republic} as whole is, \emph{inter alia}, a description of the philosopher.} These two had many followers.









%What the \emph{Oath} is to the Hipocratic doctor and elegance is to the mathematician, \emph{Enkrateia} was to the philosopher.%, and it should not surprise us that that they would have guarded her as jealously as other guilds guard their principles. Their professional pride was, in fact, where the philosopher took his inspiration.%\footnote{[[INDIA ANC CHINA]]} 


%[[CENTRAL PART OF THEIR KIT

%Philosophers are by definition the attendants to \emph{Enkratei}.

%$\phi\iota\lambda$\emph{o}$\sigma$\emph{o}$\phi\acute{\iota}\alpha$

Not that philosophers have been alone in their consternation about akrasia. Anyone can sympathize with Paul's famous lament to the Romans:\begin{quote}I have the desire to do what is good, but I cannot carry it out. For what I do is not the good I want to do; no, the evil I do not want to do---this I keep on doing\sdots I see another law at work in the members of my body, waging war against the law of my mind\sdots What a wretched man I am!\footnote{Romans 7:18--24 \citepalias{packer2001esv}.}\end{quote} \emph{Sympathize}, yes---but before we suppose that this is just an eloquent statement of a universal sensibility, let us note Paul's remark elsewhere that ``the Greeks seek wisdom, but we preach Christ crucified,''\footnote{I Corinthians 1:20--23.} and then pay more attention to the next two sentences of Romans: ``who will rescue me from this body of death? Thanks be to God through Jesus Christ our Lord.'' We mustn't mistake this for mere piety; Paul understood that philosophers were offering knowledge as a rival means of salvation, and he has every reason to represent akrasia as not only \e{bad}, but as a depravity so great that we can only hope for divine deliverance. %And besides the power the passage has in it's own right, Paul's thought entered philosophy itself through Augustine.\footnote{See e.g.\ \emph{Confessions} VI.xi--xii.}







%[[A WORLD WHERE PEOPLE EXPECT...]]

%PLUS AS WE'VE SEEN THERE'S THIS GENERAL TENDENCEY



%maybe not so surprising that for a long time nothing...



%l---like Augustine and others who followed him---for making akrasia sound 




% Paul finds ``another law waging war against the law of my mind and making me captive to the law of sin that dwells in my members.'' 

%
%
%It is no accident that someone like Paul 





%
%When you look at the role of this idea about akrasia in philosophical systems, not only in wet but India and China, it's bascially the same thing. [[[WIlliams.]]




%The apostle Paul thought that he knew how people could be saved, but it wasn't through knowledge:

%\begin{quote}Where is the one who is wise? Where is the scribe? Where is the debater of this age? Has not God made foolish the wisdom of the world? For since, in the wisdom of God, the world did not know God through wisdom, it pleased God through the folly of what we preach to save those who believe. For Jews demand signs and Greeks seek wisdom, but we preach Christ crucified, a stumbling block to Jews and folly to Gentiles.\footnote{\emph{1 Corinthians} 1:\ 20--23. Translation: English Standard Version.}\end{quote}Since we've seen what ``wisdom'' meant to the Greeks, we know that Paul isn't praising ignorance but rather rejecting the idea of \emph{salvation} through knowledge---through philosophy, or a science of living.\footnote{cf. [[XX]]} And it's not just knowledge as the Greeks understood it that comes in for criticism. ``I was advancing in Judaism beyond many of my own age among my people, so extremely zealous was I for the traditions of my fathers.''\footnote{\emph{1 Corinthians} 15:\ 14.}  So Paul tells us---and yet he has a very low view and the practical value of all his study:\begin{quote}I have the desire to do what is good, but I cannot carry it out. For what I do is not the good I want to do; no, the evil I do not want to do---this I keep on doing. . . . In my inner being I delight in God's law; but I see another law at work in the members of my body, waging war against the law of my mind and making me a prisoner of the law of sin at work within my members. What a wretched man I am! Who will rescue me from this body of death?\footnote{\emph{Romans} 7:\ 18--19, 22--24}\end{quote}That is a rather strong statement about the role of \emph{akrasia} in human life, even allowing for a certain amount of rhetoric.\footnote{There is also a suggestion that there is more than one agent at work here, which admittedly could complicate the question of how Paul understands the psychological events here. I'll assume that there is a firm acknowledgement of \emph{akrasia} here, whatever else there may be.} 

% But how could such wisdom assure a good life if akrasia is possible? The function of toy-model mentioned above is exactly to ward off this potential threat to Socrates' hoped-for salvific knowledge. Plato may have allowed the possibility of akrasia, but [[]] [[THUS BETTER FITTING THE MAJOR MODERN APPROACH]]. And such was the path of many ancient thinkers.
%, where Glaucon says that 

%Some philosophers have professed themselves so baffled by the thought that anyone might deviate from \emph{Enkrateia}'s guidance that they have declared the thing altogether impossible.\footnote{Socrates, famously, thought the notion of akrasia to be incoherent, and was followed in this by many philosophers of antiquity (\emph{Protagoras} 352b--357e). \citet[pp.\ 165--70]{hare1952language} thought that evaluative judgments had an essential action-guiding role which precluded the possibility of akrasia. Even Davidson's influential paper does not really account for the ``wide-eyed'' cases of interest to us, apparently for similar reasons. Again, it has sometimes been felt that akratic `action' may not after all \emph{be} adequately subject to reflective control to really count as acton.} The rest have soundly denounced any [[shortcoming...]]]

%Plato sees in the \emph{akrates} a diseased soul. Aristotle considers him impaired, like a man drunk or mad. Augustine finds himself ``bound'' and in need of divine deliverance.\footnote{\emph{Republic} IV, 439c ff. (trans.\ \cite{grube1992republic}); \emph{Nichomachean Ethics} VII.3; \emph{Confessions}.}

Nowadays, `philosophers' do not claim to offer salvation, nor do they much seek it. One will expect, then, that discussions of akrasia will reflect the lowered stakes. And that does generally seem to be the case. Davidson argues in his classic paper that akrasia is not a moral failing, and that the \emph{akrates} need not be overcome by passion or even anything we could properly call `temptation.' But he mistakes the nature of his innovation when he says that he knows ``no clear case of a philosopher who recognizes that incontinence is not essentially a problem in moral philosophy, but a problem in the philosophy of action.'' In fact Socrates' discussion in the \e{Protagoras} concerns the relationship between knowledge and action quite generally,\footnote{There is however much talk of being ``overcome by pleasure,'' and this kind of appeal to temptation may be part of what Davidson has in mind as `moral philosophy.'} and while Plato treats akrasia as revealing a moral failure (\ie that dissolution of character which constitutes the vice of `injustice'), that is his \e{aim} rather than an \e{oversight}, and Plato thereby inaugurates a long (and still well-represented) tradition of treating moral failure as a species of practical failure \e{simpliciter}.\footnote{See \citet{korsgaard1999self} for a sympathetic discussion of the relationship between internal order and moral theory in the views of Plato and Kant.} Davidson is really noticing the loss of the broadly ethical (and later theological) urgency which the topic once had.\footnote{It is telling that the topic was largely ignored by philosophers for centuries (Gosling, \citeyear[ch.\ 7]{gosling1990weakness}), and that \citet[pp.\ 175--7]{sidgwick1893unreasonable}, in bringing it back to their attention, already understands it as ``voluntary action contrary to a man's deliberate judgment as to what is right or best for him to do,'' informing us that `judgment' is \e{not} to be read as `moral judgment.'}

%; he does not even bother with the suggestion that it might stand between us and salvation. 

%STILL LIKE THIS LINE: So far have our aspriations fallen that the question ``what is your philosophy'' is to us a joke


``But what, on this analysis, is the fault in incontinence?'' Absent the old answers, Davidson's question is the pertinent one. His answer is that\begin{quote}the incontinent man acts, and judges, irrationally, for this is surely what  we must say of a man who goes against his own best judgment.\footnote{\citet[p.\ 41]{davidson2001weakness}.}\end{quote} This is apparently intended another way of phrasing a further point:\begin{quote}There is [a] principle the rational man will accept in applying practical reasoning: perform the action judged best on the basis of all available relevant reasons. It would be appropriate to call this the principle of continence.\end{quote} But these two ideas are not equivalent. In the second quote, Davidson says (more or less) that \enk\ is a principle of reason, and that much is indisputable. But how much can that tell us about the \ak\ himself? Davidson supposes that the \ak\ thereby acts irrationally---\e{against} reason, we might say---but that is not the right way to think about principles in general. For example, ``be parsimonious'' and ``be perspicuous'' are indisputably good principles for a writer. Hardly any book could better demonstrate their importance than does the \e{Critique of Pure Reason}. Can we suppose that, since the \e{Critique} is thus a poorly written book, \e{Kant} showed himself a poor writer in writing it, or that on this occasion he wrote poorly, or like a poor writer?\begin{quote}Although the book is the product of nearly twelve years of reflection, I completed it hastily, in perhaps four or five months, with the greatest attentiveness to its content but less care about its style and ease of comprehension. Even now I think my decision was correct, for otherwise, if I had delayed further in order to make the book more popular, it would probably have remained unfinished.\footnote{Letter to Moses Mendelssohn, August 16, 1783 (trans.\ \citetalias{zweig1967kant}); cf.\ \e{CPR} B.xxxvii ff.}\end{quote}Insofar as there is truth to what Kant says, the excuse is excellent---hardly any other book's content could so well justify the stylistic and expository shortcomings of the \e{Critique}, and hardly anyone but Kant could have set out such thoughts in a mere six months. Kant produced a poorly written book, then, but we would need further evidence before we could say whether \e{Kant wrote badly} in writing it; perhaps few stylists could have produced such a monument in six months and made it \e{more} perspicuous and parsimonious than is the \e{Critique}.

%We wish to be careful, then, in what we infer about an agent merely from the fact that he has violated some principle which 




What about the present case? Consider one of Davidson's own examples:\begin{quote}I have just relaxed in bed after a hard day when it occurs to me that I have not brushed my teeth. Concern for my health bids me rise and brush; sensual indulgence suggests I forget my teeth for once. I weigh the alternatives in the light of the reasons: on the one hand, my teeth are strong, and at my age decay is slow. It won't matter much if I don't brush them. On the other hand, if I get up, it will spoil my calm and may result in a bad night's sleep. Everything considered I judge I would do better to stay in bed. Yet my feeling that I ought to brush my teeth is too strong for me: wearily I leave my bed and brush my teeth. My act is clearly intentional, although against my better judgment, and so is incontinent.\footnote{\emph{Ibid.}\ p.\ 29.}\end{quote} This is indeed a clear example of akrasia. But is it equally clear that Davidson here acts irrationally? Davidson sketches the case as though he gets up out of a concern for the negative repercussions health-wise of not brushing his teeth \e{on this particular occasion}. But that really would be unintelligible; a single occasion, as Davidson acknowledges, really doesn't matter at all. What gets Davidson up is a deliberately and quite sensibly habituated sense of duty---a sense of duty the very point of which is to make it hard for him to excuse himself, since almost every occasion is an occasion on which you could excuse yourself. Unless we think that habituated action of this sort is irrational just as such, the temptation to see Davidson as irrational here must stem simply from the fact that he does, indeed, violate \enk.






%All we know, then, is that Kant produced a poorly writen book, but for all we know thus far, he showed himself a master stylist master in writing it.  %Not every departure from principle .



%Might not the same be true in our case? If all that the \ak does is to fail to satisfy \enk, might that not be excusable, as in 

%complain of him Kant as writer. [[not sharp---emphasize that qua writer...]]]





%This part of what Davidson says is (approximately) right:\begin{squote}There is [a] principle the rational man will accept in applying practical reasoning: perform the action judged best on the basis of all available relevant reasons.\footnote{\citet[p.\ 41]{davidson2001weakness}.}\end{squote} This says that \emph{Enkrateia} is a principle of reason. 


Davidson's tone grows still darker.\begin{quote}What is hard is to acquire the virtue of continence, to make the principle of continence our own. But there is no reason in principle why it is any more difficult to become continent than to become chaste or brave. One gets a lively sense of the difficulties in St.\ Augustine's extraordinary prayer: `Give me chastity and continence, only not yet.'\end{quote} The passage from Augustine is interesting as an instance of the dynamic I mentioned earlier. In the very previous sentence Augustine tells us how, ``when upon the reading of Cicero's \e{Hortensius}, I was first stirred up to the study of wisdom,'' and it is clear that philosophy has created in Augustine a desire which it cannot satisfy; it has turned out that incontinence is not merely a vice---he calls it a ``disease''---but that Augustine's subjugation to bodily desire is so complete that only God could provide deliverance.\footnote{\e{Confessions} VIII.vii (trans.\ \citetalias{watts1960confessions}).} The passage thus represents an admission of the Pauline thought into philosophy itself.\footnote{Cf.\ \e{Confessions} II.vii: ``What man is he, who upon consideration of his own infirmity, dares so far to ascribe his chastity and innocency to his own virtue, as that he thereupon should love thee less; as if thy mercy, by which thou forgivest those that turn unto thee, had been less necessary for him?''} Augustine's rhetoric makes little sense in the present context, given Davidson's observation that akrasia need not involve succumbing to any temptation, let alone moral temptation. In particular, it is not desirable---it would not manifest virtue---to be able to pass on brushing one's teeth whenever one thinks it not to merit the inconvenience.% Such a capacity would certainly not be a virtue, if we can even make sense of such a thing.





%No doubt akrasia is sometimes accompanied by this sort of distress, but we cannot assume that such distress has any essential connection with akrasia as \emph{Davidson} understands it.


%And this is an unfortunate passage for Davidson to have chosen. 

%[[But concern for health never bids one rise and brush \e{on one particular occasion};  If Davidson could always resist the urgings of this kind of habitual sense of duty, then habits would lose their point for him. Such self-control would be extremely unfortunate, if it's even intelligible---certainly it is not an aspect of any kind of self-control we would call a \e{virtue}.]]


%There is no reason to suppose that this rhetoric will be appropriate in any but a small range of Davidson's cases, nor even to suppose that the depravity hinted at \emph{can} be overcome in teh way Davidsn suggests.

%In the same passage Davidson implies that akratic action manifests a vice, but this cannot be warranted by what he has said about the rational man's principles, for the reasons already indicated in the remarks about Kant.



Indeed the language of virtue and vice is altogether out of place here. It is hard, admittedly, to \e{avoid} that language. `\g{>Akras'ia}' is Aristotle's word---Plato's dialogues employ only cognates---and for Aristotle, `\g{>akras'ia}' is indeed a weakness of \emph{character};\footnote{Although for Aristotle continence is not exactly a \e{virtue}, properly speaking.} likewise, `\g{>egkr'ateia}' was a positive trait (Plato uses this word, but not in the earlier `Socratic' dialogues).\footnote{\citet[p.\ 503 n.\ 15]{vlastos1983historical}.} By extension, akratic actions will of necessity manifest a weakness of character. Accordingly we translate `\g{>akras'ia}' as `weakness of will' or `incontinence.' But this is somewhat unfortunate, because those terms then import the suggestion of a character flaw into the contemporary discussion which is addressed firstly to actions rather than traits---i.e.\ an `akratic' action is one that does not conform to one's considered judgment, \emph{whether or not} this manifests some special character flaw. Aristotle himself, precisely \e{because} he was treating akrasia as a vice, rejected the idea that every failure to act as you think you ought manifests that failure. Firstly, akrasia proper was a weakness with respect to a particular range of pleasures; other kinds of weakness might be blameworthy, and we might call these other varieties of weakness `akrasia' or `akrasia with respect to X' by analogy, but each form of `akrasia' would not then manifests the \e{same} weakness. Secondly, not every failure to do as you think you ought will be vicious or otherwise bad in any way at all.\begin{squote}If continence makes a man ready to stand by any and every opinion, it is bad, i.e.\ if it makes him stand even by a false opinion; and if incontinence makes a man apt to abandon any and every opinion, there will be a good incontinence, of which Sophocles' Neoptolemus in the \emph{Philoctetes} will be an instance; for he is to be praised for not standing by what Odysseus persuaded him to do [i.e.\ win the trust of Philoctetes (who proves an extremely sympathetic character) in order to betray him to Odysseus], because he is pained at telling a lie.\footnote{\emph{NE} VII.2, 9.}\end{squote} Leaving aside whether Aristotle is right about whether violations of \enk\ must show \e{some} kind of fault (\ie some fault of reasoning), he surely \e{is} right that a capacity for perfect conformity with \enk\ (leaving aside mere procedural failings) would not amount to some \e{one} virtue, nor really a \e{virtue} at all; again, akratically getting up to brush one's teeth manifests no fault of \e{character} at all, let alone the \e{same} fault as, say, Augustine's.

%So Aristotle would not allow that akrasia in \emph{Davidson}'s sense is akratic in his sense. 

%Reflection on the tooth brushing case, and on the difference between this case and Augustine sexual weakness, should make that clear.



%[[One of Davidson's own examples will help to illustrate these points.]] [[But concern for health never bids one rise and brush \e{on one particular occasion}; a single occasion, as Davidson acknowledges, really doesn't matter at all. You could always excuse yourself. If Davidson could always resist the urgings of this kind of habitual sense of duty, then habits would lose their point for him. Such self-control would be extremely unfortunate, if it's even intelligible---certainly it is not an aspect of any kind of self-control we would call a \e{virtue}.]]

%We have here virtue, not vice.

%[[certainly a great variety of cases---not all properly a single vice here]]] --- so reimports all this old stuff...

Davidson concludes his essay as follows.\begin{squote}Why would anyone ever perform an action when he thought that, everything considered, another action would be better?\ldots\  If the question is read, what is the agent's reason for doing \e{a} when he believes it would be better, all things considered, to do another thing, then the answer must be: for this, the agent has no reason. We perceive a creature as rational in so far as we are able to view his movements as part of a rational pattern comprising also thoughts, desires, emotions, and volitions\sdots But in the case of incontinence, the attempt to read reason into behaviour is necessarily subject to a degree of frustration.

What is special in incontinence is that the actor cannot understand himself: he recognizes, in his own intentional behaviour, something essentially surd.\end{squote} Applied to the akratic brushing of teeth this is beyond hyperbole. Even Augustine does not go this far, though his weakness is a source of real frustration. We know why he prays as he does because he tells us, precisely answering the question why he acts as he does rather than in the way he judges best:\begin{squote}For I was afraid that thou wouldst hear me too soon, and too soon deliver me from my disease of incontinency; which my desire was, rather to have satisfied than extinguished.\end{squote}And if Augustine is circumspect here, he has recently spoken more frankly:\begin{squote}I thought I should be too miserable, should I be debarred of the embracements of a woman.\footnote{VI.xi.}\end{squote} There is only too little mystery here. 

%[[Davidson's dictinstion here not very plausible...]] 


And again it quite mistaken to suppose that when we \e{are} opaque to ourselves, the fault must be in our actions rather than in our understanding. Augustine tells us that as the same young man praying that extraordinary prayer, he was still dabbling in Manichaeism, and would tell himself he was delaying a serious pursuit of continency because ``there did not appear any certain end, which I was to direct my course unto.'' But eventually he recognized that he \e{preferred} to find things uncertain, for exactly the reason that he preferred incontinence.\begin{squote}Now was the day come wherein I was to be set naked before myself, and when mine own conscience was to rebuke me: ``Where is thy tongue? Surely thou wert wont to say, how that for an uncertain truth thou wouldst not yet cast off the baggage of vanity. See, certainty hath appeared now, and yet does that burden still overload thee\ldots .'' Thus felt I a corrosive within, yea most vehemently confounded I was with a horrible shame.\end{squote} Augustine had been incontinent, but to the extent he had not understood himself, it was not because his action could not be made intelligible but because he wished to avoid knowing himself too well. That there is no mystery in his incontinence is acknowledged here by Augustine himself, who stands `naked' before his own conscience, and is ashamed. He is \e{ashamed} because he sees quite clearly that the man he loathes is the man \e{he made himself into}.\begin{squote}With what scourges of condemning sentences lashed I not mine own soul, to make it follow me, endeavouring now to go after thee! And it drew back: it refused, but gave no reason to excuse its refusal by. All its arguments were already spent and confuted, there remained a silent trembling; and it feared, like the death, to be restrained from the sore of custom, which made it pine away even to the very death.\end{squote} Ultimately Augustine's soul ``gives no reason,'' but what Augustine means by this is just that what he has become accustomed to do is no longer something he sees any merit in. But this refusal does \e{not} mean that his actions are \e{unintelligible}, only that they are, in fact, habituated.


%In toothbrush case it's quite absurd; in Augustine's case it quite misses the actual issue, which is far graver---it's only too intelligible what has happened: he made himself into something that he loathes



%, except to the extent that habituation itself is unintelligible. 


%and in another paper, he says we would wish to know the `cure' for the mental partitioning which akrasia implicates.\footnote{\citet[p.\ 42]{davidson2001weakness}; \citet[pp.\ 300--1]{davidson1982paradoxes}.} 


%[[YES IT REFUSES HIM...

%UNLESS HABIT SIMPLY FALLS OUTISED THE REALM OF REASON...









% Moreover, weakness here is an essentially \emph{relative} rather than an essentially \emph{deficient}. 

%For example, supposing that the pride which inclines you to stay in the line of battle wins out over the fear which inclines you to run. Fear is then weak, but that is not a denigration of either the attitude or the person. This understanding is more suitable to the contemporary discussion, and is the one I adopt: the \emph{akrates} is not (or not \emph{firstly}) a weak \emph{person} but someone whose \emph{judgment} as to what he ought to do does not carry the day.

%Likewise, in the \emph{Gorgias}, Socrates concedes that the testimony of the orator, who knows how to play on the emotions, will, among the ignorant, be more compelling than the testimony of a genuine expert.



%Beliefs have many sources besides one's own expertise: wishful thinking, sadness or anger, nearby pleasures or faraway pains, optical illusions, the advice of experts or of charlatans. These forces sometimes compete, and some are stronger than others.


% But, according to Socrates, knowledge (if you possess it yourself) dominates all other sources of belief. That is what he means by saying that knowledge is \emph{strong,} as he puts it in the 




% [[The habitual sense of duty which forces Davidson out of bed here is exactly what Davidson There is no sense in which this is vicious. It is also quite unclear whether [[anything intelligible in the thought.]]

%The case is exactly as intelligible as habit.

%The case is perfectly intelligible because we all know that one point of forming such habits is precisely to avoid allowing ourselves excuses all the time. [[can't understand everything in isolation.]] 




Davidson requires a more Socratic understanding of akrasia. Socrates' thesis was first of all that \emph{knowledge} is the strongest (\g{kr'atistos}) thing there is---stronger, that is, than fears, pleasures, and anything else that might otherwise lead you to the wrong view of things. We see from this, firstly, that we could equally well talk about the strength or weakness of any attitude or emotion or disposition which might lead you to a view of things; secondly, that weakness is here relative; thirdly, that `weakness' is not essentially a term of disapprobation as regards either the attitude or the agent who possesses it. For example, in their speeches in the \emph{Symposium,} Phaedrus and Agathon argue that fear would not drive you from battle while you fought beside your beloved---love, then, will be stronger than fear, but this does not entail that there is anything wrong with fear, or with feeling fear, just as such. So while Aristotle discusses the \emph{agent} who is unable to rule himself according to his judgment, Socrates asks whether knowledge rules \emph{him}, or whether knowledge is rather, as the vulgar suppose, ``dragged about like a slave.''\footnote{This is why, in the \emph{Protagoras} (352b), Socrates says that without a \emph{techne}, your beliefs are at the mercy of other forces; if you possess a \emph{techne}, you will think the right way about the appropriate cases. Thus \citet{penner1997ssk} is right to insist that the ``knowledge is strong'' thesis is not implied by the denial of akrasia in \e{our} sense. Cf.\ also \citet[pp.\ 257--8]{allen1960sp} and my discussion in \citet{ritchiesocrates}.} This is nearer to the conception of akrasia we should be employing insofar as we are interested first of all in principle and actions rather than traits. In particular, the `Socratic' understanding which I adopt as most suitable to the contemporary understanding of akrasia is this: the \ak' judgment as to what he ought to do is what is weak, in the sense that this judgment is not decisive in action. What we should say about the \ak\ himself---i.e.\ whether \e{he} is weak or otherwise errs from the perspective of rationality---remains an open question.



Allow me to briefly motivate the idea that the weakness \e{of one's considered judgment about what ought to be done} is not the same as the weakness of the agent \emph{per se}---or equivalently, the idea that akrasia, as an agential failure of \enk, is not necessarily rationally blameworthy or culpable. There are in fact an enormous variety of cases here, and a great variety of verdicts which may be rendered regarding the \ak\ himself; I have already gestured at the case of habituation, and there was also Aristotle's `good akrasia.' Take now another example not so far from that of Neoptolomus:\begin{squote}\e{Impossible Choice.} God has commanded Abraham to kill Isaac, his only son.\end{squote} Suppose that Abraham had been less resolute, and had shied away from the deed before God stopped his hand. It is quite unnecessary to suppose he would then have been somehow inadequately responsive to the demands of reason. A proper sensitivity to the \enk\ could quite as well be demonstrated by aguish over the choice---however it is made---as by conformity.





%Nevertheless, Davidson's severe view is widely shared among contemporary philosophers. Akrasia is routinely said to be \emph{eo ipso} irrational, or worse. T.\ M.\ Scanlon says that ``akratic actions (and irrational thoughts) are cases in which a person's rational capacities have malfunctioned.''\footnote{\citet[p.\ 40]{scanlon1998we}.} Christine Korsgaard says that the \emph{akrates}' ``volition is defective: he has performed an abortive act of will.''\footnote{\citet[p.\ 34]{korsgaard2009self}.} Elsewhere we find that akrasia is one of the ``distinctive defects'' or ``pathologies'' of practical reason;\footnote{\citet[p.\ 224]{wallace1999three}; \citet[p.\ 53]{pettit1993practical}.} it is ``perverse,'' and ``a form of self-alienation.''\footnote{\citet[p.\ 37]{buss1997weakness}.} Expressions of such sentiments are innumerable. In her article on ``Weakness of Will'' for the \emph{SEP}, Sarah Stroud is voicing a common sentiment when she says that showing akrasia to be ``a puzzling, marginal, somehow defective instance of agency'' is a reasonable \emph{constraint} on a theory of practical reason.\footnote{\citet{stroud2008weakness}.}

%I believe the same confusions I identified in Davidson's case are responsible for this this consensus. Firstly, it is routinely supposed that if \enk\ is a principle of reason, the the \ak\ must be irrational. Often this thought is expressed by saying something along the lines of: ``it is irrational to defy one's own judgement.''\footnote{This is \citet[pp.\ 242--3]{kolodny2007does}, though he is here discussing akratic belief rather than akratic action.} Even those few scholars who have argued that the irrationality of akrasia had been overstated are often apt to grant that akrasia displays \e{some} such defect of procedural reasoning. Thus Robert Audi, though suggesting that akratic action need not always be irrational, also thinks ``we can agree that one who acts incontinently is not functioning normally, or even that such agents are not functioning at all well.''\footnote{\citet[p.\ 276]{audi1990weakness}. Likewise, \citet[pp.\ 285--6, 302]{mcintyre2006wrong}, who thinks it is misleading to say that the \ak always acts irrationally, also says akrasia exhibits ``practical defects,'' both procedural and substantive. See also \citet{arpaly2000acting}, \citet{audi1990weakness}, \citet[pp.\ 188--90]{frankfurt1988rationality}, \citet{jones2003emotion}, \citet{mcintyre1990akratic, mcintyre2006wrong}, and \citet{tappolet2003emotions}. These defenses (or partial defenses) of the possibility of reasonable akrasia appeal to broadly substantive considerations. So for example it has been observed that the emotions or sympathies which may lead us to act akratically are sometimes better attuned to our values and concerns, or, more broadly, to whatever considerations are relevant, than are our considered judgments. On such views, akrasia might on such occasions manifest some form of incoherence or other shortcoming, but it might be less probematic than other non-akratic alternatives, and as such it may be that the \ak demonstrates no very special failing, and the censure of `irrationality' may be too strong. My approach is quite different. I assume that \enk is the paramount principal of practical reason, and nothing I say could diminish the importance or authority or centrality of reflection. Though this may seem paradoxical, my claim that some instances of akrasia manifests \emph{no irrationality or procedural dysfunction whatever} is a mark of the \emph{higher} place my account assigns (or certainly permits) to \enk. If violations of \enk not always irrational, that is only because \emph{no} procedure could be such that agential failure is just \emph{ipso facto} a culpable failure.} But, as I have pointed out, it is \e{not} trivial to infer, from an agent's violation of a principle, some culability  or shortrcoming on the part of that agent.

%Secondly, there is no question that the distinction between akrasia as a weakness of character and akrasia as the inefficacy of a judgment is frequently elided. For example, R.\ M.\ Hare, who has been considering ``the state of mind called `thinking that I ought','' tells us that\begin{squote}Our morality is formed of principles and ideals which we do not succeed in persuading ourselves to fulfil. And this \emph{inability} to realize our ideals is well reflected in the highly significant names given in both Greek and English to this condition: Greek calls it \emph{akrasia}---literally `not being strong enough (sc.\ to control oneself)'; and English calls it `moral weakness' or `weakness of will.'\footnote{\citet[p.\ 77]{hare1963freedom}, his emphases.}\end{squote} Recall that \emph{Socrates} was worried about the possibility that your fear, say, might undermine your ability to discern the truth, the latter ability in that case being weak. But he never allowed that the kind of loss of control Hare is interested in was possible at all. The ``highly significant'' names reflect the Aristotelian understanding, applied, in an un-Aristotelian fashion, to \e{all} cases of `thinking one ought.'






%But again, these characterizations will frequently be inapt if we keep in mind that, on the contemporary understanding, akrasia should be understood (at least first of all) as a weakness \e{of one's considered judgment about what ought to be done} rather than as a weakness of the agent \emph{per se}. What could it mean, for example, to say that an akratic Neoptolomus has ``performed an abortive act of will,'' or that his action is ``perverse,'' or that he suffers some ``pathology'' or ``self-alientation''?

Again, some things are just extremely hard.\begin{squote}\e{Audacious Ambition.}\ Nick Vujicic was born without arms or legs. As a child he was suicidal. As a teenager he read John 9:3, in which Jesus says that the man to whom he has just given sight was born blind ``that the works of God might be displayed in him.'' Vujicic took this to mean that he was supposed to live a full life and find ways to inspiring hope in others. There was still much frustration and anger in the years that followed, but Vujicic has learned to do a surprising number of things without limbs and is now a successful motivational speaker.\end{squote} A failure to live and act as he though he ought after reading John 9:3 would have been far less surprising than is Vujicic' success, which is nothing short of heroic. Had he given up in frustration, it would be a cruel and absurd insult to say that his failure was irrational or unintelligible.



%success is heroic. Failure would be much more readily intelligible than is the success, as suggested by Ralston himself in a newspaper report:\begin{squote}``I'm not sure how I handled it,'' Mr. Ralston said. ``I felt pain and I coped with it. I moved on.''\footnote{``Climber Speaks Of His Ordeal In Utah Canyon,'' \e{NY Times}, May 9, 2003.}\end{squote} Vujicic and Cheng were not so far from suicide at various points, and that outcome would likeiwse have been less surprising. Had one of these men given up in frustration, most of the epitaphs above would constitute only cruel and absurd insults.



%\noindent \e{Aron Ralston:} In order to free myself, I will have to amputate my good arm with this pocket-knife.\footnote{I have in mind the climber Aron Ralston.}


%\noindent \e{Zhang Cheng:} Blind teenagers in China---like myself---have it rough, so I should start a free vocational school for them.\footnote{``A School in China Gives Hope to the Blind,'' \e{Toronto Star}, June 1, 2010.}\end{squote}

Some people perform heroic feats, and others possess that ``heroic and divine kind of virtue'' which lies beyond continence. Aristotle says it ``is seen in few people and seldom,'' because he was not acquainted with Chinese women.\footnote{\e{NE} VII.1, 9.}\begin{squote}\e{Child of a Tiger Mom.} Hui Hong may be found at her desk in Gund Hall 16--18 hours a day, six days a week. It used to be seven days a week, but during a recent summer break (spent volunteering) she became a Christian, and thereafter came to understand that she was \e{obligated} to take a day of rest---though she has adopted the Jewish conception of a `day,' which allows her to begin working again at 6pm, so she still gets in six or so hours. Her only other break (besides sleep, when she has time for it) is the half hour a day she spends at a a caf\'e with some fellow students. The trips to the caf\'e are almost always akratic; she really doesn't think she can afford the time.\end{squote} Hui Hong thinks that her will is weak, since she was raised never to be satisfied with her efforts. But if that were really so the notion would be meaningless.



%It seems that scholars generally pass silently over Aristotle's suggestion that there is a counterpart to the \ak\, namely the person possessed of a  But Aristotle is right that there is something beyond continence. He says this is overlooked becasue this ``extreme is seen in few people and seldom,'' but we have less excuse for overlooking the godlike characters because we know more Chinese women than Aristotle did.


At a far remove is the perverse-minded Edmund Pevensie:\begin{squote}\e{Spite.} Edmund has been ridiculing Lucy's talk of a magical land called `Narnia,' while Peter and Susan have been worrying about her mental health. Then Edmund discovers Narnia for himself. He could reassure Peter and Susan, but then he would have to apologize to Lucy, and he'd really rather keep ridiculing her and generally acting superior.\end{squote} Edmund knows perfectly well what he \e{ought} to do, and does otherwise precisely what he ought to do would be unpleasant. This shows a moral perversity, but not a shortcoming in terms of his \e{reasoning}. It is his clearheaded and willful disregard of his brotherly duties that makes Edmund's behaviour so nasty.


Such examples indicate to me that nothing much can be inferred from the fact that someone has acted akratically. Until we know exactly what happened, we have no basis for seeing any kind of fault in him. We may then sum up the discussion by saying that in his focus on individual actions, Davidson employs a `Socratic' conception of akrasia, but in his verdict on akrasia he relies on the Aristotlean conception of akrasia as a defect of character, and perhaps also on Augustine's conception of akrasia as a depravity. But in many cases, it is simply not correct to say that the \ak\ (in our sense) manifests some irrationality, or acts in an unintelligible manner, or manifest a vice, or anything of that sort.



Nevertheless, Davidson's severe view is widely shared among contemporary philosophers. Akrasia is routinely said to be \emph{eo ipso} irrational, or worse. T.\ M.\ Scanlon says that ``akratic actions (and irrational thoughts) are cases in which a person's rational capacities have malfunctioned.''\footnote{\citet[p.\ 40]{scanlon1998we}.} Christine Korsgaard says that the \emph{akrates}' ``volition is defective: he has performed an abortive act of will.''\footnote{\citet[p.\ 34]{korsgaard2009self}.} Elsewhere we find that akrasia is one of the ``distinctive defects'' or ``pathologies'' of practical reason;\footnote{\citet[p.\ 224]{wallace1999three}; \citet[p.\ 53]{pettit1993practical}.} it is ``perverse,'' and ``a form of self-alienation.''\footnote{\citet[p.\ 37]{buss1997weakness}.} Expressions of such sentiments are innumerable. In her article on ``Weakness of Will'' for the \emph{SEP}, Sarah Stroud is voicing a common sentiment when she says that showing akrasia to be ``a puzzling, marginal, somehow defective instance of agency'' is a reasonable \emph{constraint} on a theory of practical reason.\footnote{\citet{stroud2008weakness}.}



%Our conception of akrasia combines Aristotle's deficiency of character with Socrates' concern for success on given occasions. It is unsurprising that the judgment rendered by contemporary philosophers is far more severe than that of either Socrates or Aristotle.

%form aristotle the character; from AUgustine the depravity; from socrates the acts

%Aristotle goes on to say that we suppose akrasia (i.e.\ lack of self control) to contrast flatly with self-control because the other extreme is rarely seen.

%[[Plato?]][[[AS SEEN ABOVE!]]] Socrates thought that we always pursue what we think best; the difficulty is knowing what that is. What he then hopes for is some wisdom to release us from the uncertainty of our condition---it is a hope akrasia would render vain.

%PEOPLE ARE ROUTINELY SHOCKED THAT *MEDICINE* DOESN'T WORK LIKE THIS!

%And as saw, [[even stronger expressions]]

%But the inference here (or is it intended merely as a rephrasing?), though commonly drawn, does not seem very plausible, unless we attend to a rather narrow range of cases---which by necessity we \emph{do}, since it is the failures \emph{bother us} that stand out the most.

%[[PHILOSOPHICAL +PAULINE MOTIVE]

%But the inference here (or is it intended merely as a rephrasing?), though commonly drawn, does not seem very plausible, unless we attend to a rather narrow range of cases---which by necessity we \emph{do}, since it is the failures \emph{bother us} that stand out the most. [[PHILOSOPHICAL +PAULINE MOTIVE]

%This part of what Davidson says is (approximately) right:\begin{squote}There is [a] principle the rational man will accept in applying practical reasoning: perform the action judged best on the basis of all available relevant reasons.\footnote{\citet[p.\ 41]{davidson2001weakness}.}\end{squote} This says that \emph{Enkrateia} is a principle of reason. But Davidson goes on to say that\begin{squote}the incontinent man acts, and judges irrationally, for this is surely what we must say of a man who goes against his own best judgement.\footnote{\citet[p.\ 41]{davidson2001weakness}.}\end{squote} And as saw, [[even stronger expressions]]

%What then \emph{is} problem? The problem is just precisely that the \emph{akrates} is irrational, ``for this is surely what we must say.''\footnote{\citet[pp.\ 29--30, 30.\ n.\ 14.]{davidson2001weakness}.}

%\footnote{\citet[p.\ 41]{davidson2001weakness}---though \citet{sidgwick1898practical} is in fact an exception to Davidison's claim. On internal order and moral theory see e.g.\ \citet{korsgaard1999self}, and compare \citet[esp.\ pp.\ xi--xii]{korsgaard2009self}.}

%Davidson's influential paper on akrasia is a peculiar one, in that it manages to show how much is \emph{not} wrong with akrasia while assuming that \emph{something} must be. It is not a matter of incoherence;%Now Davidson also says that the rational man accepts this principle: ``perform the action judged best on the basis of all available relevant reasons.''\footnote{\emph{Ibid.}\ p.\ 41.} That is true (once we properly interpret the principle, anyway), but, as I have explained, it does not license the inference that a man who does not do what he judges best is irrational, because that depends on \emph{why} he does not end up doing what he thinks is best. That is how procedural norms work. As such, we are left with the fact that \emph{nothing} is wrong with akrasia \emph{per se}; what we are \emph{trying} to say is that \emph{Enkrateia} is a principle of reason. 

%  But Descartes seems to deny its possibility; on Leibniz' view, or Kant's, akrasia would seem to necessitate not merely irrationality but a diminution of agency; Mill says that ``men often, from infirmity of character, make their election for the nearer good, though they know it to be the less valuable,'' though he also says that ``to desire anything, except in proportion as the idea of it is pleasant, is a physical and metaphysical impossibility.'' Sidgwick, attempting to reintroduce the the topic of akrasia, entitles his paper ``Unreasonable Action,'' and speaks of it's ``awful irrationality.''\footnote{\emph{Protagoras} 352b--357e for Socrates, and see for instance \citet{joyce1995early} on Stoicism; \emph{Republic} IV, 439c ff. (trans.\ \cite{grube1992republic}); \emph{Nichomachean Ethics} VII.3; \emph{Confessions} VI.xi--xii; Descartes, Letter to Mesland, May 2nd, 1644; for `post-medievals' in general see \citet[ch.\ 7]{gosling1990weakness}; \emph{Utilitarianism}, chs.\ II, IV; \citet{sidgwick1893unreasonable}.} 

%\footnote{\citet[p.\ 41]{davidson2001weakness}---though \citet{sidgwick1898practical} is in fact an exception to Davidison's claim. 

%On internal order and moral theory see e.g.\ \citet{korsgaard1999self}, and compare \citet[esp.\ pp.\ xi--xii]{korsgaard2009self}.}

% Plato's close association of akrasia with moral vice is an \emph{oversight}; he is deliberately approaching moral theory through his psychology, and we find broadly similar approaches still today.

%was strictly a problem of action

%nor even necessarily a matter of succumbing to temptation of \emph{any} sort.\footnote{\citet[pp.\ 29--30]{davidson2001weakness}.} 

%Austin 198: ``Yet unfortunately, at least when in the grip of thought, we fail not merely at these stiffer hurdles.''

%``Or we collapse succumbing to temptation into losing control of ourselves,''

%\begin{squote}Plato, I suppose, and after him Aristotle, fastened this confusion upon us, as bad in its day and way as the later, grotesque, confusion of moral weakness with weakness of will. I am very partial to ice cream, and a bombe is served divided into segments corresponding one to one with the persons at High Table: I am tempted to help myself to two segments and do so, thus succumbing to temptation and even conceivably (but why necessarily?) going against my principles. But do I lose control of myself? Do I raven, do I snatch the morsels from the dish and wolf them down, impervious to the consternation of my colleagues?  Not a bit of it.  We often succumb to temptation with calm and even with finesse.\end{squote}

%HARE QUOTES OVID AS WELL AS PAUL ON 78--9:
%Her struggling Reason could not quell Desire. 
%``This madness how can I resist?'' she cried; 

% [[WHICH WE DO, FOR OBVIOUS REASONS.]]

%This is a common inference, and routinely treated as trivial. In fact it is probably more often thought of as a rephrasing. But it is not at all trivial. That a principle instructs something does not in general entail that you are somehow culpable if you cannot that thing off; hence we distinguish mere want of success from incompetence or culpability. And I do not see that the inference is very plausible on this case either, unless we have [[BRIEFER FOR NOW?]]

I believe the same confusions I identified in Davidson's case are responsible for this this consensus. Firstly, it is routinely supposed that if \enk\ is a principle of reason, the \ak\ must be irrational. Often this thought is expressed by saying something along the lines of: ``it is irrational to defy one's own judgment.''\footnote{This is \citet[pp.\ 242--3]{kolodny2007does}, though he is here discussing akratic belief.} Even those few scholars who have argued that the irrationality of akrasia had been overstated are often apt to grant that akrasia displays \e{some} such defect of procedural reasoning. Thus Robert Audi, though suggesting that akratic action need not always be irrational, also thinks ``we can agree that one who acts incontinently is not functioning normally, or even that such agents are not functioning at all well.''\footnote{\citet[p.\ 276]{audi1990weakness}. Cf.\ \citet{arpaly2000acting}, \citet{audi1990weakness}, \citet[pp.\ 188--90]{frankfurt1988rationality}, \citet{jones2003emotion}, \citet{mcintyre1990akratic, mcintyre2006wrong}, and \citet{tappolet2003emotions}.} But, as I have pointed out, it is \e{not} trivial to infer, from an agent's violation of a principle, some culpability or shortcoming on the part of that agent.

%, and nothing I say could diminish the importance or authority or centrality of reflection. Though this may seem paradoxical, my claim that some instances of akrasia manifests \emph{no irrationality or procedural dysfunction whatever} is a mark of the \emph{higher} place my account assigns (or certainly permits) to \enk.

%This is supposed to be a trivial point, but it is not, and in many cases is hardly even an intelligible charge. But 

Secondly, there is no question that the distinction between akrasia as a weakness of character and akrasia as the inefficacy of a judgment is frequently elided. For example, R.\ M.\ Hare, who has been considering ``the state of mind called `thinking that I ought','' tells us that\begin{squote}Our morality is formed of principles and ideals which we do not succeed in persuading ourselves to fulfill. And this \emph{inability} to realize our ideals is well reflected in the highly significant names given in both Greek and English to this condition: Greek calls it \emph{akrasia}---literally `not being strong enough (sc.\ to control oneself)'; and English calls it `moral weakness' or `weakness of will.'\footnote{\citet[p.\ 77]{hare1963freedom}, his emphases.}\end{squote} Recall that \emph{Socrates} was worried about the possibility that fear (say) might undermine your ability to discern the truth, the latter ability in that case being weak. But he never allowed that the kind of loss of control Hare is interested in was possible at all. The ``highly significant'' names reflect the Aristotelian understanding, applied, in an un-Aristotelian fashion, to \e{all} cases of `thinking one ought.'






%But again, these characterizations will frequently be inapt if we keep in mind that, on the contemporary understanding, akrasia should be understood (at least first of all) as a weakness \e{of one's considered judgment about what ought to be done} rather than as a weakness of the agent \emph{per se}. What could it mean, for example, to say that an akratic Neoptolomus has ``performed an abortive act of will,'' or that his action is ``perverse,'' or that he suffers some ``pathology'' or ``self-alientation''? We only abuse otherwise useful notions when we speak thus. There is in fact an enormous variety of cases here, and a great variety of verdicts which may be rendered regarding the \ak\ himself. Let us briefly indicate a few of these.



%Why must we suppose that \emph{Neoptolus} is at fault somehow? Davidson must imagine that Neoptolomus must \emph{unintelligible} to himself, but that seems a stretch. 









%that is obscured if we simply view akrasia as specifically a failure of reason.





%But treating \emph{Enkrateia} as a principle and akrasia as the failure to act as that principle instructs, these harsh condemnations are out of order. --- diverse cases and how do the epitaphs apply?








%\e{Dilemmas.} Begin with two examples not so far from that of Neoptolomus:\begin{squote}\e{Abraham's Dilemma.} God has commanded me to sacrifice my son, Isaac.

%\noindent\e{The Quaker Tested.} I am obliged always to tell the truth, even to this murderer at my door.\end{squote} If we imagine an akratic Abraham, or an akratic Quaker, it is quite unnecessary to suppose that they would be somehow \e{blind} to reason, as `irrationality' suggests. A proper sensitivity to the demands of \enk\ could quite as well be demonstrated by aguish as by conformity.


%\e{Arduity.} Some things are just extremely \e{hard}.\begin{squote}\e{Nick Vujicic:} I was born without limbs, but according to John 9:3, the man to whom Jesus gave sight was born blind ``that the works of God might be displayed in him,'' so I guess I'm supposed to live a full life and find ways to inspiring hope in other.\footnote{I have in mind Christian motivational speaker \href{http://en.wikipedia.org/wiki/Nick_Vujicic}{Nick Vujicic}.}

%\noindent \e{Aron Ralston:} In order to free myself, I will have to amputate my good arm with this pocket-knife.\footnote{I have in mind the climber \href{http://en.wikipedia.org/wiki/Aron_Ralston}{Aron Ralston}.}

%\noindent \e{Zhang Cheng:} Blind teenagers in China---like myself---have it rough, so I should start a free vocational school for them.\footnote{``\href{http://www.thestar.com/article/813790--a-school-in-china-gives-hope-to-the-blind}{A School in China Gives Hope to the Blind},'' \e{Toronto Star}, June 1, 2010.}\end{squote} In these cases the very ambition is audacious, and the success is heroic. Failure would be much more readily intelligible than is the success, as suggested by Ralston himself in a newspaper report:\begin{squote}``I'm not sure how I handled it,'' Mr. Ralston said. ``I felt pain and I coped with it. I moved on.''\footnote{``\href{http://www.nytimes.com/2003/05/09/us/climber-speaks-of-his-ordeal-in-utah-canyon.html?scp=1\&sq=ralston+\%22felt+pain+and+i+coped\%22\&st=cse\&pagewanted=print}{Climber Speaks Of His Ordeal In Utah Canyon},'' \e{NY Times}, May 9, 2003.}\end{squote} Vujicic and Cheng were not so far from suicide at various points, and that outcome would likeiwse have been less surprising. Had one of these men given up in frustration, most of the epitaphs above would constitute only cruel and absurd insults.







%As Ralston's comments suggests, success is no more intelligible than a falure would be, and perhaps less so. [[so with ]]

%\item[iii] Eighteen hours of every day spent at my desk in Gund Hall just isn't enough; I can't afford fifteen minutes for cof%fee with my friends.

%\end{itemize} 


%heroic cases---often sort of stable charcter traits as with Huihong




%Scholars tend either to pass silently over Aristotle's suggestion that there is `good incontinence, or  If his view is inconsistent, that is, I should think, a tribute to Aristotle's integrity as a philosopher.

%Aristotle has said that the \emph{akrates} has a lamentable lack of self control. Should Neoptolomus have better self-control? Aristotle thinks: he's just right as he is---in fact he deserves \emph{praise}. Aristotle is right, too, about the possibility of having \emph{too much} self-control;





%We'll have to take up cases in more detail, of course, but Aristotle's verdict strikes me as a plausible one. I believe he is right that some are \emph{too} self-controlled, as well. If someone close to you is very conscientious you will know that it is often necessary, for their own good, to lure them in akratic action (and often to sooth them during and after the episode as well). That does not mean that you would anything about them to change. (FIND THESE PEOPE MORE IN CHINA)

%FULLER QUOTE: Temperance (which is one of the virtues on the human level) is described111 as involving the entire absence of bad desires, and there is no room for a superhuman virtue beyond this. [[HOW DOES THIS MAKE SENSE?]]


%\e{Superhuman willpower.} Some people perform heroic feats, and others have an heroic character. It seems that scholars generally pass silently over Aristotle's suggestion that there is a counterpart to the \ak\, namely the person possessed of a ``superhuman virtue, a heroic and divine kind of virtue.''\footnote{\e{NE} VII.1. It is sometimes said (perhaps rightly) that this suggestion does not fit the rest of his account. Thus \citet[p.\ 230]{ross1995aristotle}: ``Aristotle's doctrine as it is worked out leaves no room for anything higher than `virtue'.'' If that is so, then I am inclined to view it a testament to Aristotle's keen observation and philosophical integrity.} But Aristotle is right that there is something beyond continence. He says this is overlooked becasue this ``extreme is seen in few people and seldom,'' but we have less excuse for overlooking the godlike characters because we know more Chinese women than Aristotle did.\begin{squote}\e{The Harried GSD Student.} Hui Hong is a student at Harvard's School of Design. Like many of her colleagues, she can be found at a desk in Gund Hall for about 16--18 hours a day, every single day. Beside sleep (when that happens), her only breaks come when, two or three times a day, she goes outside for a cigarette, and once a day when she goes to a caf\'e with some fellow students. She enjoys the breaks, and recognizes the value of getting out a bit, but thinks that she ought to give up both coffee and smoking on account of their being bad for her health. However she wouldn't in fact get outside or go out with friends except that she is habituated, or addicted, to coffee and tobacco. But `addiction' is a bit strong: on summer breaks, and then upon graduation, she gives up both coffee and tobacco. So her habits, with the `compulsions' or promptings they entail, function to give Hui Hong some release from her high-pressure environment.\end{squote} Hui Hong herself might suppose that her will is weak; her tiger-mother ensured that Hui Hong would always think she was falling short. But if the notion has any meaning, then Hui Hong is not weak of will.% To imagine a firmer degree of self-control is to imagine a student who is institutionalized upon graduation.


%ASIAN WOMEN: <2.5% OF US; >10% OF HARVARD UNDERGRAD (2009--10)


%Can the GSD student who defies \enk for a coffee break really be described as weak of will? Most people would consider these students to have an almost superhuman \e{strength} of will. Their akratic episodes manifest standards I would never even have \e{imagined} setting for myself. Without akratic breaks for coffee and for cigarettes, institutionalization would be a formal part of GSD graduation process. If \e{these} people are weak-willed, the notion is meangingless.



%[[Aristotle also notes the difference between stubbornness and continence.\footnote{\e{NE} VII.9.}]]


%\e{Spite.} Far removed from that divine creature we have:\begin{squote}\e{Edmund and his Perverse Thought:} I could tell Peter and Susan that I've found out Narnia is real, though then I would have to apologize to Lucy, and I'd rather keep ridiculing her\ldots \footnote{ \e{The Lion, the Witch, and the Wardrobe}, ch.\ 5.}\end{squote} When Edmund decides to ridicule Lucy rather than to apologize to her, he may know perfectly well what he \e{ought} to do, and do otherwise precisely because what he ought to do would be unpleasant. This no doubt shows a moral perversity, but why any shortcoming in terms of his \e{reasoning}? Isn't it his clearheaded understanding and willful disregard of his brotherly duties what makes Edmund so nasty?


%The same sort of thing may be said for J.\ L.\ Austin stealing segments of bombe.\begin{squote}I am very partial to ice cream, and a bombe is served divided into segments corresponding one to one with the persons at High Table: I am tempted to help myself to two segments and do so, thus succumbing to temptation and even conceivably (but why necessarily?) going against my principles. But do I lose control of myself? Do I raven, do I snatch the morsels from the dish and wolf them down, impervious to the consternation of my colleagues?  Not a bit of it.\footnote{\citet{austin1979plea}.}\end{squote}





%not


%\begin{squote}\e{Brother's Keeper.} I owe it to my brother to help him overcome his drug addiction.\end{squote}Suppose [[]] makes valiant effort to help his brother beat addiction, and sustains the effort over years. But eventually he becomes rather discouraged, and in addition has gotten married and had children. He might still think he ought to do it, and it might still be doable, but have largely stopped making much effort; we may suppose that this rather torments him sometimes. It would be quite unfair to think that [[]] is \emph{not properly attentive} to the demands of reason. That he worked so hard and still is tormented by the thing shows just how sensitive is to the consideration that imposed the requirement upon him in the first place. [[]] is akratic, to be sure, but to say of him what philosophers say of the \emph{akrates} would only be to abuse him for his conscientiousness. Similar observations would be apt regarding akrasia in the following cases:


%(Interestingly, Socartes himself seems to have worried---rightly, I would say---that his rejection of akrasia didn't take care of this case.\footnote{See especially the \emph{Hippias Minor}.})


%CASE OF ELIOT IN CHAP 3 OF DAMASIO WOLD BE INTERESTING, AS NO DOUBT WOULD SOME OTHERS


%I should go congratulate him---though I could ridicule him to make myself look better.




%The example of akratically getting back out of bed to brush your teeth is of course drawn from Davidson, who later tells us that the \emph{akrates} ``cannot understand himself: he recognizes, in his own intentional behavior, something essentially surd.'' 


%\item I should stop treating [[]]

%\item I must win her love and make her my wife.



%\item I ought not to act so coldly towards her.

%\item I should just stay in bed and forget about my teeth for tonight.

%\item TRUST

%\item I need to spend less time fretting about [[]]

%\item It's tempting to step out for [[huihongs breaks]]

%\item I ought to leave my wife, but her tears always melt my heart.

%[[acting coldly from eg jealousy]



%\item It is of course wrong to deprive someone of their share of the bombe, but I am partial to ice-cream, and would very much like two segments for myself\ldots





%\item I should go for a run but I don't really feel like it.

%\item I ought to learn \emph{Ge'ez}.

%\item I ought to make more friends.

%\item I need to tell my wife that I've been cheating on her.

%\item My five-year-old daughter needs to know that her mother has been killed.


%\item I am going to dress more stylishly from now on.

%\item I'm going to stop walking with this funny gait.

%\item I shouldn't keep staring at his weird mole.

%\item I need to stop fidgeting like this.

%\item I SHOULD DO X FOR REASON Y

%\item I must not drop bombs if, in doing so, I intend for civilians to die.

%\item I'm not going to argue with her anymore.

%\item As a Christian I must act in a loving fashion towards everyone at all times.

%\item I cannot turn down a friend.

%\item I really should forgive him.



%Suppose that in each of these cases the indicated action is required by \enk , and also that the person in question does not in fact do as \enk requires. 


%These examples indicate to me that nothing much can be inferred from the fact that someone has violated \enk\ except that he has, in fact, acted akratically. Until we know exactly what happened, we have no basis for seeing any kind of fault in him. He may not have acted irrationally; still less can we suppose that his behavior was pathological or perverse, or unintelligible to himself. %If the characterizations ring so false, why has akrasia so universally been thought so problematic?





%Suppose that a failure to act in the indicated way is akratic---i.e.\ that the person in question acts in some waythat does ot accord with

%tells him to do that thing. But suppose he then doesn't do it.






%Again, if a man knows he need to tell his young daughter that her mother has died but keeps putting off, why isn't it enough to observe that such things are \emph{very hard}? To say that such a failure manifests a failure of reaso---or some pathology, or a self-alienation---would not at all be to the point.




%An obvious response would be to that these failures will be criticizable insofar the outcome is `within one's power' in suitable way. Akratic action is some sense \emph{actual action}; merely being prevented from doing what you think you ought some external force doesn't count. We can then say, for example, that if it's just not in my power to make new friends or to win her, being as I am rather deficient in social ability, then I will not be akratic, so long as I make some suitable effort of whatever sort I can manage. Even in the case of sacrificing one's son, where the action itself is in some sense perfectly manage, there may, at least for some people, a sort of psychological impossibility in the thing, in which case again the fault, if any, is something other than the akrasia.

%I accept and indeed insist on this point, and also that for just this reason we should draw a distinction between \emph{mere} failures to satisfy \emph{Enkrateia} and properly akratic action.  The distinction will not, however, save the thought that akrasia is necessarily irrational or otherwise dysfunctional. The thought here that when it is in your power to do as \emph{Enkrateia}, and you still fail to do it (i.e.\ are akratic), then you have been in some way \emph{inadequately sensitive} to the demands of reason. But this inference could only reflect the bare observation that he didn't \emph{in fact} do as it instructed, because looking back on the examples, it is not a very plausible suggestion.

%[[THE IDEAL]]


%[[ARG THIS MORE INS]]

%TAKEN UP IN MORE DETAIL IN \ref{rae} \& \ref{rea}

%NOTHING TO CURE HERE---no ideal that is intelligible for \emph{us}

%inherited this ideal from Plato

%for a lot of us a curse---all to easy to sympathize with Paul

%some people would be beneffited by more of it

%habits

%trust

%emotions

%conscientiousness

%being good or bad as a separate thing...

%conflicting things generally

%instinct---not being a sociopath


% Simpler solution: we think the cases aren't akratic cause there not problematic. They aren't weak of will, as it were.---Davidson right about all the things that aren't wrong with AK, lets go further

%To say that `\g{F }' covers an extraordinarily heterogenous variety of cases is an understatement. 

%[[for example often overlooks fact that doing something is part of some other thing]] 

%But to suppose that the \emph{akrates} acts \emph{against} reason, as it were, is not right. There are many ways in which, and reasons why, one might then fail to \g{F }. 

%TO SAY: some reflective malfunction, or practical defect, or that he exercises only a marginal agency or manifests some perversity---as applied to many cases [[must be a philosophical result, and the terms must have technical sense if any sesne]]

%temptation to set cases aside not genuine is a cheat

%Perhaps notable that Aristotle didn't offer principles for practical life...

%YOU MIGHT HAVE THOUGHT THAT DOING THE THING WAS HARD.

%or an overblown understanding of the powers of reason...

%[[CANNIBALIZE A BIT OF THIS---(COMMENTED OUT)]]

%But, fourth, accounts of practical reasons do sometimes exclude the possibility of rational akrasia at a fairly deep level, and even by design. 


% Socrates' view collapses distinctions I wish to make, but it would be peculiar to say that he's \emph{conflating} or \emph{confusing} them, as if he were doing it by oversight. Plato and Kant offer much less \emph{ad hoc} accounts of action that would ensure that akrasia will some degradation or dysfunction of action.\footnote{See \citet{korsgaard1999self} for a sympathetic discussion of this feature of the views of Plato and Kant.} And there is every reason to suppose that many philosophers, even those with other quite different views, would be inclined to view this as an advantage. 

%But why should this be considered a likely general \emph{constraint}? Stroud herself associates failures to satisfy this constraint with ``views that do not assign a privileged place in rational deliberation and action to the agent's overall assessment of her options,'' and threaten to break the expected connection between evaluative judgment and guidance of action. But the view that overall assessments have a a privileged place in deliberation is the view that \emph{ekrateia} is a preeminant principle of practical reason, while the constraint tells us that failures to satisfy \emph{enkrateia} must be agential failures, or worse. So adopting this constraint amounts to starting with the confusion and then trying to secure it.\footnote{As for the relationship between evaluative judgement and action, my view is the straightforward one indicated already: \emph{enkrateia} is a principle of reason. The thesis that excuses exist does not change that. Nor could the supposition that akrasia, or failures of \emph{enkrateia}, \emph{must} be irrational secure a firmer connection between knowledge and action, because no such firmer connection should exist. The only question here is whether certain categories of justification or excuse can be employed with regard to \emph{enkrateia}.}

%The philosophical history of Stroud's constraint---and indeed of philosophical interest in akrasia---is surprisingly sordid. When I say supposing that akrasia must always be irrational is like supposing that an unsuccessful doctor must always be incompetent, it's actually not merely that I think the structure of the confusion is the same, but that in many cases, most notably in the \emph{Protagoras} and the \emph{Republic}, it is quite literally the \emph{exact same} confusion---motivated, moreover, by the same professional pride. That doesn't show that the standard view is wrong. But it does show that univocal support for that view is of less philosophical significance than one might have supposed.




%Often as in sports or writing (Tolkien started a chapter with `but') excuses are pretty easy. If nothing else is at stake, then ``I was just trying something new'' will cut it. In other cases not---eg etiquette.

%enkrateia is always legit, but it's the easiest thing to excuse yourself from. If the stakes are low, then `I didn't feel like it' is god enough. Are there other cases where `I didn't feel like it' or `I changed my mind' will work? In fact `I changed my mind' is \emph{always} good enough? Probably not anywhere else...





\


I understand \enk\ to be a genuine principle of rationality. My thesis does not amount to saying that \enk\ is somehow only \emph{prima facie} a principle, or that it can be overridden, or its demands denied, or that it is sometimes inapplicable to us. It as much a principle of rationality as anything could be a principle of anything. To say that failures of \enk\ do not \emph{always} represent agential failures, and that agential failure are not \emph{always} irrational simply means that the \ak\ corresponds to the unsuccessful doctor rather than to the incompetent doctor. It means that there are exculpatory considerations of certain characteristic sorts in this case as elsewhere. In other words, there is no realm, however circumscribed, in which it is always possible to attain our aims.

% I now set about arguing these claims in more detail.

%But one might suppose that, properly formulated, a failure to satisfy \enk\ will after all be irrational in the way we have come to expect. I believe that such a hope is vain, for entirely general reasons concerning the nature of principles. I will explain this in \S\ref{eai}.


%, but we may foreshadow that discussion with the observation that the instruction ``intend to \g{F}'' is either the same as ``\g{F}'' or else much less interesting. I have often had to tell my roommates that I had \emph{meant} to take out the trash, but I do not recall that it has ever been accepted as a complete exoneration.

%The more interesting question, which I then set about addressing in more detail, is whether the \ak\ properly speaking necessarily acts irrationally. I approach the question first by broadening the scope of the discussion so as to include emotion and belief as well as action. Special complications arise when we consider the possibility and nature of akratic action and belief, but the extension to these cases, which has often been suggested, seems appropriate so long as it makes sense to say, with Joseph Raz, that\begin{squote}we should believe only when reason warrants, be afraid only when in serious danger, feel pride only when we have something to feel proud about.\footnote{\citet[p.\ 5]{raz1999ourselves}. \citet[pp.\ 40--1]{davidson2001weakness} sees an analogy between rational action and rational belief formation, as does e.g.\ \citet[pp.\ 221--2]{korsgaard1997normativity}. \citet[pp.\ 6, 36--40]{gibbard1992wise} observes that when we reflect upon the rationality of our responses, we are as  much interested in emotions as in beliefs or actions. \citet{smith1995moral} argues that our desires ought to conform to our evaluative beliefs. For \citet[pp.\ 20, 25]{scanlon1998we}, `judgment-sensitive attitudes,' which include ``beliefs, intentions,  hopes, fears, and attitudes such as admiration, respect, contempt, and  indignation,'' are all, precisely in virtue of their sensitivity to  reflective appraisal, susceptible to akratic irrationality.}\end{squote} Emotions and beliefs can fit or fail to fit the relevant considerations, and, accordingly, it makes sense to subject both emotions and beliefs to critical scrutiny, and to expect at least some degree of responsiveness---more responsiveness, often, than we actually get. Since we speak of the authority of reflection in this way, we can also speak of akrasia.%\footnote{Specifically, the notion of akratic attitudes does not presuppose that attitudes are `voluntary' in the manner of actions; I do not assume that we can form attitudes at will.}

% (though he  wonders about the significance of the non-voluntary nature of belief.)

%: ``the thought or action they involve is in an obvious sense `contrary to (the  person's own) reason': there is a direct clash between the judgments a person makes and the judgments required by the attitudes he or she  holds.''


%I shall suppose, then, that there are facts or reasons proper to various attitudes, and to which these attitudes may better or worse conform; these attitudes are conformable to our considered judgments, the (or a) role of which is precisely to so shape them. An attitude does not conform to one's judgment about the status of the relevant considerations; call such a case `akratic'.


%The point in broadening the topic is to have available a richer variety of cases, contrasts, and modes of failure. Identifying failures of \enk, cases of akrasia, and cases of reasonable akrasia is easiest in the case of emotion, and hardest in the case of belief. Thus my more detailed illustrations of rational akrasia will come first from cases of akratic emotion, which I take up in \S\ref{rae}. The lessons learned more easily in this case can then be employed in an attempt to identify and account for cases of rational akratic belief, which attempt I undertake in \S\ref{rea}. I will not in fact return in any detail to the case of akratic action, assuming that if the analyses broached in the other cases are compelling, the extension to action will fairly obvious, especially in conjunction with the types of examples I have already indicated.















%\subsubsection*{Akratic attitudes} [[PROBABLY MOVE THIS TO AN OUTLINE IN THE INTRODUCTION]] WHOLE CHUNK IS STILL A MESS Although akratic \emph{action} is the classic case, there is no obvious reason why we should not speak of akrasia in a broader range of cases. For example, 








%Again, we need not suppose that this ``seeing'' amounts to belief; Austin's example of watching the Headless Woman on stage might sometimes be a better analogy.\footnote{\citet[p.\ 14]{austin1962sense}. Cf.\ \citet[pp.\ 39--40; ch.\ 7]{gibbard1992wise}.}







%\footnote{[[CITE]]} THE IMPORTANT THING TO NOTICE IS THAT WE WIND UP WITH THE SAME SET OF OPTIONS.



%Although some accounts of practical judgment make akratic action (or intention) impossible, the reality of the phenomenon is widely accepted, though with much disagreement about it's nature and severity as a rational failing.











%stop treating someone...

%Huihong's break

%Abraham killing Isaac

%turn someone over to their death

%maybe Huck Finn 

%Have to help my brother whatever---drugs or something...

%starting stopping habits

%physical painful stuff?

%lots of stuff that doesn't seem partiucularly insensitive...

%make a million dollars

%win her over















%disordered state amounts a disease of the soul.


%good sign of an agenda...


%The same sets of options seems to have been available historically. If it is not irrational---likely gravely so---then it is impossible.  For Plato, akrasia represents a ``rising up of a part of the soul against the whole,'' and this disordered state amounts a disease of the soul.  


% For a good while after the medievals, there seems not to have been much discussion or notice of akrasia. But Descartes seems to deny its possibility; on Leibniz' view, or Kant's, akrasia would seem to necessitate not merely irrationality but a diminution of agency; Mill says that ``men often, from infirmity of character, make their election for the nearer good, though they know it to be the less valuable,'' though he also says that ``to desire anything, except in proportion as the idea of it is pleasant, is a physical and metaphysical impossibility.'' Sidgwick, attempting to reintroduce the the topic of akrasia, entitles his paper ``Unreasonable Action,'' and speaks of it's ``awful irrationality.''\footnote{\emph{Protagoras} 352b--357e for Socrates, and see for instance \citet{joyce1995early} on Stoicism; \emph{Republic} IV, 439c ff. (trans.\ \cite{grube1992republic}); \emph{Nichomachean Ethics} VII.3; \emph{Confessions} VI.xi--xii; Descartes, Letter to Mesland, May 2nd, 1644; for `post-medievals' in general see \citet[ch.\ 7]{gosling1990weakness}; \emph{Utilitarianism}, chs.\ II, IV; \citet{sidgwick1893unreasonable}.} 


%Hence it is unsurprising to find nearly universal agreement that akrasia---to the extent that it is even \emph{possible}---is an instance of what we may call `procedural irrationality': akratic action manifests a failure to proceed as reason dictates, and as such, may even be thought to manifest only a marginal or diminished form of agency. 


%******MAYBE KEEP SOME OF THIS AS A FOOTNOTE?******** 

%Now our judgement as to how we ought to act may be mistaken 

%But we suppose that reflection, or the judgments at which we thereby arrive, ought to have authority over our decisions and action. Thus Davidson:\begin{squote}There is [a] principle the rational man will accept in applying practical reasoning: perform the action judged best on the basis of all available relevant reasons.\footnote{\citet[p.\ 41]{davidson2001weakness}.}\end{squote}If akrasia is irrational, that irrationality lies precisely is in our failure to properly exercise our capacity for reflection.

%We are capable of reflecting on the merits of our actions and of shaping our behavior accordingly, and yet sometimes we fail to act as we think we ought, even realizing \emph{as} we act that this is so.

% Since in such a case we fail to act as we take ourselves to have reason to act, akrasia is thought to be a variety of irrationality.

%This is a principle everyone who reason at all has to accept. So when 

%Whenever anyone at all thinks that he ought to do something, no matter what, \emph{enkrateia} is there to tell him to do it.

%When someone thinks he ought to do something, the principle of \emph{enkrateia} directs him to do that thing. That is, when an agent thinks ``I ought to $\Phi$,'' the principle of \emph{enkrateia} tells him ``$\Phi$.'' Hence:Henceforth the capitalized `\emph{Enkrateia}' will refer to this formulation, and the uncapitalized \emph{enkrateia} will refer to the principle that forbids akrasia, however it \emph{should} be formulated. A suitable variation for \emph{not} $\Phi$-ing may be understood as given throughout. 











%\subsubsection*{The standard view of akrasia} 


%Suppose, for instance, that a mother is told that she can choose to save one of her two children, the other of which will be shot, and that if she declines to choose then both children will be shot. Suppose she thinks that she has decisive reason to choose one child or the other---perhaps by flipping a coin---but cannot bring herself to do so, knowing that, in effect, she will be choosing to condemn one of her own children, and both children are thus shot. Now the sense in which she `can't' do it is not the sense in which the addict `can't'; what holds her back is simply horror at what is in fact horrible. So although she [[VIOLATES CONDITIONS]], the torment and guilt that she feels both then and long afterwards shows quite well that her thoughts perfectly well reflect the judgments she made about the force of the relevant considerations.

%Again, suppose Mark is thinking about moving, and Mandy promises that she will help if he does. Then [[]] asks Mandy for help [[m-something]] on the following Monday, and Mandy says she will. The next morning, Mark calls up announcing that his move will occur on the same Monday. Being a conscientious woman, Mandy finds herself torn between two binding commitments. She finally makes up her mind by flipping a coin. [[HOW SHE DISPLAYS THE SENSITIVITY]]

%Third. Suppose I dislike a colleague for some petty reason---perhaps I find his gait or tone of voice grating, and make fun of it behind his back. One day I learn that he  commits substantial time to some worthwhile volunteer activity. Knowing I should feel admiration for his behavior, and shame at my own, and feeling in myself some movement in that direction, I instead make the spiteful decision [[MAKE ACTION-ORIENTED]] to imaginatively re-enact some of my best efforts at mocking mimicry of his mannerisms, and continue to do so until any trace of admiration or shame is gone, firmly replaced by the original sense of contempt. I am akratic in this, knowing as I do that I ought to feel admiration rather than contempt (I know, that is, that admiration and not contempt fits the circumstances); it is quite within my power to feel admiration rather than contempt, and while there may be in some broad sense be an unreasonableness in what I do and in what feel, it is not the irrationality that we are supposed to find in akrasia; my reflective faculties were not \emph{weak} but rather ill-used.]]] [[HOW THIS DISPLAYS THE SENSITIVITY.]]






%\footnote{By contrast, Socrates and Aristotle were first of all concerned with the efficacy of \emph{knowledge}.}




%Thus Donald Davidson tells us that\begin{squote}the akrates does not\ \ldots hold logically contradictory beliefs, nor is his failure necessarily a moral failure. What is wrong is that the incontinent man acts, and judges irrationally, for this is surely what we must say of a man who goes against his own best judgement.\ \ldots There is [a] principle the rational man will accept in applying practical reasoning: perform the action judged best on the basis of all available relevant reasons.\footnote{\citet[p.\ 41]{davidson2001weakness}. The irrational judgment to which Davidson refers is [[COMPLETE]].}\end{squote}
%Such sentiments are widespread. [[SHARPEN THE CLAIM]].\footnote{For example: \citet[pp.\ 404, 413--15]{broome1999normative}; \citet[ch.\ 4]{gibbard1992wise}; \citet[p.\ 222]{korsgaard1997normativity}; \citet[p.\ 25]{scanlon1998we}; [MORE CITES].}


% A similar sentiment is expressed in even stronger terms by Christine Korsgaard:\begin{squote}The rationality of an action\ldots depends upon the agent's being motivated by her own recognition of the rational necessity of doing the action.\footnote{\citet[p.\ 222]{korsgaard1997normativity}.}\end{squote}Many such examples could be adduced.% And for these reasons akrasia is commonly considered a \emph{paradigmatic} form of irrationality.




%“Even though the akratic agent does not believe that she is doing what she has most reason to do, it may nevertheless be the case that the course of action that she is pursuing is the one that she has … most reason to pursue” (McIntyre 1990, p. 385)



%THIS AS GOING TOO FAR: “A theory of rationality should not assume that there is something special about an agent's best judgment. An agent's best judgment is just another belief” (Arpaly 2000, p. 512). 


%[[McIntyre implicitly takes her earlier self to task for having neglected the procedural aspect of rationality and equated what is rational with what we have most reason to do (McIntyre 2006, p. 289; p. 299).
%SO SEEMS MCINTYRE ALLOWS THAT THERE IS A CERTAIN PROCEDURAL DEFECT IN 2006 PAPER:
%McIntyre holds, however, that it would be overstating the case to say that because weakness of will involves this procedural defect, it is always irrational (McIntyre 2006, p. 290; pp. 298–9; p. 302).]]



% Sarah Stroud suggests that showing akrasia to be ``a puzzling, marginal, somehow defective instance of agency'' is a reasonable \emph{constraint} on theory.\footnote{\citet{stroud2008weakness}.} [[ALSO THE VERDICT OF HISTORY]]

%[[EXPLAIN THE REQUIREMENT]] (see e.g.\ Watson, \citeyear{watson1977skepticism}).


%Thomas Hill: a failure of resolution---not sure if that's akrasia proper

%Smith and Pettit?

%Hurley?

%stuff from scanlon and others about alientation?\

% [[PP 21--22 IN SCANLON FOR ATTRIBUTIBILITY, RESPONSIBILITY ETC]]



%[[Davidson seems to endorse the idea that incontinent action must be free (p. 22, n. 1) and, in particular, uncompelled (p. 29), even if his official characterization does not use these terms explicitly. Whether and how weakness of will can be distinguished from compulsion has been a subject of much debate in the literature: see Gary Watson’s classic paper on the subject (1977) and also Audi 1979, Mele 1987, ch. 2, Buss 1997, Tenenbaum 1999, Wallace 1999, Kennett 2001, ch. 6, Mele 2002, and Smith 2003 for discussion of this issue.]]



%*******MAYBE KEEP SOME OF THIS AS A FOOTNOTE?******** The same sets of options seems to have been available historically. If it is not irrational---likely gravely so---then it is impossible. Socrates, famously, thought the notion of akrasia to be incoherent, and was followed in this by many philosophers of antiquity. For Plato, akrasia represents a ``rising up of a part of the soul against the whole,'' and this disordered state amounts a disease of the soul. Aristotle sees the akratic as impaired, like a man drunk or mad. Augustine likewise finds himself ``bound,'' requiring God for his deliverance. For a good while after the medievals, there seems not to have been much discussion or notice of akrasia. But Descartes seems to deny its possibility; on Leibniz' view, or Kant's, akrasia would seem to necessitate not merely irrationality but a diminution of agency; Mill says that ``men often, from infirmity of character, make their election for the nearer good, though they know it to be the less valuable,'' though he also says that ``to desire anything, except in proportion as the idea of it is pleasant, is a physical and metaphysical impossibility.'' Sidgwick, attempting to reintroduce the the topic of akrasia, entitles his paper ``Unreasonable Action,'' and speaks of it's ``awful irrationality.''\footnote{\emph{Protagoras} 352b--357e for Socrates, and see for instance \citet{joyce1995early} on Stoicism; \emph{Republic} IV, 439c ff. (trans.\ \cite{grube1992republic}); \emph{Nichomachean Ethics} VII.3; \emph{Confessions} VI.xi--xii; Descartes, Letter to Mesland, May 2nd, 1644; for `post-medievals' in general see \citet[ch.\ 7]{gosling1990weakness}; \emph{Utilitarianism}, chs.\ II, IV; \citet{sidgwick1893unreasonable}.} And, of course, the very labels themselves---`akrasia,' `incontinence,' `weakness of will'---imply a deficiency.



%Aquinas has two account...


%OH: GET PARFIT
 
%







%

%I will simply assume that the irrationality which akrasia is supposed to manifest is not a matter of the substantial merits of our considered judgments, either in respect of our own values and beliefs, or in respect of the facts as they actually stand. 



%[[AVOID WILD CASES?]]


% the possibility of akrasia does not turn on whether there is some direct logical conflict among attitudes.










%This is indeed suggested by the classic discussion in the \emph{Protagoras}, in which Socrates is concerned foremost to show that \emph{knowledge is strong}---ie. \emph{kratos}, \emph{akrasia} being literally \emph{weakness}---meaning that while the ignorant are fragile and changeable not only in their behaviour but also in their beliefs and affects, the wise man is able to maintain a steady view in the face of dangers and temptations.\footnote{\emph{Protagoras} [CITE]; I develop an interpretation of this passage in \citet{ritchiesocrates}.}
 

 
 
 
 



%\footnote{See especially \citet[p.\ 25ff.]{scanlon1998we}, who argues that akratic attitudes are the clearest, and in a strict sense the only, manifestation of irrationality.} The view that akrasia \emph{is} at any rate irrational---in some degree, and to the extent that it occurs---is nearly universal. [THINK I'D LIKE A BIT OF GIBBARD FOR ANOTHER EXTREME]]




%Even a sense in which we only cloud the issue by focussing on really problematic judgments.




%FIX FOR PRACTICAL CASE:

%I shall suppose, then, that there are facts or reasons proper to various attitudes, and to which these attitudes may better or worse conform; these attitudes are conformable to our considered judgments, the (or a) role of which is precisely to so shape them. An attitude does not conform to one's judgment about the status of the relevant considerations; call such a case `akratic'.



%Suppose, then, that actions can be better or worse justified, whether in light of the actual facts or the facts as you see them (I do not believe it much matters here what view of reasons we adopt), and that our actions are conformable to our considered judgments about their merits, the (or a) role of such reflection being precisely to so shape them. As practical agents, we are then called upon by reason to act as we think fit.



%This consensus reflects a confusion. It is true that there is a principle of reason which calls upon agents to act as they see fit. I will call this the principle of \emph{enkrateia} (i.e.\ `continence' or `self-control'). The \emph{akrates} fails to conform his behaviour to this principle. But to say the \emph{akrates} then suffers some reflective malfunction, or practical defect, or that he exercises only a marginal agency or manifests some perversity---that he is in some such fashion \emph{irrational}---is to say something stronger, namely that the \emph{akrates} 

%At a first pass, the difference corresponds to the difference between failure and incompetence. Although failure is \emph{often} the result of incompetence, failures never \emph{necessarily} implicate incompetence. Likewise failing to do as you judge you ought is not \emph{always} irrational.


%WHAT I HAVE JUST SAID IS A CONFUSION IS AN INFERENCE DRAWN ALL THE TIME EG DAVIDSON


%In a representative passage, Davidson tells us, reasonably enough, that the rational man accepts the practical principle ``perform the action judged best on the basis of all available relevant reasons,'' and infers from this that the \emph{akrates} is irrational---that is, he says, ``surely what we must say.''\footnote{\citet[p.\ 41]{davidson2001weakness}.} Since I do not this is clear at all, I refer to it as a confusion. [[MODUS PONENS]]



%FREQUENTLY SEE JUST EXACTLY THAT INFERENCE.





%Second, one might suppose that, whatever the general utility of the distinctions I've drawn, \emph{enkrateia}, and perhaps rational principles generally, will be a special case. Aristotle said that virtue cannot be an art, because arts can be abused while virtue cannot, and one might suppose that something similar is true of reason as well.\footnote{\emph{NE} II.iv.} But most often the question is simply not properly entertained. 








%that is not always the case, and perhaps not even typically the case. That is so here as everywhere else.

%However, I believe that the unanimous view that akrasia, to the extent that it is possible, reflects some kind of procedural irrationality---at best a dysfunction and at worst a merely marginal exercise of agency---reflects a confusion. It particular, it reflects a conflation of three distinct varieties of failure. That there is a failure is correct. That the failure \emph{sometimes} manifests some irrationality, malfunction, or deficiency, is also correct. That it \emph{always} does so is not.

\bibliographystyle{apalike}
\bibliography{Dropbox/Dissertation/bibliography}

\end{document}
