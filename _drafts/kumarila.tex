\documentclass[11pt]{amsart}

\usepackage[greek, english]{babel}

\usepackage[usenames,dvipsnames]{color}

\newcommand{\tab}{\hspace*{1em}}

\author{Brendan Ritchie}
\title{Kum\=arila Bha\d t\d ta on Epistemic Entitlement}


\usepackage{verbatim}

\usepackage{bigfoot}
\AtBeginDocument{\RestyleFootnote{default}{para}} 

\usepackage{natbib}
\bibpunct{(}{)}{;}{a}{,}{,}

\usepackage[pdfborder={0 0 0}, colorlinks=true, linkcolor=RoyalBlue, urlcolor=RoyalBlue, filecolor=RoyalBlue, citecolor=Black]{hyperref}

\begin{document}

\maketitle

\thispagestyle{empty}

%\={x}, \={\i}, \d{x}, \.x, \~x, \'x, \b{x}

%\begin{abstract}ABSTRACT TEXT\\ \end{abstract}

%Kathleen Norris --- Akrasia ---accidie

%\doublespace

%Evaan's #: 604537

%Lily's friend Imelda 604-727-0819

%check with Taber: stuff on difficulties of I.i.5; something that gives a good general sense of Kumarila's importance

%got to discuss Patanjali and Plato o eternality of language - see Taber eternality article

%Kevin's # 4166272095

%private photos: a variation of dear lily

%LILY: ARRIVES 3PM AC169

%Will need to talk to Parimal about translations:
	%first: just check to make sure my adjustments are ok
	%second, problematic verses: 53 (Taber); 52--4 in general re my interpretations; 80, 52 and TS 2904 re Taber's passage of time suggestion
	%regardning my note on v.38 in the appendix, have to check with Parimal whether we can actually tell whose voice stuff is in, or whether anything indicates objections etc.




%----------------------
%1. INTRODUCTION
%----------------------


\section{Introduction: The M\={\i}m\=a\d ms\=a Inquiry into Dharma}

The P\=urva M\={\i}m\=a\d ms\=a was first of all a school of exegesis. The great bulk of their bulky founding text, the \emph{M\={\i}m\=a\d ms\=a S\=utra}, set down in the 1st or 2nd century AD and credited to one Jaimin\={\i}, treats interpretative difficulties in the Vedas, discusses their study, or details rules for settling procedural questions regarding sacrificial ritual. But by the first century, the heyday of ritual was long past, and there was much skepticism as to whether M\={\i}m\=a\d ms\=a's constitutive aim---according to \emph{MS} I.i.1 the inquiry into \emph{dharma} (i.e.~pursuit of the highest good)---was best pursued in the M\={\i}m\=a\d ms\=a fashion. Besides Buddhists, Jains, and the plainly nihilistic C\=arv\=akas, there were rival exegetical schools, and many Hindus who acknowledged the authority of the Vedas attributed them to gods---unlike the atheistic M\={\i}m\=a\d ms\=akas, who claimed that the scriptures had simply always existed. It would have seemed appropriate to establish some philosophical foundation for the school.

And so indeed several epistemological theses are set out in the first chapter of the \emph{M\={\i}m\=a\d ms\=a S\=utra}, beginning with v.~I.i.4:\small \begin{quote}Perception is knowledge which one has by the senses coming in contact with the soul. It is not the cause of duty [\emph{dharma}] by reason of acquiring knowledge of the thing existing.\footnote{Trans.~\citet{jaimini1923mimamsa}. For style, clarity, or to remove or add glosses, I've lightly altered most of the older translations I've quoted. Meaning has not (I hope) been affected.}\end{quote}\normalsize Indian texts are characteristically cryptic. Pithily rendered verses are meant to be committed to memory and elucidated orally by teachers, or studied with the aid of commentaries (themselves cryptic). But the \emph{M\={\i}m\=a\d ms\=a S\=utra} is difficult even by these standards. Some psychological theory is obviously being appealed to in this verse---the \emph{MS} may only have been written around the 1st century, but it codifies ideas of much great antiquity.\footnote{For a survey of the school see e.g.~\citet{keith1921karma}.} But we must be satisfied---as M\={\i}m\=a\d ms\=akas of later generations had to be---with the text itself and the only slightly less difficult 5th-century commentary of \small\'S\normalsize \=abara.

But, following \small\'S\normalsize \=abara, the import of the present verse is reasonably clear. For M\={\i}m\=a\d ms\=akas, \emph{dharma} concerns a causal system relating actions and future outcomes, so that to perceive \emph{dharma} would be to perceive the future. And this verse points out that sense perception makes contact only with what exists presently. So \emph{dharma} can't be learned about through perception. (The Buddha, for example, was supposed to have done just that.)

Happily for us, the Vedas attest to \emph{dharma}:\small \begin{quote}\emph{MS} I.i.5. Certainly there is eternal connection between the word and its meaning; its knowledge is injunction; it is never erroneous in matters invisible; it is authoritative in the opinion of B\=adar\=aya\d na\footnote{Credited with the \emph{Brahma} (or \emph{Ved\=anta}) \emph{S\=utras}.} by reason of its not depending on others.\end{quote}\normalsize This is yet more puzzling than the previous verse, and unfortunately \small\'S\normalsize \=abara offers little more than a paraphrase, though he notes the importance of ruling out other potential ways of knowing \emph{dharma}, for example by inference or analogy.\footnote{\emph{\'S\=abarabh\=a\d sya} \emph{ad loc.}.} We seem to be confronted, then, with the claim that testimony (i.e.~Vedic testimony) is infallible with regard to ``matters invisible'' (specifically \emph{dharma}).\footnote{The restriction to matters invisible was taken seriously by M\={\i}m\=a\d ms\=a: what the Vedas said about anything other than \emph{dharma} was reinterpreted or disregarded.\label{invisiblenote}} The reason for this is, apparently, that Vedic testimony about \emph{dharma} does not depend upon ``others'': it does not ultimately rely on any of the other legitimate sources of knowledge catalogued by \small\'S\normalsize \=abara.\footnote{I leave aside the part about the eternal connection between words and meanings, which would take us too far afield. Suffice it to say that the M\={\i}m\=a\d ms\=akas spilt a good deal of ink developing an ontology and a philosophy of language to make room for the idea of testimony that is authoritative but authorless. See \citet{tabereternality}.} In other words, testimony, which in the ordinary case---and M\={\i}m\=a\d ms\=a joins every other Indian school on this point---can only convey knowledge which was originally acquired in some other way, is, in the special case of \emph{dharmic} knowledge, an infallible authority, precisely \emph{because} no other means of acquiring knowledge will do the job. And thus I.i.2:\small \begin{quote}Duty is a purpose having injunction for its sole authority.\footnote{Trans.~\citet{jha1907skb}.}\end{quote}\normalsize

We should remember we don't have much idea how these ideas might have been fleshed out in the earlier centuries of the M\={\i}m\=a\d ms\=a school. We ought also to allow for the distance between ourselves and these medieval atheistic scriptural exegetes who take the authorless but eternal and transcendent texts of the Hindu Vedas as the only possible authority for their ritualistic morality. (Although atheists do still walk among us.) We note in particular that India worked hard not to record any history: the names of the authors of \emph{many} Indian texts are little more than conventions, and \small\begin{quote}we can read thousands of pages of Sanskrit on any imaginable subject and not encounter a single passing reference to a historical person, place, or event---or at least to any that, historically speaking, matters.\footnote{Thus \citet[p.~606]{pollock1989mima}, who argues that M\={\i}m\=a\d ms\=a itself encouraged the rejection of history. But there was already fertile soil, e.g.~in the cyclical conception of the universe.}\end{quote}\normalsize The \emph{M\={\i}m\=a\d ms\=a S\=utra}'s bold epistemology of \emph{dharma} is at least a little more intelligible given such facts and the attendant conceptions of the cosmos and our place therein. Indeed the idea that language has a natural or eternal status seems to have been anticipated in earlier grammatical texts.\footnote{\citet{tabereternality}.}

But it hardly need be said that this doesn't \emph{vindicate} M\={\i}m\=a\d ms\=a epistemology. Nor were the philosophers of other Indian schools much impressed. There can still be little wonder, then, if even one of the most serious and sympathetic recent scholars of Indian philosophy has found in this aspect of M\={\i}m\=a\d ms\=a ``a kind of  fundamentalism.''\footnote{\citet[p.~32]{matilal1986perception}; see \citet[pp.~589--91]{arnold2001ivr} on the modern reception of the doctrine of \emph{svata\d h pr\=am\=a\d nya}.}

\

Enter Kum\=arila Bha\d t\d ta, a 7th century M\={\i}m\=a\d ms\=aka. His \emph{\small\'S\normalsize lokav\=artika}, comprising well over three thousand verses, corresponds to just the first thirty-two verses of the \emph{M\={\i}m\=a\d ms\=a S\=utra}---i.e.~those verses which had set out the philosophical basis of M\={\i}m\=a\d ms\=a.
In this paper we are primarily concerned with the second chapter of the \emph{\small\'S\normalsize lokav\=artika}, the \emph{Codan\=as\=utra}, or chapter on injunction. The \emph{Codan\=as\=utra} is framed as a commentary on \emph{MS} I.i.2 (see above).\footnote{Testimony (\emph{\'sabda}) is sometimes treated under the heading of `injunction' (\emph{codan\=a}) because the M\={\i}m\=a\d ms\=a school is concerned with Vedic testimony insofar as it offers prescriptions for living, whether in a grammatical form or not (see also n.~\ref{invisiblenote} above. In our passages, Kum\=arila is explicitly treating \emph{\'sabda} in general (vv.~7--8).} Kum\=arila defends the authority of the Vedas on matters of \emph{dharma} by way of a quite general epistemological thesis, the `\emph{svata\d h pr\=am\=a\d nya},' or `intrinsic validity' thesis:\small \begin{quote}The validity of all valid cognitions is to be understood as intrinsic, since a potency not existing intrinsically cannot be brought about by something else.\footnote{II.47--8, trans.~\citet[p.~207]{taber1992dkb}.}\end{quote} \normalsize The interpretation of this thesis occupies the bulk of this paper. But it's uncontroversial that Kum\=arila is claiming that ``cognitions'' (`\emph{j\~n\=ana}' approximates to occurrent judgment) is (in some way) epistemically legitimate by default, or immediately. This will ultimately support the authority of the Vedas, since the ideas conveyed by Vedic testimony will also have immediate legitimacy.\footnote{Thus v.~68: ``Then, too, in the case of the Veda, the assertion of \emph{freedom from reproach} is very easy to put forward, because there is no speaker in this case; and for this reason the unauthoritativeness of the Veda can never even be imagined.''} And there is no danger that anything will undermine this legitimacy, since in principle no other mode of knowledge acquisition (no other \emph{pram\=a\d na}, in the Sanskrit terminology) provides knowledge of \emph{dharma}.\footnote{vv.~13--18.}

%\footnote{I have appended Jh\=a's translation of vv.~1--88 of the \emph{Codan\=as\=utra} to the end of this essay.}


The application of the doctrine of \emph{svata\d h pr\=am\=a\d nya} to the question of Vedic authority has understandably provoked suspicion. But as John Taber has pointed out, the thesis doesn't in its own right entail the authoritativeness of the Veda. That requires accepting the Vedas' authorlessness, which demands a good deal of theorizing about language and the ontology of reference, the nature of the other \emph{pram\=a\d nas}, and so on. And this is Kum\=arila's own understanding of the dialectical situation. So one could accept the M\={\i}m\=a\d ms\=a school's most general epistemological theses while rejecting Vedic authority. And I believe Taber is also right in arguing that, whatever its provenance, the \emph{svata\d h pr\=am\=a\d nya} thesis has merit.
But I'm also inclined to go further. If we may take \emph{dharma} as corresponding roughly to \emph{morality}, or to practical concerns of the highest order,\footnote{The difficulties here will be roughly analogous to the difficulties in comparing modern morality theory with ancient (Greek) ethical theory.} then most elements of the M\={\i}m\=a\d ms\=a outloook, \emph{including} the relationship between epistemology and \emph{dharmic} (or moral) authority, are not terribly foreign. The M\={\i}m\=a\d ms\=a see \emph{dharma} as a casual system which, properly understood, allows one to realize one's ultimate good. That idea that moral or ethical life allows one to realize one's good has had an enormous influence in Western thought; the suggestion that morality is a system of broadly instrumental rules is not foreign even to the contemporary discussion.\footnote{I think especially of Philippa Foot's ``Morality as a System of Hypothetical Imperatives'' \citep{foot1972morality}---but she has repudiated that view, for example in \citet{foot2003natural}.} A general worry about how moral epistemology is possible, manifest in the M\={\i}m\=a\d ms\=a attacks on other philosophical schools as well as in their own recourse to scripture, is a pervasive theme in contemporary metaethics.\footnote{The classic statement being \citet{mackie1977ethics}, though he is much motivated by the purportedly non-instrumental nature of moral demands.} The view that moral properties are in some way non-natural, and would thus require some special variety of epistemological insight, likewise retains enormous influence.\footnote{See again \citet{mackie1977ethics} for the skeptical view. For the positive view, \citet{moore1903principia} is the \emph{locus classicus}; cf.~more recently \citet{huemer2005ethical}. But the terms of metaethical debate of the last century \emph{in general} were set by Moore.} The idea that moral knowledge, or some moral knowledge, has ultimately a divine provenance also has its contemporary defenders. None of this is to say that these positions must therefore be plausible (I myself see nothing at all persuasive in some of these views), and there is also the (hardly trivial) matter of combining these various ideas into a coherent theory. The point is rather this: once we realize that the only element of the M\={\i}m\=a\d ms\=a view that will not be familiar from even a relatively limited exposure to moral philosophy in a contemporary philosophy department is the specific means---namely the Vedas conceived of as authorless---by which moral knowledge is acquired---and that is only an accident of history---then we can understand M\={\i}m\=a\d ms\=a epistemology as, in part, constituting \emph{an attempt to reconcile a general epistemological theory with a moral epistemology.} The M\={\i}m\=a\d ms\=akas will be no means have been the first to espouse a conception of knowledge that answers to practical as well as theoretical considerations; the thought that what constitutes reasonable belief may well depend in part on the demands laid upon us as practical creatures is again a familiar one, and this time it's surely a reasonable one as well.\footnote{See further \citet{clooney1987vhn} and \citet{tabereternality}.}

Kum\=arila's \emph{svata\d h pr\=am\=a\d nya} thesis \emph{can} be considered independently of the matter of moral knowledge. That is just what I'll do in \S3. But to appreciate the theory's merits and motives, it is necessary to discuss Kum\=arila's epistemology more broadly, which I'll do in \S3. But first, in \S2, I identify some interpretative difficulties, and some interpretative constraints.


%----------------------
%2. INTERPRETING KUMARILAS EPISTEMOLOGY
%----------------------

\section{Interpreting Kum\=arila's Epistemology}

\subsection{Interpretative Issues} \label{interpretativeissues}

Some difficulties in reconstructing Kum\=arila's epistemology, including the extraordinary concision of the texts, are characteristic of Indian philosophy. That concision allowed plenty of scope for commentaries on Jaimin\={\i} and \small\'S\normalsize \=abara to rival Kum\=arila's; later commentators on Kum\=arila were likewise able to differ greatly in interpreting \emph{him}.

The difficulty is compounded by the commentarial nature of the Indian traditions. A school is associated with certain views, and conspicuous novelty is avoided. Thus all of the members of a given school will defend nominally the same answer to what is nominally a single question of inter-school interest. In fact different schools, and different writers within a school, may be answering different questions.\footnote{Cf.~\citet[p.~1]{mohanty1989gangesa}.}

The \emph{svata\d h pr\=am\=a\d nya} thesis is a case in point. Every school had a position on whether \emph{pr\=am\=a\d nya} was \emph{svata\d h} or \emph{parata\d h} to cognitions. In English this is often glossed as a question about whether the `truth' or `validity' (\emph{pr\=am\=a\d nya}) of a cognition is intrinsic (\emph{svata\d h}) or rather extrinsic (\emph{parata\d h}) to that cognition. A scholar might well begin a survey by saying that the Ny\=aya school holds that truth is extrinsic to cognitions---where the Ny\=aya view is roughly that there is never a feature of a cognition which establishes that cognition's `truth'---and then go on to explain how the Buddhists think that both truth \emph{and} falsity are extrinsic to cognitions, even though that would be an absurdity on the Ny\=aya understanding of `truth.' But Buddhists can say that \emph{pr\=am\=a\d nya} and \emph{apr\=am\=a\d nya} (`invalidity' or `falsity') are both \emph{parata\d h} because they have a quite different understanding of \emph{pr\=am\=a\d nya}. Buddhists and Naiy\=ayikas are different enough that no-one is much confused about this, but M\={\i}m\=a\d ms\=a is closer to Ny\=aya and less understood. And just how little value there is in saying that the M\={\i}m\=a\d ms\=a believe that \emph{pr\=am\=a\d nya} is \emph{svata\d h} is only further brought out by the observation that there were plainly different understandings of the thesis in the different branches of M\={\i}m\=a\d ms\=a, and also among Kum\=arila's own later followers.\footnote{So for example Prabh\=akara before Kum\=arila, and Umbeka, P\=arthas\=arathi, and Sucarita after him---see \citet{chatterjea2003svatah} and \citet{taber1992dkb} for brief accounts.} `Intrinsic validity' as a gloss for ``the'' M\={\i}m\=a\d ms\=a thesis has a long history, and is well-represented in recent discussions,\footnote{\citet{keith1921karma} and \citet{radhakrishnan1927indian}, \citet{taber1992dkb} and \citet{arnold2001ivr}.} so that even if there were a better gloss it would be silly to change now. But the phrase must not be supposed to convey much.

\subsection{Rival Interpretations} \label{rivalinterpretations}

Let us return to Kum\=arila's thesis:\small \begin{quote}The validity of all valid cognitions is to be understood as intrinsic, since a potency not existing intrinsically cannot be brought about by something else. And [in general] things depend on [other] causes in arising, but once they exist they exercise their functions by themselves.\footnote{vv.~47--8; cf.~e.g.~vv.~6, 80.}\end{quote} \normalsize To start, we can say at least that validity, which is intrinsic, contrasts with invalidity, which is extrinsic. In the following verses, Kum\=arila supports these claims with a regress argument: if cognitions were not \emph{intrinsically} valid, they would have to derive validity from further cognitions. Those other cognitions must themselves have to be valid if they are to help, but then, by the hypothesized extrinsic nature of validity, there would have to be yet further cognitions validating \emph{them}, and so on \emph{ad infinitem.}\footnote{E.g.~vv.~49--51.} Thus validity must be intrinsic. But while other cognitions cannot make a cognition valid, they \emph{can} make it \emph{invalid}, and hence invalidity is extrinsic.\footnote{E.g.~vv.~52--3, 56, 86--7.}

Notwithstanding the interpretative difficulties noted above, this reasoning allows us to eliminate at least some possible readings of `validity.' So, for Buddhist philosophers, the world is made up of fleeting particulars, and association or categorization of them, as found in thought, is \emph{ipso facto} a falsification. The notion of `validity,' then, if it is to have utility, cannot be understood in terms of a correspondence between mind and world, but will rather be made out in pragmatic terms: a cognition is valid if the expectations it generates are not frustrated. Validity will then be \emph{extrinsic} to cognitions, since frustration or satisfaction of resulting expectations comes only after the cognition itself. So to say that validity is \emph{intrinsic} rules out at least such a view.\footnote{But even with Kum\=arila it's not trivial to rule out a pragmatic understanding. Karl \citet{potter1984die, potter1992presuppositions} has argued that `knowledge' has \emph{always} more a pragmatic than a theoretical import in Indian philosophy---but see \citet{mohanty1984pramanya}. A pragmatic understanding of validity was read into the \emph{SV} itself by Sucarita Mi\'sra, a later M\={\i}m\=a\d ms\=a commentator---see \citet[p.~52--53]{chatterjea2003svatah}. My own interpretation gives a role to pragmatic considerations; cf.~also \citet[pp.~XX]{bhatt1962epistemology}.} But that doesn't narrow the options enormously.

\subsubsection*{Validity as truth}

A popular interpretation of Kumarila's thesis has been this: cognitions are true unless shown otherwise.\footnote{Assuming a correspondence understanding of truth. Scholars of Indian philosophy often use `truth' as a label of convenience, and thus may describe Kum\=arila's thesis in these terms without subscribing to the interpretation now under consideration.}  (Alternatively, validity may only \emph{entail} truth.) K. K. Dixit, who adopts this interpretation, responds as one might then expect:\small\begin{quote}the difficulty with Kum\=arila's argumentation is that a piece of cognition not proved to be invalid is not necessarily valid, it might be valid but it might as well be otherwise.\footnote{\citet[p.~5]{dixit1983slokavarttika}; for validity as truth see also e.g.~\citet{chatterjea2003svatah}, and \citet{hiriyanna1932outlines}; among traditional authors see e.g.~Ga\d nge\'sa, \emph{PV} [CITE]. \citet{bhatt1962epistemology} interprets validity as truth-entailing, but understands intrinsicality in a way that would avoid Dixit's complaint; classically, cf.~Umbeka \emph{SVVT} [CITE].} \end{quote}\normalsize That would indeed seem to settle the matter. But this is an uncharitable interpretation, and is anyway at odds with the text. Kum\=arila several times allows that a false cognition may be valid.\footnote{For example v.~6 (Jh\=a's translation): ``Even the unauthoritative Means would, by itself, lead to the conception of its object, and its function could not cease unless its falsity were ascertained by other means.'' Cf.~e.g.~vv.~85--6.}

\subsubsection*{Taber on intrinsic validity}

John Taber has defended another interpretation, the basic lines of which he attributes to P\=arthas\=arathimi\'sra, another M\={\i}m\=a\d ms\=a commentator.\footnote{\citet{taber1992dkb}.} According to Taber, validity is not truth but rather as it were \emph{committment:} thus ``every cognition is accompanied by an initial sense of conviction; one is initially inclined to believe of every cognition that it is true.''\footnote{p.~216; cf.~p.~210. \cite{arnold2001ivr, arnold2001ivs} is broadly supportive of the Taber-P\=arthas\=arathi interpretation. For other suggestions in terms of presumptive or \emph{prima facie} truth see \citet[p.~18]{keith1921karma}, \citet[pp.~372--5]{dasgupta1957history}, who offers a coherentist spin, and \citet[p.~131ff.]{bhatt1962epistemology}. \citet[p.~52]{chatterjea2003svatah} and \citet[p.~8]{mohanty1989gangesa} also suggest that---at least in P\=arthas\=arathi's case---the claim is a psychological one. On P\=arthas\=arathi's interpretation cf.~\citet[pp.~50--55]{chatterjea2003svatah} and \citet[pp.~129--45]{bhatt1962epistemology}.} But Taber rightly allows that Kum\=arila is concerned not only with psychology but also with what is \emph{reasonable}. Verse 60 tells us that when a cognition is not undermined, then ``there is no reasonable ground for doubt.''\footnote{Cf.~\emph{TS} 2872--4, noted by Taber.} So we are supposed to be \emph{right} in accepting our beliefs as a guide to how things actually are. ``In the end,'' says Taber (to give a fuller version of the quote just above),\small \begin{quote}it appears that Kum\=arila adopts a common sense position. Every cognition is accompanied by an initial sense of conviction; one is initially inclined to believe of every cognition that it is true. One continues to be so inclined as long as the cognition is not called into question. Almost certainly, however, if the cognition is never overturned, it \emph{is} true. Falsehood cannot conceal itself forever. If, over the long run, the cognition is not shown to be false, then on the basis of its initial, intrinsic validity one is certainly \emph{justified} in believing that it is not false, that it is really true.\footnote{p.~216.}\end{quote}\normalsize This, for Taber, shows Kum\=arila adopting a kind of hard-headed empiricism: one may believe only that for which one has evidence; but in addition, to have \emph{no} evidence for something (to have no evidence that your view is wrong, for example) is to be warranted in thinking that no such evidence exists.\footnote{pp.~205, 216.} To ask for more is in any case futile, since the demand only leads to regress.

Taber's is one of the most sympathetic and careful interpretations of Kum\=arila on intrinsic validity, and there is much to like about it. But how can we take seriously the idea that a cognition never undermined is ``almost certainly'' true, a view that Taber says is ``common sense''? Dixit's comment seems apt again: it might be true and then again it might not. Suppose, for instance, that, from a distance, I mistake for a woman a man whom I am unlikely to encounter again, and would not recognize if I did. How is the passage of time likely to help me see my error? Taber himself raises roughly the same worry about Kum\=rila's defense of the Vedas, namely that they seem to come out unscathed only by default.\footnote{p.~217.} The problem is more general, as my example shows, and it is also a more general problem about practical life: it's often hard to know whether one has acted well. I could not possibly have a general expectation that advice---to buy fair-trade coffee, say---will, if bad, always be recognized as such.\footnote{A complicating factor here is the M\={\i}m\=a\d ms\=a adherence to the view (very contentious in the Indian setting) that `absence' (\emph{abh\=ava})is a sound means of acquiring knowledge. That is, one may rightly infer from the absence of any indication of something that it doesn't exist. But this principle makes no appeal to the passage of time, and Kum\=arila is liberal and reasonable about what counts as an indication of a real or potential difficulty about evidence. Taber does not rely on an appeal to \emph{abh\=ava} in his defense of \emph{svata\d h pr\=am\=a\d nya}.}

%[[A POINT FROM CHATTERJEA 51: it;s the idea of absence as a valid source of cognition that allows K to say what he does --- but note that clearly we'll have to treat validity in this case in terms of warrant or validity and not truth; because obviously sometimes something \emph{does} come up later.]]

The difficulty here is also particularly acute given Taber's view that \emph{all} cognitions are intrinsically valid, barring the surfacing of some reason for doubt. On Taber's understanding, \emph{every judgement} has \emph{prima facie} justification or warrant. There is no good reason, on view, why a judgment should not be warranted, even if it's formed on the basis of a dream or wishful thinking. Taber may say that Kum\=arila's defense of the Vedas is unsuccessful because it makes Vedic injunctions valid just by default, but that's tantamount to an objection to the \emph{svata\d h pr\=am\=a\d nya} thesis itself, on Taber's interpretation thereof. The whole \emph{point} would be to make cognitions justified by default, despite the fact that many types and instances of judgements---including quite empirical judgements---cannot be supposed likely to be corrected if faulty.\footnote{Taber (p.~215, n.~56) cites v.~80 in support of the idea that the passage of time improves confidence, but while the verse shows that Kum\=arila is trying to make more than a psychological point, I cannot see in it any indication that he is thinking about the passage of time, except in the trivial sense that a counter-indicating cognition would come after the initial cognition. Taber also adduces \emph{TS} 2904, but again I see in that verse only the claim that cognitions are valid if they're not upset by other cognitions. Kum\=arila refers in this verse to cognitions occurring at ``other times and places,'' but that doesn't mean that length of time (or travel experience, for that matter) makes a difference to warrant except insofar as counter-indication cognitions would, by their nature, have to occur at other times.\label{tabertextual}}

Taber interpretation draws attention to crucial but often overlooked features of Kum\=arila's epistemology, including the lack of coincidence between truth and validity, the importance of warrant or justification. But Taber does not explain how warrant enters the picture; the interpretation falls into the same trap that P\=arthas\=arathi's does, and which other scholars have noted in that context: the theory reduces to a theory of judgement.\footnote{Thus \citet[p.~8]{mohanty1989gangesa} and \citet[p.~52]{chatterjea2003svatah}.} Taber's has provided an argument for his claim that a cognition ``compels our assent''\footnote{p.~216.} only if that means merely that judgment \emph{entails} assent. We are offered no reason to suppose that it would be unreasonable simply to stop believing something upon realizing that one doe believe it but can see no particular evidence supporting that belief. Since Kum\=arila explicitly draws that conclusion\footnote{v.~60; cf.~vv.~[XX], \emph{TS} 2872--4.\label{reasonableness}}---and Taber himself acknowledges the importance of the point---Taber's attempted defense of Kum\=arila's epistemology must be judged a failure. Moreover, while Taber is right to say that the [\emph{svatah pramanya}] thesis is separable from the doctrine of Vedic authority, the thesis is plainly motivated in great part by a desire to defend, in effect, the possibility of moral knowledge.\footnote{[CITES.]} Even if there are independent arguments---from the threat of regress, most obviously---it does not seem to me that we do Kum\=arila's view much of a service if the realm of knowledge of most interest to him fares least well on his own epistemological theory. Besides which, it is hard to imagine that Kum\=arila would rest his thesis on the idea that the passage of time makes a judgement more likely to be correct when this line of thought so clearly could not, \emph{in principle}, apply to the Vedic case, and which, indeed, would make validity \emph{extrinsic} in an obvious sense, since it would accrue with time. But Taber's interpretation relies on this claim, for which the textual evidence is ambiguous at best.\footnote{See n.~\ref{tabertextual}.}

Nevertheless, Taber \emph{does} draw attention to important and neglected points in Kum\=arila's thought. By drawing out yet more critical features of the view, I believe it is possible to show that Kum\=arila's view is genuinely compelling.

\subsection{Interpretative Constraints} \label{constraints}

Two points in particular will help in constructing a better interpretation. The first stems from a bit of a puzzle. So far as the above interpretations go, it sounds like just any thought or belief whatever is supposed to be possessed of validity. Taber, for example, insists quite explicitly that \emph{all} cognitions are valid, unless sublated or otherwise brought into doubt by some further cognition.\footnote{p.~[XX].} And that is a common reading of Kum\=arila. But the M\=im\=a\d ms\=as, like other schools, provide a list of the sources of knowledge or valid cognition---for the M\=im\=a\d ms\=as the list includes, among other things, perception, inference, and testimony.\footnote{The six M\=im\=a\d ms\=a [PRAMANAS] are [LIST].\label{thesix} } But what is the need for such a list if every belief is automatically valid? In fact Kum\=arila does \emph{not} think that every cognition is valid---mere fancies or wishful thoughts, for example, are not valid.\footnote{vv.~94--5; cf.~also vv.~85 and 128.} In fact, Kum\=arila notably excludes memorial judgements from the realm of validity---only cognitions or judgments with novel contents are apt for validity.\footnote{\emph{SV} II.22--3, V.11.} And also memories are, on Kum\=arila's view, generally recognizable as such,\footnote{[CITE - CHECK BHATT 90--3]} we don't always obvious whether a judgment is memorial, since on the Mimamsa view mistaken judgements---say formed as a result of a dream---are effectively memorial events or collocations of memorial elements.\footnote{[SL CITE]; [SECONDARY CITE].} So we cannot say that every judgement has intrinsic validity, or that any commitment we have is warranted barring some counter-indication. So we'll need to find a way to understand how cognitions can be \emph{intrinsically} valid without severing the connection between \emph{valid sources of cognitions} and \emph{valid cognitions}.

The second point is this. The M\={\i}m\=a\d ms\=as are strict about the distinguishing the mental state which provides some awareness of the world (I will call this a first-order cognition) from the mental state which provides awareness of that first-order cognition (I'll call this the second-order cognition).\footnote{\emph{SV}, \emph{\'S\=unyav\=ada} v.~64ff.; for the school see ~\citet[pp.~20--1]{keith1921karma} and citations there.} In some schools of Indian thought, one cognition can serve both roles: awareness is itself present to consciousness as it reveals its object. Not so in M\={\i}m\=a\d ms\=a.\footnote{See \citet[pref., chs.~IV, X]{sen1984concept} for discussion.} But the implications of this point for the \emph{svata\d h pr\=am\=a\d nya} thesis have not been properly appreciated. In effect, we have \emph{two} theses.\footnote{cf.~\citet[p.~41]{chatterjea2003svatah}.} One thesis concerns the status of the first-order cognition, and the second concerns the apprehension of that status by a second-order cognition.

In my view the failure to appreciate point is partly responsible for the failure to appreciate the first. Tara Chatterjea, for example, though observing that the \emph{svata\d h pr\=am\=a\d nya} thesis is really two theses, says that Kum\=arila makes ``contradictory suggestions'' concerning the scope of the \emph{svata\d h pr\=am\=a\d nya} thesis, and identifies v.~53 as apparently claiming that \emph{all} cognitions are intrinsically valid, in contrast with e.g.~v.47 which says only that \emph{valid} cognitions are intrinsically valid.\footnote{\citet[p.~51]{chatterjea2003svatah}.} (Taber also relies on v.~53 in claiming that all cognitions are valid.\footnote{\citet[p.~171, n.~27]{taber2002mo}, \citet[p.~212]{taber1992dkb}.}) But if v.~47 concerns the status of a cognition while v.~53 concerns the \emph{assessement of} the status of a cognition, then that is too quick. It might be, for example, that only some cognitions are valid, but that every cognition, when apprehended, is apprehended as (i.e.~ taken to be?) valid. We may thus hope that care regarding these two points will make an adequate interpretation possible.




%----------------------
%3. KUMARILA ON INTRINSIC VALIDITY
%----------------------


\section{Kum\=arila on Intrinsic Validity}



After describing my interpretation of Kum\=arila's \emph{svata\d h pr\=am\=a\d nya} thesis, in \S3.1, in \S3.2, interpretation in hand, I work through a portion of the \emph{Codan\=as\=utra}.


\subsection{The interpretation in outline.} \label{interpoutline}

`Validity' has, for Kum\=arila, as for most Indian philosophers, both a causal or psychological aspect and also a normative aspect; in India, as in the West, different understandings of perception were taken to have implications concerning what we can know, and epistemological commitments motivated views about mind and perception.

Beginning with the psychological side, consider the idea, common in the history of philosophy, that experience confronts us with certain \emph{appearances} (or \emph{ideas} or \emph{impressions} (though these appearances need not always be pictoral or visual in nature). One may hope that these appearances at least sometimes resemble or represent the world in some way. Locke, for example, thought that they sometimes did, and that we could sometimes be confident of this; the Stoics had a comparable view. Certain ancient skeptics thought that indeed our impressions \emph{might} resemble or represent the world, but they argued that we can never \emph{know} whether, and should not believe that, they do. And then Hume and Kant, in different ways, decided that the whole question was nonsense. Parallel positions existed in India concerning what we have called `cognitions.' As in the early modern case, skepticism was more threat than reality, but Locke, Hume and Kant have their counterparts among Naiy\=ayikas, Buddhists, and idealists. Thus far, then, we might think of cognitions as being roughly equivalent to appearances or impressions.

Now one might think of cognitions or appearances offer themselves for acceptance or assent, or, equally, for rejection. This is the view of some of Kum\=arila's opponents; hence an objection considered by Kum\=arila to the effect that cognitions do \emph{not} present themselves as \emph{authoritative}.\footnote{v.~85.} Now Taber says that ``the intrinsic validity of cognitions is vested for the M\={\i}m\=a\d msaka in the sense of conviction with which they typically arise.''\footnote{\citet[p.~172]{taber2002mo}; cf.~\citet[p.~203]{taber1987review}: validity ``is inherent insofar as it is vested in the essential `presenting' nature of cognition.''} And it is right that, for Kum\=arila, in a cognition, the subject is not simply presented with a view or thought which he might or equally well might \emph{not} accept as his own. To form the cognition is indeed, as Taber says, \emph{already} to view the world in the way portrayed by the cognition. But this does not go far enough. The problem is that, in Kam\=arila's case, the language of \emph{portrayal} is out of place altogether; so also, I think, is Taber's willingness to make use of the language of appearances, as does when he sees in Kum\=arila ``a kind of rigid philosophical realism which insists strictly on the reality of appearances,'' and when he says that the ``emphasis on the inability to distinguish true from false cognitions expresses what seems an almost skeptical attitude on the part of Kumarila, viz., that we can never establish with absolute certainty that a cognition corresponds with reality.''\footnote{\citet{taber1992dkb}, p.~206; 215; cf.~pp.~207, 216, 218, 221.}

Consider on the one hand a (representational) painting. One may appreciate it without supposing that what is represented there ever existed elsewhere. For most of us, on the other hand, a photograph or recording invites us to suppose that we are getting a bit of a glimpse of another time and place. Indeed it might be only when something seems off about the photograph that we feel the need to inspect it more carefully---when what is displayed seems very improbable, say, or when a face on the cover of a magazine is suspiciously free of wrinkles. But Kum\=arila's cognitions are not like paintings \emph{or} photographs. A photograph, like a painting, is something we examine it its own right, and which we may call into question. But, as was pointed out above---and this is uncontroversial---Kum\=arila does not believe that cognitions are items to be inspected in their own right. A better analogy for Kum\=arila's cognitions would be a (very clean) \emph{window}. The existence of a cognition is \emph{inferred} from what is shown us by it, just as we may realize that there is a mirror or a window at the end of a hall because we see \emph{ourselves} there at the end of the hall, or buildings beyond it.\footnote{\emph{SV} [CITES]. Compare the familiar idea that, in order to figure out what one believes, one looks, as it were, not \emph{inward}, but \emph{outward}, towards the world.}

\subsubsection*{Intrinsic validity as a psychological thesis.} It was noted earlier that the `intrinsic validity' thesis is really \emph{two} theses, one about a first-order cognition and one about a second-order cognition. There are likewise two aspects of the thesis still considered as a psychological thesis. First, as regards the first-order cognition, the claim is that it is the nature of cognitions to provide a \emph{window on} the world, rather than a \emph{picture of} it---to have a cognition \emph{just is}, at least in the basic case, to have a view of the world; it is not something that can be inspected itself.  Regarding the second-order case, the point is that one reads (it's a matter of inference) the existence of the first-order cognition off of the portion of the world thus revealed (you see the buildings through the window, and because you can see buildings you may infer that there is a window). The first-order cognition does not become an object of thought directly; it can be contemplated, indirectly as it were, without, as it were, \emph{stepping back from the cognition}, or without regarding the cognition as something now in question, or as something which might now be rejected, since it is the very having of the view that makes it possible to recognize the cognition.\footnote{We must leave aside the details of Kum\=arila's theory of perception, which would also take us into his metaphysics and philosophy of language, but a translation of the relevant chapter of the \emph{SL} (the \emph{Pratyak\d sapariccheda}) has recently been offered, along with extensive notes and discussion, by \citet{taber2005hindu}.}

Kum\=arila also claims that \emph{invalidity} is \emph{extrinsic}. Remember that validity does not entail truth or accuracy. If cognitions manifest the world rather than representing it, Kum\=arila nevertheless thinks that cognitions may err in various ways. We might again compare a window, which, even if very clean, so that we cannot inspect it directly, may be warped or tinted so that although we see the world through it and not in it, we see the world through it \emph{wrongly}. The thesis that invalidity is extrinsic emphasizes the point that even when a cognition portrays the world wrongly, we cannot consider it directly and note features of that cognition that are misleading, the way you could point out parts of a photo that may mislead. What you can do is \emph{form new cognitions}, and these may reveal that earlier cognitions were defective. (Kum\=arila identifies two cases. First there is the case where you directly apprehend something as being different than you had thought: this is like looking out a new window, seeing that it's bright outside,  and realizing that the first window had been tinted. Second there is the case where you realize that the first cognition was formed in some defective way: this is like being told that the windows are tinted and realizing it's brighter out than you thought.)\footnote{\citet{kataoka2002validity} outlines the structure of invalidity; cf.~\citet[p.~132]{bhatt1962epistemology}.}

But the extrinsic invalidity thesis, still regarded as a psychological thesis, may again be divided into two parts, regarding first- and second-order cognitions respectively. First, since there is nothing in a cognition itself which could indicate a problem with it, a problem can only be indicated by another cognition, or by something turned up by another cognition (either because it reveals the world differently or because it reveals that the first cognition was formed in a problematic way). And second, for the very same reason, one can only \emph{recognize} a problem with a cognition on the basis of what is turned up by some further cognition. The difference between the two cases is that, in the first case, one might never even notice that one had possessed a cognition that was problematic; you might simply have adopted a new view without ever noticing an evolution in your thinking, for example, or you might wind up holding onto inconsistent views (in which case you would not, of course, have yet actually given up. In the second case, the new cognition becomes your new way of seeing the world; to see the old cognition as defective (as invalid) is already to have set it aside.


%Again, this interpretation of \emph{svata\d h pr\=am\=a\d nya} differs from the interpretation according to which it is the thesis that a cognition or judgement is characteristically attended by conviction, or, perhaps better, according to which a cognition presents the world a certain way?\footnote{The second gloss is preferable, I think, since it does not suggest that cognitions are appearances or mental images.}

So again, on my interpretation of \emph{svata\d h pr\=am\=a\d nya} we must actually distinguish between the capacity of a cognition to manifest the world (whether accurately or not) and validity. They are not the same, nor are validity and manifestation always found together. Now Kum\=arila frequently appeals to the fact that a cognition is present and manifesting the world in a certain way as a reason for saying that cognitions are intrinsically valid,\footnote{See e.g.~vv.~6, 47, 85.} and this has no doubt contributed to the thought that manifestation and validity are the same thing. But Kum\=arila also says that even invalid cognitions can manifest the world, at least until the error is noticed.\footnote{v.~85, cf.~vv.~25, 94--5 [XX].} This point seems to have been overlooked (i.e.~by interpreters who think that \emph{all} cognitions are---at least initially---valid) primarily on account of a failure to distinguish the first- and second-order cases.\footnote{The other factor is perhaps the failure to see the connection between the validity of means of knowledge and the validity of cognitions. See \S\ref{constraints} above and \S\ref{textualinterpretation} below.} Kum\=arila assumes that once an error has been acknowledged, the erroneous cognition is set aside; thus a cognition will manifest the world in a certain way until it is recognized to be invalid. Thus it has been supposed that validity and manifestation go hand in hand. But that conflates the first- and second-order theses. A cognition could continue to manifest the world a certain way even after its invalidity has been revealed, if that invalidity has not been noticed. And there is another way that manifestation may come apart from validity. It is only cognitions derived from certain sources that count as valid: in particular valid cognitions derive from valid modes of knowledge.\footnote{See \ref{constraints} above.} But you could see the world some way without having a view that is even apt for validity. (Memorial beliefs, for example, are not validity-apt.\footnote{\emph{SV} [CITE].}) A cognition is intrinsically valid, then, neither in virtue of being accurate nor in virtue of manifesting (or pretending to manifest) the world. A cognition is intrinsically valid when it constitutes a genuine (though perhaps distorted) window on the world, in virtue of having a psychological or causal basis which allows it to be such a window. Cognitions arising from another source, however much they present or seem to present the world, and however accurately they do so, are not and cannot be valid, intrinsically or otherwise.




%\footnote{It may be that Kum\=arila actually has this case primarily in mind at v.~85.}


\subsubsection*{Intrinsic validity as a normative thesis.} Although Kum\=arila generally describes cognitions and validity in psychological or causal terms, he takes his account to have (what we would consider) normative implications.\footnote{See n.~\ref{reasonableness} above.} We can indeed consider Kum\=arila's thesis to have a normative or evaluative aspect.

On the normative side, then, we may consider validity as the status a cognition has when one is rationally entitled to it: I'll call it \emph{warrant} or \emph{entitlement}. And we can again split this into two points, one regarding the first-order cognition, and the other a second-order cognition. For the first-order case, the claim is that the warrant of a cognition does not depend on its being buttressed by further cognitions. (Thus, for example, if you read in the \emph{Times} that Hirut is beautiful, and come to believe that Hirut is beautiful, this belief does not require, for its warrant, that you have any other supporting belief, for example that the \emph{Times} is a reliable source.) Second, one may realize that one possesses no evidence (or further supporting beliefs) for some belief without being rationally required to suspend or abandon that belief. Indeed, barring undermining evidence, it is irrational to reject your beliefs.  (So, for example, suppose you realize that you believe Hirut is beautiful, but can't remember how it is that you came by that belief. If you're reasonable, and barring some more definite source of doubt, you will stand by this belief regardless.) To say that invalidity is extrinsic again means two things. First, a warranted cognition loses that warrant when some further cognition undermines it. (The belief that Hirut is beautiful will lose its warrant if you read that the reporter has been fabricating stories.) Second, a cognition is properly adjudged unwarranted when you can see that there is something to undermine it.

However, a cognition is not simply warranted \emph{by default.} The warrant of a \emph{warranted} cognition is intrinsic in the sense described. But to have that status in the first place depends on the causal or psychological origin of the cognition. Hence the list of `valid' \emph{sources} of cognition, or warrant-conveying sources of cognition. Thus a cognition arrived at through testimony, inference, perception, or other sound basis of knowledge is warranted---though it may not be \emph{correct}, since not every belief derived from perception (for example) is in fact veridical. A cognition originating merely in a dream, for instance, was never warranted in the first place, though it may very well masquerade as a valid cognition, and as a cognition from a good source.

\

How are the psychological and the normative aspects of \emph{svata\d h pr\=am\=a\d nya} related? We may note first that it's characteristic of the Indian tradition to (as we would see it) transform normative or evaluative questions into epistemological or metaphysical questions, so far as possible. Notoriously there seems to have been virtually no direct philosophical inquiry into final ends---no ethics or moral philosophy in the Western sense. The introduction has already given an illustration of the kind of thing that happened instead: debates about final ends were fought by proxy, ramifying out into epistemology, ontology, perception, language, and so on, with each school developing an increasingly sophisticated philosophical apparatus to buttress their approach to duty or liberation. And in our present case, it's quite unsurprising that there should be a normative or evaluative side to the intrinsic validity thesis, since, as a psychological theses, it offers resistance to the idea that some objects should stand between ourselves and the world, an idea that the M\={\i}m\=a\d ms\=a plainly felt led to skepticism.\footnote{See further \citet{sen1984concept}. Of course this sort of thing is familiar in the West as well.} 

And that is Kum\=arila's strategy. In his view, the various \emph{pram\=a\d nas} are ways of opening up windows (in the form of cognitions) on the world. They don't merely represent it, but reveal it. This is what distinguishes the \emph{pram\=a\d nas} from other sources of cognition such as wishful thinking or dreams---also, in Kum\=arila's view, even from memory, which is parasitic on genuine \emph{pram\=a\d nas}. But then it would be a mistake to distance yourself from your cognitions, as if they merely \emph{represented} the world, or as if in looking around your surroundings you were only looking at something in your own head. A confused philosophical perspective on these matters leads to skepticism---it did so, or was thought to, in India as in the West---and a clear view ought to ward off that skepticism. Now of course sometimes you're wrong about things---Kum\=arila doesn't pretend otherwise. One often learns this in the ordinary course of things, as when one gets a closer view or a view in better light. But it is because cognitions open up the world to us the way they do that allows us to correct ourselves in that manner.

\

Kumarila's claim that cognitions are grasped as valid---the second-order intrinsic validity thesis---gives the impression that Kum\=arila is trying to just help himself to evidence. But we are in a position to see that grasping a cognition as being valid corresponds more closely to \emph{reasonably believing something in a conscious and occurrent manner} than it does to \emph{believing that a belief is reasonable.} The latter---believing that a belief is reasonable---looks like it ought generally to be a response to some kind of evidence one can marshall regarding the belief in question or its content. But that kind of reflective attention involved in \emph{evaluating} a belief is different from the reflective attention involved in simply \emph{exploring} one's beliefs. Consider the familiar idea that, in order to figure out what one believe, one looks \emph{outward} towards the world---or the world as one sees it. If I'm asked whether Hirut is beautiful (or whether I believe that she is), and supposing this time that I've seen her at some point, then I'll think back to what she looked like, or try to frame her image in my mind, and then say ``yes'' or ``no'' depending on what I see there. There need be no question in this case of assessing evidence in any ordinary sense, or of, say, reviewing the visual conditions under which I first saw her. This is the case Kum\=arila is concerned with. And Kum\=arila's claim then seems fair enough. This kind of reflection just assumes the reasonableness of the original belief, and it's hard to see how else it could be. In Kum\=arila's view there is no need to be concerned with the further question about the basis of one's beliefs, so far as no particular cause for concern arises; indeed, if there were such cause for concern, you could never answer this simple question about whether Hirut is beautiful if the moment you thought back on Hirut your earlier cognitions of her must be open to question. Compare J.~L.~Austin on demanding how one knows:\small\begin{quote}If you say 'That's not enough', then you must have in mind some more or less definite lack. 'To be a goldfinch, besides having a red head it must also have the characteristic eye-markings': or 'How do you know it isn't a woodpecker? Woodpeckers have red heads too'. If there is no definite lack, which you are at least prepared to specify on being pressed, then it 's silly (outrageous) just to go on saying 'That's not enough'.\footnote{\citet[p.~84]{austin1961other}.}\end{quote}\normalsize 

That also helps us to make sense of the idea that \emph{all} cognitions may be taken as valid just in virtue of being cognitions. Sure enough, not every cognition \emph{is} valid. But accepting it as valid doesn't \emph{make} anything valid, either: Kumarila's does not think that reflection renders a cognition valid any more that it renders cognition invalid; nor does taking a cognition to be valid in any way \emph{establish} that it is valid, or even imply that any evidence favors that cognitions, unless it's what's manifested in that cognition already. It does not mean that warrant or evidence has \emph{accrued}. To put the point differently, we might say that Kum\=arila is interested rather in \emph{being} justified than in \emph{having} a justification.


%; not clear in a way that K ever deals with the other thing, not that he thinks there's anything wrong with it. Just doesn't affect whetehr you have justification.]]

Again, Kum\=arila's view is that reasonable belief does not \emph{require} inquiry into evidence; it is not his view that there is anything \emph{wrong} with such inquiry. We saw above that Taber attributed to Kumarila the idea that if no contrary evidence comes up then there must not be any. But in fact Kumarila' need only claim that, absent problematic evidence, one is not rationally required to suspend or reject belief.

There is no willful blindness in the approach, thought the M\={\i}m\=a\d ms\=as' larger aim of defending Vedic authority may create that impression. Kumarila view is not that testimony (for instance) ought always to be accepted unquestioningly. There are sorts of reasons for distrusting testimony, or the deliverances of any other cognitions-forming mechanism for that matter. Kumarila gives no ground to skepticism---the fact is that quite often we know or can easily know plenty about the various ways we form our beliefs and about when further investigation is required. Kumarila is not confronting the skeptical worry that we can never get outside our own beliefs to see how well they fit the world by saying baldly that we just \emph{can} accept our beliefs as generally sound. He is, rather, refusing to allow the skeptical worries to set the agenda at all.

% [[HERE WE CONFRONT AN ISSUE ABOUT `ABSENCE']].




%The position is quite intelligible. It may be reasonable to believe based on testimony, and unreasonable to believe only from wishful thinking. And it may be that a reasonable person will shed a belief once he realizes that it reflects only wishful thinking. But neither of those facts entail that it is unreasonable to accept those beliefs at face value, since after all I may have no reason to suspect that the beliefs \emph{are} rooted in wishful thinking.

%\begin{quote}It is in the case of objection (2) that you would be more inclined to say right out 'Then you don't know'. Because it doesn't prove it, it's not enough to prove it. Several important points come out here: 

%(a) If you say 'That's not enough', then you must have in mind some more or less definite lack. 'To be a goldfinch, besides having a red head it must also have the characteristic eye-markings': or 'How do you know it isn't a woodpecker? Woodpeckers have red heads too'. If there is no definite lack, which you are at least prepared to specify on being pressed, then it 's silly (outrageous) just to go on saying 'That's not enough'. \end{quote}





\subsection{Working through the text.} \label{textualinterpretation}

As noted earlier, the \emph{Codan\=as\=utra} is a commentary on \emph{MS} I.i.2, which says that ``duty is a purpose having injunction for its sole authority.'' A number of the initial verses of the \emph{CS} are concerned with minor clarifications of this thesis, but vv.~5--6 are close to a statement of the chapter's main theses:\small\begin{quote}5.~Inasmuch as authoritative character [regarding duty] is possible only to the Word, [\footnotesize\'S\small \=abara] has also pointed out the incapacity, with regard to such objects [necessary for knowledge of duty], of Sense-Perception and the rest, which is to be described hereafter. 6.~Even with regard to purely non-existing objects, the Word brings about some conception. And consequently, in the absence of any discrepancy, authoritative character must be accepted to belong to it by its very nature.\footnote{Jh\=a refers to `cognitions' as `conceptions'; `validity' is `authoritativeness'; `intrinsic is `inherent'; Jh\=a also uses `means of right notion' to refer to a valid cognition.}\end{quote}\normalsize The latter parts of the \emph{CS} are concerned more directly with the role and value of testimony as opposed to the other sources of knowledge mentioned here in v.~5, and the earlier portions (up to around v.~102) focus more on the the general epistemological theses indicated in v.~6.

The latter verse picks up on a common M\={\i}m\=a\d ms\=a theme, namely the independence of testimony as a source of knowledge:\footnote{See \S1 above.} testimony produces cognitions, including of things non-existent, hence which could not have been produced by sense perception---and thus is a \emph{primitive} producer of cognitions---and so it is a \emph{pram\=a\d na}. Now this seems rather problematic in a few ways. On the one hand Kum\=arila will not want to count just anything as a \emph{pram\=a\d na} merely because it leads us to think things. But if the point is rather (as I think it is) that testimony can (at least in the case of Vedic testimony regarding duty) provide us with a window on things that other \emph{pram\=a\d nas} cannot, then the claim will be question-begging. But this is only a preliminary statement of the view (and fortunately we are not concerned with the case of testimony specifically), so we may simply note that a cognition-producing mechanism is \emph{eo ipso} a valid mechanism, i.e.~a producer of \emph{valid} cognitions.

The starting-point for the discussion of \emph{svata\d h pr\=am\=a\d nya} is an objection to the M\={\i}m\=a\d ms\=as doctrine that an authorless text is our authority on \emph{dharma}, the nub of which objection Kum\=arila presents himself:\small\begin{quote}22.~It is always an object perceived by other means of knowledge that is got at by the Word [testimony]; and, like memory, no authority can belong to it by itself.\end{quote}\normalsize When one accepts testimony, it is because ``it is an assertion of a person who is believed to have perceived the object.''\footnote{v.~23.} In other words, testimony is not a primitive source of knowledge like perception or inference are, and the objector adds that we sometimes disregard testimony when we \emph{don't} judge the source reliable---but how could that be, if a reliable source isn't in fact essential to testimony as a source of knowledge?\footnote{v.~30.} The opponent also makes the objection I worried about a moment ago:\small\begin{quote}25.~``Thus then, as even when producing a conception, Fancy, \&c., are no authorities by themselves, so we may apply the same rule to the case of Veda also.''\end{quote}\normalsize So, again, we do not want to say that just any psychological mechanism which produces some judgment is therefore to be accorded authority.

Kum\=arila responds by broadening the inquiry, turning away from the specific case of testimony:\small \begin{quote}33.~With regard to all conceptions, you must consider the following question: ``Is the authoritativeness or unauthoritativeness due to itself or to something else?''\end{quote} \normalsize There are three possible positions besides Kum\=arila's own:\small\begin{quote}34.~Because those that are by themselves false cannot by any means be proved to be true. Some people attribute both (authoritativeness and its contrary) to (the conception) itself. Others attribute them to the proved excellences or discrepancies of its origin.\end{quote}\normalsize In the first sentence Kum\=arila briefly hints at his reason for rejecting the view that invalidity is intrinsic and validity extrinsic---he will return to it at v.~38ff. That leaves the possibility that both validity \emph{and} invalidity might be intrinsic to cognitions, or that both might be extrinsic to cognitions. The next verses are devoted to rejecting these possibilities. Regarding the first possibility, Kum\=arila says that\small \begin{quote}35.~Both cannot be due to itself, because the two are mutually contradictory\ldots.

\noindent 36.~How can it be possible that any one thing, independent of all extraneous agency, should have contradictory characters? \ldots

\noindent 37.~If ``non-contradictoriness'' were possible with regard to different conceptions---even then, if nothing else is taken into consideration, it cannot be ascertained which is which, and where.\end{quote}\normalsize We might, uncharitably, see the argument of vv.~35--6 as turning on a confusion of type and token---why shouldn't some cognitions have the one character and other cognitions the other? Verse 37 would then acknowledge the confusion and offer a better argument. But that would be an unhelpful procedure, and v. 36 offers some textual basis for rejecting this reading. Verse 36 asks not just how cognitions could have contradictory characters, but specifically how anything could have contradictory characters \emph{independently of all extraneous agency.} If we were talking about a single cognition it would not matter whether or not extraneous agencies were implicated---a single thing could not be valid and invalid at the same time regardless. That suggests that even in vv.~35--6 Kum\=arila has in mind cognitions in general rather than particular cognitions---some cognitions will be valid and some invalid but the difference has to be explained by further, external factors.

Why should we have to appeal to extraneous factors to explain the difference? And what is the real objection in vv.~35--6 to the possibility that both validity and invalidity might be intrinsic? A reasonable possibility is that Kum\=arila thinks that either validity or invalidity should be in some sense \emph{natural} to conceptions. But then \emph{only one or the other} could be natural to cognitions in general, not both. Thus if validity were natural or intrinsic to cognitions then the invalidity of a given cognition would require some further explanation appealing to extrinsic factors. This interpretation would fit the language employed in the string of objections that are shortly posed. Thus 38: ``for those that hold the authoritativeness of conceptions to be natural\ldots''; v.~40:  ``if authoritativeness were inherent or natural\ldots''; v.~44: ``while unauthoritativeness, being natural\ldots .'' So it seems reasonable to say, on the basis of vv. 35--6, that, for Kum\=arila, the question of the intrinsic or extrinsic nature of validity or invalidity is a question about the natural state of cognitions in general, the point then being that there could be (at most) only one such natural state. And that seems a natural view to hold if the questions is in part a question as to whether cognitions are presentative or rather \emph{re}presentative.\footnote{See \S\ref{interpoutline} above.}

But in v.~37 Kum\=arila seems to set aside the inquiry into cognitions in general. Now he does consider the possibility that cognitions should be considered individually. Might some then be intrinsically valid and others invalid, just as there are both natural redheads and natural blondes? Kum\=arila now objects that ``it cannot be ascertained which is which, and where.'' The idea, I think, is this. Focussing on the psychological side of the issue, the suggestion at hand is that some cognitions are presentative in nature and others representative.\footnote{At least that's how Kum\=arila, a realist, would look at the matter: Buddhists, for example, who might actually hold the view that both validity and invalidity are extrinsic, will generally think about the matter in very different terms---see \S\S\ref{interpretativeissues} \&\ref{rivalinterpretations} above. It's not clear that Kum\=arila is really trying to seriously respond to such views as opposed to just considering all of the possibilities in a realistic spirit. Kum\=arila's principal target is evidently the view that validity is extrinsic and invalidity intrinsic} But since the question has implications for the question of warrant---there remains an open question about the accuracy of a representation which does not arise in the case of a presentation---then the suggestion being considered in v.~37 is effectively a suggestion that sometimes a cognition \emph{arrives} warranted, as it were, while other times a cognition is warranted only if supported or certified by another cognition. The ``which is which and where'' objection seems very much to the point then, the thought being that this suggestion undermines the possibility of intrinsic validity altogether---for how could you be warranted in accepting the intrinsically warranted cognitions if you can't tell them apart from the ones that need to be certified by appeal to another cognition? A cognition that requires either buttressing or discarding must be discarded if you can't buttress it; but if you can't tell the cognitions to which that requirement applies apart from the cognitions to which it doesn't, how can you be rationally required to stand by the latter, given that for all you need they must be either buttressed or discounted? True, there just could still \emph{be} intrinsically warranted beliefs after all---i.e. cognitions to which you are entitled even if they are not supported by other cognitions. But that entitlement would seem to disappear upon reflection.\footnote{At present we are asking a question about the warrant of the first-order cognition, and not a question about how it can be \emph{determined} whether a first-order cognition is warranted. But one part of the suggestion in the contemplated scenario is that some cognitions may be warranted only \emph{if} assessed in a certain way.} The epistemological benefit of understanding cognitions as presentative would thus disappear if it is only in the nature of \emph{some} cognitions or types of cognitions to be presentative.

\

The first halves of vv.~35--6 having argued that validity and invalidity cannot both be intrinsic to cognitions, the latter halves of these sections argue that they cannot both be \emph{extrinsic} to cognitions either.\small \begin{quote} 35. \ldots \ Nor can both be due to something else, because in this latter case, there would be no definiteness in the conception.

\noindent 36. \ldots \ And when devoid of both these characters, of what form could the conception be?\end{quote}\normalsize it looks like Kum\=arila has here restricted his attention to individual cognitions, so if `being devoid of both characters' would mean that a cognition would be neither presentative nor representative, neither warranted nor unwarranted, then it's unsurprising that Kum\=arila would dismiss this possibility. This seems to be confirmed when we look at what Kum\=arila means by saying that these cognitions would lack `form' or `definiteness.' We may for example consider v.~168, which comes in the course of a discussion of sentence meaning and the importance of reliable speakers:\small \begin{quote}168. Though the meaning [of a sentence] may have been comprehended beforehand, yet it depends for its definiteness upon the fact of its originating directly from the [trustworthy] speaker's cognition; hence such a fact can only be comprehensible through the comprehension of the meaning. But in the matter of authenticity, it takes the first place.\end{quote}\normalsize This seems to say that while \emph{understanding} a sentence is simply a matter of sentence meaning, definiteness and authenticity (ie.~validity) are a matter of the source of the belief in question. In any case the ``definiteness'' of a cognition seems to depend somehow on the causal origin of that cognition. What would that mean? It seems plausible to suppose that Kum\=arila is gesturing at the difference between \emph{merely} having a thought for just whatever reason, and having a thought that is delivered by some belief-producing mechanism which lends it weight. (Specifically, for Kum\=arila, these will be cognition-producing mechanisms.\footnote{Cf.~\citet[ch.~1]{sen1984concept} and eg.~\emph{SV} IV.v.78.}) One might for example just entertain the thought ``Hirut is beautiful.'' This thought does not offer itself for acceptance or rejection---nothing \emph{asserts} Hirut's beauty, as it were. Alternatively one might \emph{read} that Hirut is beautiful, which in effect presents Hirut to you as beautiful. In this case the thought would be ``definite.''



%\footnote{BUT SPECIFICALLY THE WINDOW THING: MORE ABOUT DETERMINATENESS OF THE OBJECT OF COGNITION?---Cf.~\citet[ch.~1]{sen1984concept}. [[notice regarding eg detemrinateness worry, prob has to do with the idrect realism --- if you thought that what we had were appearances you wouldn't worry the same way. Anyway for example the Nyaya consider both extrinsic]] --- IV.v.78 seems to be relevant to the function question: basically to "specify" an object (as Jha translates)]]}

Considered form the perspective of warrant, the proposal is arguing against is that a cognition is warranted only if you can buttress it with some other cognition; and moreover the cognition is only unwarranted if it is undermined by some other cognition. Now suppose that were true. Now you are reading a newspaper, and you read that a certain Hirut is beautiful. And suppose you have no particular reason for doubting this or for distrusting the newspaper. But you can only rationally adopt the belief that Hirut is beautiful if you can ground it in some other cognition. But also you can only reject it if you can oppose it with some other cognition. But then in what sense could this report be offering itself to you for acceptance, if you could accept it only by checking somewhere else for support first, but at the same time can't rationally reject it either? Of course we could say that the newspaper is just offering you some things for consideration, but this would be exactly Kum\=arila's point: you might just as well have spent your morning daydreaming, if that's all reading the newspaper gets you.% So the view would would make a mockery of the very notion of reporting: it would make reporting like daydreaming.

\

The big question is whether validity is intrinsic and invalidity extrinsic, as Kum\=arila thinks, or the other way round. Verses 38--46 set out the position of Kum\=arila's opponent, and some of the opponent' reasons for saying that invalidity is intrinsic and validity extrinsic. One objection to Kum\=arila's view is offered in v.~40:\small \begin{quote}40.~``If authoritativeness were inherent or natural and its absence artificial, then dream-cognitions would be authoritative, self-supported; for what is there to refute this?''\end{quote}\normalsize This is nearly a restatement of the objection of v.~25 within the new dialectical context---we have shifted from the case of \emph{modes of knowledge} to the case of \emph{cognitions}. Kum\=arila had claimed that testimony (and thus Vedic testimony) is a source of knowledge on the ground that it produces cognitions, and in v.~25 the opponent says that cognitions are produced by many other (and plainly more problematic) mechanisms as well.% Here Kum\=arila is adopting the view that cognitions are intrinsically valid, and the opponent is observing that this will apply to cognitions from problematic sources.

Certainly from the perspective of warrant, the opponent's view is intuitive: the thought is that, given all the defective ways cognitions can be formed, one can be warranted in thinking something only if one can recognize that one came by that thought in some sound way. My confidence in what I think about this or that should be affected, for example, by whether I read it in the \emph{National Enquirer} or on Wikipedia or in the \emph{NYRB}, and that requires that I should have some appreciation of how my views originated.

But Kum\=arila's response is also intuitive:\small \begin{quote}49--51. If even on the birth of conception, the object thereof be not comprehended until the purity of its cause has been ascertained by other means, then in all cases we should have to wait for the production of another conception from a new source; for until its purity has been ascertained, the conception would be equal to nothing. And this second conception too, would be true only on the ascertainment of the purity of its cause; and so on and on, there would be no limit.\end{quote}\normalsize The opponent thinks that the soundness of your view should turn on your appreciation of the soundness of the origins of your view---but to this sort of verification there is no end, for how can you trust that you have indeed properly assessed the origins of your original view?\footnote{Cf.~\citet{alston1980level} on ``level confusions.''} Your assessment itself might be wrong, and indeed according the opponent the assessment itself should be in doubt until verified. At that rate we will never be entitled to confidence about anything.\footnote{cf.~also v.~56.}

Though the issue of warrant is obviously important here, Kum\=arila is more directly making a psychological point, as is especially evident from the two preceding verses.\small \begin{quote}47. You must understand that authoritativeness is inherent in all Means of Right Notion. For a faculty, by itself non-existing, cannot possibly be brought into existence by any other agency;

\noindent 48. since it is only for the sake of its birth that a positive entity requires a cause. And when it has once been born, it's application to its various effects proceeds naturally out of itself.\end{quote}\normalsize A valid cognition must of course have some causal basis, but once produced it provides comprehension of its object---that is what I take the reference to `facuty' and `application' to mean. Kum\=arila thinks that if the validity of a cognition---that, say, Hirut is beautiful---depends on some other cognition, then we would not after all have a comprehension of Hirut being beautiful until such a cognition is available. In line with the interpretation proposed in \S\ref{interpoutline} above, I understand the point here as follows. Suppose you pick up a copy of the \emph{New York Times.} In it you read a story about Ethiopia. The story is about a singer named Hirut and it reports that she is beautiful. Now suppose you know essentially nothing about the \emph{Times}, but you have no reason to doubt it or to to doubt it's reliability in this particular arena (compare asking someone on the street for the time). And in fact you come to believe, as a result of reading the story, that there is an Ethiopian singer named Hirut who is beautiful. In fact the \emph{Times} is reasonably reliable in these matters; the reporter is an honest and careful one, with sound aesthetic sensibilities. Now hasn't the \emph{Times} opened up a bit of the world for you? Doesn't your new belief give you a bit of a window on how things are, in particular with regard to Ethiopia and one Ethiopian singer named Hirut? This is what Kumarila claims is the case: a valid cognition is a window on the world, and nothing more than what has been described is required to open up that window.\footnote{Compare \citet[pp.~XX]{kripke1972naming} on Jonah.}

Why does Kumarila say that once it has originated, the cognition thereby acts as this window, and that nothing could make it act as a window if it couldn't do so of it's own accord? If we considered the matter in terms of \emph{thoughts} or \emph{beliefs}, this would seem to be a mistake. Suppose another man reads the story about Hirut, but only in a detached fashion, forming no beliefs, because he dismisses the \emph{Times} as a part of the reactionary old-media elite. The report that Hirut is beautiful opens no windows for him. But then he reads that the reporter in question supports some bold progressive cause like open borders, and thus decides he must be a trustworthy sort after all. He now remembers how this reporter said that a certain Hirut was beautiful, and now believes it. Doesn't the initial thought about Hirut now serve as a window for him, having been thus opened? Here we must remember that Kumarila's cognitions are not just the same as beliefs as we understand them---or as thoughts, depending on how those are understood. A cognition is anchored to its genetic origin---to the particular episode of receiving testimony or of perceiving that gave rise to the cognition---in a way that a belief is not. Or again, validity, unlike justification, cannot be provided later for a cognition that lacked it. A cognition either captures a view of the world or it doesn't, and if it doesn't nothing can change that later. You could later come by a new cognition that gives you what the first one might have, but that will just be a different cognition.

If the validity of a cognition depends on other cognitions, then we'll have a regress; Kum\=arila's view avoids that difficulty:\small\begin{quote}52.~In case, however, authoritativeness be accepted to be due to (the conception) itself, nothing else is wanted (for its cognition). Because in the absence of any cognition of discrepancies, falsity (unauthoritativeness) becomes precluded by itself (ie. without the help of any extraneous Means).\footnote{Cf.~v.~56: ``The fact of mere Unauthoritativeness being due to discrepancies does not lead to any \emph{regressus ad infinitum}---as is found to be the case with the theory of the cognition of excellences (being the cause of authoritativeness)---for us who hold the theory of ``self-evidence'' [\emph{svata\d h pr\=am\=a\d nya}].''}\end{quote}\normalsize Note that Jh\=a (rightly) glosses `falsity' as `unauthoritativeness' (ie.~invalidity). We return to `falsity' in connection with v.~54 below.

Verse 47 above said that validity is intrinsic to all \emph{valid} cognitions. But v.~53 has been cited by Taber to support the view that \emph{all} cognitions are intrinsically valid, and vv.~47 \& 53 have adduced by Tara Chatterjea to show that Kumarila's work contains ``contradictory suggestions'' concerning the scope of the \emph{svata\d h pr\=am\=a\d nya} thesis.\footnote{See \S2.3 above.}\small\begin{quote}53.~Therefore the authoritative character of a conception, cognised through the mere fact of its having the character of ``cognition,'' can be set aside only by the contrary nature of its object, or by the recognition of discrepancies in its cause.\end{quote}\normalsize But while the discussion (mine and Kum\=arila's) has thus far primarily concerned the first-order cognition, verses 52--53 connect up the first- and second-order elements of the intrinsic validity thesis. We remember that for Kum\=arila, a cognition is not---as it would be for other Indian philosophers, or for Descartes, say---itself an object in consciousness, though it manifests some other object to consciousness. If you are aware of a cognition as opposed to its object, that is because it is itself the object of a further cognition. And the \emph{svata\d h pr\=am\=a\d nya} thesis in this connection concerns the further question of the relationship between cognizing a cognition and cognizing that cognition as valid, and it amounts to the claim that a cognition, when apprehended by another cognition, is thereby apprehended \emph{as being valid}.\footnote{\citet[p.~3]{mohanty1989gangesa} offers a useful note on the difference between (and different positions on) grasping `\emph{prak\=a\'sa}' and grasping `\emph{pr\=am\=a\d nya},' and see \citet{sen1984concept} for a more extensive treatment. [SV CITES]} That is what v.~53 is asserting.

Why are cognitions apprehended as valid? Because to reflect upon a cognition just is to reflect upon the world which that cognition reveals; the reflection presupposes that the initial cognition constitutes your view of the world. It would be possible, of course, to say that reflection does throw open the matter---that to reflect on a belief is to step back from it, and to now require some new basis for accepting it, if one is to continue to accept it. So suppose that, upon recognizing and attending to the belief that Hirut is beautiful, you were to apprehend it as not valid---whether actually invalid or as somehow neither valid nor invalid. Then that very second-order cognition would undermine the first cognition. Even granting that, up to that point, the initial cognition was valid, there would now be a new cognition to the effect that the first cognition was invalid or false or doubtful or whatever the case may be. Of course this new cognition might itself be invalid, but, by Kumarila's own theory, that can only be so if some other cognition throws some some specific doubt on \emph{it}. And since there are generally going to be legitimate ways of coming to recognize one's own cognitions, this condition will often not hold. So reflection will be a way of undermining your cognitions, except to the extent that you are able to identify some support for them. In that case the two forms of the \emph{svata\d h pr\=am\=a\d nya} thesis match nicely, and both are supported by the appeal to the threat of infinite regress.% If validity were not intrinsic, then cognitions could never be valid, because each cognition would (\emph{ex hypothesi}) have to confirmed or supported by a further cognition. And if in reflection one could not see one's cognitions as valid, one would again have to identify an unending chain of supporting cognitions in order to see any cognition as valid.

In v.~54 Kumarila identifies several varieties of invalidity:\small\begin{quote}54.~Unauthoritativeness is three-fold: as being due to falsity, non-perception, and doubt. From among these, two (falsity and doubt) being positive entities, are brought about by discrepancies in the cause.\end{quote}\normalsize With the reference to `positive entities,' Kum\=arila responds to qualms of a pan-Indian variety about what sorts of facts or entities ought to be explained in what way; that need not detain us.\footnote{These were mooted by the opponent at vv.~38--9.} The reference to `discrepancies' is more important: the proponent of intrinsic invalidity will claim that a cognition is rendered valid when the mechanisms which produced it are seen to have been sound, whereas Kum\=arila claims that a cognition is shown invalid when the mechanism is shown to have been defective, i.e.~to have had `discrepancies.' In v.~54 Kum\=arila is thus cataloguing the ways in which a cognition can be rendered invalid and saying that two of these ways are indeed reducible to the the identification of discrepancies.\footnote{Notice it's not enough for there to \emph{be} discrepancies---they must be turned up somehow. Otherwise those cognitions would not have been valid in the first place, and Kum\=arila is claiming that they were.} (The third, non-perception, will be a special case.)

By `falsity,' Kumarila means not the actual falsity or inaccuracy of the initial cognition but rather the case in which some such inaccuracy is turned up, or in which some defect in the mechanism behind the cognition is turned up, so that the initial cognition is altogether overturned. The case of doubt requires not so much an overturning as an unsettling; thus in a stock Indian example you might be unsure whether you're looking at (or saw) a piece of silver or rather a bit of shell. Kum\=arila is considering these cases in vv.~57--61:

\small\begin{quote}57.~Unauthoritativeness (falsity) is got at directly through the ``Cognition (of its contradictory).'' For, so long as the former is not set aside, the subsequent cognition (of its contradictory) cannot be produced.

\noindent 58.~Though the cognition of the discrepancy of the cause is known to refer to a different object (\emph{i.e.}, not the object which is the effect of tlie cause), yet we have co-objectivity (of the two cognitions) as being implied thereby; and hence we have the preclusion of the former,---as in the case of the ``milking-pot.''\footnote{Kum\=arila adduces an example from ritual: Jh\=a \emph{ad. loc.}}

\noindent 59.~But this rule applies only to those cases in which (with regard to the second conception) there is neither cognition of any discrepancy, nor any contradictory conception. In those cases, however, in which we have any of these two factors, the second conception becoming false, the first comes to be true. 

\noindent 60.~But in that case too, the authoritativeness is due to the conception itself, in the absence of any cognition of discrepancies. And in a case where there is no such cognition of discrepancies, there is no reasonable ground for doubt. 

\noindent 61.~Thus (in this manner) we do not stand in need of postulating more than three or four conceptions. And it is for this reason that we hold to the doctrine of ``Self-evidence'' [i.e.~\emph{svata\d h pr\=am\=a\d nya}].\end{quote}\normalsize The `falsity' case well illustrates one form of the extrinsic nature of invalidity. It's in having the \emph{new} cognition that the earlier one is set aside and rendered invalid. (And should the new cognition itself prove faulty, the original cognition is valid again---not because it's gained new support but because the objections have been lifted---thus the validity of a valid cognition is intrinsic even if it loses and regains that status.\footnote{vv.~59--60.} This is not just a psychological point. It may be that, as a psychological matter, it's impossible to credit both the old idea that Hirut is beautiful and the new idea that she's plain, but as a psychological matter one could reject the new information rather than the old. But for Kumarila, the new cognition, is, barring some further problem with \emph{it}, itself intrinsically valid, just as the old one had been valid up to this point. The process of discarding the old way of looking at things is, moreover, an inferential process, not merely a causal one, and this is clearest in the case where a cognition is not just straightforwardly at odds with the earlier cognition in terms of its content, but where the new cognition apprehends something problematic about the causal basis of the original cognition, which is the case Kumarila considers here:\small\begin{quote}58. Though the cognition of the discrepancy of the cause is known to refer to a different object (i.e. not the object which is the effect of the cause), yet we have co-objectivity (of the two cognitions) as being implied thereby; and hence we have the preclusion of the former,---as in the case of the ``milking-pot.''\footnote{With the reference to the milking-pot Kum\=arila adduces an example from ritual to illustrate his point: Jh\=a \emph{ad. loc.}}\end{quote}\normalsize For this case we could imagine that rather than seeing Hirut and finding her plain, you learn that the \emph{Times} reporter in question has abnormal aesthetic sensibilities. Finally, v.~60 above was explicit that there is a question of rationality here: it would be a \emph{mistake} to give up your view of Hirut's beauty barring some specific reason for doing so.

When in the case of `falsity' a cognition is set aside, it is, in a sense, not really even apt for validity any more, since it is no longer constitutes one's view of the world anyway. Likewise with the case of doubt; the view has been suspended in a way that would make it inappropriate to ask, for example, whether the subject is justified in holding it.\footnote{A doubt carries no commitment in any direction in the first place---this point fits with my earlier discussion where I supposed that validity presupposses \emph{definiteness} as I interpreted that term; cf.~\citet[pp.~93--5]{bhatt1962epistemology}.} The same point would seem to hold doubly for the case of non-perception, the case in which you were simply not in contact with the world at all, as might be the case in dreams or the aforementioned case of `fancy,' or wishful thinking.\small\begin{quote}55. In the case of non-perception, however, we do not admit the action of such discrepancies. Because for us all non-perception is due to the \emph{absence} of cause,---just as you have asserted.\end{quote}\normalsize This case important because it illustrates a claim I've made a couple of times---and in doing so address the earlier objections about dream cognitions and so on---namely that cognition must have a certain causal origin if it was ever to have validity in the first place. `Fancy' and wishful thinking, for example, are not sources of valid cognition,\footnote{vv.~94--5; cf.~v.~128.}, but only special cases of non-perception. We remember that v.~48 had said that it was only for its birth that a positive entity required a cause---here there is no (sound) cause in the first place, and no positive entity (ie. no valid cognition). This implicitly offers a response to the dream objection of v.~40.

%---cognitions without a source in a valid source of knowledge aren't viable candidates for validity in the first place.

%That means that Kum\=arila thinks that you are justified in believing something so long as you believe it on the basis of some approved means of knowledge such as testimony, or perception, \emph{whether or not} the specific instance of testimony or perception is sound (unless of course you can actually see that it isn't sound). 

%Other psychological basis for cognitions are merely 

%[[MAYBE us this P to connect above P with the whole issue of independence etc.]] 

The non-perception case also allows us to appreciate more clearly that validity cannot be anything like \emph{truth.} If I daydream the beautiful Hirut into existence, then later forget it was a daydream and actually come to believe that there was a beautiful Ethiopian singer named Hirut, the belief may or may not in fact be \emph{true}; Ethiopia has it's singers, Hirut is a reasonably common name there, and it is a land of beautiful women; some woman may very well be all of those things. But there is no question or possibility of \emph{validity} here. Validity requires a causal basis (such as testimony or perception) that can (though may not always) mediate some kind of contact between you and the word. If there's no actual chain between the world and your thoughts about it, you can't have valid cognitions about it, however accurate your speculations might be.% (Remember too that a cognition is anchored to it's origin; new causal bases can be formed between a \emph{belief} and the world, but not so in the case of cognitions. Hence Kumarila's claim---noted earlier---that a cognition cannot be rendered valid by some other cognition, but must rather have that potential in it's own right, if it is ever to be valid at all.) 

\

Kum\=arila applies his conclusions to the case of testimony:\small\begin{quote}64.~In (truthful) human (speech) we find two (factors)---\emph{absence of discrepancies,} and (\emph{presence of}) \emph{excellence}; and we have already explained that authoritativeness cannot be due to \emph{excellence}.

\noindent 65--6.~Therefore excellences must be held to help only in the removal of discrepancies; and from the absence of these latter (discrepancies), proceeds the absence of the two kinds of unauthoritativeness; and thus the fact of (authoritativeness) being inherent in Words remains untouched. And inasmuch as the word gives rise to a conception, its authoritativeness is secured.\end{quote}\normalsize When Kum\=arila says it has already been explained that validity cannot be due to the excellences of testimony, he must mean that a cognition is not valid in virtue of \emph{recognizing} that the source of knowledge, in this case testimony, was free of fault. He exploits this point to respond to the objection that his own view does face a regress, despite earlier protestations:\small\begin{quote}66. ``If the absence of discrepancies be held to result from excellences, then there is the same \emph{regressus ad infinitum} (that you urged against us.''

\noindent 67. (Not so): because at that time (ie. at the time of the conception of the absence of discrepancies), we do not admit of any active functioning of the excellences, though they continue to be recognised all the same;---because in the conception of the absence of discrepancies they help by their mere presence.\end{quote}\normalsize As we have seen before, one need not actually have an appreciation of the soundness of the basis of a conception for it to be valid. There is also a slight puzzle here, though, because v.~67 seems to suggest that in fact a valid cognition \emph{must} have a \emph{sound} basis. But that cannot be right, because it would mean that testimonial cognitions are not after all valid just as such, and as we can see even here (from the the last sentence of vv.~65--6) validity \emph{does not} depend on the actual faultlessness of the testimony in question: testimony just as such \emph{is} a source of valid cognitions. I therefore take v.~67 to mean that the \emph{presumed} excellences of the source of a cognition suffice (so far as the source is of an approved variety: testimony, perception, etc.).

This is confirmed by the next verses, as we turn to the case of the Vedas, where Kum\=arila claims that the very testimonial nature of Vedic-derived cognitions imbues them with validity:\small\begin{quote}70.~Hence the mere fact of the Veda not having been composed by an authoritative author, ceases to be a discrepancy. \ldots 

\noindent 71. It is only human speech that depends for its authority upon another Mean of Right Knowlede; and hence in the absence of the latter, the former becomes faulty; but the other (ie. Vedic sentences) can never be so (on that ground).\end{quote}\normalsize We have returned here to the second-order case: testimonial cognitions may be presumed sound just on the basis of being testimonial cognitions, though in the the typical or human case, where testimonial knowledge is parasitic on other sources of knowledge, that presumption would be overruled by the observation that the testifier had no way of coming by his supposed knowledge. In the case of the Vedas, so such difficulty can arrive, and the presumption stands.

Verse 72 returns us to the matter of independence:\small\begin{quote}72. Thus then, the very fact of the incompatibility of the Veda with other Means of Right Notion, constitutes its authoritativeness; for if it were not so incompatible, it would only be subsidiary (to other such means).\end{quote}\normalsize The point here is that Vedic testimony produces cognitions---which are intrinsically valid---which could never be called into question by any other source of knowledge, the Vedas are an authority: they produce cognitions which must be trusted. 

This may seem to reopen our question about dreams and fancies. Perhaps you have dreams about the future, and come to have beliefs about the future in that way, and these beliefs cannot be undermined by perception or inference. Does that make dreaming authoritative concerning the future? Kum\=arila does not think so, but not, I think, for deep reasons of epistemological theory proper; it is just that dreaming and Vedic testimony are different cases. On Kum\=arila's view, dreams and fancies just take the matter of perception and mix it up in arbitrary ways; they are not new sources of information. One could, of course, say the same about the Vedas---but Kum\=arila wouldn't agree. He thinks that language and meaning are deep and primitive aspects of the cosmos, that the cosmos is eternal, that it's reasonable to suppose that the Vedas were never written, and that they tap into eternal meanings somehow. Those are matters separate from epistemology, nor does Kum\=arila suppose himself to have defended any of those claims at this stage.\footnote{Verse 70 in its entirety runs thus: ``Hence the mere fact of the Veda not having been composed by an authoritative author, ceases to be a discrepancy. Of the syllogistic arguments urged against us, we shall lay down counter-arguments hereafter.''}

We pass over vv.~73--81, which continue discussing the independence of Vedas, and arrive at an objection in v.~82:\small\begin{quote}82.~(Objection): ``But Sense-Perception and the rest are not comprehended as that `these are authoritative'; nor is it possible to carry on any business by means of such perceptions, when they are not comprehended as such.''\end{quote}\normalsize The opponent objects that since in forming cognitions we do not yet have any confidence in those cognitions, they cannot be a basis for action or further inference. In responding, Kum\=arila provides a helpful recapitulation of his view.\small\begin{quote} 83. (Reply): Even prior to comprehension, the Means of Right Notion had an independent existence of their own and they come to be comprehended subsequently (as such), through other cognitions.\end{quote}\normalsize A cognition is valid and serves as a basis for action and inference even without any appreciating its existence; when its existence is appreciated, this is in a new cognition.\small\begin{quote} 84.~Therefore the fact of its being comprehended as such, does not in any way help the authoritativeness (of the Means of Right Notion); because the idea of the object is got at through the former alone.\end{quote}\normalsize The second-order recognition of a cognition as valid is not necessary for the validity of that cognition, nor can it confer validity upon it.\small\begin{quote}85. Even the unauthoritative Means would, by itself, lead to the conception of its object; and its function would not cease unless its falsity were ascertained by other means.\end{quote}\normalsize Even an invalid cognition manifests the world, though it may do so wrongly (it may for example reveal a real shell only as a piece of silver\footnote{Jh\=a \emph{ad loc.}}), but until it is revealed as misleading it is accepted, and forms a basis for further inference. Notice here that a cognition can be invalid but still manifest the world; hence validity cannot be manifestation. We should also see here a connection to action: so far as v.~85 itself goes, we might think that the `function' of a cognitions was restricted to manifestation, and that may indeed be the primary meaning, but in the context of the objection of the objection at v.~82, Kum\=arila evidently means that, in virtue of the exercising its function, a cognition (even an invalid one) functions as a basis for action.

Finally we end with a reiteration of the intrinsicality of validity and the extrinsicality of invalidity:\small\begin{quote}86.~The falsity of an object is not, like its truthfulness, perceived by its very first conception. For the recognition of unauthoritativeness, the only cause is one's consciousness of the falsity of its subject itself, or of he faultiness of the cause thereof.

\noindent 87.~Thereby alone is falsity (of a conception) established; and by no other means. And the truthfulness (or authoritativeness of a conception) is proved to \emph{belong to the state of its birth} (\emph{i.e.}, is natural or inherent in it).

\noindent 88.~Therefore even in cases where falsity is proved by other means, these two (causes of falsity) should be noted, and not only certain points of similarity (with another false idea).\end{quote}\normalsize 




%But the point is also, I think, that this belief is a basis for action, and this is because the original cognition gives you a view of the world on which to act, without any further assessment of that view. 
%	- connect with reasonableness of action

%Consider the next verse:

%`Truth' and `falsity' cannot, presumably, mean actual truth or falsity in a correspondence sense---the point seems rather to be the one that Taber had urged: a belief gives you an outlook on the world, and you don't need some further view \emph{about} that view before this becomes true. This gives us the needed response to the objection in v. 82, because a belief just is a basis for action.





%RELEVANT QUOTES FROM SLIV

%32. Just as ordinary Fancy, independently of Sense-perception and the other (Means of Right Notion), is not able to give rise to any definite idea, so also would be the Imagination (or Intuition) of the Yogi. 

%34. Specially, because, Duty is not perceptible, prior to its performance ; and even when it has been performed, it is not perceptible, in the character of the means of accomplishing particular results, 

%[[DOES THIS SUGGEST THAT DEFECTIVE CIGS JUST DON'T COUNT?]] 38-39. The word " >Sam " is used in the sense of " proper (or right) " ; and it serves to preclude all faulty ' prayoga.' And by " prayoga " is here meant the "functioning" of the senses with reference to their objects. In the case of the perception of silver in mother-o-peavl, the functioning of the Sense-organ i« faulty ; and hence such perceptions become piecluded (by the prefix ' Sam ' ). 

%53-54. By " cognition-production " is meant that " cognition becomes authoritative as soon as it is produced." In the case of all causes, we find that their operation is something apart from their birth (or manifestation). In order to preclude such character from the Means of Right Notion (Cognition), the word " production " has been added. 

%55. Not even for a moment does the cognition continue to exist ; nor is it ever produced as doubtful (or incorrect) ; and as such, it can never subsequently operate towards the apprehension of objects, like the Senses, \&c. 

%56. Therefore the only operation of Cognition, with regard to the objects, consists in its being produced ; that alone is Right Notion (Prama) ; and tbe cognition itself as accompanied by this Right Notion is tlie Means (of Right Notion: Pramana).

%60. The Means of Right Notion may be (1) either the sense, or (2) the contact of the sense with the object, or (3) the contact of the mind with the senses, or (4) the connection of the mind with the Soul, or (5) all these, collectively. 

%61. In all cases, cognition alone would be the Result ; and the character of the Means would belong to the foregoing, on account of their operating (towards cognition) ; for when there is no operation of these, then the Result, in the shape of cognition is not brought about. 

%71-72. When the object of cognition is the qualification itself, then the abstract (or undefined) perception subsequently gives rise to a definite cognition; and in this case the character of Pramana belongs to the undefined Perception, and that of the Result, to the subsequent definite (or concrete) cognition. 

%**** 72. When, however, there is no definite cognition, then the char- acter of Pramana could not belong (to the foregoing undefined perception); because of its not bringing about any definite idea with regard to any object. ****

%***** 78. The result being the specification of the object, the character of Pramana belongs, according to us, to that which immediately precedes it ; and so, if the cognition be said to be the Pramana, then the Result must be held to be something else. ****	FROM 71--2 I TAKE THE RESULT TO BE A COGNITION. THE FIRST HALF THEN MEANS THAT A A PRAMANA AND A COGITION ARE ABOUT SPECIFYING AN ACTUAL THING.






%----------------------
%4. KUMARILA'S CONCEPTION OF KNOWLEDGE
%----------------------

\section{Kum\=arila's Conception of Knowledge}


%[[WILL MAKE COMPARISONS WITH CONTEMPORARY PHIL FREQUENTLY --- BEG NOT TO ASSUME MORE THAN WHAT I NOTE, AND DON'T START POSING THOSE QUESTIONS. CONTEMPORARY EPIST IS WONDERFULLY SOPHISTICATED BUT NARROW IN HISTORICAL TERMS ---LIJKEWISE MENTIONS OF HISTORICAL STUFF SHOULD NOT BE TAKEN TOO FAR; AIM IS TO USE THE FAMILIAR AS AN AID TO UNDERSTANDING THE FOREIN, AND CERTANLY NOT AS A STANDARD OF VALUE]]\footnote{Thus Robert Audi, in a recent introduction to epistemology \citep[p.~1]{huemer2002ecr}, can say flatly that ``knowledge is constituted by belief (of a certain kind).''}

%[[INTERESTING RE QUESTION OF CERTAINTY THAT KUMARILA TRADES ONE VARIETY OF CONFIDENCE FOR ANOTHER---CAN'T HAVE PROOF SO GOES WITH INHERENT WARRANT --- THAT HELPS US RESPOND TO A CONCERN STEMMING ROM EG CHATTERJEA'S CLAIM THAT CERTAINTY IS A PART OF THE PICTURE --- ALSO AGAIN A COMPARISON WITH PHILO THERE.]]

%[[BUT I'M STILL INTERESTED IN UNDERSTANDING AND NOT SELLING THE VIEW]] - if I were then: Bonjour's comment about the radical departure epistemology makes... - though of course it arguably goes even further back, perhaps eg to Philo]] 


%\subsection{PHILO COMPARISON}


Taber compares the intrinsic validity thesis with Descartes' idea that some ideas are ``clear and distinct,'' so that their truth is self-evident; and again he compares Kum\=arila's thesis to the Stoic notion of [phantasia kataleptike], or ``kataleptic impressions,'' impressions or appearances which contain some mark guaranteeing their truth.\footnote{p.~217.} Taber points out that unlike kataleptic impressions or clear and distinct ideas, intrinsically valid cognitions are not uncommon; on Taber's interpretation all cognitions are at least initially intrinsically valid. The more important disanalogy, which Taber must himself insist on, but nevertheless seems really to \emph{spoil} the analogy, is that intrinsically valid cognitions needn't be true, whereas the whole point of the Stoic and Cartesian notions is to isolate a \emph{special class} of impressions or ideas which have a special warrant. J.~N.~Mohanty is right in saying that the whole spirit of Kum\=arila's view is the rejection of the notion of self-evidence; as he puts it,\small\begin{quote}it is precisely the contention of the \emph{svata\d hpr\=am\=a\d nya} theory\ldots that \emph{there is no criterion of truth, though there are criteria of error.}\footnote{\citet[p.~22]{mohanty1989gangesa}, his emphasis.}\end{quote}\normalsize 

Taber's comparison with Reid's commmon-sense philosophy is probably more apt;\footnote{pp.~217--18.} and I think that Philo would be a better Hellenistic analogue of Kum\=arila than the Stoics. According to what seems to be the dominant reconstruction, Philo, who led the Academy for some time, was first of all a radical skeptic in the manner of Clitomachus, later adopted the mitigated skepticism of Metrodorus, and finally in his (unfortunately not extant) `Roman Books' advocated a sort of fallibilism.\footnote{\citet{frede-sceptic}, \citet{barnes1989antiochus}, \citet{striker1997academics}, \citet{brittain2001philo}; see \citet{sep-philo-larissa} for a survey.} The Academics seem to have generally accepted (if only for the purposes of launching counter-arguments) the Stoic understanding of knowledge, which required the afore-mentioned \emph{phantasia kataleptike}: knowledge requires an accurate impression appropriately caused by what it represents, and which contains some mark which shows that it meets those two conditions. The Academics argued that there were no such impressions, since any true impression could be mimicked by a false one. The skeptics view, like the Stoics, thought that one ought not to assent to what one does not know; thus a radical skeptic would ideally refuse ever to assent to anything. A mitigated skepticism, on the other hand, would allow the reasonableness of forming some beliefs even without allowing the existence of a criterion of truth; some appearances might at least constitute good, if never conclusive, evidence. But Philo ultimately redefined the notion of a kataleptic impression, accepting the requirements of truth and of a suitable causal basis, but removing the requirement that such an impression guarantee its own truth, which he still considered to be an impossible standard to meet. This meant that it would not only be reasonable to form beliefs but that kataleptic impressions \emph{were} after all possible, and one might rationally accept these without qualificaton. But since the requirement for a criterion of truth had been abandoned, one might very well be mistaken about whether one actually possessed a kataleptc impression.

Philo's innovation provoked a good deal of hostility. He had, in effect, given up the idea that knowledge must be certain, something assumed by dogmatists and skeptics alike. Kum\=arila's validity does not require truth, but, like Philo, he decided that the reigning epistemic standard was impossible for any cognition to achieve, and thus redefined it so that it could be claimed as a matter of course. As with Philo, this was met by complaints from outside his own school and also with disbelief from inside it. Thus a latter commentator, Umbeka, found a way interpret Kum\=arila as making validity entail truth after all. In India, as in the Hellenistic world, knowledge was supposed to be certain. Kum\=arila's innovation, like Philo's, lies precisely in giving up that requirement as in any case impossible.

In both cases, the reinterpretation of knowledge is at least in part a response to the demands of practical activity. A major criticism of the radically skeptical position was that it made action and virtue impossible, since both require belief. Efforts were made to reconcile action with the skeptics' traditional refusal to assent, but the mitigated skeptical position amounted to a way of allowing for assent without allowing for knowledge or certainty. Since the mitigated position winds up buying into dogmatic conceptions of knowledge in a way that radical skeptics could claim to avoid, the Philonian position can be understood as an attempt to just find a new understanding of knowledge that was a better fit with the practical considerations driving the mitigated skeptics' concessions in the first place.

Likewise Kum\=arila reinvents the notion of knowledge to allow for the demands of practice. Moral or \emph{dharmic} knowledge will be impossible without the authority of Vedic texts---the Buddhists, for example, are wrong to suppose that perception will suffice. But Vedic testimony cannot meet the standards imposed by the epistemological theories of other schools, which would require either a kind of self-evidence of which no cognition is capable, or verification which is in this case impossible. How then would one pursue \emph{dharma}? Human beings must act, and they must have moral guidance. It cannot be unreasonable to act as best you can, relying in the best resources available, even if the soundness of these resources cannot be established. That is part of what it is to be reasonable, and the standards of theoretical reason should not say otherwise.




%\

%maybe go back to the worry i had about why saying testimony is legit - maybe cause how else are we to get by?


%note probably not the case that there will always be a mark of invalidity either. Nor indeed really any answer at all to a question like how do we establish validity or invalidity

%Taber 217 - the bigger disanalogy, I'd think, is that valid cognitions needn't be true, unlike with Stoics or Descartes

%remember that Philo's fallibilism seems to have been shocking to all academics, but also v different from just tons of people all together who seemed to generally agree with Socrates that its shameful to believe without knowledge, and who aimed for certainty basically
%	notice also that no reason for doubt needn't entail certainty in a big sophisticated sense...

%a diff from Philo would be perhaps the leaving out of the truth requirement for the good status of interest - 

%also openness to correction....

%two truths vs sceptics trying to distinguish different kinds of acceptance or whatever\footnote{[Nagarjuna for instance \citet{nagarjuna1998dialectical}; See the essays in \citet{burnyeat1997original}.]]}


%\


%WE DON'T START WITH APPEARANCES - THINK THAT'S WORTH POINTING OUT

%Seems unfortunate for Taber to say that the Mimasas stick strictly to ``appearances'' --- something like a disjunctivist model would be more appealing here (look into Travis here again)

%Taber talks about our ``inability to distinguish true from false conceptions'' on K's view - p.215
%	looks like here I would hope to get Kumarila's other book: TS 2904 - regarding this also look at 80 and 52

%OK: HERE'S MY APPROACH: INSTEAD OF METAETHICS, I CAN SHOW HOW KNOWLEDGE GETS MODELLED AFTER SORT OF PRACTICAL STUFF

%NOVELTY CONDITION SEEMS TO STEM FROM PRAGMATIC INTEREST/VEDIC INJUNCTION SETTING. (KATAOKA)
%	72 relevant on novelty

%AND REMEMBER THAT POLLOCK EXPLAINS HOW ALL KNOWLEDGE COMES TO BE MODELED ON VEDIC KNOWLEDGE

%THE PRAGMATIC ELEMENT IS WHAT HELPS US TO GET A GRIP ONTHE NORMATIVE SIDE --- WHAT MAKES IT REASONABLE TO ACT ON THESE BELIEFS, AND WHAT MAKES VALIDITY MORE THAN A PSYCHOLOGICAL ISSUE

%NOT INTERESTED APPARENTLY JUST IN FACTS

%on novelty check autpattika 12ab as well and kataoka (with Umbeka) and Franco 1997 (with sucarita perhaps?) - disagrement seems to be about whether novelty here is just in teh cntext if speech - but even taken in that con text it looks to me like the reasoning is is precisely that pramanya has to be novel. ---yeah, basically there seems to be an issue about novelty in SK as opposed to BK (Kataoka p96 and n21, Franco p62 and n.38,  but no problem. --- also sounds like Kumarila works through each pramanya explaining how it gives new stuff. Also I think 80 may have the character of a definition in a way, but may also be about the second-order case. And that would explain why memory is left out: you might not know whether a cognition was memorial

%also anyway have to look at the brhattika --- will have to look into whether there are any epistemological disgareements between that book and Slok.

%Reconsider Potter a bit?

%ALOS COME BACK TO THE AUTHIRITY OF THE VEDAS --- IF THERE'S REALLY NO OTHER WAY OF KNOWING ABOUT DHARMA, THEN MIGHT IT NOT INDEED BE REASONABLE TO ACCEPT THE VEDIC INJUNCTIONS, *REMEMEBERING THAT WE'RE IN PART INTERESTED IN THE REASONABLENESS OF ACTING ON THIS STUFF*
%	could also compare philo and eg Socrates and plato on bending epistemology to the needs of practice --- or of moral epistemology




%\subsection{VERY GENERAL THOUGHTS}


%it is, I take it, about refusing the skeptical worries altogther - not about answering them.

%no wonder the ancient skeptics could never say ``it is this way'' or ``it isn't this way'' - they had to always step back from their beliefs.

%can't drive a wedge between how things are and what you think - believing isn't a matter of looking in, which is unproblematic, as opposed to looking out which is hard

%a window you use to look through the first window

%would be a bit like the cartesian case where you both see the world and the belief, except for K they can come apart.

%of course nothing bars Kujmarila from saying that generally you can get some idea of the origins of your beliefs. This isn't some wild skeptical case in question. --- connect this with Mohanty's point about self-evidence, perhaps

%generally is, I think, how assessment work: after all you wouldn't be contemplating the belief as a belief if you still didn't look througfh it, as it were - you look out to look in. That means still supposing it's worth something. Maybe look at some self-knowledge stuff on this. In other words reflection may mean stepping back from teh belief in a sense, but it doesn't mean stepping back from believing. just have to be careful how divy up validity over the two cogs. 

%Kind of like Bhatt's way of putting it (p73): we come to distinguish truth and falsity because stuff doesn't work out. Needn't sit there worrying before hand

%Bhatt 139: ultimately your evidence has to some down to internal stuff --- at this point he seems to be talking about proof and evidnce, so I'm finding this a bit odd... but maybe that's srt of right - look this is our evidence...









%\subsection{SOME PROBLEMS}


%WILL HAVE TO ALLOW THAT MY INTERP HAS A BIT OF A PROBLEM OF THE SORT I ATTRIBUTED TO TABER'S INTERP, NAMELY MAKING TOO MANY COGNITIONS VALID --- STILL I THINK A HUGE IMPROVEMENT DESPITE BEING CONSISTENT WITH THE TEXT --- NOTICE THIS IS A PROBLEM WITH EXTERNALISM IN GENERAL.









%discuss keith point that Kumarila doesn't consider ths possibility of taking truths as a system etc. But here I think we don't nec want to think that way - the point is to a greater extent that that suggests, to get direct cognition, not merely justification




%\subsection{POTENTIAL CONNECTIONS WITH CONTEMP PHIL}

%going to borrow from the sen book a bit --- compare disjunctivism. in this case a bit diff - not that sometimes there's an appearnace, excpet in sense that memory or whatver revelas some misalgned thing somehow etc...

%might want to bring in disjunctivism in the discussion
%	with disjunctivism goes another natural isea: that a belief purports in teh first place to constitute a window. the bad, but defective case is the ``it wasn't a window'' case.

%evidently part of the justificatory worK is moved into to the sort direct realism and denial of self-illumination

%discuss Taber 214 - does seem this becomes an issue of what a judgement is, and what maybe perception say delivers to us (compare eg. Travis' SIlence of the Sense for some discussion)

%\

%the internalism-externalism examples of eg extra-sensory perception

%externalism

%\

%level confusions

%[[[suspecting that a lot of the stuff from around 100 may be turning more to this other question about checking - perhaps I could say giving an explanation or justification? - rather than the original state. I hadn't separated this enough before, I suspect. So I'm going to need to work that in as another aspect of this whole thing.]]

%\

%specification of the sources

%\

%various sort of good qualities of beliefs; seems to me that Kumaril is picking up on one important feature...

%\

%an issue about testimony as opposed to the particular episode

%\

%what about example of signs evolved to the level of language; or the fact that a sentence can means omething even if someone doesn't mean to say that thing.... 

%\

%perhaps a dicscussion of relationship to knowledge in our sense? cf Taber's sabdapramanya article p 169 - and esp 171

%check on lit that downplays the importance of knowledge - maybe in fact some externalists? check maybe the goldman SEP article and other SEP articles.

%ALSO ISSUE OF KNOWLEDGE AND BELIEF AS DIFFERENT MENTAL STATES---I COULD THEN TIE THIS IN WITH TRAVIS, WILLIAMSON, AND PLATO

%I suppose Arnold's bizarre discussion does show that I ought to say somthing about the modes of knowledge...

%Also re Arnold notice that he's trying to get foundational stuff out of this and I think it's probably just not there.

%interesting comparison from n.23 of Taber's sabda article: (from p169, seems this should correspond to valid cognition, at least in the testimonial case - but different terminology is used.)

%24. To what does the second level of awareness correspond in Western philosophy of language? I would suggest: belief, of the sort that Thomas Reid, e.g., holds arises from verbal testimony. We are naturally inclined, writes Reid, to accept and not merely consider, what others tell us. A statement uttered by another person, unless there is some question about his or her reliability, naturally tends to cause in us a judgement that matters are as he or she says they are. Reid refers to this as the “credulity principle,” and maintains that without it we would not survive from infancy to adulthood. See Inquiry into the Human Mind on the Principles of Common Sense, ed. Keith Lehrer and Ronald E. Beanblossom (Indianapolis: Hackett Publishing, 1983), pp. 95–97.


%Taber (sabda, p171): Moving on to Mohanty’s first criticism of [sabdapramana], viz., that it is unlikely that words just by themselves, independently of other [pramanas] like perception and inference, can cause knowledge (in the Western sense) of actual states of affairs – I would like to suggest that M¯ım¯am. s¯a philosophers, at least – Bh¯at.t.a and Pr¯abh¯akara alike – never thought that language does cause knowledge (invariably) in that sense. This may strike some scholars as a rather dubious claim, but I am fairly confident about it. For what is a pram an. a according to M¯ım¯am. s¯a; and what is it to say that the Veda, in particular, is a pram ¯ an. a in regard to things that the senses and reason cannot access? I think that it is clear from the early M¯ım¯am. s¯a texts, especially the ´ Slokav ¯ arttika (Codan ¯ as ¯ utr ¯ adhikaran. a), that a pram ¯ an. a is a cognition, or a karan. a or means that produces a cognition, that prima facie appears true. When Kum¯arila says that every cognition is intrinsically valid (note that he says, or at least implies, that this is the case for every determinate cognition, not just true cognitions!27 ), what he means is that every cognition arises accompanied by a sense of its own truth. Perceptual cognitions, in particular, are self-warranted, but not infallible. If a cognition is refuted by another (intrinsically valid) cognition, or undermined by the realization that it was produced by a defective cause (a damaged cognitive mechanism or an object presented under non-ideal conditions), then its intrinsic validity is removed. That is why the doctrine of intrinsic validity for the Bh¯at.t.a does not at all entail that there are no false cognitions – that objection, raised by classical thinkers and modern scholars alike against the Bh¯at.t.a svatah. pr ¯ am ¯ an. ya-v ¯ ada, is simply based on a misunderstanding.28 

%Taber (sabda, p. 172): must certainly involve some kind of judgement or belief that a certain state of affairs exists, hence be a ni ´ scaya32 – can be overturned.33 We do distinctly judge that “This is silver”, yet this judgement can easily turn out to be false. Thus, the intrinsic validity of cognitions is vested for the M¯ım¯am. saka in the sense of conviction with which they typically arise, which is to say, their truth for us; both schools agree that the validity of a cognition is not established by confirming it, by determining that it corresponds to reality.34 However, the longer a cognition goes undefeated, the more it seems that it is not merely true for us, but really true – that is, for all practical purposes, it is true. And that is how the Veda, which is an authorless discourse, thus language by itself, can be seen to be true for the M¯ım¯am. saka; for the sentences of the Veda produce in us cognitions which, like all cognitions, have intrinsic validity. As ´ S¯alikan¯atha puts it, they give rise to ni ´ scay ¯ ah. (v. 49). Since it is impossible for those statements ever to be overturned by other pram ¯ an. as – for they concern matters that none of the other pram ¯ an. as can tell us about – or that one will ever ascertain a defect in their author – for the Veda is eternal, without any author – the cognitions to which they give rise will forever retain their intrinsic validity; they will always seem true. Thus, for all practical purposes, they really are true.35 Mohanty’s first criticism of ´ sabdapram ¯ an. a, then, also fails. It is true that language of itself cannot produce beliefs that are guaranteed to be true; but that is not what is required for something to be a pram ¯ an. a. Even in contemporary Western philosophy, that is more than we require 



%\subsection{POTENTIAL CONNECTIONS WITH ANCIENT PHIL}

%OBVIOUS COMPARISONS TO SOCRATES, TO HELLENISTIC SCEPTICS WHO LIBERALIZED OVER WORRIES ABOUT ACTION --- EXACT SAME STUFF IS AN ISSUE FOR EG NYAYA --BHUDDISTS --- M\={\i}m\=a\d ms\=aS

%Kind of Nice compared to eg Socrates too because you have vedic standards sort of spilling out onto other stuff --- not only a sort of eternality also novelty (Kataoka memory paper; upamana section in Kumarila is relevant here; 


%\subsection{MORE DIRECTLY INTERPRETATIVE MATTERS}

%UH-OH: CHATTERJEA SAYS THAT THE BAHATTAS DON'T ACCEPT FALSE OBJECTS OR APPARENT TRUTHS (53) ---BUT ACTUALLY I'M NOT SURE I GET THIS. THEY DO ACKNOWLEDGE INVALID COGS, SO... IS THE IDEA THAT ONCE INVALID THEY DON'T MANIFEST ANYMORE? --- SEE ALSO MOHANTY 8, and compare Taber's sabdapramanya article, esp. p169, for at least the special case of a sort of testimonial comprehension possibly held in abbeyance if trut is not secured

%INTERESTING IF MEMORY IS EXCLUDED FROM REALM OF VALIDITY --- MEANS AGAIN THAT TRUTH CAN'T REALY BE THE ISSUE BUT RATHER THIS MANIFESTATION THING



%\subsection{CONNECTIONS WITH CONTEMPORARY TESTIMONY LITERATURE}


%SEP ARTICLE ON TESTIMONY IS GOING TO OFFER SOME REALLY HELPFUL STUFF ON PRESUMPTIVE SUFFICIENCY OF TESTIMONIAL EVIDENCE

%[[COULD ALSO THINK IN TERMS OF EPISTEMOLOGY BEING DONE FROM THE STARTING POINT OF TESTIMONY? NOT QUITE RIGHT, THOUGH. WOULD MORE BE SO GIVEN HE CONTEMPORARY DEBATE PERHAPS....]]





%\subsection{SOME POINTS I SORT OF LIKE FROM AN EARLIER READING NOTE}

%One perhaps more interesting topic is how to handle the fact that people evidently make mistakes in identification. On this question Kumarila's basic stance is that perception has not played you false, as it were, if you mistake, say, a pig for a peccary simply because you don’t know the difference. This seems to me a fair conclusion---of course conceptualized perceptions can do no better than work with the conceptual scheme you have, and here the problem is the poverty of your conceptual scheme. But there are other cases: suppose the lighting has been rigged so that the piggish features of the peccary within are accentuated. Now even if I am an expert on mammals of medium size, the indications here may be misleading. The issue here is neither perceptual malfunction (peccaries just do look like pigs in this lighting) nor lack of discriminatory ability, but I will still wind up misidentifying the creature before me. (Taber thinks that Kumarila has this sort of case in mind at some points---see for example the comments on verses 237cd-241ab. But as far as I can see those verses deal only with error stemming from lack of expertise.) Kumarila could fall back here on the idea that a cognition is valid so long as it isn't refuted by some further cognition, but that seems pretty weak here. Surely we don’t want the correctness of the cognition to turn on whether I actually happen to look around a little more carefully. Perhaps Taber is right to call this a emph{reductio} of Kumarila's position (p. 219, n. 43).

%The concern with means of knowing, or \emph{pramanas,} which are, \emph{as such,} reliable, seems to be shared by the various disputants on the Indian scene. The issue then is the identification and description of trustworthy causal pathways by which we come to have thoughts. In a way I’m really sympathetic to this: the question ``what can we know and how'' seems to me more important than, say, ``what is a reason for belief,'' which preoccupies a lot of contemporary philosophers. I’m doubtful, though, whether it is right to think that there could be any type of cognition (whether we identify types by origin or in some other way) such that a given thought is credible just for being of that type. Certainly we sometimes know things by perception: I may know that there is a pig in the room because I’m looking at it. But sometime I also know what I'm seeing and that I'm seeing it, but my knowledge does not stem from perception (or not only). Suppose I look into a darkened room and see something moving around in the corner. Now looks alone may not settle whether it is a pig or a peccary, but I may still know that there is a pig in the room which I am looking at because a credible source tells me he just put a pig in the room.

%(Admittedly it isn't clear from our reading what this ``intrinsic validity'' idea amounts to. Does it just amount to the thought that pramanas are after all reliable? If not, is it that any thought whatever have the validity? Which ones can cancel which others? Why would pramanas even matter?)

%One possible lesson for philosophy of religion (or philosophy generally) is to be drawn from something I initially raised as a worry. This is that in discussing philosophical questions, including those that cut across points of serious disagreement on religious questions, we should start with everything (we think) we know. Then we can pose the following questions: how can we make sense of these things? What sorts of things are these which we know? How, in fact, would we come to know such things? If some fact is the kind of thing that could be known by, say, perception, what would perception have to be like? Does it make sense to think about perception that way? And so on. Our rivals can do the same and the two sides will sharpen each other's thoughts. The case of reformed epistemology shows that this way of thinking has recently found some place in contemporary analytic philosophy of religion. But perhaps the Indian case shows more clearly (because it was more developed) that it is possible to conduct a fruitful debate along such lines.




%one might want to carve this more narrowly - but I uppose that sort of the point of K taking about tristworthiness of diff people....





%76-7 explicitly ties the mode and the cognition together in terms of reliability

%82-4 suggest (well, say, basically) that you aren't necaware \emph{that} the stuff is authoritative.
	%I think my interp of which is which and were is perhaps screwing stuff up....

%86 truth and falisty must be taken in a psychological sense...

%91 shows that non-perception is across varieties of K basically - point is just that there is no basis for a cogition in any source of K.


%***************94-5 goes back to fancy, and sure-enough, wishful thinking won't count

%97 explains why you don't have to postulate n author but this seems pretty weak.


%***************128 makes the point that just saying stuff about whatever doesn't at all count - eg humans saying stuff about the transcendent

%ok 130 is extremely worrisome- - or is it mockery? yeah, probably that. ooh, but 131 seems to belie that ....

%but 132 helps....

%144 eople are untrustworthy

%a passage like 154 could easily be misunderstood - but even 155 goes on to make a parallel between Vedic stuff and *perceptions due to faultless cognitions* 0 not totally clear what that means but has to be anchored or else goig to q-bing


%remmeber it's very well to say just broadly that people are unreliable, but that's no good - doesn't in any sense undercut a given conception - so clear that can't really be what's at stake, unless Kum\=arila is just cheating

%162 also seems to confirm my view...

%and see 166 on defending it - you point to the source. defense is possible - it's not like tghere's nothing to say - so I've really gone to far in saying that there can be no checking for K - there can be in the sense that I guess you are supposed to have some conception of the the origin - what you don't have to have is a sense of the excellences? excpet it sort of seems like later the excellences really are coming in after all... (I mean in this general are - maybe need to reread)

%ok I seriously have to figure out 167 - something crucial going on here - like checking means hifting your grounds a bit... (think about that in intuitive terms.).
	%ok I think the pooint is this: now you aren;t going just on the authority of the cognition - because now you rely directly on the person for the warrant. But this makes manifest the very arrant that the cognition had in the first place. and this is exactly as checking should be: You invited Hirut to accept your word as you accepted what you heard; when she challenges it you offer the authority whence the authority of your own position derived. - compare that other passage about setting aside one source for another source.

%I find 168 puzzling, but it may be crucial for understanding the notion of DEFINITENESS. Specifically looks likethe definiteness comes somehow from having heard it from a speaker, and NOT from just understanding the thing. This suggests perhaps, in line with my interp, that definiteness means (at least in part?) COMMITMENT. Just having a thought doesn't get you that but testimony does. (but in reality don't you often not accpet testimoy? I thnk actually you really do tend to be inclined to accept what you hear - one likes to avoid hearing people's opinions before seeing eg a movie precisely because it tends to take a reflecive effort to reject  what you are told. (Or at least a reflective position hostile to some given speaker or set of positions or whatever). 



%HAVE TO CHECK WITH PARIMAL OR RAJAM ON THE LANGUAGE



%34: WHat does the first half mean? "because those that are by themselves false cannot by any means be proved true." 

%2nd part of 34 offer two initial possibilities: both intrinsic, both extrinsic. 35-7 rejects these two possibilities.

%Actually first halfs of 35 and 36, and then all of 37, are against both being intrinsic.

%AGAINST BOTH INTRINSIC. In 35-6 looks really bad at first and then to parry with 37. Alternatively there's some less obviousl point being made in 35-6 which I would hope because otherwise why bother?


%

%36 seems like the "extraneous" puts pressure in the "generous" direction - ie a further argument: if a kind of thing has contrary characters you've got to look outside it in some sense for an explanation. And a second half which seems to repeat the arg of 35: about having a definite form. 

%


%Second halfs of 35 an 36 are against both being extrinsic. 37 perhaps applies to this as well as to other side.

%35: perhaps my notion of necessity of a default fits with the definiteness idea - if it couldn't be one or the other it couldn't be taken as anything - this would just mean interpreting definiteness in terms of having to have a default - but why should it have to have a default? perhaps because if the conception weren't either true or false that would be because it didn't have a suitably fixed content. That's perhaps satisfactory. Ok, but how to put them together? Any particular conception has to have taht def, but why should there have to be a general default?


%Jha ix: in the fact of bringing about a definite cognition consists the authority of Ved injunction...
%	but then that would also allow fancy etc....




%Jh\=a and Taber (also Arnold?) describe this as a sort of physcological process - a matter of how we actually think of the soundness of a cognition or how we get the idea of reliability.

%But as I've pointed out, Kum\=arila is not just concerned with a psychological point; he thinks that it is unreasonable and hopeless to require a check on a cognition before it can be accepted as valid. And often it is, though not always. Suppose I say that I saw a crow. But maybe it was a raven, someone suggests---there are a lot of ravens around here. I confess that I have no idea how to distinguish a crow from a raven, and will reasonably acknowledge that I don't really know that it was a crow. But someone might also ask me how I know that my vision didn't suddenly fail me and that I only hallucinated a crow, or that I didn't actually see an alien disguised as a crow. Well I can't rule that those possibilities either. And as a psychological point I may very well feel, at least when in my arm-chair, that knowledge is, in the face of these skeptical scenarios, impossible.

%But whatever the psychological reality, it is unreasonable to be moved by just any far out possibility. I do not need to worry about the threat hallucination all the time. I would be foolish to think that I needed to check up on my eyes every time I think I see something. It is reasonable, in general, to go along with what I come to believe on the basis of sight. But at the same time the battle-lines seem to be drawn to sharply here. So	qmetimes the demand for some kind of verification is unreasonable, but sometimes it's not. 

%22 we get objection that testimony is not a basic source of K


%For Kum\=arila, 


%It is surely right to treat testimony as a basic source of knowledge in the sense that for any given individual, much knowledge derives directly from testimony---we do learn things from testimony and in many of these cases testimony provides the only, and only possible, access we can have to that information. However in another sense it is not basic: what s testified to is ultimately learned in some other way. At some point testimony goes back to, say, an eye-witness. But this means that we could recognize another variety of defect in testimony: namely not ultimately being based on another form of knowledge. But knowledge of \emph{dharma} is rooted in Vedic utterances and nothing else. However this objection is question-begging in the epresent context: for the M\=im\=a\d ms\=a, language and meanings and the Veda are eternal---in which case the M\=im\=a\d ms\=a has a metaphysical view that allows him to treat testimony as a basic source of knowledge in both of the above senses.

%We might also feel that language could not be interpreted except in the mouth of a speaker---that an appreciation for a context of utterance is what makes interpretation possible. But again, given the M\=im\=a\d ms\=a view of language, such an objection would have to be developed within the philosophy of language.

%But an objection which is 



%Further issue is that K thinks that in particular it's not reasonable to refuse to believe X if testified to but not by any particular person. But once we see that sometimes checking is reasonable it's not obvious hwy that shoudl be so in this particular case....

%really has to push everything back on the metaphysics....

%because a reasonable grounds for doubt surely has to do content etc as well.... well anyway I think I want to make the point that this could hardly just settle everything....




%Taber goes too far here (p 207)

%Thus the doctrine of svatah pramanya can be seen, simply, as the claim that we should take our cognitions at face value, like everything else. If in every way they appear true, then they are!





%Chisolm, The Problem of the Criterion

%Having these good apples before us, we can look them over and formulate certain criteria of goodness. Consider the senses, for example. One important criterion—one epistemological principle—was formulated by St. Augustine. It is more reasonable, he said, to trust the senses than to distrust them. Even though there have been illusions and hallucinations, the wise thing, when everything seems all right, is to accept the testimony of the senses. I say “when everything seems all right.” If on a particular occasion something about that particular occasion makes you suspect that particular report of the senses, if, say, you seem to remember having been drugged or hypnotized, or brainwashed, then perhaps you should have some doubts about what you think you see, or hear, or feel, or smell. But if nothing about this particular occasion leads you to suspect what the senses report on this particular occasion, then the wise thing is to take such a report at its face value. In short the senses should be regarded as innocent until there is some positive reason, on some particular occasion, for thinking that they are guilty on that particular occasion. One might say the same thing of memory. If, on any occasion, you think you remember that such-and-such an event occurred, then the wise thing is to assume that that particular event did occur—unless something special about this particular occasion leads you to suspect your memory. We have then a kind of answer to the puzzle about the diallelus. We start with particular cases of knowledge and then from those we generalize and formulate criteria of goodness—criteria telling us what it is for a belief to be epistemologically respectable. 

%not sure yet exactly where this is going, but this sure looks not far off the sort of thing Kum\=arila could be saying - no reason to go either with Arnold to the comparisons with religious epistemology (well, not sure I'm right there) or with Taber...

%and I'm not just thinking of the credence but of the listing reliable canons, etc

%see esp also section 15 in Chisolm's paper - indeed rules come from the confidence in general types of cases - should look for evidence of that line of thought in Kum\=arila

%section 15 perhaps also interesting re question of whetehr or how this is actually a resonse to skeptic - but hardly crucial.





% Get this: Karl Potter, "Does Indian Epistemology Concern Justified True Belief?" Journal of Indian Philosophy 12 (1984): 307-327 at pp. 317-318. 


%Umveka's view ties together truth and causal stuff



%TABER WORRIES THAT THE VEDIC STUFF GETS THROUGH JUST BY DEFAULT, BUT SO WHAT? TESTIMONY OFTEN DOES AND SHOULD GET BY BY DEFAULT. 

%ARNOLD SEEMS TO HAVE PUT TOO MUCH EMPHASIS ON *VEDA* BUT THAT'S SUPPOSED TO BE JUST A SPECIAL CASE OF TESTIMONY






\begin{comment}


Note, incidentally, that in the last sloka Kumarila carefully 
avoids saying that one must establish that there is no dosa. 
The subsiding of any concern about the falsehood of the 
cognition occurs automatically, "without effort." L. Schmithausen 
has argued that the svatah prdmanya theories of both Kuma- 
rila and Mandanamigra 
are undermined 
by a stipulation that in 
order for a cognition to be considered valid it must be 
posi- 
tively ascertained that there are no hetudosas or badhakajnia- 
nas. (See VV, 199-200, with regard to Kumarila, and 232 
with regard to Mandana.) Schmithausen cites in support of his 
claim about Kumarila, SV codana 67 cd: dosdbhdve tu 
vijieye 
[gunadh] 
sattamatropakdrinah. 
This states, roughly, that gunas 
function to exclude dosas merely by their presence in the 
causes of cognition, but not insofar as they are known. Thus, 
one needn't be aware of the presence of gunas in order for 
cognition to be valid (which would lead to the regress of the 
theory of extrinsic validity), but one should be aware of the 
absence of dosas (dosdbhdve 
vijneye). But Schmithausen may 
be taking the word vijiieye in this verse more literally than 
Kumarila intended. In light of 11.52, cited above, Kumarila 
seems to mean not that one must be positively aware or estab- 
lish that there are no defects, but only not aware that there are 
any (nivartate hi mithydtvam dosajandat...). Similarly, Ku- 
marila says with regard to a cognition that contradicts an ear- 
lier cognition, i.e., a 
badhakajnana (codana 60): 
svata eva hi tatrdpi 
dosajnidnt pramdndtd 
dosajnine tv anutpanne na sahrkyd 
nispramdnatd. 
I believe that the same analysis applies to Mandana, even 
though his assertion that one should "know" that there are no 
dosas is more definite: tato gate 'bhdve hetudosanam tatha- 
khydtiviniscayah (VV 1 
14d- 115b). 

\end{comment}




\begin{comment}

THIS POINTS FROM GOLDMAN'S SEP ARTICLE SEEMS WORTHWHILE (AND LOOKS THEN LIKE INTERNALIST LEVEL CONFUSIONS ARE ALIVE AND WELL): ``Another way to reflect on Steup's proposal is this. No matter what criterion C of justifiedness is chosen, it will always be possible (assuming the fallibility of justifiedness) 
for a person to undergo events that justify him in thinking falsely that he satisfies or fails 
to satisfy C in a particular case. Hence, he would be justified in believing falsely that a 
particular (first-order) belief of his is justified or unjustified. Now consider the possibility 
that externalist reliabilism is the correct criterion of justifiedness. Then the subjects Steup 
describes would be justified in their perceptual beliefs if their perceptual processes are 
reliable, even if they have evidence contrary to this conclusion. However, as we have seen 
above, having evidence might be equivalent to having (propositional) justifiedness for a 
proposition. So in the scenarios in question, a reliabilist would say that subjects are 
justified in believing that their perceptual beliefs are not justified even though, in fact, they 
are justified. This is perfectly in order and consonant with reliabilism. It's compatible with 
really being justified in believing P that you are justified in believing that you aren't 
justified in believing P. Thus, an externalist reliabilist can say that Steup confuses iterative 
unjustifiedness with respect to perceptual beliefs (J~J(P)) with first-order unjustifiedness 
with respect to these beliefs (~J(P)) (see Goldman, forthcoming). 
	FORTHCOMING PAPER IS ``Epistemic Relativism and Reasonable 
Disagreement'' - SHOULD BE ABLE TO GET A PDF FROM G'S SITE

GOLDMAN'S BASIC DESCRIPTION OF HIS ORIGINA RELIABILISM IN THE SEP ARTICLE REALLY DOESNT LOOK SO DIFF FROM Kum\=arila - PROCESSES CARVED PRETTY BROADLY - EG PERCEPTION ETC NOT WISHFUL THINKING ETC

MOVES SEEM CLEAR HERE: YOU REJECT JJ, HAVE LEVELS SEPARATED, THEN YOU WORRY ABOUT WHAT HAPPENS IF A PERSON DOES HAVE COUNTER-EVIDENCE AVAILABLE
	G PROPOSED THIS: Instead of requiring the subject to have a reliably caused meta-belief that her 
first-order belief is reliably caused, it proposes a weaker condition intended to cover 
evidence that undercuts reliability. It says that there must be no reliable process available 
to the subject that, were it used by the subject in addition to the process actually used, 
would result in her not believing P.
		COMPARE THIS WITH K'S VIEW WHERE SOMETHING WEAKER IS PERHAPS TRUE AT THE FIRST LEVEL: THERE CAN'T BE ANY OTHER-RELAIBLY PRODUCED BELIEF AT ODDS WITH THE FIRST ONE. NOT SURE THAT THE HIGHER LEVEL THING GETS ADDRESSED
		
SHOULD TAKE UP SOME EXAMPLES - CAN START WITH SOME FROM GOLDMAN'S SEP ARTICLE
	ALSO USE BONJOUR'S CLAIRVOYANCE EXAMPLE
	
BOOTSTRAPPING OBJECTION ALSO INTERESTING

Goldman says that the generality problem is just a problem for all epistemologists because it affect sthe basing relationship. Roughly, any connection bwteen whatever it is you care abot and the target beliefs will be subject to devient causal chains. So you want to know what sort of causal chains will work... and then you have your problem again... (and indeed perhaps another place where you would need something like reliabilism anyways?

CAN OFFER K'S OWN PROGRESSION. A POINT ABOUT HOW TESTIMONY IS ALWAYS DEPENDENT. THEN WE GET THE POINT ABOUT INTRINSIC VALIDITY - and generally attend to the issue of speaker reliability (eg v. 23 and then a whole bunch later on)\\

SO SHOULDN'T THERE BE AN ISSUE SPECIFICALLY ABOUT TESTIMONY, THAT EVEN IF COGS ARE VALID, BUT AS LONG AS WE CARE ABOUT SOURCES, THEN WE STILL OWE A *FURTHER* ANSWER TO THE OBJECTION THAT TESTIMONY IS ALWAYS DEPENDENT ON OTHER SOURCES OF COGNITION?\\

IT'S POSSIBLE THAT THE BEST WESTERN ANALOGUES WOULD BE, ODDLY ENOUGH, SCEPTICS LIKE CARNEADES OR PHILO. 1. THERE YOU ALSO DON'T HAVE BOTH WARRANT AND TRUTH IN THE PICTURE SEPARATELY, REALLY (THOUGH HERE BECAUSE TRUTH HAS BEEN REJECTED AS A POSSIBILITY - BUT PERHAPS Kum\=arila WOULD ALSO FIND THE IDEA OF SOMETHING BEYOND VALIDITY OTIOSE - SOMETHING NOBODY EVER DID OR COULD HAVE ANYWAYS, AS THE SCEPTICS MIGHT SAY - THOUGH I SUSPECT THIS MIGHT BE TOO WEAK? GUESS MY Q IS TO WHAT EXTENT VALIDITY MUST BE UNDERSTOOD IN TERMS OF TRUTH-TRACKING; BUT OF COURSE THAT WAS AN ISSUE IN THE SCEPTICAL CASE TO AND A SOURCE OF DIVISION - THEN THE ISSUE S WHETHER TRUTH IS MORE IN THE PICTURE ON THAT SIDE? EITHER WAY LOOKS LIKE A GOOD CASE FOR A COMPARISON HERE! AND LOTS OF FUN CITATIONS!; GUESS PART OF POINT HERE WAS THAT PERSUASIVE IMPRESSIONS WERE UNDERSTOOD AS ONES THAT SEEMED TRUE, SO TRUTH CAME IN IN A WAY OBJECTIONABLE FOR AT LEAST SOME SCEPTICS); YOU DO HAVE THE WARRANT IDEA, PLUS EVEN ISSUE ABOUT REVERSAL; IS INDEED A QUESTION OF RATIONALITY AS WELL AS APPEARANCE (SEE BURNYEAT P29 IN D&D WHERE IT LOOKS LIKE AT LEAST CARNEADES FOLLOWERS TOOK THIS AS A CLAIM ABOUT REASONS VS JUST DESCRIPTIVE - THERE IS AN ISSUE ABOUT TO WHAT EXTENT WE COULD TALK ABOUT ASSENT, THOUGH, WHICH Kum\=arila WOULD PERHAPS WANT AND CARNEADES WOULD REJECT - BUT THEN I THINK PHILO *DID* ALLOS ASSENT; ALSO MAYBE ANALOGOUS RE SORT OF CONTENTS IN QUESTION? NOT SURE. ANYWAYS, THINK I SHOULD DEVELOP THAT A BIT. I WONDER ABOUT ELSEWHERE IN HISTORY OR MAYBE CONTEMP STUFF? MAYBE PRAGMATISMS OF SOME FORM WOULD HAVE A SIMILAR SHAPE?\\

GOLDMAN ``EXPERTS'' PP 86--7

There are other approaches to the epistemology of testimony that lurk in 
Hardwig’s neighborhood. The authors I have in mind do not explicitly urge 
any form of skepticism about testimonial belief; like Hardwig, they wish to 
expel the specter of skepticism from the domain of testimony. Nonetheless, 
their solution to the problem of testimonial justification appeals to a mini- 
mum of reasons that a hearer might have in trusting the assertions of a 
source. Let me explain who and what I mean. 
The view in question is represented by Tyler Burge (1993) and Richard 
Foley (1994), who hold that the bare assertion of a claim by a speaker gives a 
hearer prima facie reason to accept it, quite independently of anything the 
hearer might know or justifiably believe about the speaker’s abilities, 
circumstances, or opportunities to have acquired the claimed piece of knowl- 
edge. Nor does it depend on empirically acquired evidence by the hearer, for 
example, evidence that speakers generally make claims only when they are in 
a position to know whereof they speak. Burge, for example, endorses the 
following Acceptance Principle: “A person is entitled to accept as true some- 
thing that is presented as true and that is intelligible to him, unless there are 
stronger reasons not to do so” (1993: 467). He insists that this principle is 
not an empirical one; the “justificational force of the entitlement described by 
this justification is not constituted or enhanced by sense experiences or 
perceptual beliefs” (1993: 469). Similarly, although Foley does not stress the 
a priori status of such principles, he agrees that it is reasonable of people to 
grant fundamental authority to the opinions of others, where this means that 
it is “reasonable for us to be influenced by others even when we have no 
special information indicating that they are reliable” (1994: 55). Fundamental 
authority is contrasted with derivafive authority, where the latter is generated 
from the hearer’s reasons for thinking that the source’s “information, abili- 
ties, or circumstances put [him] in an especially good position” to make an 
accurate claim (1994: 55). So, on Foley’s view, a hearer need not have such 
reasons about a source to get prima facie grounds for trusting that source. 
Moreover, a person does not need to acquire empirical reasons for thinking 
that people generally make claims about a subject only when they are in a 
position to know about that subject. Foley grants people a fundamental 
(though prima facie) epistemic right to trust others even in the absence of any 
such empirical e~idence.~ 
It is in this sense that Burge’s and Foley’s views 
seem to license “blind” trust.\\

AG IS LESS SCEPTICAL :``A case might be made that children are in a 
position to get good inductive evidence that people usually make claims 
about things they are in a position to know about.'' - BUT SOCRATES WORRY IS APPROPRIATE - PEOPLE TAKE THEIR EXPERTISE TO RANGE MUCH FURTHER THAN IT DOES\\

P 88: ``Of greater concern to me is the recognition that a 
hearer’s evidence about a source’s reliability or unreliability can often bolster 
or defeat the hearer’s justifiedness in accepting testimony from that source. 
This can be illustrated with two examples.''\\
	MATTERS TO OVERALL ENTITLEMENT - HERE K CAN AGREE

GOLDMAN 88-9 - THE CASE OF EXPERT DISAGREEMENT SHOULD BE A TOUGH ONE FOR K BECAUSE GOING FOR 

BIT IN N6 ABOUT PROPRIETARY VS TRANSMISSIONAL WARRANT OF THE TESTIFYIER IS INTERESTING..

Apparently Unger's book Ignorance doesn't try to dislodge belief despite a sceptical restructuring - at least that's how Burnyeat sees it with inter on 32\\

[[is just any counter evidence gathered by anyone supposed to count? Parimal said something that suggested this. Seems a bit worrisome to me, though, because how would you know whether this had happened? Issue here might be that we want in some sense to say that other peoples beliefs are not warranted....]]\\

John Pollock p101 "A Plethora of Epistemological Theories" discusses coherentisms where everything os sort of default warranted - so get this\\

Alston in Kornblith volume discussing internalism, p75: justification can be conferred locally only by other beliefs that already have it - but for coherentism, at a holistic level beleifs must suffice\\

Alston 79 distinguishes the activity of justifying a belief and the state of a belief's being justifed (is there perhaps a third thing: being justified in a belief?) - makes it sound like an understood distinction whih is perhaps frequently trodden over\\


for Alston the strongest line of argment puts justification in a sort of deontological setting
	in which case I should also get Alston's "concepts of epistemic justification" - he gives other refs in n24\\

think about a weaker sort of standard (looking at Alston p 98): being justified requires that you can find out\\

a lesson from Alston is perhaps that internalism has lots of regress issues - not just about circularity but higher level regresses - like being able to tell *that* this feature is a justifying feature, etc etc\\

***Great, Alston 103--4 raises the issue of the two varieties of juistification - on the one hand tempted to go pure externalist, but on the other admits we don't want to admit wierd beliefs unless we have some sense of the source. THis he takes as a thing that arises becuase we want to be able to meet challenges. Anyways seems to fit with my interp of K, in that basically Alston also has it in terms of a general externalism plus mainly a worry about types of sources.\\

Sosa 149 in the Kornblith volume quote a distinction in Goldman between strong and weak justification: these look to me like they may very well match up usefully with some things Kum\=arila distinguishes.... specifically everything (from what you take as legit sources) will have weak justification.




\end{comment}










%VARIOUS ITEMS WORTH LOOKING AT

%Frede, M., ‘Stoic Epistemology’, in K. Algra, J. Barnes, J. Mansfeld and M. Schofield (ed.), The Cambridge History of HellenisticBrittain, C., Philo of Larissa (Oxford 2001), 296–342. 

%Frede, M., ‘Des Skeptikers Meinungen’, Neue Hefte für Philosophie, Aktualität der Antike 15/16 (1979), 102–29 [= ‘The Skeptic's Beliefs’, in M. Frede, Essays in Ancient Philosophy (Minneapolis 1987), 179–200; reprinted in M. Burnyeat, and M. Frede (ed.), The Original Sceptics (Indianapolis 1997), 1–24.] 

%Frede, M., ‘The skeptic's two kinds of assent’, in his Essays in(Cambridge 1990), 184–303. reprinted in M. Burnyeat, and M. Frede (ed.), The Original Sceptics

%Brittain, C., Philo of Larissa (Oxford 2001), 73–168.

%Barnes, J., ‘Antiochus of Ascalon’, in M. Griffin and J. Barnes (ed.), Philosophia Togata (Oxford 1989), 51–96. 

%Glucker, J., ‘The Philonian/Metrodorians: Problems of method in ancient philosophy’, Elenchos 25.1 (2004), 99–153. [A review of Brittain's Philo of Larissa.] 

%Striker, G., ‘Academics fighting Academics’, in B. Inwood and J. Mansfeld (ed.), Assent and Argument (Leiden 1997), 257–276. 

%G¨ ohler, Lars (1994). “The Concept of Truth at Kum¯arila Bhat .t.a and K. R. Popper: A Comparison,” in R. C. Dwivedi, ed., Studies in M¯ım ¯ am . s ¯ a: Dr. Mandan Mishra Felicitation Volume (Delhi: Motilal Banarsidass), pp. 79–86. 

%Jha, Ganganatha (1964). P ¯ urva-M¯ım ¯ am .s ¯ a in its Sources. Varanasi: Banaras Hindu University Press. (Second edition.) 

%Matilal, Perception

%Mohanty, J. N. (1992a). “Indian Epistemology,” in Jonathan Dancy and Ernest Sosa, eds., A Companion to Epistemology (Oxford: Blackwell), pp. 196–200. 

%Mohanty, J. N. (1992b). Reason and Tradition in Indian Thought. Oxford: Clarendon Press. 

%Saksena, S. K. (1970). Essays on Indian Philosophy. Honolulu: University of Hawaii Press. 

%Verpoorten, Jean-Marie (1987). M¯ım ¯ am . s ¯ a Literature. A History of Indian Literature (ed. Jan Gonda), Vol. VI, Fasc. 5. Wiesbaden: Otto Harrassowitz. 

%Williams, Michael (1996). Unnatural Doubts: Epistemological Realism and the Basis of Scepticism. Princeton University Press. 

%?lokav?rttika. Edited by K. S?mba?iva ??str?, 1990, in M?m??s??lokav?rtika with the Commentary K??ik? of Sucaritami?ra (Parts I & II), Trivandrum: CBH Publications (reprint of Trivandrum Sanskrit Series, Nos. 23, 29, 31; 1913). 

%there are a couple oher editions as well. (See Arnold's SEP article)

%Bhatt, G. P., 1962, Epistemology of the Bh???a School of P?rvaM?m??s?, Varanasi: Chowkhamba Sanskrit Series Office.


%Taber, forthcoming, “M?m??s? and the Eternality of Language,” in The Columbia Guide to Classical Indian Philosophy, Matthew Kapstein (ed.), New York: Columbia University Press. 

%Verpoorten, Jean-Marie, 1987, M?m??s? Literature (A History of Indian Literature, ed. Jan Gonda, vol. VI, Fasc. 5), Wiesbaden: Otto Harrassowitz. 

%Shastri, Pashupatinath, 1980, Introduction to the P?rva M?m??s? (2nd ed., edited and revised by Gaurinath ??str?), Varanasi: Chaukhambha Orientalia. 

%Ram-Prasad, Chakravarthi, 2001, Knowledge and Liberation in Classical Indian Thought, New York: Palgrave. 

%Mohanty, J. N., 2007, “Dharma, Imperatives, and Tradition: Toward an Indian Theory of Moral Action,” in Indian Ethics: Classical Traditions and Contemporary Challenges, vol. 1, in Purushottama Bilimoria, Joseph Prabhu, and Renuka Sharma (eds.), Aldershot: Ashgate, pp. 57–78. 

%Halbfass, Wilhelm, 1983, Studies in Kum?rila and ?a?kara, Reinbek: Verlag für Orientalistische Fachpublikationen.

%Arnold, Dan, 2005, Buddhists, Brahmins, and Belief: Epistemology in South Asian Philosophy of Religion, New York: Columbia University Press. 

%Arnold, Daniel, 2002, “M?m??sakas and M?dhyamikas against the Buddhist Epistemologists: A Comparative Study of Two Indian Answers to the Question of Justification,” Ph.D. dissertation, University of Chicago. (Appendix I, pp. 345–370, contains a complete translation of P?rthas?rathimi?ra's Svata?pr?m??yanir?aya.)

%Jha, Ganganath, 1983, ?lokav?rtika: Translated from the Original Sanskrit with Extracts from the Commentaries “K??ik?” of Sucarita Mi?ra and “Ny?yaratn?kara” of P?rtha S?rathi Mi?ra, Delhi: Sri Satguru, 1983 (reprint; first published in Calcutta, 1900).

%Jha, 1986 The Tattvasa?graha of Sh?ntarak?ita with the Commentary of Kamalash?la, Delhi: Motilal Banarsidass (reprint; first published in Gaekwad's Oriental Series, 1937–1939).

%–––, 1998, Tantrav?rttika: A Commentary on ?abara's Bh??ya on the P?rvam?m??s? S?tras of Jaimini (2 volumes), Delhi: Pilgrims Book (reprint; first edition, 1924).

%Ksitish Chandra Chatterjee, "Misconceptions about some terms in M?m??s? literature", IHQ 4, 1928, 783-787

%C.Kunhan Raja, "In defence of M?m??s?", ALB 16, 1952: 115, 168

%Vachaspati Upadhyay, Theory of Self-Validity of Knowledge in M?m??s? Philosophy. Ph.D.Thesis, University of Calcutta 1967

%Erich Frauwallner, Materialen zur ältesten Erkenntnislehre der Karmam?m??s?. Wien 1968

%V.N.Jha, "On the M?m??saka's general definition of pram??a", CinSasVol 16-22

%Purushottama Bilimoria, "The idea of authorless revelation (apauru?eya)", in Roy W. Perrett (ed.), Indian Philosophy of Religion (Dordrecht 1989), 143-166

%Purushottama Bilimoria, "Hindu-M?m??s? against scriptural evidence on God", Sophia (Victoria) 28.1, 1989, 20-31

%Hiroshi Marui, "What prompts people to follow injunctions? An elucidation of the correlative structure of interpretations of vidhi and theories of action", Acta Asiatica 57, 1989, 11-30

%Subodh Kumar Pal, "A note on the M?m??s? conception of ap?rva", VJP 25.2-26.1, 1989, 50-52

%N.N.Sarma, Verbal Knowledge in Pr?bh?kara M?m??s?. Delhi 1989

%V.K.Chari, "?abdapr?m??ya: an analysis of the M?m??s? argument", JOR 55- 62, 1986-92, 96-105

%Daya Krishna, "M?m??s? before Jaimini: some problems in the interpretation of ?ruti in the Indian tradition", JICPR 9.3, 1992, 103-112

%K. Acharya, "Knowledge representation in M?m??s?", IndS 162-167

%PM127 G.P.Bhatta, "M?m??s? as a philosophical system: a survey", StudinM 3-26

%Hajime Nakamura, "Problem of categorical imperative in the philosophy of Pr?bh?kara school: a brief note", StudinM 169-185

%Purusottama Bilimoria, "Authorless voice, tradition and authority in the M?m??s?: reflections in cross-cultural hermeneutics" Sambhasa 16, 1995, 137-160

%Sabujkoli Sen (Mitra), "The Bh???a definition of pram? and the problem of dh?r?v?hika pratyak?a: an analysis", VJP 32.1, 1995-96, 9-95

%A. Ramanna, "Pram??a-M?m??s?", ResIn 142-147

%Ujjwale Panse, "Turning points in M?m??s? epistemology", TPIST 34-41

%Ujjwala Jha, "Some Recent M?m??s? Works in Sanskrit", SWIII 287-296

%Kei Kataoka, "The M?m??sa definition of pram??a as a source of new information", JIP 31, 2003, 83-103

%Dan Arnold, "Nobody is seen going to heaven: toward an eppistemology that supports the authority of the Vedas", BBB 59-114

%K. T. Pandurangi, "The epistemology of P?rvam?m?m??", PIPV 53-100

%G. Misra, "Scop and limits of ?ruti as a pram??a: perspective from P?rva M?ma?s? and Advaita Ved?nta", SPIP 108-118

%Manjulika Ghosh, ed., ?abdapram??a in Indian Philosophy. New Delhi 2006

%Dretske, Knowledge and the Flow of Information

%Goldman, Epistemology and Cognition

%Alston, An internalist Externalism

%Steup, Internalist Reliabilism

%\footnote{Cf.~\citet{austin1962sense}












%----------------------
%END MATTER
%----------------------




\small

\section*{Abbreviations}

\noindent\emph{CS}\tab\tab\tab \emph{Codan\=as\=utra} of Kum\=arila (=\emph{SV} ch.~II)

\noindent\emph{MS}\tab\tab\tab \emph{M\={\i}m\=a\d ms\=a S\=utra} of Jaimin\={\i}

\noindent\emph{PV}\tab\tab\tab \emph{Pr\=am\=a\d nyav\=ada} of Ga\d nge\'sa

\noindent\emph{SV}\tab\tab\tab \emph{\small\'S\normalsize lokav\=arttika} of Kum\=arila

\noindent\emph{SVVT}\tab\ \ \emph{\small\'S\normalsize lokav\=arttikavy\=akhy\=at\=atparya\d t\={\i}k\=a} of Umbeka

\noindent\emph{TS}\tab\tab\tab \emph{Tattvasa\.ngraha} of \'S\=antarak\d sita (quotes Kum\=arila's lost \emph{B\d rha\d t\d t\={\i}k\=a})


%\section*{Text of Codan\=as\=utra Verses 1--88}

%For the reader's convenience, I include here the text of vv.~1--88 of the Codan\=as\=utra, in Ga\d{n}g\=an\=atha Jh\=a's translation.\footnote{\citet[pp.~21--35]{jha1907skb}. The full text of the \emph{SL} is available in a variety of formats \href{http://www.archive.org/details/cu31924022991818}{here}.} The translation is unmodified, and the glosses and formatting are Jh\=a's, though I have grouped the verses into sections.




%The translation is in the public domain and the full text of the \emph{SL} is available in a variety of formats \href{http://www.archive.org/details/cu31924022991818}{here}; Jh\=a also provides many helpful extracts from later commentaries. Though as far as I know, the text versions have not been proof-read, excepting the selection presented here.} 

% The footnotes are also mine. I have tried to supply cross-references, but lay no claim to exhaustiveness.

%A PDF of this selection, in a larger font and with a bibliography and further notes, is available \href{http://www.people.fas.harvard.edu/~britchie/files/Codanasutra.pdf}{here}.

%from \href{http://www.people.fas.harvard.edu/~britchie/}{my personal webpage} at

%Jh\=a refers to `cognitions' as `conceptions'; `validity' is `authoritativeness'; `intrinsic is `inherent'; Jh\=a also uses `means of right notion' to refer to a valid cognition. Other terminological notes will be provided as necessary.

%\begin{center} $\star$ \tab\tab $\star$ \tab\tab $\star$ \end{center}

%\begin{center}APHORISM II.\end{center}

\noindent\textsc{Aphorism II:} ``Duty is a purpose having Injunction for its sole authority (means of conceivability)'' (I-i-2).\footnote{I.e.~verse I.i.2 of the \emph{MS} of Jaimin\={\i}, upon which the following verses of Kum\=arila (as well as the rest of the \emph{Codan\=as\=utra}) are a commentary.}\footnotesize\input{textofthecodanasutra}\small


\nocite{jha1937tattvasangraha}
\nocite{sepkumarila}

\bibliographystyle{apalike}
\bibliography{bibliography}

\end{document}