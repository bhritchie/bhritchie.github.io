\documentclass[11pt]{amsart}

\usepackage[greek, english]{babel}

%\usepackage{microtype}
%\DisableLigatures{encoding = *, family = * }

%\usepackage{setspace}
%\usepackage{fullpage}

\usepackage{verbatim}
\usepackage{natbib}
\bibpunct{(}{)}{;}{a}{,}{,}


%\renewcommand{\rmdefault}{ptm} 

%\usepackage[usenames,dvipsnames]{color}

%\usepackage[pdfborder={0 0 0}, colorlinks=true, linkcolor=RoyalBlue, urlcolor=RoyalBlue, filecolor=RoyalBlue, citecolor=Black]{hyperref}



\usepackage{bigfoot}
\AtBeginDocument{\RestyleFootnote{default}{para}} 



\author{Brendan Ritchie}
\title{Socrates' Conception of Knowledge}


\begin{document}

\maketitle



%Rep II 359e--360a Glaucon makes the observation about a craftsman knowing what he can and can't do - obviously takes this to mean actually succeeding. - he's explicit about failure entailing ineptitude - ``complete injustice''


%\doublespace






\begin{comment}

COMMENTS FROM JON:

MAYBE BIT OF STUFF IN CONCLUSION COULD BE MOVED TO START

SPECIFIC:

p4: subject matter in extensional terms - maybe true but example doesn't really show it
	medicine examples not good etc...

p6: for right without success: should it be said that it makes the good outcome more successful

say more about the category theory example - seems important and underdeveloped....

maybe theory or concept - worry is about what he could have explicit

more in the way of a stalking horse or stalking horses

\end{comment}




%\greektext t'eqnh\latintext 

%\greektext t'eqnai\latintext 

%\greektext >idi'wths \latintext

%\greektext teqn'iths\latintext






%HAVE TO SORT OUT A BUNCH OF SINGLE QUOTE ISSUES AND REPLACING TRANSLITS WITH GREEK (ESP PERHAPS IN SECTION 4


%FROM GOLDMAN'S SEP ARTICLE ON RELIABILISM: ``The difference between the expert and novice bird-watchers evidently resides in the differences between the cognitive processes they respectively use in arriving at their bird identification beliefs. The expert presumably connects selected features of his current visual experience to things stored in memory about pink-spotted flycatchers, securing an appropriate “match” between features in the experience and features in the memory store. The novice does no such thing; he just guesses. Thus, the expert's method of identification is reliable, the novice's is unreliable.''






%bunch of relevant notes at top of chapter 1 file


%SOME LIKELY MOVES:\\

%PERHAPS SOME OF THE MENO STUFF (specifically 5.1 and 5.3) CAN BE INTEGRATED INTO THE DISCUSSION OF THE ACCOUNT OF KNOWLEDGE? AND MORE BRIEFLY? AFTER ALL I DON'T THINK I CARE SO MUCH NOW ABOUT LINKING UP THE EARLIER DIALOGUES WITH THE REPUBLIC. THAT CAN BE A MORE SECONDARY CONSIDERATION NOW.\\

%AND THE TEACHING AND RECOLLECTION SECTION ON THE MENO  (5.2) COULD PERHAPS BE MOVED INTO THE CONCLUSION. SHOULD ALSO THINK HERE ABOUT WHETHER I DISAGREE WITH SMITH BUT I THINK I PROBABLY NEEDN'T. ANYWAYS THEN I CAN ALSO SAY LESS IN THE INTRO ABOUT FORWARD LOOKING STUFF. SO SECTION 6 SHOULD GO ALTOGETHER\\

%ANOTHER GENERAL ISSUE IS THAT I THINK PART 1 ISN'T ALL THAT INTERESTING ANYMORE GIVEN THAT WOODRUFF AND SMITH HAVE DONE MOST OF IT ANYWAYS... HARD TO SAY - FOR A MAINSTREAM JOURNAL I WOULD WANT TO GO SHORTER, BUT PERHAPS FOR EG OSAP.... WELL, MAYBE TRY TO MAKE AS MANY OF THE SAME BASIC POINTS AS POSSIBLE BUT MORE BRIEFLY.\\

%[THINK I SHOULD BE MOVING THE VIRTUE SECTION INTO PART 3 WITH THE GENERAL QUESTION ABOUT POSSIBILITY OF OTHER VARIETIES OF KNOWLEDGE. THEN IT CAN JUST PICK UP ON THE POINT I'VE ALREADY MADE ABOUT THE RULE OF THE REJECTION OF AKRASIA - MAYBE SAY SOMETHING ABOUT ORDER OF EXPLANTION? I SUPPOSE TECHNE CONCEPTION WOULD HAVE TO BE MORE INGRAINED THAN THE REJECTION OF AKRASIA... - THEN BASICALLY THINK OF SECTIONS 3 AND 4 AS CONSOLIDATING POINTS, GENERALIZING MY THESIS, DRAWING OUT IMPLICATIONS, ETC.]]


%CONTEMPT FOR HISTORY: HIPP MAJ 285D--286A



%\section*{Introduction}





%Sometimes at dinner parties people make the mistake of asking me what my dissertation is about, and even compound the error by resisting my polite efforts at redirection. But I try to draw them into the topic a bit by going around the table and asking everyone to give examples of things they know. My (possibly 



Unscientific polling on my part has revealed that, when asked for examples of things they know, modern people always give examples in (or appropriate to) propositional form: ``I know that Addis Ababa is the capital of Ethiopia'' or ``God loves me''---that sort of thing. Nobody has ever said anything like ``carpentry'' or ``how to play the piano.'' That is the situation in contemporary philosophy as well, where epistemologists emphasize knowledge of particular facts; Thus Robert Audi, in a recent introduction to epistemology, can say flatly that ``knowledge is constituted by belief (of a certain kind).''\footnote{\citet{huemer2002ecr}, p. 1.}

But it doesn't take much acquaintance with Plato's earlier dialogues to know that Socrates \emph{loves} examples of the other sort. In particular, examples of practices of the skilled, often professional variety which were called ``\emph{technai}.''\footnote{By ``earlier dialogues,'' I mean the \emph{Apology, Charmides, Crito, Euthydemus, Euthyphro, Gorgias, Hippias Major, Hippias Minor, Ion, Laches, Lysis,} and \emph{Protagoras}; I also draw on the \emph{Meno} and \emph{Republic} I. (But similar examples are also prominent in later dialogues.)} Socrates' contemporaries certainly noticed this: Callicles complains that Socrates never lets up with his ``continual talk of shoemakers and cleaners, cooks and doctors''; Xenophon even records the Athenian tyrants warning Socrates that he had better lay off discussing carpenters and herdsmen and blacksmiths and the like.\footnote{\emph{Gorgias} 491a; \emph{Memorabilia} 1.2.}

%His examples of knowledge are examples of \emph{technai}---crafts or forms of expertise, though no English word is very suitable. 

%[[want to observe how Vlastos---a great and I think sympathetic interpreter---thought that Socrates was no epistemologist - apparently because S's epistemology s just in some sense so foreign! - and expalin a bit how this could be missed - eg fact that often there are just these free propositional uses.]]



Modern readers do not exactly miss these examples, but the \emph{significance} of \emph{techne} in the early dialogues is often ill-appreciated. It is not always seen that Socrates' scattered reflections on \emph{techne}---which constitute a very significant part of the dialogues---constitute, in fact, reflections on knowledge. Thus so great and sympathetic an interpreter as Gregory Vlastos was able to say that Socrates was ``exclusively a moral philosopher.''\footnote{\citet[p.~41]{vlastos1991socrates}.}


On occasion Socratic epistemology \emph{is} taken up in its own right, but then \emph{techne} generally gets too little emphasis. Hugh Benson is by no means alone in saying that for Socrates, knowledge is more like understanding than like the (for us) more familiar justified true belief.\footnote{\citet[11]{benson2000swm}.} He offers some ``paradigmatic examples of the kind of understanding [he has] in mind'':\small\begin{quote}We say things like Einstein understands gravity, Richard Feynmann understands quantum mechanics, R. G. Collingwood understands history, or James Boswell understands Samuel Johnson.\footnote{pp.~211--212.}\end{quote}\normalsize But Callicles does not complain that Socrates never leaves off talking about cosmological theories; the tyrants do not warn Socrates that he should stop talking about historical figures or battles. To talk only of ``understanding'' leaves us in the dark as to why this should be. We would do better to put more emphasis on \emph{techne} in particular.

Scholarly interest in \emph{techne} has instead mainly been due to interest in the so-called ``craft analogy''---the supposed Socratic analogy between virtue and \emph{techne}. But even then the \emph{distinctiveness} of Socrates' conception of \emph{techne} often goes unnoticed. Terence Irwin, author of perhaps the most important recent book on Plato's moral theory, tells us that ``Socrates does not say exactly what he takes to be implied by saying that something is a craft [i.e. a \emph{techne}] or is similar to a craft.'' He goes on:\small\begin{quote}Aristotle, however, has a fairly clear and explicit view of the character of a craft. It is useful, then, to replace the rather imprecise question `Does Socrates treat virtue as a craft?' with the more precise question `Does Socrates treat virtue as the sort of thing that Aristotle regards as a craft?'\footnote{\citet[70]{irwin1995pse}.}\end{quote}\normalsize But Socrates certainly says enough about \emph{techne} to distinguish his conception thereof from Aristotle's, and it is a mistake to suppose that two conceptions are close enough for interpretative purposes.


%But at least the last two examples seem to me profoundly unSocratic, and the first two, though describing areas of expertise, fall rather outside the sort of areas Socrates himself deals with. 

%\footnote{R. E. Allen is right, I think, to suggest that Socrates has no place for historical ``knowledge.'' \citet{allen1989dpe} p. 326.} 

%They \emph{do} warn him that he had better lay off discussing carpenters and herdsmen and blacksmiths and the like.\footnote{Xenophon, \emph{Memorabilia} 1.2.} We just don't see Socrates dealing with the sorts of examples Benson adduces, but Benson's account leaves us in the dark as to why this should be.



%Benson's suggestion, as hinted, is that Socrates gives us, in effect, an account of \emph{understanding.}\footnote{p. 11 - Similar suggestions have been made by other scholars. [cf. NEHAMAS 1985, MOLINE 1981, MORAVCSIK 1978, BURNYEAT 1981, ANNAS 1981 pp. 192-3]} He starts by offering some ``paradigmatic examples of the kind of understanding I have in mind'':\begin{quote}We say things like Einstein understands gravity, Richard Feynmann understands quantum mechanics, R. G. Collingwood understands history, or James Boswell understands Samuel Johnson. (211-212)\end{quote}No doubt \emph{we} say things of this sort, but at least the last two examples seem to me profoundly unSocratic, and even the first two, though describing areas of expertise, fall rather outside the sort of area Socrates himself deals with. Callicles does not complain that Socrates never leaves off talking about cosmological theories; he \emph{does} complain that Socrates never lets up with his ``continual talk of shoemakers and cleaners, cooks and doctors'' (\emph{Gorgias} 491a). The tyrants do not warn Socrates that he should stop talking about historical figures or battles.\footnote{R. E. Allen is right, I think, to suggest that Socrates has no place for historical `knowledge.' \citet{allen1989dpe} p. 326.} They \emph{do} warn him that he had better lay off discussing carpenters and herdsmen and blacksmiths and the like.\footnote{Xenophon, \emph{Memorabilia} 1.2.} We just don't see Socrates dealing with the sorts of examples Benson adduces, but Benson's account leaves us in the dark as to why this should be.









%[[And since knowledge is \emph{techne}, the analysis of knowledge, to the extent that Socrates offers such, concerns questions about the systematicity, reliability, and rationality of \emph{technai}. \S1 of this essay is devoted precisely to describing the Socratic conception of \emph{techne}.]] 




%This is a revision of \citet{irwin1977pmt}.


%[[It is true that Socrates does not often, or in detail, discuss the nature of \emph{techne}. But it would be a mistake to suppose that Socrates' conception of \emph{techne} is the same as Aristotle's, or close enough for our purposes.]]

%Then, even when Socrates' discussions of \emph{techne} are treated as epistemology, scholars may think that \emph{techne} is only one variety of knowledge for Socrates, or that the \emph{techne}-model is abandoned, perhaps from the {Meno} on.  [[ANGELA SMITH AND PAUL WOODRUFF]]


%ALSO EG. PENNER AND BENSON FOR EXAMPLES WHERE IT'S OVERBROAD



We moderns do talk about knowledge of non-propositional varieties all the time. And Socrates talks about propositional varieties of knowledge. But, generally, once we start thinking \emph{about} knowledge, we start thinking about knowledge in propositional form. And Socrates, when he starts to think about knowledge, thinks about \emph{techne}. It is not too much to say that, for Socrates, knowledge just \emph{is} \emph{techne}. And so in this essay I describe the Socratic conception of knowledge, and argue that it \emph{is} the Socratic conception of knowledge that I'm describing. In other words, I describe the Socratic conception of \emph{techne}, explaining how it is different from other conceptions of \emph{techne} (in \S1), and I explain why that is tantamount to describing the Socratic conception of knowledge altogether (in \S\S2--4).






%[[In this talk I'll be treating---too briefly---\emph{techne} in the early dialogues---what we might call the Socratic (or early-Platonic) view of knowledge. Specifically, I'll explain what Socratic \emph{techne} is, and I'll explain why Socrates' conception of \emph{techne} represents his conception of knowledge \emph{generally}.]]



%A \emph{techne} is a skilled, often professional, practice---examples could include medicine, carpentry, and mathematical or musical arts. \emph{Technai} are prominent in Plato's early dialogues, because Socrates often appeals to them for illustrations. Scholars have been interested in \emph{techne} mainly because Socrates seems to treat \emph{virtue} as a \emph{techne}, but this has often entailed some neglect of how Socrates might have understood \emph{techne} in its own right. 







%Callicles does not complain that Socrates never leaves off talking about cosmological theories; 





%The tyrants do not warn Socrates that he should stop talking about historical figures or battles.




%\footnote{R. E. Allen is right, I think, to suggest that Socrates has no place for historical `knowledge.' \citet{allen1989dpe} p. 326.} 






%In Plato's early dialogues, the basic contrast is between, not knowledge and belief, but the \emph{technites}, the expert, and the \emph{idiotes}, the layman or private individual---or, alternatively, between these two figures' respective cognitive states: knowledge and ignorance.\footnote{e.g. \emph{Protagoras} 344c--d; cf. Isocrates \emph{Against the Sophists} [cite]. Compare the contrast between the \emph{demiourgos} and the \emph{idiotes} at \emph{Protagoras} 312b and 322b, though this is perhaps a narrower contrast: \citet[154--5, 175--6]{lyons1972ssa}.} And the central controversies concern the features a practice must have in order to qualify as a \emph{techne}---i.e. craft knowledge or expertise---and how those features are to be understood, and which domains are governed by \emph{technai}. Sceptical worries arise here, too, but they are different: what if there are not, and perhaps could not be, any experts on a given subject? And even if there are, how do non-experts identify them?




%The reason for the difference between the contemporary concerns and the Socratic concerns is in each case the same. In the early dialogues, knowledge just is, at least in the first place, \emph{techne}, and a \emph{techne} is not a likely candidate for an analysis along the lines of ``true belief plus something.'' There is no belief, certainly no \emph{one} belief, possession of which constitutes knowing carpentry, say.






%For Socrates, a craftsman has reliably correct views about the matters in which he is expert, which is a way of saying that he has \emph{knowledge} of facts related to his field. Here `knowledge' takes on a secondary sense within the Socratic picture, and a sense which fits the contemporary understanding of knowledge. This is now \emph{atomic} knowledge, constituted by those beliefs which are manifestations of the expert's \emph{techne}. In \S 2 I will set out this relationship between craft knowledge and atomic knowledge in more detail.

%But we sometimes find Socrates avowing, or attributing to others, knowledge even where it is plain that it has no connection to anyone's actual possession of a subordinating \emph{techne}. Accordingly, I consider, in \S 3, whether we ought to distinguish different varieties of knowledge in the early dialogues. But I argue that the diverse uses of epistemological vocabulary in the early dialogues do not license the attribution to Socrates of any conception of modes of knowledge other than \emph{techne}. Moreover, as I argue in \S 4, although Socrates doesn't \emph{distinguish} different varieties of knowledge, \emph{techne} is at the center of his epistemological thought, and we can fairly say that, for Socrates, knowledge \emph{is} \emph{techne}.\footnote{I assumed this conclusion just above, but it does have to be argued.}

%Or, insofar as Socrates thinks \emph{about} knowledge, he thinks in terms of \emph{techne}

%This essay is thus an exposition of the Socratic conception of knowledge. But in the course of the discussion I have also made points about the broader importance of the topic, primarily for the Socratic understanding of virtue and moral psychology, and for the interpretation of later dialogues.









%HAVE T SCRAP THIS "PROPOSITIONAL" STUFF - PUT IT JUST IN TERMS OF ISOLATED FACTS, OR MAYBE THINGS ETCs






%SOME POSSIBILITIES FOR THE INTRO:\\


%Quote the ancient indan folktale - powerful arts are dangerous - got it from Nouwen's Wounded Healer pp5--6 - from van Buitenen Tales of Ancient India 50-1\\


%MIGHT WANNA USE TIM 51D--E\\

%A QUESTION: WHY IS TECHNE SO INTERESTING TO SOCRATES? Ok, yeah, this is something I never really figured out - I mean GS was of course right that this was what people were going around claiming. So that's part of the issue. But why was techne the only possible model here? That seems like a further issue. So why was he hooked on that?\\
	%think of the way people make inappropriate appeals to science as a model for all knowledge


%think about Plato's epist in light of Alston's conceptions of epistemic justification - arguably sort of moves both deon and reliability into a sort of external thing\\


%Goldman has a paper called ``Experts: Which ones Should you Trust''
	%also there's an volume (with the G paper) called The Philosophy of Expertise ed Selinger and Crease 2006


%is his intro really popular? check. maybe at least he's a very established figure?




%Contemporary epistemologists emphasize knowledge of particular facts---let us call this `atomic' knowledge, i.e. the kind of knowledge which is constituted by a belief.\footnote{Thus Robert Audi, in a recent introduction to epistemology, can say flatly that ``knowledge is constituted by belief (of a certain kind).'' \citet{huemer2002ecr}, p. 1.} And the contrast between knowing and believing is important in contemporary epistemology because a central question is: what distinguishes knowledge from belief \emph{simpliciter}---that is, under what conditions does a belief amount to knowledge? And a traditional concern here is how and whether the appropriate requirements can be met.




%\footnote{[[but Williamson]]}

%[[going to have to address fact that some people reject this picture: so maybe: the kind of knowledge which has the same content or object that a belief has? something like that.]]

%It will be generally agreed that the belief must be \emph{true}. The more controversial question is what \emph{further} qualities a belief must possess. But a person will have some set of beliefs, some of which are true and some of which are false, and a subset of his true beliefs will in turn have the right sort of origin, or cohere properly with his other beliefs, or satisfy whatever other requirements the account in question imposes, and those beliefs will be atomic instances of knowledge. And a traditional concern here is how and whether the appropriate requirements can be met.


%And a traditional concern here is that our beliefs may in fact never meet the appropriate conditions.

%The elementary contrast in contemporary epistemology, then, is between belief and knowledge, and the principal arena of controversy is the question of what is needed beyond true belief for the possession of knowledge, and how and whether those further requirements can be met.













%\footnote{No English word is a particularly good substitute for `\emph{techne}'. One reason for using English glosses at all is that Socrates (or Plato) employs a variety of Greek words to indicate this same form of knowledge---among them `\emph{sophia}' and `\emph{episteme},' or corresponding verbs like `\emph{epistasthai}' (though these words do not \emph{always} indicate \emph{techne})---and so I have sometimes wished to avoid implying that Plato has actually used the \emph{word} `\emph{techne}' in some passage, although I think that \emph{techne} is in fact the topic.}


%In Plato's early dialogues, too, there is a contrast between knowledge and belief, and also between true belief and knowledge.\footnote{As ``early dialogues,'' I have in mind the \emph{Apology, Charmides, Crito, Euthydemus, Euthyphro, Gorgias, Hippias Major, Hippias Minor, Ion, Laches, Lysis,} and \emph{Protagoras}; I also draw on the \emph{Meno} and \emph{Republic} I.} But the former contrast is not the central one, and the latter contrast, far from being a major point of difficulty, is not discussed explicitly before the \emph{Meno.} Rather the basic contrast is between, not knowledge and belief, but the \emph{technites}, the expert, and the \emph{idiotes}, the layman or private individual---or, alternatively, between the respective cognitive states of these two figures: knowledge and ignorance.\footnote{eg. \emph{Protagoras} 344c--d; cf. Isocrates \emph{Against the Sophists} [cite]. Compare the contrast between the \emph{demiourgos} and the \emph{idiotes} at \emph{Protagoras} 312b and 322b, though this is perhaps a narrower contrast: \citet{lyons1972ssa} pp. 154--5, 175--6.}






%[[Apparently Aristotle treats ethical aspects of Euthydemus as Socratic: EE1247b11-15; Pol 1260a22-4]]


%are there other terms used the same way besides techne episteme and sophia?




%There are various ways of not knowing something: you might have no view on the matter; you might have a mistaken view; you might only have made a lucky guess. But a particularly natural contrast for us is the contrast between believing and knowing---we are interested in the difference between holding some view of things and holding, in a properly responsive fashion, a view that fits the way things really are, and so also in what responsiveness and fit amount to. There are many ways of not having craft knowledge, too---you might be totally ignorant of the very existence of a field or its subject matter, or have some amateur views about it, or you might have your own craft which is intimately connected with this other one, as with the makers and players of flutes in \emph{Republic} X; you might even have built up, by experience, some routines that provide you much of the success an expert would enjoy.\footnote{cf. \citet{brickhouse1994pss}, p.10: ``Many things are explicitly contrasted with craft:'' [].} But the basic contrast is between, not knowledge and belief, but the \emph{technites,} the expert, and the \emph{idiotes,} the layman or private individual,\footnote{eg. \emph{Protagoras} 344c-d, Isocrates \emph{Against the Sophists} [CITE]. Compare the contrast between the \emph{demiourgos} and the \emph{idiotes} at \emph{Protagoras} 312b and 322b; this is perhaps a narrower contrast: \citet{lyons1972ssa} pp. 154-5, 175-6.} and the central questions are about what expertise looks like, who has it, and how you can attain it yourself.

%It is sometimes said that Plato does not, before the \emph{Meno,} distinguish between knowledge and true belief.\footnote{eg. recently by \citet{fine2004ktb}, p. 41. (At least this seems to be the implication of what she says in the note; the remark in the text does not go so far.)} I do not believe this, but since the contrast between knowledge and belief is not the central contrast when it comes to craft, I also don't think it's very important just how explicit the earlier dialogues were about this. It would be important to draw attention to the difference between knowledge and true belief \emph{if} you had in mind a kind of knowledge an instance of which would be constituted by a belief (of the right sort).







%[[Instead the \emph{technites}---\emph{idiotes} contrast gives us a contrast between \emph{people} who have reliably correct views about, and act reliably with regard to, matters within some domain, in virtue of their \emph{techne}, and \emph{people} who merely have beliefs (or not) about that domain. Or, to put it a different way, we get a contrast between the understanding and facility an individual has regarding matters in which he has some expertise and the sort of views he has regarding matters in which he has no expertise.]]


%[[add: and how to identify experts - generally this is the form scepticism takes - and can cite the goldman paper - might be interesting to speculate as to how G ends up in the same place as S... in fact should ask him... would have to do with eg specialization - you know now I actally really like the idea of moral knowledge as being something that some ad other might have.... a social role like others - after all what a preisthood might have been but our thinking is very protestant...]]


%but notice that in Goldman the issue is "intellectual guidance" and *testimony* - whreas surely when you turn to a doctor its not typically for intellectual guidance. so still v bound up in factual/propositional knowledge

%Goldman does note the CHarmides but apparently not aware that this is basically the kind of issue that defines Socratic epistemology

%AND SEE FROM 90 THAT GOLDMAN'S ``EXPERTISE'' COULD BE UNDERSTOOD BROADLY ENOUGH TO INCLUDE EYEWUTNESSES - CERTAIN NOT GOING TO COUNT FOR P - IN GENERAL THIS OVERLAPS A LOT IN GOLDMANS'S CASE WITH ISSUES ABOUT TESTIMONY

%ok, on 91 he actually sets aside the more practical sorts of expertise as not so naturally the interest of epistemology... 

%But he begins (on 91) by defining expertise - what it is to be an expert (here again seems not to be aware of how much P already addressed this...)
	%more true and less false beliefs in a domain than others, but meeting at least some threshold; capacity to exploit and delpoy the information, forming new true beliefs in respnse to new questions etc; knows a lot of secondary stuff about evidence and views within field (this criteria has a sort of odd status); 
	
%his testing methods are on 93 - says not exhaustive, and indeed lack a number of S criteria: questioning for consistency, coherence, extensivness, (what else)

%not really clear why we went straight to "relative credibility" of experts, when a very pressing issues, rightly raised by S, is often whether there is an kind of expertise to be had here - not just in terms of virtue, but literature criticism etc...

%notice also that G is interested in how you can get K from an expert, or be justified in believing something, whereas P is more interested in just relying, not assuming you'll get k or even JB

%only read up to p93 in Goldman so far....

%AH EXAMPLES: FREEMAN DYSON ON CLIMATE CHANGE - WHERE MANY EXPERTS AGREE! OR ECONOMICS - WHERE AGAIN YOU SOMETIMES WORRY THAT NOBODY REALLY KNOWS ANYTHING THERE... S HAS CONTRIBUTIONS TO MAKE HERE IF WE THINK ABOUT K AS A SOCIAL ENTERPRISE








%So besides the \emph{technites}---\emph{idiotes} contrast we get the more distinctively contemporary contrast between beliefs and the atomic knowledge constituted by beliefs (at least in the expert's case).


%[[BUT ALSO BEHAVIOUR? AN ANALYSIS OF K-HOW?]]





%(If Socrates \emph{has} any analysis of knowledge then it was presented in the last chapter.) 

%Still the difference between (mere) belief and knowledge \emph{is} important for Socrates, only for a different reason, namely that belief issues in action, while (craft) knowledge ensures that you have the right beliefs in the first place. In \S 2 I describe this connection between knowledge and belief in more detail. Part of explicating this connection will be allowing for a sense in which Socrates can after all speak of knowledge of particular facts, which knowledge might indeed be constituted by beliefs. Briefly: those beliefs which are secured, or delivered, by craft knowledge can themselves be treated as instances of (atomic, belief-constituted) knowledge. 

%I suggested that one contrast between belief and craft knowledge was that a body of craft knowledge would have to be constituted (to whatever extent) by many beliefs, so that the content of a single (true) belief might be a single, isolated fact or proposition, while the content of craft knowledge would be or include many facts or propositions. But, because you can ``know health'' or have craft knowledge, and then also, in virtue of that craft, know that such-and-such a treatment is appropriate for a certain sort of ailment, or that some person has a certain ailment, knowledge in the early dialogues is often knowledge of a particular, isolated fact, so that knowledge need not, in this sense, have a different sort of content than belief. This may be a source of confusion, but does not undermine the suggestion that, for Socrates, knowledge is, at least in the first case, a \emph{body} of (craft) knowledge.

%[ALSO THE TB/K CONTRAST IS IMPORTANT IN THE MENO - BUT THERE I THINK THE Q IS MORE WHY DOES K MATER SO MUCH AND HOW CAN YOU GET TO K]



















%******BEGIN LIST OF ITEMS FOR POSS INCLUSION*********

%Finally I conclude with some remarks about the role of the [techne]-conception of knowledge in some of Plato's other dialogues. It is more common to discuss Socratic epistemology with Platonic metaphysics in mind than to discuss Platonic metaphysics with Socratic epistemology in mind. But the [techne]-conception continues to play an extremely important role in the development of Plato's thought, and I will want to at least set aside a couple of barriers to a clear appreciation of that point.

%[[THINK I SHOULD SAY AT LEAST THIS: OK CLEARLY PRIMARILY EXPOS, AND SPECIFICALLY K. BUT IMPLICATIONS FOR OTHER ASPECTS OF INTERPRETING S AND P. ]]

%WHAT I HOPE TO ACCOMPLISH IN THREE PARTS - 1 INTERP 2 IN RESPONSE TO OTHER INTERPS 3 DRAWING ATTENTION TO SOME POINTS OF SPECIAL CONTEMPORARY INTEREST

%As that outline of my essay shows, my task is first of all an interpretative one. I aim to describe Socrates' conception of knowledge. But I hope to do more than that as well. 

%mention controversies about whether S even has got this view here etc, and generally relative nelect even re early dialogues - even sympathizers see it going out from eg Meno. My ``reading more backwards than forwards'' remark is apt here as well.

%SOME PEOPLE DON'T THINK EPISTEME IS KNOWLEDGE AT ALL - EG MOLINE (GOOD QUOTE P.5) - HIS MAIN REASONS ARE 1. EXPLANATORY CONDITION AND 2. PUTATIVE PRACTICAL EFFECT

%FOR WHITE PKR IT'S LIKE PLATO IS GRASPING AROUND FOR AN EPISTEMOLOGICAL THEORY - CERTIANLY LIKE WE JUST PASS OVER ALL THE SIMPLER CASES WHICH ARE BASICALLY ASSUMED TO BE PERFECTLY SOUND

%AND OF COURSE FOR V IT'S LIKE THERE'S NO EPISTEMOLOGY UNTIL THE MENO

%[[want to observe how Vlastos---a great and I think sympathetic interpreter---thought that Socrates was no epistemologist - apparently because S's epistemology s just in some sense so foreign! - and expalin a bit how this could be missed - eg fact that often there are just these free propositional uses.]]

%[[not perhaps an intrinsic interest in K to same extent as P - contrast eg Theaetetus with early dialogues, and yet...]]

%[[[Basically this is an account of Socrates conception of knowledge. But I like to think that it is more - given the delving into the akrasia stuff - one of the more famous episodes in the history of philosophy, but not I think well understood or seen in the sympathetic light it deserves. At connections with issues in contemporary epist.]]

%Allen and Prior with earlier theory of forms - A does I guess have this as sort of basis of a moral expertise?

%It is frequently and rightly said that Plato's epistemology and metaphysics have to be understood together. But what this usually means is that his epistemology has to be understood in terms of his metaphysics. Less rarely is the emphasis on 

%of course often felt S has no epistemeology

%even people interested in Socratic epistemology are thinking in terms of defs usually - eg Benson

%even people thinking of Socratic epist in terms of craft think this is sort of altering with meno or somewhere like that (Woodruff, Smith) or somehow breaking in the Republic (Kube - well actually already in the Meno I guess for reasons similar to Woodruff maybe... - actually complicated because he also says that it's the only true techne in some sense... - seems to thnk it breaks in some way that has the techne model being itself used to subordinate the original examples? ie even the demand for an account is somehow something that in a certian sense the other crafts can't offer...) - check also Bambrough - Reeve 8, 19

%Reeve does see the forms as emerging from the techne epistemology - see \citet[212]{reeve2000rot}

%[[[AND GENERALLY WHAT I HOPE TO ACCOMPLISH WITH THIS ESSAY.]]

%GENERALLY HAVE THIS STUFF IN A PARA ON THINGS OF CNTEMP INTEREST I HOPE TO DRAW OUT:

%for two reasons I will dwell on the relationship between craft knowledge and atomic knowledge for longer than would strictly be necessary simply from the perspective of understanding Socrates' conception of knowledge. The first is that the elucidation of that relationship requires a discussion of a passage in the \emph{Protagoroas} in which Socrates denies that there is such a thing as \emph{akratic} action. Socrates' rejection of akrasia is one of the more famous episodes in the history of philosophy, but it is poorly understood, and can be made more sympathetic than might be thought, once seen in connection with the Socratic conception of knowledge. The second reason is that while Plato has generally been credited with being first person to clearly distinguish knowledge from true belief, it has not been noticed that he can also be attributed with an analysis of (atomic or propositional) knowledge: a belief is a case of atomic knowledge when its production or maintenance is a manifestation of craft knowledge. This seems especially worth drawing out a bit in light of recent suggestions that knowledge and belief are or entail different mental states (so that knowledge is not a variety of belief). Because although I have described atomic or propositional knowledge as knowledge constituted by belief, Plato's actual view is, at least arguably, that beliefs are the judgments of non-experts, knowledge in the atomic sense represents the judgment of the expert---and these two form of judgment need amount to the same sort of mental state. 

%[[so P would have one articualtion of this idea ---- but give less away here, also I really would rather have this in the end so that the paper seems more to culminate in this - NOTE ALSO I HAVE A PARALLEL WITH VIRTUE EPIST THERE]]

%MAYBE ALSO THE ANSCOMBE ISSUE?

%******END LIST OF ITEMS FOR POSS INCLUSION*********
















%Interesting: Halbfass' ``The Therapeutic Paradigm'' discusses medicine/philsophy analogy in India - also further references. I seem to remember some general craft/philosophy analogies in CHinese stuff and I'll bet there's some literature - I don't know about medicine specifically...



\section{Techne in Plato's Early Dialogues}

%SOME POSSIBLE SHIFTS HERE:\\

%1. MAYBE A BIT MORE TIME EMPHASIZING SOME SHIFTS THAT WILL OCCUR IN EG PHAEDRUS AND REP

%2. MAYBE MORE CARE RE CRAFT VS CRAFT POSSESSION?\\

%3. ONE THING PROB WORTH EMPHASIZING HERE OR IN CHUNK ON VIRTUE (ACTUALLY MAYBE A REASON TO KEEP THEM TOGETHER) IS SOME TENSIONS THAT EXIST HERE - THAT WAY I OFFER MORE ABOUT HOW TO GO FORWARD WITH THIS ETC, AND IT BECOMES MORE THAN THE EPISTEMOLOGICAL THING\\

%4.ONE THING ABOUT THE EUTHYDEMUS THAT'S PERHAPS MORE PLATONIC IS THE IDEA OF A SORT OF UNVERSAL TECHNE RESTRICTED TO FEWER PEOPLE - THOUGH I GUESS THE SPECIALIZATION WAS THERE IN APOLOGY - BUT PERHAPS ONLY AS A HINT AT AN ISSUE\\


%\subsection*{Introduction}


%And we can do better than to make that supposition.

%---just as theoretical topics in general are not part of his repertoire.


%Their questions---what would it mean for virtue to be a \emph{techne}? and how seriously is the comparison meant?---tend to focus more on virtue than on \emph{techne}. 



%CUTTING: And it does matter that we do as well as we can. Besides making a difference to how we understand the Socratic understanding of virtue, it will matter for other points of interpretation regarding, for example, akrasia, and the various elements of Socratic epistemology generally, as we will see. A proper understanding of craft knowledge and the role it plays in the early dialogues will also prepare us for a better understanding of epistemological themes in the \emph{Republic} and other dialogues. So we should settle for nothing less than a concrete understanding of \emph{Plato's} (or Socrates') conception of craft, rather than somebody else's. And characterizing the conception of craft found in the early dialogues is thus the task of the present section.


%wait - are oratoray and rhetoric diff things?



As `\emph{techne}' was a term of commendation, there were disputes about whether given practices deserved the title.\footnote{Isocrates' \emph{Against the Sophists} and the Hippocratic \emph{On Ancient Medicine} discuss oratory and medicine respectively. \citet{heinimann1961vpt} treats the situation prior to Plato.} Socrates himself engages in such disputation: he views medicine as a genuine \emph{techne}, but raises questions about the status of poetry, `cookery,' and rhetoric. Nor is it only Socrates raising such questions in the early dialogues: in the \emph{Laches}, Nicias assumes that fighting in armour is a \emph{techne}, while Laches has doubts.\footnote{181e--184c. Neither Laches nor Nicias actually uses the word `\emph{techne,}' when speaking of fighting in armour; rather they refer to it as a (possible) subject (\emph{mathema}), or as knowledge (\emph{episteme}). Now `\emph{episteme}' is a frequent synonym for `\emph{techne}' in the early dialogues. And Socrates is treating `\emph{mathema}' as implying a \emph{techne} at 185e, and when he begins speaking of \emph{techne} back at 185a, Nicias assumes that the \emph{techne} in question would be fighting in armour (185c).}

Disagreements as to which practices were \emph{technai} were complicated, and engendered, by the fact that different people had different ideas about what a \emph{techne} should be or do. For example, when you hire a craftsman, should you be able to expect that he deliver what you hire him for? That would be a natural expectation of a builder, but what about a doctor? And should a genuine \emph{techne} be theoretical, and in what way theoretical? Or might a mere body of experience also constitute a \emph{techne}? Thus, to take one example, we find the author of the Hippocratic \emph{On Ancient Medicine} (hereafter \emph{VM}) arguing that medicine's status as a \emph{techne} does not depend on incorporating into it the sorts of physical theories expounded by the natural philosophers.\footnote{\citet[63--4]{schiefsky2005} argues for dating \emph{VM} to the late fifth century.}

%Socrates, too, takes positions on such questions: especially notable is his claim in the \emph{Gorgias} that \emph{techne} is explanatory.

%, contrasting his view with that of Polus, who had said that \emph{techne} was a product of experience.

By attending to the debates about the nature of \emph{techne} which we find in the contemporary literature, and to various remarks about \emph{techne} in the early dialogues, we can identify a distinctively Socratic (or early-Platonic) conception of \emph{techne}. Here in \S1 I discuss three important features of \emph{techne} as Socrates understands it, which together give us a good general picture, and each of which emerges in discussions about whether certain practices constitute \emph{technai} or whether certain individuals possess a \emph{techne}. I begin with the scope of craft knowledge, in \S1.1. Within its proper domain, Socrates says that ``wisdom would never err'' (\emph{Euthydemus} 280a), so I try, in \S1.2, to explain what that means. For Socrates the reliable attainment of genuine goods is in part a result of the craftsman's rational and articulate understanding, which I discuss in \S1.3. In \S1.4 I deal briefly with some problems stemming from the Socratic treatment of virtue as a \emph{techne}.\footnote{A few scholars have recently taken more of an interest in the Socratic conception of \emph{techne} in its own right, and as a matter of \emph{epistemology} rather than just ethics---most notably Paul \citet{woodruff1990pse} and Angela \citet{asmith1998}. A useful discussion with more attention to the contemporary views about \emph{techne} may be found in \citet[37--45]{reeve1989sita}, though Reeve's discussion generally gives the (in my view mistaken) impression that there was \emph{a} going view of \emph{techne,} albeit one which underwent some development.}

%And, as it turns out, the conception we can identify is not in all particulars the same conception of \emph{techne} which we find in other contemporary literature, nor is it the same notion of craft as is assumed by Socrates' interlocutors, nor is it what we find in Aristotle, nor is it even exactly equivalent what we find in Plato's own later dialogues.


%In \S1.4 I discuss the relationship between the rational character of Socratic craft and Socratic inquiry, and some possible implications for the teaching of craft. %I conclude with some remarks about the relationship between craft and virtue (\S1.5).%[[LAST BIT POSSIBLY GONE]] %BUT MAY INCLUDE IT IF I WANT TO SHOW THAT THE CONCEPTION OF K HAS SOME TENSIONS IN IT - ACTUALLY THAT MIGHT BE A GOOD USE OF 1.5!! ISSUE THEN WILL BE HOW TO RELATE THIS TO THE AKRASIA THING

%Have to decide what I think about the whole intellectual aspect - ie how to hold on to it but to satisfy Gisela at the same time.

%[I have chosen the topics I have thinking them the most important for an understanding of the Socratic view of craft, but also with an eye to their further significance, either to the other matters dealt with in this essay or simply to Plato's epistemology in general. For example, knowledge continues, after the early dialogues, to be treated by Plato as in some sense reliable, but the early dialogues best help us understand what that reliability might amount to. Again, Socrates' understanding of craft teaching is important because it highlights a development in later dialogues in the direction of greater emphasis on less intellectual aspects of craft and virtue. The question of the value of craft is important for deciding whether Socrates thinks that virtue is a craft.]



\subsection{The Scope of Techne}

One may get at the nature of a \emph{techne} by way of its subject matter. Thus Socrates asks Gorgias: ``with which of the things there are is oratory concerned?''\footnote{449d. Trans. Zeyl.} And a \emph{techne} is systematic, so that its subject matter (or \emph{pragma}) has a natural integrity.\footnote{On systematicity see also \citet[71--2]{woodruff1990pse}, \citet[135--6]{asmith1998}.} In the \emph{Laches} the proposal that courage might be knowledge of future goods and evils is rejected on the grounds that there is no such thing as knowledge concerned with farming \emph{in the future} in particular, or with health or strategy \emph{in the future.} The topic will just be health \emph{period}, or strategy, or farming, or goods and evils generally (198d--199d). Similarly Ion is taken to task for the suggestion that he knows and can interpret only the works of Homer, but that as far as other poets go he ``has no power to contribute anything worthwhile'' (\emph{Ion} 532b-c). One learns, or should learn, ``a subject as a whole'' (532e); i.e. poetry, or sculpture, not specific poets or sculptors.


%This criterion is invoked especially in the \emph{Laches} and the \emph{Ion}.

%CRAFT K VS CRAFT POSSESSION IS ISELF ONE OF THE SHIFTING POINTS FROM HERE TO REPUBLIC, OF CORUSE



%[HEINIMANN], [KUBE], OTHERS? can be added re arithmetike as representing an ideal


%---There is some tension between, on the one hand, this idea that every craft participates in \emph{arithmetike}, which is itself a craft, and, on the other hand, the earlier idea that no craft is possessed by everyone. Again \emph{arithmetike} is an exceptional case, but regardless the tension does nothing to undermine the earlier point, which was that as far as the early dialogues go teaching seems to be fundamentally a theoretical endeavor. If anything it supports that point.

%449e: "Rhetoric is the knowledge of what?"


%Systematicity may be understood as having what we might call a `vertical' as well as a `horizontal' direction. Horizontally the issue is generality---poetry, not poets; health, not health at a given time. Vertically, knowing a subject---health, let us say---means not just being able to make the sick well but being able to explain the principles of the \emph{techne}, to teach the \emph{techne}, and to act as an arbiter on questions about the domain governed by the \emph{techne} (eg. \emph{Gorgias} 450). And these capacities come as a package, it seems: even to claim be able to ``speak well'' on a given topic is \emph{thereby,} in Socrates' view, to claim craft knowledge in that domain. Because of this vertical direction of systematicity, Ion fails not only on generality but also, like Homer, gets his fingers into too many pies. He claims to speak authoritatively on sailing and racing and military affairs and so on (all the subjects Homer addresses), but of course nobody could really be expert in all of these things at once.%\footnote{Woodruff says that the `specialization' criteron leads to paradox, because [DETAILS - P71-2 - JUST DON'T SEE IT MYSELF.]}


%HAVE TO DECIDE HOW TO FIX FOLLOWING PARAGRAPH

Socrates characterizes the subjects of craft knowledge in fairly intuitive terms: the odd and the even, health, houses---often it's a product the craftsman can bring about. Socrates tells us in the \emph{Ion} that when he finds forms of knowledge dealing with different subjects (\emph{pragmata}), then he calls them different \emph{technai}, while the same subject means the same \emph{techne} (537d--e). He illustrates the point by observing that both he and Ion know that Socrates has five fingers on a hand, and it's by the same \emph{techne}---namely arithmetic---that they know this. But this idea about distinct subject matters shouldn't be interpreted too strictly.\footnote{cf. \citet[xx]{kahn1996pas}.} In the \emph{Gorgias} both gymnastics and medicine concern the body, but the one provides preventative care while the other is restorative. Similarly, at \emph{Gorgias} 451b--c we find that both \emph{arithmetike} and \emph{logistike} have as subjects ``even and odd''; the further difference seems to be the \emph{way} these \emph{technai} deal with number, or the use they put it to. So we are working here with the intuitive idea that each \emph{techne} has, like a science, its own proper field, along with its own particular principles and point.\footnote{Arithmetic is a special case, representing an ideal of precision (\emph{akribeia})---see \citet{roochnik1994cnp, roochnik1996aaw} and \citet[13--18]{schiefsky2005}. Indeed Plato will later say that every \emph{techne} must participate in \emph{arithmos} and \emph{logistike} (\emph{Republic} 522c; cf. \emph{Philebus} 55e ff.). But exceptional or not, my point holds.}

% Another example: in the \emph{Charmides}, Socrates says that the subject of calculation is the odd and the even and that the subject of weighing is the heavier and the lighter (166a--b), but of course this doesn't mean that weighing doesn't also involve knowledge of the odd and the even---thus in the Protagoras it follows from the employment of measurement that arithmetic is also employed (357a).

%The basic idea here is an intuitive one as well, namely that a craft, like a science, is inherently integrated and general. While economics and psychology might be genuine sciences, there is no science of, say, the life and times of Haile Selassie specifically. That would be a topic at once too particular and too amorphous for any one science. And likewise Homer and his poems are not a suitable subject matter for a craft.

\subsection{The Reliability of Techne}

%WIll have to fill this out with the Allen stuff

%Within the area of their expertise, craftsmen may be relied upon to perform the services the offer. But just \emph{how} reliable craftsmen should be was a matter of dispute.\footnote{\citet{allen1994} provides an excellent short survey of this topic.} High standards seem to have been the norm, and Plato's dialogues are no exception.

A \emph{techne} was expected to bring about success within its proper area, and proneness to failure could call the \emph{techne}-status of a given practice into question.\footnote{\citet{allen1994} provides an excellent survey of this topic.} Nevertheless some forms of failure could be thought compatible with \emph{techne}. One possibility is that, whatever might ideally be accomplished through \emph{techne}, any given craftsman might make mistakes, as a result, perhaps, of inexperience or lack of aptitude. Socrates allows for the possibility of this kind of failure, for example at \emph{Euthydemus} 279e--280a, where he contrasts wise and foolish doctors or generals. But so far that says nothing about potential limits on \emph{techne} proper. 

Another type of failure comes out in a supposed contrast between building (\emph{tektonike}) and medicine, the former of which seems to have been an early paradigm of \emph{techne.}\footnote{\citet[19--20]{roochnik1986}. Cf. Isocrates (\emph{Against the Sophists}) on writing out letters.} There was a trope to the effect that when you call the builder you will get a house---and you won't get a house otherwise. But other putative craftsman aren't always successful. Doctors don't always restore people to health, and, what's more, people are sometimes restored to health without the aid of doctors. On such grounds some denied that medicine was a \emph{techne} at all. The orator or sophist, too, was susceptible to such charges, especially given his sometimes extravagant claims to be able to teach virtue. Again it was observed that these teachers were not always successful (how interesting that they insist on securing payment up front!), and that their students were sometimes surpassed by the untrained.\footnote{For this sort of objection see e.g. \emph{Protagoras} 319a--320b, \emph{Laches} 185d--e, Isocrates \emph{Against the Sophists} 14, \emph{Dissoi Logoi} 6.5--6, \emph{De Arte} 4; cf. \citet[123--6]{heinimann1961vpt}.}

%might check Schiefsky p10, Diseases and VM

%\footnote{[CITES for the trope about building- could add the the letters example]} 


%fix the refs in forgoing ¶ trope was from Allen perhaps? Or Roochnik?s

In the Hippocratic \emph{De Arte} the response to such charges is as follows. Nature (\emph{phusis}) sets limits on what is possible for the medical art. For example, fire is the ultimate cauterizer available to medicine---if this cauterizer is insufficient in some cases, that is no fault of medicine. The doctor still knows what can and cannot be helped, and what to do in cases that \emph{are} amenable to treatment. Now it is important that limitations are blamed on \emph{phusis} here. \emph{Techne} was often viewed as a way of mastering or overcoming \emph{tuche}---i.e. chance or luck.\footnote{\emph{Gorgias} 448c, \emph{Euthydemus} 279ff., \emph{De Arte} 4--7, \emph{VM} 1, 12, \emph{Places in Man} 41, 44, 46, Xenophon \emph{Memorabilia} III.ix.14--15; cf. \citet[54--6]{vlastos1946eap}, \citet[108, 123--4]{heinimann1961vpt}, \citet{allen1994}.} So we get a distinction between the vicissitudes of fortune, against which medicine, like any good \emph{techne}, is insurance, and the inexorable workings of nature, which the doctor will understand but could not be expected to resist. (As for the fact that some people get better without a doctor, that may be credited to their doing what the doctor \emph{would} have recommended.\footnote{\emph{De Arte} 5--6.}) Now \emph{De Arte} still offers us an impressively demanding standard for \emph{techne}: Allen observes that the defender of this view ``seems to envisage . . . a physician who will at no point undertake cures with less than complete confidence of success'' (85).


%Less ambitious attitudes towards \emph{techne} and \emph{tuche} may be found in other Hippocratic writings.

%[[KEEP THIS COMPARISON TO BUILDERS??]]And to that extent, the doctor compares favourably with the builder. Perhaps you cannot summon a doctor and expect to wind up healthy the way you can summon a builder and expect to wind up with a house, but nature places limits even on the builder---he cannot pull houses out of the air, after all.




%More techne-tuche (but have to check): Anthologia Graecae 7.135 - in Loeb 2:77-9 (some kind of epitaph for Hippocrates - got it from Jouanna p37
	%Also 54b13
	
%Plato laws 889b-c (techne-tuche and the divine craftsman
	

Socrates' understanding of the reliability of \emph{techne} seems to accord with the sort of treatment found in \emph{De Arte}. On the one hand, Socrates treats medicine as a paradigmatic \emph{techne}, and there's no reason to suppose that he or anyone else would have thought that a doctor could heal anyone no matter what their injuries or age or health. So Socrates could hardly have a \emph{more} stringent view than the one found in \emph{De Arte}. On the other hand, his view seems to be no \emph{less} stringent either. In the \emph{Euthydemus}, Socrates, adducing a variety of \emph{technai}, deploys the notion that \emph{techne} is a guard against bad \emph{tuche} (279d ff.), finally going so far as to say that \emph{sophia} (here a synonym for \emph{`techne'}) \emph{is} (good) \emph{tuche} (279d), or at any rate provides it (282a). Here Socrates, like the author of \emph{De Arte}, tries to push luck out of the picture altogether.\footnote{At \emph{Protagoras} 344c--d, Socrates says misfortune (\emph{sumphora}) might incapacitate a craftsman but not an ordinary person (who has no \emph{techne} to start with). Here misfortune interferes with \emph{techne}, but by destroying it, not by preventing it from attaining its goals.} At \emph{Euthydemus} 280a, `being lucky' is glossed as `never making mistakes,'\footnote{``Wisdom makes men fortunate in every case [of \emph{techne}], since I don't suppose that she would ever make any sort of mistake but must necessarily do right and be lucky---otherwise she would no longer be wisdom.''} and at least this latter characterization of \emph{techne} is found elsewhere, for example at \emph{Protagoras} 357d--e. It is also found in \emph{Republic} I, where Thrasymachus says that craftsmen, insofar as they are craftsmen, never make mistakes (340e), and Socrates falls in line with this, too (342b). The context is the question whether a ruler will ever mistakenly issue laws which in fact turn out to be detrimental to himself, so the point seems to be that if a \emph{techne} is being practiced properly \emph{then the craftsman should always be successful} (within the realm of possibility, presumably). So again the possibility of sheer misfortune is pushed out of the picture.\footnote{cf. Allen pp.~85--6. This view, and the thesis that virtue suffices for happiness which it is meant to support, is open to the sorts of objections raised later against the Stoics by Alexander (\emph{Supp. De Anima} 160.1 ff.). The argument in the \emph{Euthydemus,} is, moreover, clearly fallacious. But the conclusion that virtue is sufficient for happiness is after all Socratic, and the case of \emph{De Arte} shows that this position was a contemporary possibility.}



%[MS on pushing out tuche---de arte ect].---COuld say that S is actually more demanding...

%A brief summary may be found in \citet{cuomo2007tac}, pp. 18-22. - [eg Might exclude eg Epidemics, Breaths - Cuomo p15][and also: see \citet{cuomo2007tac}, pp. 16-17 on harmony of nature and techne.]



%Now the argument here is clearly fallacious, and this is a largely comic dialogue. And the argument is actually capped at 280a-b with Socrates' slightly peculiar remark that ``we finally agreed (I don't quite know how) that, in sum, the situation was this: if a man had wisdom, he had no need of good fortune in addition.'' The ``I don't quite know how'' might suggest that Socrates is more invested in the intended conclusion about the sufficiency of virtue for happiness than in the argument from crafts which he has appealed to. 






%On the limits being set by Phusis see also Heinimann 122


Later Plato will take a more generous view. While in the \emph{Gorgias} Socrates insists that mere experience and guessing are the marks of something \emph{less} than \emph{techne} (462b--c, 465a, 501a), the \emph{Philebus} has him distinguishing a class of ``less precise'' \emph{technai} which rely on guesswork (\emph{stochasmos}) rooted in experience, which would thus be less reliable (55e--56a). A later development still, with Aristotle and other writers, was the notion of a stochastic \emph{techne} (which need not be untheoretical).\footnote{On stochastics \emph{technai} see \citet[88 ff.]{allen1994}.} Here we would distinguish between competent application of the methods associated with a \emph{techne} and actual successful outcome, or between the exercise of a \emph{techne} and the aim of that \emph{techne}. This would seem to offer a more appropriate model for medicine: we would no longer expect the doctor to know in every case whether or not the treatment will succeed; medicine would provide no \emph{absolute} guarantee against luck. But, again, these are later developments.

%\footnote{Aristotle \emph{Topics} 1.3 (101b5-10); \emph{Rhetoric} 1355b10-11, 25-6 [check and maybe merge with prev note]} 

%Check what Allen said about this - he might have generally seen Plato as more tolerant.



%Treating these as equivalent leaves no room for stochastic crafts, and at least the latter characterization of crafts is found elsewhere, and 3) accounts of stochastic crafts seem to have been a later development in any case.


%We find this conception of medicine described by Alexander of Aphrodisias, to take one example.\footnote{[cites - check on tuche - aristotle. Also other Hipp authors more cautious.]} Let us then take this notion of stochastic craft as providing our most generous conception of the sort of failure compatible with genuine \emph{techne}. And of course either this or the Hippocratic conception is compatible with failure attributable rather to the nominal craftsman than to the craft itself. Our question, then, is what kind of fallibility is countenanced in the early dialogues.

%It may be, for example, that the materials with which the stochastic craftsman works are liable to variability which is undetectable prior to the application of his methods. 



%Now the debate about stochastic crafts seems to be a later one, emerging as a notion particularly with Aristotle and more sharply in later writers, and it would be pointless to press the dialogues too far for Plato's answer to a question he may never have asked; nevertheless it does seem that Plato would generally resist recognizing stochastic crafts and align himself rather with the Hippocratic picture. Stochastic crafts are \emph{compatible,} as Allen observes, with the sorts of demands made at \emph{Republic} 340d, 341d, and 345d, but the remarks there do not go so far as showing that Plato would in fact consider stochastic crafts real crafts. Allen also says that ``critics of the stochastic practices were sometimes willing to grant that there was a weaker and more extended sense of the term according to which the stochastic practices qualified as arts,'' and cites \emph{Philebus} 55e-56a, which distinguishes a class of ``less precise'' crafts, which rely on guesswork (\emph{stochasmos}) rooted in experience. 



%Now on one understanding of stochastic craft it is indeed based on experience, but on another conception of stochastic craft it might be just as theoretical as any other craft.\footnote{On these understandings see Allen, esp. \S\S V-VII.} The \emph{Philebus} passage seems indeed to grant a qualified craft status even to stochastic crafts on the former, less theoretical understanding.

%The early dialogues, however, would seem to reject any putative craft based only on experience: Socrates insists that that mere experience and guessing are the marks of something \emph{less} than craft (\emph{Gorgias} 462b-c, 465a, 501a), as against Polus who declares that ``it is experience that causes our times to march along the way of craft, whereas inexperience causes them to march along the way of chance'' (448c; cf. 462b-c). Socrates thus shows less deference than Plato does in the later \emph{Philebus.}




%Again fix cites and check stuff besides De Arte = MS says place in man I think as well has this notion?






\subsection{Techne and Explanation} For Socrates, \emph{technai} are reliable because they are rational. In the \emph{Gorgias} Polus says that ``it is experience that causes our times to march along the way of craft, whereas inexperience causes them to march along the way of chance,'' and while we see the common contrast between \emph{techne} and \emph{tuche} here, Socrates finds this statement inadequate.\footnote{448c; cf. 462a and Aristotle \emph{Metaphysics} 981a.} Thus he later says that when some practice ``has no rational account by which it applies the things it applies, to say what they are by nature, so that it cannot say what is the explanation of each thing,'' then that practice is merely a product of experience (\emph{empeiria}), and not a \emph{techne}. Socrates goes on: ``I don't call anything a craft which is unreasoning'' (\emph{alogos}---465a). He derides Polus' own `\emph{techne}' of rhetoric, comparing it to ``cookery,'' which ``by habit and experience . . . keeps only memory of what usually happens.''\footnote{501a. A view of \emph{techne} according to which it did just consist in experience was influential in later centuries. But quite likely Polus doesn't himself mean to be defending a peculiar and disputable view of craft (\citet[346--8]{schiefsky2005}). Aristotle mentions Polus' statement as if it were an unobjectionable truism (\emph{Metaphysics} 981a).}

%, aided by skeptical philosophies which rejected the possibility of anything more theoretical. 


A craftsman then understands the nature of the things he is dealing with. A doctor, for example, would know ``what sight itself is'' (\emph{Laches} 189e--190a). The same expectation is strongly intimated at \emph{Gorgias} 459c--e. It has been observed that the orator may be more persuasive concerning medicine even than a doctor, at least among the ignorant, but he ``isn't knowledgeable in the thing in which a doctor is knowledgeable'' (459b). Socrates asks whether\small\begin{quote}the orator is in the same position with respect to what's just and unjust, what's shameful and admirable, what's good and bad, as he is about what's healthy and about the subjects of the other crafts? (459d)\end{quote}\normalsize The doctor, by implication, \emph{would} know health and whatever else his \emph{techne} concerns, and a rhetorical \emph{techne} ought to be similar.\footnote{cf. \emph{VM} 20: doctors know ``what the human being is in relation to foods and drinks, and what it is in relation to other practices, and what will be the effect of each thing on each individual.'' \citet{dodds1959pg} and \citet{irwin1979pg} think the use of `\emph{prospherein}' at 465a suggests Plato had medicine in mind there too, which is perhaps borne out by 501a.}

In the \emph{Gorgias,} the contrast between \emph{techne} and \emph{empeiria} is treated as a contrast not just between understanding and mere experience, but as a contrast between \emph{aiming at something good with understanding} and \emph{aiming at mere pleasure by guesswork} (464a--465d, 500a--b). I take the point to be that if you don't understand the nature of your aims and of your subject matter, then you're in no position to know what would be the ``good condition'' of the things you treat.\footnote{464a, cf. \citet{irwin1979pg} 500e--501a \emph{ad loc}, and \emph{Laches} 189e--190a.} Thus you cannot aim at that good condition but only, and only by guesswork, at what will please some person at some time. We might contrast an idealized cobbler, who knows the support a foot needs and how to provide it, with a shoe designer, who is concerned with aesthetic pleasure rather than comfort or health. The cobbler's principles are always the same; the trends are ever-changing and unpredictable.\footnote{Cf. \citet{moss2006pai} on the relationship between pleasure and illusion in Plato.}

%The explanation criterion thus incorporates an ``aiming at the good'' criterion.

%\begin{greektext}prosf'erein\end{greektext}

%VM 22 was related


Socrates' interest in \emph{what things are} has been described as an interest in real as opposed to nominal definitions. This is fair in the sense that Socrates expects experts to be be able to articulate their knowledge. Doctors and laymen alike \emph{talk} about health, and can point to examples of healthy people, or give general characterizations of health---as ``good condition of the body,'' for example. But the layman relies on more superficial signs, and is liable to error, while the doctor can tell healthy people apart from those who merely appear healthy (\emph{Gorgias} 464a--b).\footnote{Cf. \emph{Euthyphro} 6e, where piety is a \emph{paradeigma} or model one may look to.} The doctor's ability to give a ``real definition'' of health is a reflection of this ability to distinguish appearance from reality. So to know what the qualities in question are is, in part, to be able to employ \emph{and explain} reliable standards of assessment. %An answer to the question ``what is X'' that did not reflect such an ability would not be an adequate answer to the question as Socrates understands it.

%Thus at \emph{Euthyphro} 6e Socrates speaks of piety as a \emph{paradeigma} or model to which one may look, so that knowledge of piety means being able to tell whether or not a given action is pious. 


%This suggests that whatever Charmides or Euthyphro could \emph{say} about temperance or piety would represent only a reflection or aspect of their knowledge.

The interest in definitions should not be construed as an interest in knowledge which may be captured by a verbal formula.\footnote{Socrates is as happy with expressions of the form ``know X'' as ``know what X is''---he thinks of knowledge as much in terms of acquaintance or perception as in terms of anything linguistic. See e.g. \emph{Charmides} 159a, \emph{Euthyphro} 6e, \emph{Gorgias} 503e.} Knowledge of a given entity or quality, in the sense Socrates is concerned with, already indicates mastery of the whole rational practice that constitutes a \emph{techne}.\footnote{See, for example, \emph{Laches} 190a--c and \emph{Gorgias} 503d--e. Indeed a phrase like ``knowing X'' can simply be used synecdochally to characterize a \emph{techne} as a whole by reference to its proper object---thus medicine is the ``knowledge of health,'' or the ``knowledge of health and disease'' (e.g. \emph{Charmides} 165c, \emph{Gorgias} []).} But because the expert can articulate his reasoning, putative experts may be interrogated. And so we have the familiar type of Socratic question: ``what is courage?'' or ``what is piety?''\footnote{This sort of test is prominent in \emph{Euthyphro,} \emph{Laches,} \emph{Charmides} and \emph{Hippias Major}.} As Angela Smith points out, requests for definitions follow declarations of expertise: the requests are a \emph{test}.\footnote{p. 149. Though some dialogues, e.g. \emph{Charmides,} have the character more of a shared investigation---here it is an idea rather than a purported expert that is being tested.} And this is only \emph{one} Socratic test for expertise. In some cases we can just \emph{tell} whether a wall or a shoe, say, is a good one. So one way of telling a craftsman is by his work; another is by his pedigree, or his students.\footnote{\emph{Euthyphro} 16a, \emph{Laches} 185b--187b, \emph{Gorgias} 514a--515a.} In several dialogues we see Socrates trying to elicit a suitable delineation of the scope of some supposed \emph{techne}.\footnote{E.g. in the \emph{Ion,} the first third of \emph{Gorgias,} and the later parts of \emph{Euthyphro} and \emph{Laches}.} But the influential notion that Socrates is primarily interested in deductive knowledge, or knowledge of definitions, seems to be a result of taking one of Socrates' investigative tools as too much representative of his whole conception of knowledge.\footnote{Cf. \citet[153]{asmith1998}: ``too much emphasis has been placed upon the formal features of the Socratic elenchus, and not enough on the purpose and spirit of it.''} 



%Socrates'\begin{quote}request for definitions is, in effect, a request for proof that these self-proclaimed experts have the knowledge that they claim to have.\end{quote}


%In asking Charmides what temperance is, Socrates tells Charmides that he must be able to gain an impression from its presence in him (159a); Socrates tells Euthyphro that he wants to ``look upon'' the form of piety (6e---cf. again \emph{Gorgias} 503e).


%A source of confusion here is the role that definitions play in Socratic \emph{investigation}, as opposed to their role in the Socratic conception of \emph{techne} proper. 


%The idea is that any supposed craftsman who claims to have quite exact knowledge about, say, piety, should be able to \emph{say} what piety is, since, as we have seen, craft knowledge is articulate, and the craftsman knows the qualities and entities with which his craft is concerned.



%Knowing the qualities appropriate to one's [techne] entails sound judgment about their instantiation; it also means being able to bring them to realization (where appropriate). Indeed the entailment goes in both directions. For the capacity to knowledge direction, we can look at the \emph{Laches}, where Socrates says \begin{quote}suppose we know that sight, when added to the eyes, makes better those eyes to which it is added, and furthermore, we are able to add it to the eyes, then clearly we know what this very thing sight is . . . (190a)\end{quote} We can see from further on in the same passage that Socrates supposes the reverse to hold as well: that if you know sight then you know its value and how to bring it about. Thus, having just offered the example with sight, he tells Laches that they had better see whether they know what virtue is, since otherwise they would not be able to advise on it (190b-c)---and being able to advise on a topic goes along with being an expert yourself, as we saw in \S1.1. The assumption then is not just that if you're an expert you'll know what the relevant qualities are, but if you know the qualities then you're an expert. The same assumption is operative in the the \emph{Gorgias} (at 503d-e), where Socrates portrays the craftsman as \emph{looking to} something, his actions guided by his target, until he imposes the appropriate form on his work.\footnote{Plato continues to express himself in these ways in later dialogues, once the primary objects of knowledge have become full-blown Forms, even in connection with mundane crafts. In the \emph{Cratylus} Socrates has the weaver looking to the Form of the shuttle in creating a new one; in \emph{Republic} X the carpenter looks to the Form of the bed (\emph{Cratylus} 389a-b, \emph{Republic} X 596b). But in later dialogues there is a greater gap between forms and the things that share in them: you can't so much realize a form in a thing as mimic it.} So to know what sight or virtue is is to be equipped with a whole set of appropriate capacities.







%pp. 129-130, and 153


%Now there are other, simpler, ways of identifying craftsmen. In at least some cases we can tell whether a wall or a shoe, say, is a good one, or whether someone is healthier than he was before. So one way of telling a craftsman is by his work; another is by his pedigree, or his students.\footnote{\emph{Euthyphro} 16a, \emph{Laches} 185b-187b, \emph{Gorgias} 514a-515a.} But generally we see Socrates dealing with more controversial cases, and investigating via dialogue. 

%Again the ability to give a verbal definition would manifest the craftsman's knowledge, rather than reveal the content of his knowledge (148).\footnote{Thus the influential notion that Socrates is primarily interested in deductive knowledge, or knowledge of definitions, seems to be a result of taking Socrates' examinations as too much representative of his whole conception of knowledge---cf. \citet{asmith1998}, pp. 129-130, and 153: ``too much emphasis has been placed upon the formal features of the Socratic elenchus, and not enough on the purpose and spirit of it.''}



%incorporating knowledge of various tools and materials and techniques---the sort of whole rational practice Socrates gestures at when he speaks of the craftsman knowing why he does what he does and why he applies or arranges what he does.\footnote{\emph{Gorgias} 465a, 503d--504a.[[support: Laches 190a--c; Gorgias 503d--e]]} 




%Also Timaeus 28s looking to a form


%A doctor will know what disease is as well as what health is, but presumably disease is not something you try to bring about? The situation is slightly awkward here. On the one hand, crafts are characterized by orientation towards the good, or goods. On the other hand, to be a doctor is also to know what disease is, which is to have a measure for disease, which is to know how to make things fit that measure: as we saw, a doctor is best equipped to harm as well as to heal. In that sense even disease is a realizable target: knowing what disease is makes causing harm quite practicable if you should so desire, even though health has a natural priority within the craft of medicine.


%[[[[CUT]]] Indeed ``knowing X'' can simply be a way of characterizing a craft as a whole [[IN A WAY THAT EMPHASIZES RATIONALITY OF THE CRAFT, PERHAPS??]]---thus medicine is the ``knowledge of health,'' or the ``knowledge of health and disease''---the craft is synecdochally characterized by reference to its principle object.\footnote{eg. \emph{Charmides} 165c, \emph{Gorgias} [].} The same thing, in the reverse direction, is happening in the \emph{Laches} and \emph{Gorgias} passages mentioned just above, in which the expert knew what sight itself was, and the doctor looked to the form of health in treating his patients, where this entailed possession of the craft as whole. 



%[[[[CUT]]]  Definition talk should not obscure this---a definition could in no sense give a full articulation of the craftsman's knowledge, nor therefore could it give a full articulation of what it means to know what something is. It might also be remembered that the explanatory requirement for craft knowledge demands more than just that accounts or definitions can be given for certain particular qualities---the expert should also be able to explain, for example, why he employs this or that treatment or instrument in a given sort of case.






%\footnote{\citet[25--7]{irwin1995pse}.} 





%EUthyphro 4d-5d would be suitable here as well





%(However the measure role and the aim role must come apart, if they apply at all, in the case of things like food or bones: a doctor may have to know what food is and how it affects health, but obviously he won't produce food let alone bones.)






%defs re Smith

%first of all Charmides case cannot really be set aside just because there is a sort of pretence that Charm is an expert.

%and why should Lysis be set aside because he never asks exactly the question?

%the context bit doesn't at all show that S doesn't expect also to get anything out of a def - ie fact he asks experts doesn't mean it's not more than a test.






%[[EXPERIMENTING WITH DROPPING WHOLE TESTING/TEACHING SECTION]]

%[[TESTING AND TEACHING]] \subsection{Testing and Teaching} [[SHOULD I CONCEIVABLY DROP THIS WHOLE SECTION?? MAYBE JUST STASH A COUPLE SIMPLE POINTS IN A FOOTNOTE IN PREV SECTION?? EG SEPARATE THE EXPLANATION ISSUE FROM THE TESTING ISSUE - AFTER ALL THEIS ISN'T IN A WAY A ``NATURE OF TECHNE'' ISSUE, PERHAPS??]] 

%[[TESTING AND TEACHING]] Because the craftsman can articulate his reasoning, craftsmen (or at least some pretenders), may be identified through interrogation. Now there are other, simpler, ways of identifying craftsmen. In at least some cases we can tell whether a wall or a shoe, say, is a good one, or whether someone is healthier than he was before. So one way of telling a craftsman is by his work; another is by his pedigree, or his students.\footnote{\emph{Euthyphro} 16a, \emph{Laches} 185b-187b, \emph{Gorgias} 514a-515a.} But generally we see Socrates dealing with more controversial cases, and investigating via dialogue. And here one of the main tests consists of probing for \emph{definitions}. Here I have in mind the examination that begins with the familiar type of Socratic question: ``what is courage,'' or ``what is piety?'' This sort of test is particularly prominent in the \emph{Euthyphro,} \emph{Laches,} \emph{Charmides} and \emph{Hippias Major.} The idea is that any supposed craftsman who claims to have quite exact knowledge about, say, piety, should be able to \emph{say} what piety is, since, as we have seen, craft knowledge is articulate, and the craftsman knows the qualities and entities with which his craft is concerned.

%[[TESTING AND TEACHING]] If the previous section is correct, a verbal definition could capture only an aspect of the caftsman's knowledge. Angela Smith points that the definition test is indeed a \emph{test}---requests for definitions follow declarations of expertise on the part of Socrates' interlocutors:\begin{quote}His request for definitions is, in effect, a request for proof that these self-proclaimed experts have the knowledge that they claim to have.\footnote{p. 149. Though some dialogues, for example the \emph{Charmides,} have the character more of a shared investigation---here it is an idea rather than an expert that is being tested.}\end{quote}Again the ability to give a verbal definition would manifest the craftsman's knowledge, rather than reveal the content of his knowledge (148).\footnote{Thus the influential notion that Socrates is primarily interested in deductive knowledge, or knowledge of definitions, seems to be a result of taking Socrates' examinations as too much representative of his whole conception of knowledge---cf. \citet{asmith1998}, pp. 129-130, and 153: ``too much emphasis has been placed upon the formal features of the Socratic elenchus, and not enough on the purpose and spirit of it.''} [[REPETITIVE?]]


%cf also Wieland11982 eg p 248 - though remarks not nec drected at techne specifically here





%As Smith also emphasizes, no single definition could constitute Socratic knowledge---she appeals to the craft case by observing that we would not become experts in shipbuilding by knowing the definitions of shipbuilding or ships. And besides any amount of theoretical understanding, you must know how to apply your theory to specific cases, and this further knowledge cannot be part of the theory itself---here experience and aptitude come in.

%Another consists of probing for \emph{delineations.}

%[[TESTING AND TEACHING]] [BRIEFER SOMEHOW?] We can also see this by comparing cases where the ``what is X'' question does not arise directly, and we see Socrates rather trying (in vain, as far as he's concerned) to elicit a satisfactory statement as to what subject matter some supposed craft deals with.\footnote{For example in the \emph{Ion,} in the first third of the \emph{Gorgias,} and in the later parts of the \emph{Euthyphro} and \emph{Laches}.} Thus we see Gorgias flounder in the attempt to say what sort of speeches oratory deals with, or whether his teaching incorporates moral training. Ion winds up claiming that in virtue of his poetic skill, which makes him fit to discuss the battles in Homeric poems, he could be Greece's greatest general---if only parochial local politics didn't get in his way. In other words, Gorgias and Ion don't even know what it is that they do. But Socrates himself is readily able to delineate fields in which he is not expert---medicine governs health, arithmetic the odd and the even, and so on. So to fail to delineate one's supposed field is to stumped at a pretty elementary level. But the attempt to delineate a field is continuous with the attempt to find definitions. In the \emph{Laches} the attempt to find a definition ends with a proposal which would entail that courage was equivalent to virtue altogether, a clear failure to identify the proper scope of the virtue. In the \emph{Euthyphro} the final attempt to determine the nature of piety has it as a ``mutual craft of commerce between gods and men'' (14e). This fails as a definition ostensibly because it conflicts with a point agreed earlier, but at the same time the failure establishes in a pointed way that although Euthyphro laid a claim to expertise (back at 4e-5a), he doesn't know what he's an expert \emph{in.} So if the definition test is continuous with the very modest attempt to delineate fields, it seems natural again to see the request for definitions as a way of trying to evaluate putative experts, a method of evaluation made possible by the explanatory criterion for craft knowledge.


%In the case of Ion and Gorgias we are evidently to suppose that there just is no craft in the area after all; in the case of virtue we have a long way to go. In any case a genuine expert would be able to do much more than point to the boundaries of his field. And it would be natural to think the same thing in the case of the definition test.

%Indeed we might suspect, with Richard Kraut, that Socrates is after a definition of just the form ``to be pious is to know \underline{\ \ \ \ \ \ \ \ \ }''---in which case the delineation and definition tests would amount to pretty much the same thing.\footnote{\citet{kraut1984sas} p. 251

%In the \emph{Laches} and the \emph{Euthyphro} the two sorts of test merge into each other.\footnote{A third but more complicated example may be found in the \emph{Charmides}, where all sorts of difficulties emerge about the scope or possibility of the kind of knowledge Critias identifies temperance with.} In the \emph{Laches} the attempt to find a definition ends with a proposal which would entail that courage was equivalent to virtue altogether, a clear failure to identify the proper scope of the virtue. In the \emph{Euthyphro} the final attempt to determine the nature of piety has it as a ``mutual craft of commerce between gods and men'' (14e). This fails as a definition ostensibly because it conflicts with a point agreed earlier, but at the same time the failure establishes in a pointed way that although Euthyphro laid a claim to expertise (back at 4e-5a), he doesn't know what he's an expert \emph{in.} Indeed we might suspect, with Richard Kraut, that Socrates is after a definition of just the form ``to be pious is to know \underline{\ \ \ \ \ \ \ \ \ }''---in which case the delineation and definition tests would amount to pretty much the same thing.\footnote{\citet{kraut1984sas} p. 251}


%[[TESTING AND TEACHING]] However the \emph{Euthyphro} could easily suggest a more important role for definitions. Socrates asks Euthyphro to\begin{quote}teach (\emph{didaskein}) me what this form [of piety] itself is, so that I may look upon it and, using it as a model, say that any action of yours or another's that is of that kind is pious, and if it is not that it is not. (6e)\end{quote}Assuming that Socrates here treats piety as a craft, Socrates is asking Euthyphro to teach him his craft---just as he had suggested a bit earlier that he should become Euthyphro's student (5a). Strictly what Socrates says is that he would like to learn what piety is in a way that will allow him to \emph{judge} whether or not any given action is pious, but this is in effect a request to learn the craft, since the good judge \emph{is} the craftsman, a point made in a number of dialogues.\footnote{Smith rightly insists that the person in possession of the sort of criterion which (in my view) Socrates asks for here will \emph{be} the expert (p. 150, n. 49).} (And the passage is reminiscent again of \emph{Gorgias} 503e where the doctor looks to the form of health.) Now the question is how Socrates envisages the `teaching' in question. Does he (merely) want a definition or does he want to enroll in a course? Given that Socrates has been trying to get Euthyphro to tell him what piety is, the request at 6e looks to be just another way of posing the request for a definition. Indeed shortly after Euthyphro tells Socrates that piety is prosecuting the wrongdoer (5d), Socrates asks him to\begin{quote}try to tell me (\emph{epein}) more clearly what I was asking just now, for, my friend, you did not teach (\emph{didaskein}) me adequately when I asked you what the pious was, but you told me that what you are doing now, prosecuting you father for murder, is pious.\end{quote}This makes it sound as though teaching were the kind of thing which might have happened over the previous minute or so of discussion. It seems no distinction is drawn here between teaching the craft and giving a (verbal) account of piety.

%[[TESTING AND TEACHING]] Of course we don't know what might have happened had Euthyphro given a definition of the sort Socrates would like---not that Socrates expects such a thing in any case. In the \emph{Gorgias}, when Socrates says that oratory is ``an image of a part of politics'' (463d) he scolds Polus for pressing him with further questions as if it were clear already what Socrates was saying. A good deal of exposition is in fact still required. Similarly a good deal more discussion would no doubt be required whatever Euthyphro said. Besides which, Socrates observes elsewhere that crafts are developed through practice, and that learning is a gradual process in which some students progress faster than others.\footnote{\emph{Gorgias} 514e, \emph{Protagoras} 318, \emph{Charmides} 159e, \emph{Euthydemus} 272c, 278d.} And that rules out the possibility that a craft could \emph{really} be communicated merely by means of verbal formulae. But in the early dialogues it seems to be possible to move from, say, medical knowledge to knowledge of health; from knowledge of health to being guided by the form of health; and from knowing what health is to being able to say what it is. Thus after Socrates has elicited a general claim of expertise from Euthyphro, he asks him: ``tell me now, by Zeus, what you just now maintained you clearly knew''---ie. that which you know, \emph{tell} to me (5c). Taken seriously this goes even further than the claim in the \emph{Laches} that if you know something, you ought to be able to say what it is. It suggests that the telling should somehow actually encapsulate, or at least reveal, the knowledge---that is to say the craft---in question.

%[[TESTING AND TEACHING]] [[AT LEAST SOME OF THIS P SHOULD MOVE, ESP IF I NOW DISCUSS A COUPLE OF THE ISSUES...]] Although the interest in definitions remains an interest in craft knowledge, I think the \emph{Euthyphro} offers a valuable reminder that the less cognitive aspects of craft knowledge receive less attention in the early dialogues than they do in later dialogues, so that we wind up with these surely misleading suggestions that craft knowledge could somehow be distilled verbally. Later, in the \emph{Phaedrus,} Socrates more deliberately draws attention to the need for practice, which brings discernment in the application of theory (271d-e). The \emph{Republic} is concerned throughout with natural ability and the kind of character a student needs to bring with him in the first place. And indeed the necessity of each of these three---talent, theory and practice---seems to have been a general contemporary theme. While Socrates is not exactly an exception, he puts relatively little emphasis, certainly in comparison with Plato, on the development of proper sensibilities and behavioral repertoires.\footnote{One illustration of Socrates' relative neglect for these themes may be seen in connection with the point (in the \emph{Hippias Minor} and \emph{Republic} I) that someone with a given craft or power or virtue will be best able not only to put that power to use for its proper ends but also for the frustration of those ends: for example, someone who knows the truth is best able to ensure that they never speak it, since they can avoid hitting on it just by accident. There is plenty of truth to this, of course: the doctor will know best how to save and how to kill. But it is not generally just as easy, as Socrates seems to imagine, to get things wrong as it is to get them right: for instance it's not easy to produce a real word-salad in your own language, as Elizabeth Anscombe has pointed out. To think that it could be seems to reflect the influence of a particularly intellectual conception of abilities, including crafts, and certainly of virtues.}


%BUT ALSO THE PHAEDRUS AT 269D HAS THE TRIAD



%Three items - or at least talent and education in Canon






%What seems to be happening is this. I said that there are broader and narrower senses in which a person might have knowledge of what a thing (health or piety) is, but that Socrates draws no actual or explicit distinction between these two. In the case of the \emph{Euthyphro} the two seem to be collapsed. Socrates seems at first to be (merely) testing Euthyphro, but the language of 6e is reminiscent of \emph{Gorgias} 503e where the craftsman is described as ``looking towards something.'' So Socrates seems not just to be trying to determine whether Euthyphro knows anything, but is also asking to be let in on whatever expert knowledge Euthyphro possesses. We may also compare this case with \emph{Laches} 190, where, as I pointed out, the broader and narrower senses of knowing what a thing is seem both to be in play. There, in connection with knowing what virtue is, and only very shortly after having indicated that knowing what sight is amounts to being able to impart it (189e-190a), Socrates says (at 190c) that ``what we know we must surely be able to tell.'' So again there seems to be no real distance between the broader and narrower readings of ``know what sight is.''





%Admittedly, whether, in this case, this is supposed to mean that you can really convey all of what you know verbally is less certain, since in this case we have less of a sense of what the practical implications of hearing the account are supposed to be. But again there seems to be no real distance between the broader and narrower readings of ``know what sight is'' here.











%\footnote{In the \emph{Laches}, in the course of asking about courage, Socrates gives an illustration of the sort of answer he is after with the example of quickness, which he says is ``the quality which accomplishes much in a little time'' (192a-b). That might make it sound like we're just after some kind of verbal formula or definition. But quickness is not an ideal example for us so far as an exposition of craft knowledge goes, since there could hardly be a craft associated with quickness in general. The point of the illustration, presumably, is just to get Laches to give answers of the right level of generality.}


%Also the Meno, where it doesn't even look like a real definition.







%What Socrates says to Charmides (at \emph{Charmides} 159a) is more modest, and seemingly more appropriate, if we may take the situation there to be one in which we imagine that Charmides might know what temperance is.\footnote{That is, if we may take being temperate to imply knowing what temperance is---but admittedly the issue is not framed in terms of knowledge.} There Socrates says only that Charmides should be able to express some view about temperance. 



%\footnote{Eg. Isocrates \emph{Against the Sophists} 10, MORE CITES - On Breaths 1 habit being best teacher; see cites on importance of experience in Cuomo p.14 n.28 --- also Cuomo c. 28---Law 2.1-12 has maybe not so much aptitude as disposition, Met 1.1 has experience nec]]}


%\footnote{\citet{anscombe1963i} p. 28.} 


%Definitions, in the sense of verbal accounts of the nature of moral qualities, are central to Socratic \emph{inquiry,} but not to his conception of knowledge or craft. And this is an important point because definitions have often been given far too prominent a role in treatments of Socrates' conception of knowledge. Nevertheless some passages seem to exaggerate the potential for teaching inherent in verbal accounts, and this does seem to reflect a more limited attention to the less cognitive aspects of learning than we find in later dialogues.








\subsection{Techne and Virtue}

I said that much of the scholarly interest in \emph{techne} has circled around Socrates' frequent comparisons of virtue with various \emph{technai}. Socrates makes these comparisons because he thinks that virtue is knowledge, and that knowledge is \emph{techne}.\footnote{The notion of a ``craft analogy'' is misleading, as it invites the question ``how far does the analogy hold?'' But Socrates treats \emph{specific} \emph{technai} as \emph{analogues} of virtues, because he thinks that medicine, arithmetic, justice, etc., are \emph{all equally} \emph{technai}. Contrast Aristotle, who appeals to \emph{techne} for a (limited) analogy with virtue, e.g. at \emph{NE} 1103a32--b2.} However this is a matter of much controversy, because many scholars think that \emph{technai} have features unsuitable to virtue.\footnote{Some scholars, such as Terence Irwin, take the analogy to be exact. Others, including Gregory Vlastos and R. E. Allen, think that the analogy has definite limits. David \citet[6]{roochnik1996aaw} says that ``Plato rejects techne as a model of moral knowledge,'' and that Plato juxtaposes \emph{techne} and virtue to show how \emph{poor} a model \emph{techne} provides.} Which leads to a problem for my central thesis, since if you think that virtue cannot be a \emph{techne}, you will probably still think that virtue is knowledge (that being an important Socratic thesis), so that it would be easy to think (mistakenly) that \emph{techne} is just one form of knowledge among others for Socrates.

And a seemingly attractive line of thought supports the idea that Socrates would be interested in the craft analogy even if he doesn't himself think that virtue is a \emph{techne}: a lot of people were running around Greece claiming to teach virtue---claiming, in fact, a moral \emph{techne}. Socrates was showing how wrong this was. Perhaps he thought that the example of \emph{techne} could be used to illuminate the nature of virtue to some extent, but it was a typically sophistic rather than Socratic idea that you could really posses a full-blown moral \emph{techne}. But that line of thought becomes less attractive once we realize how Socartes' conception of \emph{techne} differs from that of his interlocutors. He never shows that Gorgias or Polus lack a moral \emph{techne} as \emph{they} understand that notion; in fact Socrates insists that they have the \emph{wrong} understanding of \emph{techne}. This is a rather frivolous exercise unless Socrates thinks that virtue really \emph{is}, or \emph{ought to be,} a \emph{techne}.\footnote{Perhaps Socrates objects to the use of the term `\emph{techne}'? But surely Socrates thinks these Sophists don't teach virtue at all, not just that it's not a \emph{techne} they're teaching.}

And perhaps attention to the distinctive features of Socratic \emph{techne} could also show us how to get out of the difficulty about unappealing features of \emph{technai}. Here is an illustration of how that might go. One worry is that if virtue is a \emph{techne}, then virtue will be of only instrumental value, and Socrates surely does not think \emph{that.}\footnote{\citet[198--9]{irwin1995pse} thinks that, for Socrates, \emph{technai} are of instrumental value, virtue is \emph{techne}, and so virtue is of instrumental value. Gregory Vlastos and George Klosko, for example, want to resist that implication, and rightly so.} Now I don't think it's clear that Socrates thinks \emph{technai} are of only instrumental value.\footnote{What about the case of draughts, say? (\emph{Petteutike}---\emph{Gorgias} 450d; see Dodds \emph{ad loc.}: Plato is ``probably thinking of a game of pure skill.'') \citet[100]{gosling1978pmt} draws attention to flute- and lyre-playing as well. But \citet[187--8]{roochnik1986} goes too far in claiming that Socrates draws a distinction between productive and theoretical crafts at \emph{Charmides} 165, so that ``the value of theoretical knowledge is not instrumental: its worth derives solely from itself.'' (cf. Smith, 137.)} But even if we suppose that he does, one aspect of \emph{techne} suggests that an exception might be made for the case of virtue. What I have in mind is \emph{techne}'s mastery of luck (\S1.2). If a \emph{techne} overcomes luck within a given domain, then one might aspire to a master-\emph{techne} which overcomes luck altogether---the sort of thing Socrates describes as being or superseding luck in the \emph{Euthydemus} (279d ff.). And once you have the notion of such a \emph{techne}, it's a short leap to the notion of a \emph{techne} which is intrinsically rather than instrumentally good, since good luck (or fortune) may itself be thought of as a or indeed \emph{the} intrinsic good: Socrates says everyone considers good fortune the greatest good (\emph{Euthydemus} 279c), and Aristotle says that good fortune (\emph{eutuche}) is or is near to well-being (\emph{eudaimonia}---\emph{Physics} 197b3--4).\footnote{This is not ultimately a workable position. It identifies the good with knowledge, but to the question what it's knowledge \emph{of}, Socrates seems to want to answer that it's knowledge of the good, which is unilluminating and violates the explicit stricture that a \emph{techne} have some object independent of itself (\emph{Charmides} 165c--166c, \emph{Clitophon} 409a--d, \emph{Euthydemus} 292a--e, \emph{Republic} 505b--c).}

Socrates recognizes the difficulties his model of knowledge presents for an understanding of virtue. But that does not mean he rejects that model of knowledge---after all, he has no other. It's interesting rather to see how that model can be adapted in different ways or how other aspects of Socrates' thought are suited to fit with his conception of of knowledge. Another example can be drawn from another way that \emph{technai} are problematically instrumental: they can be misused, put to work thwarting their natural ends as well as advancing them, as Socrates points out in the \emph{Hippias Minor}. Aristotle gives this as a reason for saying that virtue is not a \emph{techne}; some scholars think Socrates is trying to make that point too.\footnote{For example \citet[29]{allen1996pih}.} But in fact, as I will explain in \S2, the denial of \emph{akrasia} ensures that virtue will not \emph{actually} be abused, so Socrates already has a way out of this problem, without giving up the view that virtue is a \emph{techne}. Plato later takes a different approach, since in the \emph{Republic} he accepts the reality of \emph{akrasia}. But the \emph{Republic} emphasizes the role of practice and experience in \emph{techne}; a substantial portion of the \emph{Republic} discusses cultivation of the character that is a prerequisite for moral knowledge. And this means that even though Plato makes conceptual space for \emph{akrasia}, his understanding of \emph{techne}-possession allows him to hold that virtue is knowledge, and that knowledge is \emph{techne}, but again without the implication that virtue might be abused. So in neither case does the \emph{techne}-model itself need to be rejected or relegated to a lesser role.\footnote{Though the situation in the \emph{Republic} is more complicated in several ways.}

%For one, there are varieties of virtue not entailing knowledge. And the claim that knowledge is \emph{techne} needs qualification.


%Another item could be the supposed impossibility of meeting the criteria: see Woodruff what a general needs to know - and I have some notes on that

%A \emph{techne} then is a specialized practice, typically with a practical point, and meeting certain standards of generality and reliability. It is rational in the sense that the practitioner can explain his aims and actions. Much or all of this, at least under some interpretation, would have been common to other conceptions of \emph{techne}. But the Socratic conception also differs from those espoused by some of his contemporaries, or by later thinkers. So, for example, Socrates distinguishes his own view from that of Polus by his insistence on the theoretical or explanatory aspect of \emph{techne}. Socrates' conception differs from Aristotle's in lacking a place for stochastic \emph{technai}. And his treatment of \emph{techne} differs even from what we find in Plato's later dialogues.

%We can illustrate some possible interpretative implications of proper attention to specific features of the Socratic conception of \emph{techne} by drawing from issues about the so-called ``craft analogy''---the Socratic analogy between virtue and \emph{techne}. Socrates frequently compares virtue to \emph{techne} or to specific \emph{technai}, and a general question is whether Socrates thinks that virtue \emph{is} a \emph{techne}, or only \emph{rather like} a \emph{techne}. The simple reason for adopting the former view is that Socrates holds the already puzzling view that virtue is knowledge, and the \emph{techne}-model provides us an interpretation of that thesis.


%It would be another matter if Socrates had developed some other conception of knowledge, but he didn't.


%\footnote{Indeed the notion of a Socratic ``craft analogy'' is misleading. It invites the question ``how far does Socrates think the analogy holds?'' But in fact Socrates treats \emph{specific crafts} as \emph{analogues} of virtues, because he thinks that medicine, arithmetic, carpentry, justice, etc., are \emph{all equally crafts.} It would be better to ask (and this sort of question comes up in the dialogues themselves) how far the analogy between virtue and a \emph{given} craft holds. For a contrast we might consider Aristotle, who really does appeal to craft for a---limited---analogy with virtue, for example at \emph{NE} 1103a32--b2.}


%Perhaps virtue is a craft no one possesses, or even could possess, but Socrates thinks of claims to virtue or moral knowledge as amounting to a claim to a moral \emph{techne}.


%such as the \emph{Phaedrus} and \emph{Republic}, since Socrates puts relatively little emphasis on the importance of experience and aptitude. [[In the next sections, I will argue that this Socratic conception of craft should in fact be thought of as his conception of knowledge generally. - integrate this sentence better somehow]]







%But the difficulty is a natural one to get into if we are trying to imagine a general \emph{techne} for living under the assumptions that \emph{techne} 1) has a distinct object and 2) renders fortune irrelevant.

%The inference seems to me bad, and for evidence Roochnik adduces passages only from Aristotle. He also finds the productive/theoretical distinction at \emph{Euthydemus} 290b--c, but that is also unwarranted, as the inclusion of hunting on the `theoretical' side shows.


%\emph{Petteutike} comes up a number of times as an example of a \emph{techne}, eg. at \emph{Charmides} 174b).





%Now even accepting that, one might easily suppose that virtue could no longer be conceived of as a \emph{techne} in the \emph{Republic}, since Plato there accepts the reality of \emph{akrasia}. 





%[[And now I'll simply ask you to accept on faith that we can always say something like that, and that a proper understanding of Socratic \emph{techne} can help us see that Socrates thinks virtue is a \emph{techne}, which then will help us see what an important role \emph{techne} plays in Socrates' understanding of knowledge.]]]

%In the \emph{Gorgias} Socrates just \emph{tells} Polus: this is what I say a \emph{techne} is and you don't have it. This, it seems to me, would be a rather frivolous exercise unless Socrates thought that virtue really \emph{is}, or \emph{ought to be,} a \emph{techne} as he understood it.





%[[KIND OF LIKED THIS, BUT...]] \footnote{He \emph{might} have done such a thing. Isocrates seems to have tried to portray oratory as a worthwhile subject even though not necessarily a craft properly speaking. But we don't find anything similar in the case of Socrates.}

%[CUT] For example: is it productive the way medicine is? Or is it more like arithmetic? Or: is it instrumental the way most crafts are? I mean mainly that these are the only sorts of questions \emph{we} should ask, but we also find examples of this sort of question coming up in the dialogues themselves, perhaps most notably at \emph{Charmides} 165.

%\footnote{In one way Socrates may have thought a moral craft genuinely impossible. For example, we saw that craft is teachable, but Socrates seems sometimes to be tempted by the idea that virtue is not: after all we find no teachers of it (\emph{Protagoras} 319a-320b, \emph{Meno} 89d-96c). But for Socrates this would seem to be less a reason to doubt that virtue is or ought to be a craft than a source of despair about whether virtue is possible (cf. \citet{kraut1984sas} pp. 247-9). (The transition from ``virtue is not taught'' to ``virtue is unteachable'' may look fallacious---as if the fact that no one (now) teaches jousting would show it to be unteachable---and this might cause one to doubt whether Socrates really supposes virtue unteachable---certainly Plato would not have accepted the conclusion. I need not take a stance on this, but I will say that what looks like fallacy if we think about the question in terms of craft may not seem so when we think about it in terms of virtue. When the Psalmist laments that ``there is no one does good, no not one,'' he is not merely making an empirical generalization but saying that we are not fit for virtue.)} After all, he demands accounts from putative moral teachers, and he frequently compares virtue to crafts---moral training to horse training, the usefulness of virtue to the usefulness of building, etc.

%So I will not try to argue specifically that Socrates thinks of moral knowledge as craft knowledge. However, I \emph{will} argue in the next chapter that, for Socrates, knowledge just \emph{is} craft knowledge---and \emph{ipso facto} moral knowledge must be craft knowledge if it is knowledge at all.

%One might ask for an argument here.\footnote{So \citet[97]{klosko1981tcv}, challenging \citet{irwin1977pmt}, who gave a positive answer to the question just posed.} But that would be, in one way, a mistake. Everyone grants that 

%In the case of an ordinary craft, it is true, good fortune would be good fortune \emph{in bringing about some further good,} but that is not inevitable: as I understand the \emph{Gorgias,} while a good state in the case of material condition and physical state is brought about by an independent craft---ie. money-making or medicine---the good state of the soul is itself justice (477c). So to learn this craft (still assuming that justice or virtue is a craft) is already to possess the most important of goods.

%cf. Laches 185e

%Though on the whole I incline to the view that Socrates does indeed think of the value of \emph{techne} in broadly practical terms---even arithmetic, for example, comes up repeatedly in a \emph{practical} guise---in the \emph{Protagoras} Socrates even goes so far as to identify arithmetic and moral knowledge.\footnotet{}

%In the \emph{Charmides,} after Critias has proposed that temperance is knowledge of oneself,  Socrates adduces several examples of crafts which have a particular product---health in the case of medicine, houses in the case of building, etc.---and asks Critias what fine result temperance produces (165c-e). Can we then add to what we have said that \emph{technai} are oriented towards some form of \emph{production}? 

%Certainly I think that \emph{techne} always has practical value. To that extent I would side with Irwin against Roochnik. But I do not see that there is any really decisive stance in the early dialogues as to the value of \emph{techne} \emph{generally,} of the sort which would force us to understand \emph{any} \emph{techne,} even virtue, as being of only instrumental value, even if we were to adopt (as I would) a quite strict understanding of the \emph{techne}-\emph{arete} analogy. 




%-------------------------------------




%BEGIN OLD CHAPTER 2










%check 345a as well


%What about the paideia of eg Prot 312? isn't this treated as non-technical, and doesn't this set=up ythe moral stuff? so perhaps the moral stuff is not technical as opposed to technical? - but then painters and carpenters invoked again... also episteme and sophia are back right away... is it considered K?
%-however issue here is also what H thinks he'll get from Prt - so techne is perhaps sort of out of picture given what H is after, even though S thinks it should be more? Then this could actually be used to illustrate point about S knowing tthat there are other conceptions of techne or of what Prot offers which is less than techne - also at 313-4 S is suggestion P doesn't know what he does himself at all... - or at any rate doesn't know what he's selling
	%but then at 317c P himself invokes a techne - this is formalized at 319a - notice that political techne and making men good are together - not sure why Parry wants to separate these..


%Euthydemus 277e-278a has a great bit bout learning or understanding through the application of K you already have - can use this for my pont about knowing the thing or via it. this assage looks back to the previous page - eg 277a-b







%In this chapter I lay out this picture of knowledge in more detail and defend various aspects of it---for example my interpretation of the passage in the \emph{Meno} which discusses true belief and knowledge. Among the question I take up are the following: how should we understand belief in the Socratic dialogues? What is the relationship between belief and knowledge in the atomic sense? What is the relationship between each of these and craft knowledge? Is it possible to have knowledge in the atomic sense without knowledge in the craft sense? To what extent is a new understanding of knowledge emerging in the \emph{Meno}? In particular, how does the craft model fit with the important new role for recollection and perception?


%maybe also general reflections on socratic K and contempt epist? eg respond to people who S S is a coheneretist, or thinks of K in terms of understanding, etc. Maybe connections with eg VE or nature of science.



\section{Techne and Propositional Knowledge}

%ITEMS I MIGHT WANT TO INTEGRATE:\\

%a few items re craft analogy: parrry 90ffm bambrough, tiles, o'brien socratic paradoxes, penner socrates on virtue and motivation
	%parry thinks you get out of the instrumentality issue by knowing good more generally - a craft directed at that - I still don't see how this helps (89-90
	%but I think arguably it's exactly the opposite - you have to have basically the right (good) dispositions to be able to see the Good in the first place - certainly doesn't look like you make contact with the good and then all the character falls into place - maybe it'll improve things but you have to already be going in the right direction etc - I mean you start by loving the good etc - suppose you had to first love health, all health and in every way etc before you could become a doctor - then why would we worry that you might abuse your skills by killing people?
	%maybe an analogy - perhaps something like counselling. Some of the people I've known who have gone into counselling are really generous people with a high degree of empathy and sympathy for people. It seems to me quite likely that 1) it would be hard to be a counsellor if you weren't like this already, although of course you also have to go through all this counselling training at the local college or whatever. But then I'm also not afraid that these people would want to use their counselling training to harm people. Obviously I don't know how far this is true, but seems at least plausible that there could be differences in some things of this sort - eg no reason someone who will excell as a tax lawer will have to be the kind of person who wouldn't use their knowledge to use their knowledge to defraud or whatever.
	%the spirited person maybe another case... why would they use their military knowledge to hide etc?
		%anways would use eg 518 to make a related point... and 514 - the *effect* of education is the getting up to the sun etc
			%was thinking of this reading Reeve - not sure how much I ight be stealing from there...\\


%EXPAND AKRASIA MATERIAL FROM M and P presentation and conference submission\\

In the \emph{Protagoras}, Socrates and Protagoras discuss the claim, attributed to the many, that one can know what one should do, and yet not do it. Socrates and Protagoras disagree with the many; they think that knowledge is ``strong'' and would not succumb to temptation that way. And this seems to be an explicit discussion of the nature of knowledge in the context of an example of \emph{propositional knowledge}. So how does the existence of such discussions fit with my claim that, for Socrates, knowledge just is \emph{techne}? To begin answering that question, it's best to begin with this very treatment of \emph{akrasia} in the \emph{Protagoras}.

%A Socratic \emph{techne} is a variety of knowledge, but, being a kind of practice, it is not a variety of belief. The relationship between knowledge and belief is important for Socrates nevertheless. To understand how Socrates envisions it, 

Socrates argues that \emph{akrasia} is impossible. The argument is framed as a refutation of the many, who \emph{do} believe in \emph{akrasia}: they believe that, thinking you should do one thing, you can be ``overwhelmed'' by fear or the prospect of some pleasure, and act contrary to your better judgement.\footnote{As observed just above, they say that \emph{knowledge} can be overwhelmed, but I leave knowledge aside for now.} Socrates first foists hedonism on his opponents---the best action, they would agree, is the most pleasant action, or the action most productive of pleasure over time. But then, given hedonism, for some pleasure to overwhelm you is for some \emph{good} to overwhelm you. And this good which ``overwhelms'' you must not be the greater good, or you would not be doing anything wrong by pursuing it. So being overwhelmed must entail choosing lesser goods over greater. But Socrates finds this unintelligible, since there is no longer a question of being deceived by pleasures and pains distinct from the goods and evils in question (255d--e).\footnote{Jessica \citet{moss2006pai} argues that pleasure is a trope for practical illusion in Plato.} Or again, by hedonism, being overwhelmed would have to mean choosing lesser pleasures over greater. But, says Socrates, there is nothing more to comparing the values of two sets of pleasures and pains than weighing them up against each other, and weighing is treated as a purely quantitative matter (256a).  So now again, Socrates thinks, there is nothing here to be overwhelmed by: it is not intelligible that a person would choose a set of pleasures which they see to be the smaller. \emph{Akrasia,} then, is impossible. Whatever you believe it best to do you will in fact do.\footnote{Like most scholars, I don't believe Socrates really accepts hedonism. But the assumption of hedonism lets Socrates deploy the above argument against \emph{akrasia} and allows for the development of a sort of toy-model for displaying Socrates' views.}%[HURLEY AND RAZ?]



%\emph{Akrasia,} then, is impossible---assuming hedonism, anyway, which Socrates here accepts or pretends to accept, at least for the purposes of this refutation.\footnote{Like most scholars, I don't think Socrates really accepts hedonism. The assumption of hedonism lets Socrates deploy the above argument against \emph{akrasia} and allows for the development of a sort of toy-model for displaying Socrates' views.} Whatever you believe it best to do you will in fact do.%[HURLEY AND RAZ?]


%[[(And other misleading psychological forces are now presumably being understood in terms of the pursuit of pleasures and avoidance of pains.)]]

%Socrates does briefly entertain the imagined suggestion that there is some difference between pleasures and pains which are near and those which are far which accounts for being overwhelmed. Perhaps, though aware that you ought to opt for some course of action which will result in great pleasures which will however be long delayed, lesser but temporally more proximate pleasures exercise an undue influence on your actions? Socrates still insists otherwise: it is unintelligible that one should voluntarily choose less over more, regardless of how far off the various pleasures and pains might be. But there \emph{is} a difference which distance makes, by analogy with physical distance. What is further off appears smaller than what is near. And such appearances may mislead. Here then seems to be the nearest thing to akrasia that Socrates is prepared to recognize, namely ignorance about what is good caused by the relative nearness or remoteness of various pleasures and pains. Pleasures and pains cannot exercise an influence which causes us to act wrong \emph{by our own lights,} but they can throw off our perception.[INCORPORATE ANY OF THIS?]


%do I need to say that this is not just value hedonism but psychological hedonism?


But in Socrates' view, to believe that some course of action is best is largely worthless from the perspective of producing that action. The ``power of appearances,'' we are told, causes us ``to wander and to change back and forth, to accept and reject the same things in actions and in choices of large and small'' (i.e. large and small pleasures; \emph{Protagoras} 356d). So we do what we believe best---but we are easily misled, and we easily change our minds. (You have resolved to avoid unhealthy foods, but the offer of a tempting dessert prompts you to make an exception, since after all it would be rude to refuse---but immediately the thing is consumed, you know that your host would have been perfectly understanding.)\footnote{In the \emph{Gorgias,} too, Socrates emphasizes the vulnerability of the ignorant to the manipulation of charlatans who play on their appetites. Cf. also \emph{Hippias Minor} 372d--e, \emph{Euthyphro} 11b--c, \emph{Meno} 97c--98b.}
 
%Emotions, personal attachments and desires can have this kind of influence, as well.\footnote{Which gets away from the influence simply of proximate pleasures and pains a bit, but Socrates clearly has other emotional states in mind and again I do not believe we should take the hedonism of the \emph{Protagoras} too seriously (see previous note).}
 
%The prospect of a donut \emph{now,} for example, may tend to \emph{produce} in us the thought that it would after all be best to eat said donut: certainly it would be enjoyable; a dollar is a trivial amount of money; there will be ample future opportunities for exercise; no one violation of my resolution to give up donuts can matter much---and so on. So whatever my prior view about donuts, that view may be at risk on entering a donut-occupied room, or if I spot a billboard advertising donuts, \emph{if} that view was \emph{merely} belief. Of course immediately I put the thing in my mouth I may promptly think of my health and curse my weakness, but then it is too late. 
 
But beliefs are not strictly \emph{always} unstable or untrustworthy. In the \emph{Crito} Socrates tells the eponymous character that we should be attentive to the beliefs of wise people rather than to those of foolish people (46d--47b). It is to the knowledgeable that things will appear as they are---a point also of the \emph{Gorgias.} Now the wise are not wise merely on account of having the right views; rather they may be expected to have the right views---i.e. the right beliefs---because they are wise. And it thus matters a great deal whether we ourselves have the wisdom that ensures we see things the right way, especially given the denial of \emph{akrasia}.

%NOTICE IF AKRASIA IS POSSIBLE THAT IT'S NOT SO IMPORTNAT WHETHER OUR VIEWS ARE RIGHT BECAUSE AKRAISA OFTEN GETS US TO DO THE BETTER THING ANYWAYS.

This is just the point Socrates goes on to make in the \emph{Protagoras.} Here the issue is knowledge of the good in general, and after describing the instability of mere belief, Socrates says that our ``salvation'' will be a ``craft of measurement,'' which would \small\begin{quote} render the appearance ineffective: by making clear the truth, it would cause the soul to be at peace by abiding in the truth, and so save our life. (356d--e)\end{quote}\normalsize Socrates says that this moral master-\emph{techne} is a mathematical one (since it involves measurement), and indeed he identifies it with arithmetic (357a).\footnote{Although Socrates promises a later consideration of just what this ``craft and knowledge'' is, he does not bring up the topic again---but perhaps he would intend to qualify the identification with arithmetic, or to say how pleasures and pains are measured?} But the general point of importance for us is the role of knowledge---and specifically \emph{techne}---in ensuring correct beliefs in particular cases.

Beliefs have many sources besides one's own expertise: wishful thinking, sadness or anger, nearby pleasures or faraway pains, optical illusions, the advice of experts or of charlatans. These forces sometimes compete, and some are stronger than others. So, in the speeches of Phaedrus and Agathon in the \emph{Symposium,} love dominates other emotions: fear would not drive you from battle while you fight beside your beloved. Likewise, in the \emph{Gorgias}, Socrates concedes that the testimony of the orator, who knows how to play on the emotions, will, among the ignorant, be more compelling than the testimony of a genuine expert. But, according to Socrates, knowledge (if you possess it yourself) dominates all other sources of belief. That is what he means by saying that knowledge is \emph{strong,} as he puts it in the \emph{Protagoras} (352b): without a \emph{techne}, your beliefs are at the mercy of other forces; if you possess a \emph{techne}, you will think the right way about the appropriate cases.\footnote{Thus Terence \citet{penner1997ssk} is right to insist that the ``knowledge is strong'' thesis is not implied by the denial of \emph{akrasia.} The denial of \emph{akrasia} is rather required to ensure we do as knowledge directs. Knowledge is strong because it ensures that you \emph{think the right way}; \emph{then} the denial of \emph{akrasia} takes over. Cf. also \citet{allen1960sp}, pp 257-8 and n. 6.}

%We have now introduced a place for propositional or atomic knowledge within the Socratic scheme. 

%For Socrates, this is first of all a challenge to the value of knowledge, and, a

Now the original dispute between Socrates and the many was actually about whether it is possible to \emph{know} what to do \emph{in a particular case} and yet not to do it. Can you know, say, that you ought to get out of bed right now, but not do it because the blankets are so warm? As we have now seen, Socrates' response is two-fold: first he argues against the possibility of \emph{akrasia}, and then he explains how a hypothetical \emph{techne} of measurement would ensure that the comfort of the warm blanket won't fool you into thinking you should stay in bed. The result is that your craft knowledge delivers you (and allows you to maintain) the correct belief that you ought to get up, or---to fall in line with the way the many originally made their claim---the (propositional) knowledge that you ought to get up.\footnote{In a sense Socrates refutes the idea that knowledge is weak \emph{twice}. If you think of knowledge in terms of knowing the right thing to do on a given occasion, then the refutation does consist in showing that \emph{akrasia} is impossible. But if we think of knowledge in terms of \emph{techne}, then the refutation is completed only later with the \emph{techne} of measurement.}

In the \emph{Crito} Socrates calls the views that stem from (craft) knowledge ``good beliefs,'' but sometimes he simply calls such views \emph{knowledge}, just as the many talked of knowledge of particular facts in the \emph{Protagoras.} Thus we see Socrates asking Ion whether each of them don't know that Socrates has five fingers on his hand ``by the same craft, that of arithmetic''? (537e). That is, because you know (the craft of) arithmetic, you can know (the fact that) there are five fingers on a hand.\footnote{cf. \emph{Hippias Minor} 265-7.} Likewise the charioteer will be the one to judge advice on racing, and so also the doctor and the pilot in their respective spheres (537). We therefore find that in different senses \emph{techne} \emph{is} knowledge and \emph{is a source of} knowledge.\footnote{Strictly there is a three-way distinction between 1) the body of knowledge, 2) the possession of the \emph{techne}, and 3) the beliefs resulting from \emph{techne}.---Hugh \citet[205--11]{benson2000swm} also says we should distinguish between knowledge as a \emph{dunamis} or power and knowledge as a state which is brought about by that power. But to call Socratic knowledge a \emph{dunamis} is to under-describe it---knowledge is not just \emph{dunamis} but specifically \emph{techne}.} But of course the fact that Socrates is often interested in knowledge which represents the exercise of a \emph{techne} should not be permitted to obscure the centrality of \emph{techne} in his epistemological thought.



%The fact that Socrates can so freely claim or attribute knowledge of---and even discuss the nature of---particular facts is liable to obscure for us the Socratic conception of knowledge. [[[]]]]But as I've said, for Socrates, knowledge just is \emph{techne}, except where knowledge is somehow a result of \emph{techne}.]]]



%And the \emph{Crito} perhaps shows that the state produced by the \emph{dunamis}) may also be thought of as \emph{doxa} (again in a state sense).

%. [though I call this into question below]]


%That is, medicine is both knowledge of health, where this satisfies the criteria described in section one above, and is also a source of knowledge about questions relating to health.


%A comparison with contemporary virtue epistemology may be useful here. Virtue epistemological theories vary in motivation and details, but I'll borrow from Ernest Sosa's recent book, \emph{A Virtue Epistemology.}\footnote{\citet{sosa2007ve}, particularly chapter 2.} According to Sosa's view, the exercise of some faculty or virtue or capacity---visual perception, for example---may yield a belief. Roughly, if this belief is indeed attributable to the faculty, which is operating properly, and if the belief is true, then it constitutes knowledge. Now suppose one of these belief-yielding capacities were numeracy---call it the craft of number. Then we would have a view rather like Socrates'. Except that both the capacity and the delivered belief may be called knowledge, although it is the capacity that constitutes knowledge in the primary sense for Socrates. And Socrates, unlike Sosa, does not, in the dialogues, display much interest in \emph{other} possible knowledge-yielding faculties (like perception). The concern is specifically capacities which constitute crafts.%\footnote{Virtue is a form of knowledge for Socrates; here we see that Socratic knowledge is also (intellectual) virtue!}



%HAVE TO WORK IN THE TWO CONTRASTS GISELA WAS INTERESTED IN
%Since atomic knowledge is a result of craft knowledge, only the craftsman has knowledge of either sort---that is why Socrates treats the contention of the many that you can do what you ought not as a challenge to the value of craft. The non-expert has only beliefs, and these are unreliable; the craftsman's beliefs, so far as they lie within his area of expertise, constitute knowledge.
%HAVE TO WORK THIS IN


%But the comparison with virtue epistemology shows


%We can now see how close we are in some places to contemporary concerns. If we wished to give a Socratic analysis of atomic knowledge, it would be, very roughly: belief rooted in \emph{techne}. Indeed the fact that Socrates can so freely claim or attribute atomic knowledge---as in the case of knowing that a hand has five fingers---is liable to obscure the distance between his central concerns and ours.



%---as in the case of knowing that a hand has five fingers---




%But for Socrates atomic knowledge exists in the orbit of \emph{techne}, and it is \emph{techne} that receives his real attention.

%\footnote{Though that leaves out whatever factual knowledge may be constitutive of \emph{technai}.} 


%If Socrates \emph{did} think about knowledge the way we tend to, then we would have to admit that Socrates does not have a great deal to say about knowledge---again, it is only in the \emph{Meno} that the contrast between knowledge and belief is taken up directly, at least as an \emph{epistemological} issue. But if knowledge for Socrates is [techne], then Socrates does have a good deal to say about knowledge---about the proper scope of knowledge, as in the \emph{Ion}, about the value and efficacy of knowledge, as in the \emph{Protagoras} and \emph{Hippias Major} and \emph{Euthydemus}, about the structure of knowledge, as in the \emph{Charmides,} and about the more theoretical or intellectual aspects of knowledge, as in the \emph{Gorgias}.





%If Plato's epistemological interests \emph{were} like those of contemporary philosophers, then his efforts may seem to have been a bit feeble---as mentioned earlier, some scholars think Plato first distinguishes knowledge from true belief in the \emph{Meno,} and even if they're wrong about that, it would still be true that these two are first contrasted sharply only in the \emph{Meno}---which might itself seem to represent an important advance or realization. The only mitigating consideration would be that Socrates was not, after all, an epistemologist.\footnote{As was Vlastos' view---a view with which I would in some sense concur.} 


%But as I've been trying to suggest, it is totally unsurprising that Socrates should not have made any fuss about the difference between true belief and knowledge. When Gorgias boasts that he has has succeeded where doctors have failed in convincing sick men to take their medicine, then of course Socrates knows perfectly well that the sick man finally takes his medicine believing truly that he ought to, although \emph{not} with knowledge. But he is not concerned to point this out; what he \emph{is} concerned about is how another orator could just as easily come along again and convince the sick man \emph{not} to take his medicine, or to drink poison for that matter, and here it matters not at all whether the sick man starts off with true or false beliefs.
 




%But this leaves open another question about the relationship between knowledge and belief, namely whether the content of the craftsman's knowledge might itself be inherited from the contents of a belief or some set of beliefs. The recognition that so-and-so has a certain ailment is not part of the content of medical knowledge, but to realize that certain factors \emph{cause} such ailments \emph{is,} we might think, part of medical knowledge, rather than merely the sort of thing you can come to know \emph{in virtue} of possessing the medical craft. In particular, I said, in the previous chapter, that knowledge of definitions seemed to form the principle content of craft knowledge; might knowledge of a definition be constituted by a belief (of the right sort) about what a given thing is?

%Take an example. Suppose I know my way around Cambridge. What does knowing my way around Cambridge amount to? Obviously it is knowledge \emph{about the layout of Cambridge}. But how is that knowledge organized? Perhaps we could think of it as a matter of knowing some series of facts---that Kirkland Street is north of Cambridge and Broadway, that there is a T stop at Porter square, etc. Certainly it seems fair to suppose that if I know my way around Cambridge then I know many such facts as these. In this case we might say that the content of my knowledge is constituted by the contents of some suitably extensive and organized body of beliefs about the layout of Cambridge. However it might be more psychologically realistic to suppose that I have some kind of internal map-like representation of Cambridge.\footnote{[CITE].} Then if I am asked whether Kirkland is north of Broadway, I can visualize the environment (or in some other way access my internal map) and produce the answer. In this case the content of my knowledge might be, in the first place, the map-like mental representation, and the content of such knowledge would then not be inherited from any beliefs. 

%Nothing said in the first chapter settles the question whether Socrates might think of craft knowledge along the lines of a set of facts or rather more along the model of the map, or in some other way. If in fact, as I said then, Socrates does not distinguish knowing \emph{that} (say) \emph{piety is \underline{\ \ \ \ \ \ \ \ \ }} and \emph{knowing piety,} then it might seem impossible to answer the present question, since treating these two as equivalent might seem to eliminate the distinction we just drew between the structure or content of knowledge and its subject matter, as if we were to simply identify what it is to know one's way around Cambridge with knowing the layout of Cambridge. We could only say some general things about the features of this sort of knowledge, as I did in the first chapter. There would be nothing further to say specifically about what sort of knowledge is involved in knowing a definition, or whether it was constituted by a belief, and whether, in turn, a craft is constituted by some set of beliefs.

%Socrates does not pose the question and no definite answer should be demanded. \emph{Charmides} 159a and \emph{Euthyphro} 6e perhaps suggest one sort of picture. Here Charmides is to form a view about temperance by considering its presence in him, and Socrates would look to the form of piety as a standard by which to judge actions. Here craft knowledge seems rather like having some internal representation of Cambridge, in the sense that all knowledge of particular facts, whether these facts seem more peripheral to the craft proper, as in the case of knowing that a given action was pious, or more central, as in the case of knowing in some articulate way what temperance is, would be a \emph{product} of craft knowledge, or a \emph{result} of exercising craft knowledge, rather than a \emph{part of} the content of craft knowledge. In that case there would be no overlap between the contents of particular beliefs and the content of craft knowledge. Or, on the other hand, does Socrates imagine, as \emph{Euthyphro} 6e, if taken quite seriously, also seems to suggest, that this knowledge can be conveyed entirely verbally? As I pointed out in the last chapter, Socrates gives us no real indication what more might be involved in teaching. But if knowledge can indeed be conveyed verbally, by means of articulated definitions, we might suspect that this is because such knowledge consists precisely in knowledge of facts which can be so articulated. Or again, is the idea only that you can somehow reveal or indicate the qualities of interest \emph{by means of} articulated definitions? \emph{Euthyphro} 6e seems to allow this possibility as well. The best we can do, I think, is to say that craft knowledge for Socrates must go beyond strictly propositionally-structured knowledge as we would understand it, since craft knowledge entails practical facility. But neither is possession of a craft knowledge-how, in Ryle's sense.\footnote{\citet{ryle1945kak}.} For Ryle know-how is not essentially articulate, but Socrates' \emph{techne} is.\\





%BUT I ALSO NEED TO ADDRESS THE ISSUE OF WHAT CONSTITUTE THE CRAFT IN THE FIRST PLACE, WHICH IS PROBABLY A GOOD PLACE TO BRING IN THE RYLE


%BESIDES THE COMPARISON WITH VE I SHOULD COMPARE WITH K-HOW - IN SOME SENSE OF COURSE IT DOES AT LEAST ENTAIL KNOW HOW, BUT OF COURSE ALSO HAS TO BE MORE SYSTEMATIC - PERHAPS A KIND OF KNOW-HOW? HAVE TO SEE HOW EG RYLE PUTS IT - PROB SECOND PARAGRAPH OF RYLE'S ESSAY SUGGESTS THAT THIS IS NOT K - NOT CRAFT - FOR PLATO - BUT 4TH PARAGRAPH I HAVE DOUBTS - DOUBT YOU HAVE TO BE THINKING SOMETHING ALL THE TIME... OR I DON'T THINK P OR S SAY ANYTHING OF THAT SORT - OK - BUT THAT'S FINE - POINT WILL JUST BE THAT IT ALLOWS YOU TO DO THES STUFF - COMPARE HIS TECHNICAL SKILLS ON P214- ON Q OF RELATIVE PRIORITY OF K-HOW AND K-THAT [P. ] WE MIGHT FIND S AND P ON DIFF SIDES... BUT EVEN IN S K-THAT IS NOT REALLY THE WAY TO PUT IT BECAUSE NOT DISTINGUSHED K-THAT FROM K-ING X. P215 ABOUT WHICH FACTS DISTINGUISH THE INTELIGENT AND SILLY I THINK S CLEARLY DOES THINK THERE ARE SOME THINGS - SEE IF I CAN FIND EXAMPLES - INDEED TAKE POINT ON 217 THAT KNOWING A RULE IS KNOWING HOW - THEN SUPPOSE S SAYS THAT KNOWING X IS HAVING THE AIM, AND SOMETHING THAT CAN BE USED ETC - THIS IS PERHAPS IN THE FIRST PLACE A FUSION OF KNOWING THAT AND KNOWING HOW. SO WE COULD SAY THAT S THINKS OF KNOWING A CRAFT AS BEING A BIT LIKE KNOWING RULES OF INFERENCE *THE WAY A LOGICIAN KNOWS THEM.* EG HE CAN BOTH APPLY THEM AND TELL YOU WHAT THEY ARE. THIS IS NODOUBT OVERLY DEMANDING, SINCE CARPENTRY SHOULD NEC REQUIRES THIS ANY MOR THAN YOU NEED TO BE ABLE TO TALK ABOUT MODUS PONENS TO USE IT. [PART OF ISSUE IS JUST WHAT YOU HAVE TO BE ABLE TO SAY AND UNFORTUNATELY S DOESN'T SAY WHAT THE CARPENTER EG WOULD KNOW.] - K-HOW JUST DESN'T SEEM TO HAVE TO BE IN ANY SENSE ARTICULATE FOR R, WHEREAS IT CLEARLY DOES HAVE TO BE FOR S AND P - SEE SPECIFICALLY THE SECOND ¶ OF P.218 - PRIORITY ISSUE ALSO COMING OUT HERE ON P218 - AND SEE 219 ABOUT EQUATION OF RATIONAL AND REASONED BEHAVIOUR - BUT THIS WOULD PERHAPS GO BEYOND WHAT WE HAVE IN S OR P - SO MAYBE THERE'S ACTUALLY A DISTINCTION R DID NOT MAKE - NOTE A REPEATEDLY FALLS FOUL OF RYLE'S STRICTURES - INTERESTING SUGGESTION THAT SOMEOE WITHOUT K-HOW COULD EVEN BE A TEACHER - POINTE REPEATED ON 221 THAT K-THAT DERIVED FROM K-HOW - IE WORKED OUT FROM THE LATTER - BUT MAYBE S COULD BE THINKING ABOUT IT IN THIS NECESSARY FOR NOVICES SORT OF IDEA? I MEAN COULD HE SUPPOSE IT'S JUST THAT? - CLEAR S COULDN'T/DIDN'T GO ALONG WITH ALL THIS STUFF ON 222-3, ISN'T IT? - WHY NOT ASSESS WHAT RYLE SAYS ABOUT S ON 223 - WOULD BE SORT OF INCLINED TO AGREE - EXCEPT THERE ARE FURTHER ISSUES LIKE S THINKS NO ONE IS VIRTUOUS SO THIS IS NOT QUITE RIGHT - WE DO NOT KNOWJUST FROM THAT WHERE S STANDS, THEREFORE - PRIORITY ARGUED ESP ON P. 224-5 - AH - THIS IS A GREAT POINT OF CONTACT, BECAUSE IN A SENSE THERE IS ALSO THIS SORT OF PRIORITY IN S AND ESPECIALLY P - INTERESTING POINT AT END THAT K-THAT REQUIRES K-HOW IN SENSE THAT YOU DON'T KNOW UNTLESS YOU CAN INTELLIGENTLY EXPLOIT THE FACT. MAYBE S TAKES THIS TOO FAR ETC. ALTHOUGH R SEEMS TO MEAN ESP FOR THEORETICAL PURPOSES MAINLY HERE


%Ok - there's a gap here-----



%The fact that Socrates is often concerned with definitions, which suggests, at least to us, an interest in knowledge of particular facts---ie. of what something is---and the fact that Socrates does speak about knowledge of particular facts such as that he has five fingers, has been a source of confusion about the Socratic conception of knowledge.



%WAS IN FOOTNOTE: Neither talk of `atomic' knowledge nor the claim that knowledge is in some way constituted by belief need mean that knowledge is \emph{propositional,} necessarily; Audi's own view is that there are some some beliefs which are \emph{of} objects and not propositional (\citet{audi2003eci}, p. []). [Not all epistemologists accept this way of looking at the matter; Timothy Williamson is a notable exception. See \citet{williamson2000kai}.---problem for saying a kind of belief etc. Perhaps because what you know you might also only have believed?]




 
%Nehamas \citet{nehamas1999si} p. 30 has this claim - but he thinks Gorg is contemporaneous with Meno - see n. 16




%\footnote{I would be happy to agree with \citet{fine2004ktb}, p. 59-60, that knowledge might be acquired otherwise than by recollection; I am only interested in the shape that knowledge must have.} 




%has been taken, as I said, as offering an analysis of knowledge. I see no reason to suppose that this is what is intended. That knowledge means a tying down of belief does not mean that it \emph{is} (a kind of) belief. The central issue in this dialogue is less the \emph{nature} of knowledge than how you get to knowledge when you don't start with it, and I take the answer to be that beliefs give you a start, and that you develop knowledge as you incorporate them into some kind of explanatory system. 


%In the \emph{Meno} as before, that's where all the real action is. The picture is just what we already found in the \emph{Protagoras,} perhaps without so narrowly moral a focus.




%incorporate this above:

%\subsection{A place for propositional knowledge.} Consider the famous passage in the \emph{Meno}---here we move beyond thoughts that can be attributed to Socrates himself, but the passage is still suggestive---in which Socrates helps the slave to ``recollect'' that one can generate, from one square, another square of twice the area by treating the diagonal of the first square as a side of the second. Socrates has to correct a few missteps on the slave's part at first, but eventually leads the slave through a proof of this point. Now though Socrates and the slave walk through proofs together, Socrates does not say that the slave has now come to know a few things about mathematics, he says rather that\begin{quote}at present these notions have just been stirred up in him, as in a dream; but if he were frequently asked the same questions, in different forms, he would know as well as any one at last. [cite.]\end{quote} Knowledge, it seems, comes in packages, and it comes with expertise and experience. But we need not say that the conception of knowledge manifested here and in the early dialogues has \emph{no} room for propositional knowledge, only that such knowledge plays a secondary role. If craft always goes right, surely the right answer will sometimes take the form of a propositional attitude, as in the earlier case of knowing that you have five fingers by the craft of counting. So we could say that you know \emph{that} you have five fingers, or \emph{that} such a square is constructed in such a way, when a propositional attitude to that effect is delivered by knowledge of the craft form. Otherwise you have mere belief.






% am inclined therefore to say that we have in the \emph{Meno} more attention to metaphysics, and some new emphases on the epistemological side, and even, in the case of recollection and teaching, some tension between different aspects of Plato's epistemological outlook, but not necessarily any more than that.

%[some transition - but after this I can have all the Protagoras, weakness of will stuff, I think...]





\section{Other Varieties of Knowledge?}



%[[Time for another admission. Socrates also talks about knowledge where no (actually possessed) \emph{techne} seems to be in question. For example, Charmides knows Greek; in the \emph{Apology} Socrates says: ``to do injustice and disobey my superior, this I know to be evil and base''; in the \emph{Gorgias} he says ``I know well that if you will agree with me on those things which my soul believes, those things will be the very truth.'' So what sorts of knowledge are appealed to in such cases?]]



%Or could there be atomic knowledge---or any other sort of knowledge, for that matter---which is independent of \emph{techne}-possession? 




But it must be allowed that Socrates also talks about, or attributes, knowledge even where no (actually possessed) \emph{techne} seems to be in question. For example, Charmides knows Greek (\emph{Charmides} 159a); in the \emph{Apology} Socrates says: ``to do injustice and disobey my superior, this I know to be evil and base'' (29b); in the \emph{Gorgias} he says ``I know well that if you will agree with me on those things which my soul believes, those things will be the very truth'' (486e). So if knowedge is \emph{techne}, what's happening in passages such as these?


%[ACK - TWO Q'S HAE TO SORT OUT WHAT EXACTLY I'M TRYING TO ANSWER. this is not really what I wind up answering - I more just say look there's no distinction drawn.]


%The question is complicated by the fact that Socrates does not himself distinguish different sorts of knowledge or senses of `knowledge.' But 

%LYONS DISCUSSES THIS CASE ON 184


Socrates' claims to moral knowledge may be thought particularly problematic. Although Socrates occasionally claims moral knowledge, he also disavows moral knowledge in sweeping terms. After examining one politician, Socrates reflects as follows:\small\begin{quote}I am indeed wiser than he. It is unlikely that either of us knows anything noble or good, but he thinks that he knows something which he does not know, whereas, as I don't actually know, neither do I think I do. (\emph{Apology} 21d)\end{quote}\normalsize In the \emph{Meno} we find him saying flatly, ``I confess to my shame that I have no knowledge at all about virtue'' (71d). I read these statements as disavowals of a moral \emph{techne}, as when Socrates denies having that \emph{techne}, claimed by others, which would allow him to train children in goodness (e.g. at \emph{Apology} 19e--20c, \emph{Laches} 186a--c). So it would seem that Socrates' confidence on specific moral questions \emph{cannot} be founded on possession of a moral \emph{techne}.

What's more, Socrates apparently supposes that you cannot know anything about e.g. virtue without knowing what virtue is (which I take to mean that without possessing the virtue-\emph{techne} you cannot know anything about virtue), and also that you cannot know whether something is e.g. admirable (\emph{kalon}) without knowing what the admirable \emph{is} (i.e. without having the \emph{techne} which takes the admirable as its object).\footnote{\emph{Meno} 71a--b, \emph{Hippias Minor} 304d--e. Whether Socrates thinks that knowledge of what a thing is is necessary and/or sufficient for knowledge of its qualities or bearers, and how these theses should be understood, has attracted much controversy, mostly under the heading of ``the priority of definition.'' (A review of construals of, and of evidence for and against, these theses may be found in chs. six and seven of \citet{benson2000swm}.) I prefer to ask whether or not it is possible to know anything if you don't possess a relevant \emph{techne}.} So not only does Socrates claim knowledge where there seems to be no \emph{techne}, he claims knowledge where he \emph{disavows} the appropriate \emph{techne}, and even says that, lacking such expertise, he \emph{can't} have any knowledge.


%Thus Socrates tells Meno:\begin{quote}You must think I am singularly fortunate, to know whether virtue can be taught or how it is acquired. The fact is that far from knowing whether it can be taught, I have no idea what virtue itself is. . . . And how can I know a property of something when I don't even know what it is? (71a--b)\end{quote} And in the \emph{Hippias Minor} Socrates laments his inability to answer an imagined questioner who asks him:\begin{quote}How do you know whether someone has spoken admirably or not, or done anything admirable whatsoever, when you do not know the admirable? (304d--e)\end{quote} 


%So how can Socrates claim knowledge in those spheres where he has no expertise? And if such knowledge is not, and does not derive from, craft knowledge, what sort of knowledge \emph{is} it?


%But in this section I will sometimes speak of the priority of definition in deference to the usage of the scholars whom I discuss. [[WILL I? IN NOTES MAYBE?]]


%\footnote{cf. \citet[210--11]{reeve2000rot}---though Reeve still puts an emphasis on definition and deduction that I would not.}
	%but also Smith, presumably? What about Woodruff?



%In my view the attention to propositional knowledge of what a thing is is undesirable (cf. \citet{penner1992sae}, p. [n. ]), but I also hope that my approach circumvents these questions to some extent.

%But besides wanting to interpret the importance of knowledge of what a thing is in terms of craft knowledge, I do not see the kind of conflict between a commitment to the priority of definition (or for me: the priority of craft knowledge) and the existence of knowledge in the absence of definitions (for me: craft knowledge) which other scholars see, as I will explain below. I think this means these debates can be avoided to some extent.


Some scholars have felt that there is a genuine \emph{problem} here---a tension or contradiction between the avowals and the disavowals of knowledge. They suggest that different dialogues reveal diverging epistemological standards, or that some apparent knowledge claims can be treated as somehow anomalous.\footnote{Richard Kraut suggests that epistemological standards rise over the course of the earlier dialogues---they are lower in the \emph{Apology,} for example, and stricter in the \emph{Meno} and the \emph{Hippias Major} (\citet[xx]{kraut1984sas}. Irwin treats the positive knowledge claims as anomalous and supposes that what Socrates avows is not knowledge but merely true belief (\citet[xx]{irwin1977pmt}; cf. \citet[28--9]{irwin1995pse}).} But such manuevers seem to me unnecessary.

Suppose someone asked me what I knew about category theory, in such a way as to make me suspect he had an interest in learning about it. Probably I would say that I knew nothing about it; maybe that ``I don't know the first thing about it.'' But if some nerdy party game required me to list the things I know about category theory, well, then: its name starts with a `C'; it's a highly abstract logical-mathematical system; its connectives are very powerful. And it is perfectly ordinary and intelligible that I should in this way be able to say both that I know nothing about category theory and that I know lots of things about category theory. %I might very well even say things like ``well, I know the connectives in category theory have unusual expressive power (my friend told me), but look, I don't know anything about this, you should ask him.''

%\citet{gentzler1995dbe}, p. 242, follows Kraut in this)

Philosophically speaking, there are different ways we might sort out my claims. We might invoke some form of contextualist semantics for knowledge attribution, whereby different standards are established by different conversational settings. Or we might distinguish different senses of `know,' so that in one sense of `know' I might know that P if I have it on good testimony; in another sense I might know that P only if I can explain why it is that P. Again, we might say that there is a difference in the kinds of things I am claiming to know: in the first case I am claiming to know a particular fact, and in the second case denying that I'm familiar with a subject matter.


%For example, knowledge might require being able to rule out alternative possibilities, with context determining \emph{which} possibilities it's necessary to rule out. 


%(and choosing the best approach might require a more careful specification of the cases)

%Similarly one of, or some combination of, these strategies might be deployed to show that Socrates is not contradicting himself when he claims, on the one hand, that he has no moral knowledge (because he has no moral craft), and also, on the other hand, that he has moral knowledge. And 


Bearing in mind plausible strategies of these sorts, I do not see any need to be very concerned about inconsistencies between Socrates' various attributions of knowledge. And indeed the issue for most scholars has been not so much to worry that Socrates' various knowledge claims and ascriptions are contradictory, but to determine \emph{what sort of knowledge} Socrates is avowing when he avows knowledge, and \emph{what sort of knowledge} he is disavowing when he disavows knowledge.\footnote{For instance \citet[esp. p.~36]{brickhouse1994pss} say that Socrates disavows systematically reliable knowledge of whole domains; but this is compatible with having knowledge of specific facts, to any degree of certainty or stability, within those domains. (I would say: Socrates has no craft---at least not of a moral sort---but this does not mean that he doesn't know particular facts which would fall within the purview of some craft.) But this approach leads them to deny that, on the Socratic view, knowledge of definitions is necessary for knowledge of specific facts (45). But you could perfectly well still say exactly that, once you have distinguished different forms of knowledge.}



%I will not attempt to do that, because my interest is in understanding Socrates' \emph{own} conception of knowledge.




%Hugh Benson, who is generally sympathetic to their account, points out that it is a mistake to see an incompatibility between saying, on the one hand, that you need definitions in order to have knowledge of particular facts and saying, on the other hand, that you have knowledge of particular facts although you don't know any definitions. That is exactly the kind of thing you \emph{can} allow once you distinguish different sorts of knowledge. It would be quite possible to say, for example, that a child knows that he has one hundred books because he was told so by his parents, but that again that he \emph{doesn't} know it---not in the sense of actually knowing how to count them up himself or having any facility with numbers.

%My category theory example might suggest the same sort of distinction: the difference between the two sorts of knowledge comes out in how I might avow knowledge about specific facts regarding category theory in some circumstances but then disavow it in others, while my friend, the actual student of category theory, will continue to espouse knowledge in circumstances where more authority or explanatory capacity seems to be invited.




%****[[[REMOVED]]]****That much I would be prepared to accept, because it represents us attempting to make sense of Socrates' \emph{usage,} and not necessarily of his own understanding of his knowledge claims. The next question then is to what extent if any we actually should attribute a \emph{conception} of craft-independent knowledge to Socrates, or to what extent Socrates \emph{distinguishes} different sorts of knowledge. Benson's approach, with which I would largely agree, is to play down the possibility of any further conception besides the craft conception (though the central form of knowledge is for Benson `understanding').


%There are in the early dialogues no such conceptions. 


But it would be a mistake to attribute \emph{to Socrates himself} a \emph{reflective conception} of any form of knowledge besides \emph{techne}. What there are in the early dialogues are attributions of knowledge which do not fit the \emph{techne}-conception. But to find in such knowledge-attributions a \emph{conception of a form of knowledge} would be to find a Socratic form of knowledge which would have to be ``characterized almost entirely in negative terms.''\footnote{Benson p.~237. Though he speaks of understanding rather than \emph{techne}.} And contrast with the fairly clear and demanding standards for craft knowledge, Socrates would apparently have \emph{no} particular criteria for any sort of non-expert knowledge.\footnote{As \citet[77]{woodruff1990pse} points out.} At best we might hope to identify some ways Socrates thinks you could come by such knowledge---by divine revelation, perhaps, or as derived from a life's experience interrogating others.\footnote{See Brickhouse and Smith pp.~39--41 for some possibilities (including these two).}

To say ``I know X,'' and thereby to claim knowledge of some form---with however much confidence or care or sincerity---is not thereby to deploy a conception of that form of knowledge. It does not amount to having a view about that form of knowledge or even to distinguishing it from others. Socrates talks about knowing \emph{technai}, people, inanimate objects, qualities, and languages, but that is not cause for supposing either that he thinks these all involve the same sort of knowledge or that he has conceptions of five different types of knowledge. Perhaps Socrates \emph{could} have distinguished, say, the kind of knowledge a craftsman has from the (propositional, factual) knowledge which the (merely) experienced person has. Later Aristotle did just that (e.g. \emph{Metaphysics} 981a28--30). But Socrates only spoke a language for which such distinctions \emph{could} be drawn.\footnote{For the same reason, Benson is wrong to suggest that expressions of knowledge which fall outside his expertise sense are ``loose'' or ``in the manner of the vulgar'' (236, 238). He worries about cases where disavowals of definitional knowledge are juxtaposed closely with avowals of non-expert knowledge. (He appears---from p.~236 with n.~34---to be thinking of \emph{Apology} 29a--b and 37b, and to a lesser extent of \emph{Euthydemus} 296e--297a and \emph{Hippias Minor} 304e.) He concludes that ``we can either take these passages as evidence for another sense of knowledge that Socrates periodically makes use of---although not nearly so frequently as we have been led to suppose---or we can take them as misstatements made in the heat of the moment or in the manner of the vulgar. In either case the account of Socratic knowledge offered in this examination escapes unscathed'' (238).} 


%Neither is it necessary to suppose that attributions of knowledge which do not fit the \emph{techne}-framework are in any way loose or careless.




%Later, Aristotle will explicitly distinguish the kind of knowledge a craftsman has from the (atomic, factual) knowledge which the (merely) experienced person has (eg. \emph{Metaphysics} 981a28--30). And Socrates already speaks (or Plato writes) a language for which someone could draw up such distinctions. But this means neither that the distinctions have already been made nor that the locutions which prompt Aristotle to make them are in any sense `loose.'

%What licenses attributing the [techne]-conception to Socrates are the identifiable, explicit, and often debatable criteria he has for it. And he has no criteria for any other form of knowledge.

We may decide that the Socratic conception of knowledge is defective in some way, or that it covers only a limited range of the epistemological phenomena, and we may even think that some of his own claims to knowledge show this. We could say that his attributions of knowledge are actually incoherent, or we could defend the coherence of his attributions by invoking a contextualist semantics. But my topic is Socrates' \emph{own} conception or conceptions of knowledge. And as far as I can see, the only variety of knowledge of which he has any conception is \emph{techne}.








%Socrates is interested in one sort of thing which goes by the name `knowledge,' he has nothing to say about whatever else may also do so; but as Audi speaks English, so Socrates speaks Greek, and language is not so tidy as philosophical views.





%Here's an illustration. I noted above, in passing, how Robert Audi takes knowledge to be constituted by belief (of the right sort). Now I come across this passage in his textbook, an example in a discussion of testimony:\begin{quote}I meet someone on a plane. She tells me about a conference in which a speaker I know lost his temper. Initially, I suspend judgment about whether he did so, since the incident is of a rare kind and I do not know her. Then, as she desribes the conference further, other details begin to fit together very well, and she notes information I already know, such as who was there. Soon I am listening in an accepting attitude, forming beliefs as fast as she proceeds. At the end, I find that I \emph{now} believe that the speaker did in fact lose his temper.\footnote{\citet{audi2003eci}, p.134-5}\end{quote}Now what are we to make of that bit about ``not \emph{knowing}'' this woman? Is knowing someone a matter of taking that person as an object of belief somehow? Perhaps this knowledge consists of some complex of beliefs? Or have we identified a second form of knowledge recognized by Audi? Could there even be a contradiction here? Perhaps `know' is ambiguous? Maybe the expression is careless? And yet surely the manuscript was carefully edited?

%I think it's fairly clear that, if we had only the evidence I've provided here, there would be \emph{nothing} to say. We just can't know from this what if anything Audi might say if we were to ask him about this. He might have some analysis of this kind of knowledge in terms of belief. He might claim `know' has a different sense here. He might admit a gap in his epistemological theory, or even an incoherence. He may already have one of these approaches in mind, but he might also adopt one only once asked, in which case there's just not yet any fact of the matter at all as to what he thinks about the case. The case is exactly parallel with our interpretative situation with Socrates. Socrates is interested in one sort of thing which goes by the name `knowledge,' he has nothing to say about whatever else may also do so; but as Audi speaks English, so Socrates speaks Greek, and language is not so tidy as philosophical views.





%I could go through Woodruff p64 a bit:

%a distinction - in some sense sure but it's ours, and all it'll have is craft on one side and everything else on the other.

%``different working conceptions of knowledge'' - only if we mean by this \emph{only} in the sense that knowledge in his mouth can mean different things - and in one important sense there's only oneworking conception - this is etrayed by tendency to think of all K in terms of craft

%`two' is arbitrary - again just craft and other - W also on eg 66 taking the other side too seriously - on the other hand 77 plays it down - no criteria etc - a good

%also wrong to say S disavows all expert K - if that's what Woodruff is doing on 65 - yeah, seems he does on 66


%Having said all that, there is one case where Socrates does actually explicitly distinguish different sorts of knowledge---it is an interesting case and a case of an exception that proves the rule. %the human wisdom - which may basically be thought of as his conception of what a craft requires








%Look at Benson's citations here too - eg Menn 1994?





%There are sources of belief besides K - are there sources of K besides craft K? Perhaps at least say finding out from an expert? Perhaps not - In the Gorgias we havethe sopst convining the patient, not the doctor.












%So we may indeed always do what we think it best to do, but that is of little significance if any passing prospect of some good may reshape our beliefs at any moment: the appearance of some new good or the sudden thought of some old good may at any moment recast our conception of what is worth trading for what. There is no such thing, except perhaps by chance, as a temporally stable (mere) belief that a donut is worth roughly so much, fitness so much more or less, and so on.\footnote{Compare Terence Penner, [Apeiron 1996, AGP 1997]; although I follow Penner in insisting that we take the appeal to knowledge more seriously than is typically done, my exposition of the argument for the strength of knowledge differs substantially from Penner's (and has here been presented much less formally). For one thing he gives no special role to the apparent hedonism which is invoked, perhaps because he takes the `hedonism' here to be simply Socratic eudaimonism. I find that suggestion dubious. Protagoras clearly is put off by hedonism; but why should he be so put off if this `hedonism' is compatible with the view that the good life is a life of wisdom and righteousness? Or why couldn't Socrates have clarified his meaning? And the hedonistic thesis is also ascribed to the many---are we to suppose that Socrates thinks that the many accept the moral views he himself espouses in other dialogues? (Compare also Aristotle's attribution of hedonism to the \emph{hoi polloi} in \emph{Nicomachean Ethics} 1.5.) Most importantly, it seems to me, as reflected in my exposition, that hedonism secures the idea that goods are subject to the simple form of measurement which renders akrasia unintelligible. This simplicity seems to me lost if `hedonism' is simply eudaimonism. But no doubt that helps explain why Penner's exposition differs from my own.}

%If that is our situation, then our good depends on being able to resist these fluxuations in our evaluative views---on being able to attain and hold to a correct picture of the importance of the various possible objects of our pursuit. And so, says Socrates, our ``salvation'' will be an ``art of measurement,'' which would \begin{quote} render the appearance ineffective: by making clear the truth, it would cause the soul to be at peace by abiding in the truth, and so save our life. (356d-e)\end{quote} Socrates clearly thinks that this is a basically mathematical art, and indeed he goes on to identify it with (the art of) arithmetic (357a). By the very fact that it is a kind of measurement that is the answer to fluctuating beliefs about the good, it is ``art and knowledge'' that is the solution. (Although Socrates promises a later consideration of just what this ``art and knowledge'' is---perhaps he would intend to qualify the identification with arithmetic, or perhaps to say how pleasures and pains are measured?---he takes up the topic again neither here nor in any other dialogue.) And that is the refutation of the \emph{hoi polloi:} we have now seen that knowledge is strong, because knowledge is precisely the answer to the difficulty of unstable evaluative beliefs. 



%******On the otehr hand in chap 2 I think we're going to see that S has moved his subtlety to another place: the dobtfulness of belief - so I don't think we should \emph{simply} say that it's this totally crude intellectualist picture






%So given the craft model for virtue, at least as found in the early dialogues, the denial of akrasia would seem to be required if the implication that virtue may be abused is to be resisted. However there are also aspects of the craft model that either do, or may have at least seemed to, make the rejection of akrasia rather natural, so that we need not see this rejection as in any way an ad hoc  effort to save the craft model. One is the notion of professionalism bound up with [TECHNE]. [EG YOU EXPECT THEM TO PRODUE], ALSO FROM TEACHING TAKE THIS AS A WAY OF ASKING ABOUT CRAFT]


%I wonder if one thing that could have made the idea that a knowledgeable person will do what's good is the professionalism aspect of techne - iie you actually do the thing. Ie Dads's not a lwayer although he has the LLB he doesn't practice law. And the virtuos person's product would f course be a good life.


%[of course the no akraisa thesis is much broader - applies to belief as well - and there is a snese here in which the view is intellectuaist. However important to see that S viewed beliefs as highly maliable/unstable, so that not clear this is really as objectionable given his way of thinking about B. THis should be the transition to chapter 2.]
























%B+S p 37 also have K as a power


%Euthydemus 293b has socrates knowing "many things but slight things" so I don't want to say that there's any definite regimentation in terms ofprop-K produced by power-K as I guess Benson would say - just that K secures doxastic outlook generally as per Crito











%I might mention Vlastos' "say what you believe" requirement or Benson's "doxastic constraint" (38) since I think it's obvious what the role of these are - and can juxapose can of ignorant here with the expert in this chater - Expertise fixes correct beliefs etc
	%- actually one problem for me might be the bit t Prot 331c-d where S says he doesn't want to deal with "ei soi dokei"s. that's odd from my perspective, though may be something tirvial having to do with the semantic range of this expression - and see eg citations at Benson 54 where this expression sems not to be a problem and indeed B takes it as a way of pointing to fact people really do believe what they're saying which I think is probably fine - so really we just have a somewhate messier situation here than I might have liked.




%Gott read Gorgias Encomium to Helena and I think there's some defense of rhetoric too - see Roochnik chapter 1 - basically he has a techne controlling doxa, and what's more all with out truth!
	%Note that it woudbe characteristically Socratic to say that ever techne controls doxa or should



%Benson's puzzle on 211 def one I want to pick up on



%Benson 206-7 gets close in some ways to the picture I have in mind of state poducing powers - ad rather off in other ways - the ambiguity he identifies at 207-8 between K as the state and K as the dunamis is one I would probably accept as well - excpet that thinking of the state as something diff from a doxa state is here pointless (if we take the Crito expressions seriously0, if that's what he's doing, which I think it is - bascically he justdoesn't have this stuff tied together in a perspicuous way - he has refs here that should be helpful as well - also cites Menn 1994?
	%also up to 209 got I think a roughly right: dunamis produces the right belief for a particular case" sort f idea


%Protagoras at Prot 327 contrasts experts with laymen in Loeb trans - should check greek on that



%Euthydemus 281 has orthos chresthai (right use) being ruled by knowledge of eg. carpentry. I want to say that 




%Gorg 454e-455a is a case where we have another belief produing dunamis, perhaps - if S would count it as a dunamis? probably so 








%Nice line at Wieland 224, point is (I think) that we think of the propositional K as fixed and don't consider whetehr it represents some knowledge which deals with it - and n my mind that's just what we should mostly be doing....
	%though of course many K claims that aren't this


%Wieland 225 suggests tom me the point: Knowledge in Plato is not the kind of mental state that could have no object - the way for us the mental state in K might just be belief, which could fail to have an object. (Would be denied by say WIlliamson - also what do I mean by object here? Cause if any belief has to have in sme sense a propositional object then ore complicated.)

%Wieland 229: nonprop K can only be had or not, possibility of error not on,  - has no opposite which is it's denial








\section{Knowledge as Techne}


%re the bit about having only the craft or nothing options might connect the question whetehr it's teachable in eg Meno (87)



%Euthydemus 294 is an example of going right to craft K, I think - interestingly Ctessipus uses an example with  "how many teeth" at 294c - interesting not SOcrates but s still  of course Plato... - knowing how to dance at e - this one is Socrates'



%A lot of this might conceivably be better n the intro....


We've seen thus far that Socrates has a distinctive and reflective conception of at least one form of knowledge, namely \emph{techne}. We've also seen that a lot of Socrates' epistemological reflections should be understood within the \emph{techne}-framework. Thus the interest in knowing ``what things are'' reflects one of Socrates criteria for \emph{techne}-possession; propositional knowledge is often the manifestation of expertise. And while many references to knowledge in the early dialogues cannot be understood in terms of \emph{techne}, they do not go so far as to indicate that Socrates recognizes different varieties of knowledge. But to establish my claim that, in Socrates mind, knowledge is \emph{techne}, I ought to show some \emph{positive tendency} on his part to interpret knowledge as \emph{techne}. I give some illustrations of just such a tendency here.

%Here are a few illustrations of that tendency to think of knowledge as such in terms of \emph{techne} specifically.

One example we have already touched on is from the \emph{Protagoras.} There Socrates sets out to show that knowledge (\emph{episteme}) is strong, rather than weak, as the many suppose (352b). This is proved precisely by showing that a \emph{techne} (specifically a craft of measurement) will ensure that you get things right (at 356a--357e). So Socrates proves his point with the assumption that `\emph{episteme}' may be replaced by `\emph{techne}.'\footnote{This point will be missed if it is not realized that the impossibility of \emph{akrasia} and the strength of knowledge are two totally different theses and that neither implies the other.} Now `\emph{episteme}' is indeed frequently synonymous with `\emph{techne}' in the Socratic dialogues, but if Socrates is thinking of \emph{episteme} in terms of \emph{techne} right from the start at 352b, that would itself be surprising.\footnote{Socrates is actually indifferent here between `\emph{episteme}' (352b), `\emph{phronesis}' (352c), `\emph{sophia}' (352d), and `\emph{gignoskein}' (352d).} The issue between Socrates and the many at that point is not the value of \emph{techne} but the effect of knowing \emph{what the right action in given circumstances} is (see esp. 352d). But Socrates \emph{treats} the view of the many as a challenge to the value of knowledge \emph{taken as} \emph{techne}.

Another example is at \emph{Charmides} 165, where Critias proposes that temperance is knowing oneself. Socrates says that if temperance is knowing (\emph{gignoskein}), then it must be some kind of \emph{episteme}, and it must be \emph{of something}; his examples are medicine and housebuilding, which are `of' health and building houses respectively. Though he then grants (at 166a) that knowledge need not \emph{always} have a product (as evidenced by mathematical knowledge), Socrates continues to assume that knowledge must be useful. So here again we have an immediate transition from knowledge to \emph{techne} (`\emph{episteme}' is here equivalent to `\emph{techne},' which also appears several times in this passage). It is notable how peculiar this transition is: Socrates apparently supposes that self-knowledge could only be interpreted as some kind of \emph{techne}, despite the fact that there is \emph{obviously} no \emph{techne} in the neighborhood---as the rest of the dialogue makes clear enough---nor any evident reason to suppose right at the start that there should be.

The \emph{Charmides} example is also interesting on account of the vocabulary employed. According to John Lyon's analysis of Plato's epistemological vocabulary, the verbs `\emph{eidenai},' `\emph{epistasthai},' and `\emph{gignoskein}' are grouped in the following way: `\emph{eidenai}' is largely interchangeably with either `\emph{epistasthai}' or `\emph{gignoskein},' but `\emph{epistasthai}' and `\emph{gignoskein}' are not themselves interchangeable.\footnote{\citet{lyons1972ssa}, chapter 7. Lyons' discussion concerns the Platonic corpus as a whole.} And `\emph{epistasthai}' and `\emph{gignoskein}' play somewhat different roles. For example, `\emph{gignoskein},' unlike `\emph{epistasthai},' is particularly associated with personal nouns,\footnote{Lyons pp.~179, 199 ff.} while `\emph{epistasthai}' is in general a ``generator'' for `\emph{techne},' in the sense that if someone knows in the sense of \emph{epistasthai}, then you can invoke a corresponding \emph{techne}.\footnote{Lyons pp.~160--1 ff. \emph{Charmides} 159a may be an exception---Socrates observes that Charmides knows Greek, employing `\emph{epistasthai},' but from the \emph{Protagoras} I do not think Socrates would speak of a \emph{techne} of Greek. (Lyons is also of the opinion that Socrates would not consider Greek a \emph{techne}: see pp.~184--5, 221.)} Now in the \emph{Charmides} passage, we move from `\emph{gignoskein}' to `\emph{techne}.' But the example is not at odds with Lyons' analysis. Socrates only speaks of \emph{techne} after eliciting from Critias some examples of what are in fact conventional \emph{technai}, but under the heading of `\emph{episteme},' and this may very well reflect the fact that `\emph{gignoskein}' is not so naturally associated with `\emph{techne}.' What we see here is an \emph{inference} on Socrates' part: we have a case of knowing something (\emph{gignoskein ti}) and \emph{therefore} also of \emph{episteme} and then \emph{techne}. And this shows that Socrates is prepared to infer a \emph{techne} not only where the sort of knowledge in question (knowing oneself) seems to have no evident connection with \emph{techne}, but also where the specific epistemic vocabulary has no particular connection with \emph{techne} in general Platonic usage.\footnote{Lyons himself discusses a similar case in the \emph{Ion} (at 540e) where `\emph{gignoskein}' is used in connection with `\emph{techne}' as part of an \emph{argument} connecting good judgment in a particular field with an appropriate \emph{techne}: pp.~188--198.}%\footnote{Other details about Plato's Greek usage are interesting but only, I think, support my thesis, and need not be considered in detail here. Given the close connection between \emph{epistasthai} and \emph{techne} identfied by Lyons, it's unsurprising that Socrates never claims to know things in a propositional sense with \emph{epistasthai}, although he makes such avowals with \emph{eidenai}. But since \emph{eidenai} and also \emph{gignoskein} are often frequently associated with, or developed in the direction of, craft knowledge (see \emph{Protagoras} 352d ff. for another example with \emph{gignoskein}), and since no other understanding of knowledge of whatever form, however associated with given verbs, is developed, it seems proper to say that in developing an account of Socrates' conception of \emph{techne} we are develping an account of his conception of \emph{knowledge}.}


The \emph{Ion} furnishes another example we've touched on. At 537c--e Socrates is discussing the way different \emph{technai} have different subject matters.\small\begin{quote}Take these fingers: I know there are five of them, and you know the same thing about them that I do. Now suppose I ask you whether it's the same craft---that of arithmetic---that teaches you and me the same things, or whether it's two different ones. Of course you'd say it's the same one.\footnote{trans. Woodruff with `craft' substituted for his `profession.'}\end{quote}\normalsize It's not surprising that \emph{techne} comes up here, because that's the topic, but the suggestion that it's in virtue of a \emph{techne} that we know something so simple as that we have five fingers is interesting---it certainly isn't the case that anyone who can count to five is a mathematician. So Socrates seems quite prepared to subsume very mundane sorts of knowledge under \emph{techne}. 

That suggests another general point: Socrates doesn't concern himself with knowledge that can come \emph{only} in bits and pieces, such as historical knowledge.\footnote{Cf. \citet[326]{allen1989dpe}.} Now the lack of discussion of such knowledge doesn't prove much on its own, since Socrates' emphasis on moral knowledge will naturally make practical varieties of knowledge more prominent.\footnote{Though such \emph{technai} as arithmetic and measurement show up plenty as well, and \emph{take on} practical forms. We might also remember Aristotle's somewhat disparaging remarks about history, on account of its dealing only with particular facts: \emph{Poetics} 1451a36-b11.} But we can sharpen the point by considering the treatment of poetry and oratory. Not only does Socrates not consider these \emph{technai}, he never considers the possibility that they might still involve impressive bodies of psychological or dramatic knowledge---though there were, after all, rhetorical \emph{manuals.} The choices seem to be: \emph{techne} or nothing. We can see this at \emph{Gorgias} 464c-465a, where it's clear that to say that rhetoric (or here ``flattery,'' of which rhetoric is a sub-type) is not a \emph{techne} is \emph{thereby} to say that it \emph{just guesses.} This is typical for Socrates: if you're having success, then it's because of \emph{techne}, or else merely a result of guessing or divine inspiration or madness.\footnote{Interestingly, Isocrates, in ``Against the Sophists,'' wants to deny that oratory is a \emph{techne} but \emph{also} to claim that it's worth teaching. In this he is more tolerant than Socrates, and yet it is clear that he is \emph{struggling} to characterize this knowledge. Whereas contrast the various forms of knowledge Aristotle tabulates at the start of \emph{NE} VI.3.}

% [[LACHES 198D--199D AN EXAMPLE; ALSO MAYBE ION'S FINGERS IN PIES, NOW THAT I CUT IT ABOVE]]

I don't know that such examples could be considered decisive. If poets and orators were claiming a special ability as educators, then perhaps it was appropriate to hold them to high standards. And perhaps the \emph{Protagoras} and \emph{Charmides} examples could be put down to Socrates' penchant for appealing to \emph{technai} whenever virtue is in question. But on the whole I think that there is a good case here for saying that, for Socrates, knowledge \emph{is} \emph{techne}. To be sure, Socrates does claim knowledge, or attribute it to others, in other contexts. And Socrates cannot be said to have distinguished craft knowledge from any other sort of knowledge. \emph{Euthyphro} 6e, for example, seems to have propositional knowledge, acquaintance, and \emph{techne} all jumbled together. But Socrates' thought tends strongly in the direction of craft knowledge, and this is really no surprise, since it's the only sort of knowledge of which he has any definite and developed conception.


%A couple of scholars have taken recently taken an interest in [techne] in it's own right---in particular Woodruff and Smith---and to a significant extent I have been following them. But I disagree with at least Woodruff about the scope of the enterprise. Woodruff, in his \citet{woodruff1990pse} expressly distinguishes expert and non-expert knowledge in the early dialogues, with his \citet{woodruff1987eka} being a more pointed attempt to distinguish the kind of knowledge Socrates avows from the kind he disavows.\footnote{A number of scholars have tried, in different ways, to distinguish different forms of knowledge in the early dialogues. A few examples may be found in \citet{lesher1987sdk}, \citet{brickhouse1994pss} and \citet{vlastos1985sdk}.} I however see only one conception of knowledge in the early dialogues: in my view knowledge for Socrates just \emph{is} craft, so that I take myself to have been describing Socrates of knowledge plainly, not just his conception of a certain kind of knowledge.





%[[BIG CUTS HERE]] Other scholars have also tried to give a single general characterization of Socrates' conception of knowledge. Some of their approaches are similar to mine in that they also see Socrates as being concerned with bodies of knowledge rather than just knowledge of particular facts. In particular I have in mind accounts which treat Socratic knowledge as some form of understanding or comprehensive knowledge. I'll conclude this section by contrasting my own view briefly with a couple views of this sort.

%I begin with the account given by Hugh Benson in \emph{Socratic Wisdom.}\footnote{\citet{benson2000swm}.} Benson, like myself, takes up the question of Socrates' conception of knowledge in its own right, abstracting away from questions of moral theory. However Benson has the notions of definition, cognitive power, and understanding at the center of his picture of Socratic knowledge, and he proceeds by defending a variety of principles and positions as Socratic, only tying them together into a ``Socratic theory of knowledge'' towards the end of his book.\footnote{In chapter 9.} Among the principles which characterize Socratic knowledge, according to Benson, are these: knowledge and true belief are distinct; knowledge is reliable; knowledge requires consistent beliefs; definitional knowledge of F-ness is both necessary and sufficient for any other knowledge about F-ness; knowledge is a kind of power (\emph{dunamis}), distinguished by its object and its relation to that object, and which produces cognitive states.\footnote{Benson's ``summary of data'' and account of knowledge as \emph{dunamis} are in \S\S 9.3 and 9.4, pp. 190-211.} 

%It is striking to me that although Benson's general strategy is quite foreign to mine, I think that all these principles are correct, though I would interpret them somewhat differently than does Benson in some cases. And specifically I think they all ought to be interpreted in terms of craft knowledge: it is recognizing that we are talking about craft knowledge that allows us to properly understand the reliability aspect; it makes the question whether knowledge and true belief are different look downright confused. That a craftsman's knowledge is consistent is obvious and trivial. The importance of consistency in the early dialogues is that seeing whether someone's views are consistent is a way of seeing whether he is an expert; to say that it is a feature of Socratic knowledge is not otherwise illuminating, and leads to wondering things like whether Socrates might be a ``sort of proto-coherence theorist.''\footnote{\citet{benson2000swm}, p. 191.} 

%Although it is not exactly my view that these principles are wrong, I think it's not clear what integrates them into a coherent or intuitive conception of knowledge. Benson himself poses this question about his interpretation:\begin{quote}Hasn't Socratic knowledge become so esoteric, so tangled, so robust, and so different from the contemporary epistemological model that any antecedent interest we may have had in it should wane? (211)\end{quote}I would submit that it \emph{has} become tangled. We have at this point a list of features---some of them, like the priority of definition, seeming rather insane---and no clear sense of what if anything motivates or unites them. Some of these features, such as the reliability of knowledge, are, in my view, a great deal more readily intelligible when understood as aspects of craft knowledge.

%Benson's suggestion, as hinted, is that Socrates gives us, in effect, an account of \emph{understanding.}\footnote{p. 11 - Similar suggestions have been made by other scholars. [cf. NEHAMAS 1985, MOLINE 1981, MORAVCSIK 1978, BURNYEAT 1981, ANNAS 1981 pp. 192-3]} He starts by offering some ``paradigmatic examples of the kind of understanding I have in mind'':\begin{quote}We say things like Einstein understands gravity, Richard Feynmann understands quantum mechanics, R. G. Collingwood understands history, or James Boswell understands Samuel Johnson. (211-212)\end{quote}No doubt \emph{we} say things of this sort, but at least the last two examples seem to me profoundly unSocratic, and even the first two, though describing areas of expertise, fall rather outside the sort of area Socrates himself deals with. Callicles does not complain that Socrates never leaves off talking about cosmological theories; he \emph{does} complain that Socrates never lets up with his ``continual talk of shoemakers and cleaners, cooks and doctors'' (\emph{Gorgias} 491a). The tyrants do not warn Socrates that he should stop talking about historical figures or battles.\footnote{R. E. Allen is right, I think, to suggest that Socrates has no place for historical `knowledge.' \citet{allen1989dpe} p. 326.} They \emph{do} warn him that he had better lay off discussing carpenters and herdsmen and blacksmiths and the like.\footnote{Xenophon, \emph{Memorabilia} 1.2.} We just don't see Socrates dealing with the sorts of examples Benson adduces, but Benson's account leaves us in the dark as to why this should be.

%As a second case we can take Terence Penner. Penner argues that knowledge for Socrates is not the knowledge of some one proposition in isolation.\footnote{\citet{penner1992sae} pp 139-147.} Knowledge for Socrates, he proposes, is a matter of knowing the \emph{things} about which one has knowledge---for example, courage would mean \emph{knowing courage,} as well as virtue, and how courage and virtue are related, and how they relate to the other virtues. Penner says this explains why poets, politicians and rhetoricians do not have knowledge: ``their supposed knowledge of some one proposition, in isolation from others, won't count as knowledge of anything'' (143).

%But Ion and Gorgias \emph{don't} know only single propositions taken in isolation---by modern standards they know many things about Homer or persuasion or whatever is in question. We do not have an explanation as to why these people lack knowledge unless we understand what Socrates \emph{counts} as having ``the whole picture,'' as Penner calls it (167 n. 73). And there I would submit again that the question is whether or not we are dealing with a \emph{craft}. Penner himself goes on to quote Frege on sense and reference (or meaning):\begin{quote}Comprehensive knowledge of the thing meant would require us to be able to say immediately whether any given sense attaches to [the thing meant]. To such knowledge we never attain.\footnote{\citet{frege1980sar} p. 58.}\end{quote}``That,'' says Penner, ``is the kind of knowledge of virtue, knowledge, power, desire, good, and so forth that I see Socrates striving for'' (Penner 147). But knowledge for Socrates is not the sort of thing to which we can never attain---perhaps the question is open in the case of virtue, but Socrates is not supposing that medicine or carpentry are forever beyond our reach. And he would certainly not suppose that a carpenter must know that, say, the sense ``Jesus' livelihood'' applies to his craft.

%Why not stick with the sorts of cases Socrates himself is always going on about? No doubt there are different reasons for different scholars---for example a failure to see the role that knowledge of what a thing is plays \emph{within} craft knowledge, or generally an emphasis on propositional knowledge, or a desire to avoid assimilating virtue to craft. But whatever the motivation the result seems to be the pursuit of a spurious generality. Benson himself points out that among the varieties of `expertise' or `understanding' is craft knowledge. And indeed that's where we should leave it, because any kind of `knowledge' or `understanding' which includes 5th-century BC medicine, 18th-century biography, and 20th-century physics, and which yet somehow excludes oratory and knowledge of Homer, is sure to be of very little interest, or indeed to be just a hodgepodge.




%[maybe i should talk about this `understanding' stuff]

%why doing this? motivations include I guess: 1. not noticing the way this is test of expertise, 2. blindness to the non-propositional, 3. desire for a spurious generality, 4. trying to avoid assimilating virtue K to techen - obviously not exhaustive here and diff ones apply to diff people.

%no interesting similarity between understanding a person and understanding a craft - 

%one point to make is: if you grant that one example of this stuff is the craft stuff then why try to abstract away like this? Eg Bensn who sort of includes it. - well loo, we can characterize it directly - what's the point of trying to give some beg general notion? - eg obviously Both Penner nd Benson trying to include craft K in what they're talking about

%It is rather like discussions in the philosophy of religion which treat the problem of evil as a problem for ``theism'' as if it doesn't matter whether we have in mind [BAAL?] who demands the blood of our children or Yahweh who refuses it. Sometimes abstraction only brings obscurity, and this is one of those cases.

	
%B+S p6: "the early dialogues are consistent in treatming all knowledge as craft knowledge" and they'll come back to this in chapt 2




%basically agreed with bnson 168 that "human wisdom" just a wa of saying he recognizes he desn't have whatever

%R's S is sort of perverse - doesn't even accept he same conception of craft as eg Gorgias.

%will want to check up more carefully on eg Sprague, Irwin, Reeve etc



%B+S talk about Wisdom around 38, saying "one aspect of this K is not propositional: it requires that one have the ability to do the right things at the right times in one's field of expertise."
	%I think it's true that object of K not in first place a proposition.
	%However suggestion might be that there's a sort of non-intellectual component - a kind of K-how besies a kind of K that - and it's really not totally clear to me that we really see this here.
		%******On the otehr hand in chap 2 I think we're going to see that S has moved his subtlety to another place: the dobtfulness of belief - so I don't think we should \emph{simply} say that it's this totally crude intellectualist picture


%B+S 39 has other ways S cancome to know things but without wisdom/explanation

%therese also this Q in the lit (eg B+S 41-2) about how far you can get with the elenchos - eg the Gorgias line about everyone always seeming ridiculous - but I wonder whether the reality isn't that much more is brought to the table in the form of the assumption that V is K and in partcular T than S gets out of elenchos



%ws echei formulation also at Euph 4e, Charm 166c-d - B+S (42-3) obviously take this expression to invoke explanatory force

%B+S43: some scholars may not even want to call this other sort of thing K at all - but seem in a way to to worry to much about this and I think this is right - think Benson sort of had same thought
	%again they stresses wrongness of thinking in terms of prop K, and stress connection to action - only cares about K that has consequences for life
		%and def right that knowing how is in some sense frontmost in the sense that it's always about crafts and never just generally about bodies of K or facts or anything, even when eg explanation is added - just that B+S perhaps risk importing a non-theoretical aspect that I don't see alot of evidence for in early dialogues. 
		







%Irwin says (69-70) that we can deal with the Hipp Min problem by accepting psychological eudaimonism. - cites penner in Exegesis and Argument volume, p142.
	%but what really is the diff between this and rejection of akrasia? seems like surely just an implication. Or rather it's denial of akrasia plus eudaimonism
	%Irwin seems to think you get denial of incontinence from having pushed emotional stuff out of K - p75 - but that's not right - again too specific - K thinks you never do anything but what you think you should do period (Protagors) so not jsut about K somehow. It's needed to shore up the indentification of V and K - it's not so much an *implication* of that identification
	
	
	

%talk about a bunch of diff distinctions people makebetween diff kinds of K in early dialogues - Vlastos B\&S, Woodruff, Lesher, etc etc


%I may want to say a bit about virtue generally here - maybe I can tie akrasia and infallibility a bit here - also generally say something about the apparent problems - if I can tie up infallibility and akrasia then 1) this could be a way of showing that basically virtue is craft and that's forcing S' hand in a specific way; 2) would be esp interesting if rejection of fallibility on the craft side more generally could be motivated by worries about implications for virtue? - def have to check that out - check on what the going notion of craft was from eg the James Allen paper and Roochnik, maybe aso Kube and whomever else? - though remember that P does seemingly try to go from infallbility of crafts t something about virtue - still might explain eg shifting to craft per se vs craftsman etc,






%In the context of the Antigone Roochnik asks (p60-1):
%"Can a techne of arete be attained? It seems not: techne looks outward, toward regions it can master, and in this sense is value neutral. Does this imply there is no knowledge of justice and the correct way of acting? Only if techne is the sole model of knowledge. And whether this is the case remains to be seen. The Antigone provokes the question, Can there be knowledge of arete? If so, what ort of knowledge would it be?"
	%agreed this is a crucial question - I think the Isocrates stuff is supposed to help show that there is a non-technical conception of knowledge about





%One thing I can say against Irwin - or way V reads him in the review - is that I don't think we can think of the point of the techne analogy as being the Socrates is on some bg reductionist project - this is just the only kind of K Socrates knows. Not unproblematic but it's all he has to go with.





%fingers thing is ion 537e




%Bluck 213-4 thinks all k in plato a matter of acquaintance?










%Checking usage of epistasthai in early dialogues - but skipping Alc i which interestingly seems to have an anomalously large number of occurances.

%Charm 159a has it for knowing a language - and for all that not at all clear he thinks a langauge is any kind of craft, given Prot
%	- oh - well, obviously he could have said this for himself







%[[NOW ALL CARVED UP AND MOVED]]\section{Some Remarks about the \emph{Meno}}


%At the same time, the earlier dialogues showed indifference between expressions of the form ``knowing what X is,'' and ``knowing X.'' I said that for Socrates a craft was based on and constituted by definitional knowledge, but that this should not be understood as meaning that Socrates was interested in definitions in the sense of propositional knowledge \emph{as opposed to} some form of acquaintance. This, it seems to me, means that the objects of craft knowledge are amenable to a more metaphysical treatment than Socrates provides; and indeed a more metaphysically inclined thinker like Plato may have been curious about the nature of these entities anyways, even besides the problem about teachers in the moral case.





%does seem to arise for (some kinds of) craft knowledge and to be readily intelligible against the background of that conception of knowledge.\footnote{Nehamas rightly draws attention to the introduction to the \emph{Protagoras,} where similar problems are raised---\citet{nehamas1999mps}, pp. 10-12. Nehamas also sees the paradox as responding to a methodological problem about the elenctic procedure of the early dialogues, namely how two people ignorant of the answer to some question could answer that question and recognize their answer as correct (p. 13).



%Put together, we have the problem that if you set your mind on becoming a doctor, and on coming to know health and disease, you will have the difficulty of figuring out who can really teach this to you. (There were, after all, rival conceptions of the medical art.) What's more, how will you tell whether you've learned about health properly even if in fact you have? If Meno could think, wrongly, that he had learned what virtue was from Gorgias (\emph{Meno} 71b-d), how would you know if you'd had the \emph{right} teacher? This is perhaps not a very serious problem, if a problem at all, when it comes to learning, say, weaving, but it matters at least for virtue, and probably for other putative crafts like medicine, at least in the ancient world. (We might compare something like trying to decide where to go to grad school.)



%You get a form of it by amalgamating the point from \emph{Gorgias} 464a to the effect that only a doctor can 









%Whether Plato is telling us, at 98a, something about how you get atomic knowledge, or is telling us rather about a feature of the more structured knowledge which constitutes craft, seems to me unsettleable, and indeed there is probably no fact of the matter. The important point is just that that 



%What I think is crucial to realize if we are to understand Socratic knowledge is just that Socrates really does not care about knowledge in the atomic sense, at least not in its own right. He cares about craft knowledge, which is our best \emph{source} of atomic knowledge.\footnote{In that sense Socrates' view is, at least in emphasis, a kind of inversion of Aristotle's, since Aristotle says that \emph{technai} are based on pre-existing knowledge: \emph{Posterior Analytics} 71a1-5.} And the passage from the \emph{Meno} does not force us to any other conclusion.\\





%\footnote{In fact I believe there are two ways of taking the case. [check re Greek.] One is to take ``\emph{eidos ten odon ten eis Larisan}'' as ``knowing the way to Larissa'' in the sense of ``knowing which way is the way to Larissa.'' That would indeed bring the knowledge and belief cases very close together; the contrast would be between \emph{knowing that this is the way to Larissa} and \emph{believing that this is the way to Larissa.} Or the Greek might be taken as ``knowing the road to Larissa,'' in the sense of ``being familiar with the road to Larissa.'' In which case it would be a stretch to say that the knowledge and the belief have the same objects here: the object of the knowledge is the road itself and the belief would (presumably) take a propositional object.}


%As part of a conclusion, or built into the meno section?





%seems lots of people don't worry about the road example too much - Scott; Nehamas p22; 

%not understanding - S not interested in figuring it out for yourself etc - or we would have seen more of that in early dialogues. Against eg Nehamas and maybe Burnyeat etc ---these are both great papers



\section*{Conclusion: Looking Forward}

The conception of knowledge I have described is easy enough to overlook. For us it's an unfamiliar conception, developed in the context of unfamiliar debates. It's a problematic conception of knowledge as well, since the treatment of virtue as \emph{techne} creates difficulties made plain even in the dialogues themselves. The role of \emph{techne} is further obscured by the room it makes for propositional knowledge, by the fact that Socrates' conception of knowledge doesn't entirely constrain his epistemological language, and by the fact that Socratic method is more prominent than Socratic theory.



%This fact has sometimes led scholars to underplay the importance of \emph{techne} even in Socrates' conception of knowledge, since Socrates plainly thinks that virtue is \emph{some} variety of knowledge.%---hence one might suppose that [techne] is not the only variety of knowledge for Socrates, nor even the most important variety.

It may also be thought that the identification of knowledge and \emph{techne} is at any rate peculiar to Plato's earlier dialogues. The \emph{Meno} in particular is (rightly) seen as a ``transitional'' dialogue, in which more distinctively Platonic, as opposed to Socratic, epistemological themes are emerging. But in fact the \emph{techne}-model is no less important in the \emph{Republic} (for example) than in earlier dialogues. Socrates and Plato do grapple with difficulties presented by the model, or by the uses they put it to. New epistemological themes do emerge. But this is only to say that Plato eventually saw the need for a \emph{modified} understanding of \emph{techne}. 





%And this must be seen if the epistemology of the middle dialogues is to be understood. The famously cryptic account of knowledge and belief at the end of \emph{Republic} V, for example, will be unintelligible if it is not noticed that the strength and reliability attributed to knowledge by Plato there is to be interpreted in the light of what Socrates says about the strength and reliability of \emph{techne} in the \emph{Protagoras}. 

To vindicate that claim I would need to defend a detailed interpretation of the epistemology of the \emph{Republic} and other dialogues. Here I will settle for illustrating one way in which Plato seems to have been motivated to rework the Socratic conception of \emph{techne}.\footnote{Cf. \citet[157--161]{asmith1998}.}

We can start with the observation of Paul Woodruff (whose view of Socratic epistemology is in many ways similar to my own) that whereas learning is in the \emph{Meno} said to be a matter of recollection, and thus involves no teaching, teachability was a mark of \emph{techne}.\footnote{\citet[81]{woodruff1990pse}.} He thinks that the \emph{techne}-model for knowledge has therefore been abandoned by the time of the \emph{Meno}.


%\footnote{Some of the issues about virtue and \emph{techne} described in \S1.4 above also seem to have motivated, or to have been reassessed in light of, shifts in Plato's conception of \emph{techne}.}


%teachability was because of rationality, I suppose, and we still have that...


%, whose view of the epistemology of the early dialogues is in many ways similar to my own, 


%[[At the beginning of this chapter I said that contemporary epistemology is concerned with knowledge of particular facts---with ``atomic'' knowledge, which is constituted by belief. This sort of knowledge does have a place in Socrates' epistemological outlook as well. A belief will count as \emph{atomic} knowledge when it is produced by or is an aspect of \emph{craft} knowledge. This form of knowledge---atomic knowledge---is important, since what you think you should do will affect what you do, and so it matters that you actually \emph{know} what to do. This importance is sharpened by Socrates' rejection of \emph{akrasia} and his pessimism about the value of mere belief. And so one crucial role of \emph{craft} knowledge is precisely to ensure that you have the \emph{atomic} knowledge you need. But atomic knowledge is \emph{secondary} to craft knowledge in Socrates' thinking. Nor is there atomic knowledge without craft knowledge---or if Socrates thinks there is knowledge in realms not governed by crafts, he says nothing about it. This picture holds for the early dialogues and up to the \emph{Meno}.]]


%[[I appealed to the \emph{Meno} as well as to the \emph{Protagoras} for the point that belief, in contrast with knowledge, is unstable. And as I said in the introduction, I will want to be arguing that there is significant continuity between the epistemological outlook of the early dialogues and the epistemological outlook of the \emph{Republic}. But the \emph{Meno} may seem to mark a departure from the earlier dialogues in several ways. First, there is the famous passage at 98a, in which it is said that beliefs become knowledge when tied down with ``an account of the reason why.''\footnote{in Grube's translation.} This will naturally seem to treat primarily of atomic knowledge, to suggest that it is analyzable partly in terms of belief, and to concern itself with what is added to belief to get knowledge. Second, it has been argued by Paul Woodruff (whose view of knowledge in the early dialogues has many affinities with my own) that the \emph{techne} model is abandoned in the \emph{Meno} with the new emphasis on recollection.  And third, two prominent examples in the \emph{Meno}, the example of knowing Meno (the person) and the example of knowing the road to Larissa, evidently fit the craft paradigm poorly. In this section I will take up these topics by turn.]]



%\citet{burnyeat1987waa} and \citet{nehamas1999mps} emphasize the role of explanation and understanding. Understanding and explanation are important here, but their view seems to me to focus excessively on propositional knowledge and not to recognize what sort of understanding is in question.



Now that seems hasty. Is geometry no longer a \emph{techne}? Does Socrates imagine a future for geometry without teachers?  Or if teachers must now be known as `people who inspire recollection,' have we abandoned the teachability criterion or only come to a new understanding of `teaching'?\footnote{Woodruff himself allows that the \emph{Meno} is not consistent on the teachability question, and says that the suggestion that there might be knowledge without teachers ``is resisted in the balance of the dialogue'' (83). The teachability criterion is important as part of the \emph{Meno'}s guiding hypothesis that if virtue is knowledge then it will be teachable.} Or couldn't Plato be toying with new ideas about how \emph{technai} are acquired?\footnote{In fact a number of familiar features of \emph{techne} reappear in the \emph{Meno}. The completeness requirement seems to show up in Socrates' insistence that the slave would have to go through many proofs repeatedly before he would have knowledge. The explanatory condition shows up at 98a, as does reliability.}


%may be acquired rather than saying that he is introducing a new conception of knowledge

%and we might also observe that the status of the teachability criterion is in doubt even in the \emph{Protagoras}.

But there is a more substantial point. The proposal that learning is actually recollection is a response to the paradox of inquiry, posed by Meno in two parts at 80d: how will you search for something if you don't know what it is? And how will you recognize it even if you find it? This paradox is a sophistical trick, as Socrates points out. And yet Plato makes use of the paradox to introduce recollection. And one reason for this is that the paradox is a genuine problem given the Socratic conception of knowledge.

%, and it makes sense that Plato would attempt to answer it seriously. That leaves open the possibility that Plato's response amounts to a rejection of the Socratic conception. But it at least means that Plato takes that conception seriously.

Varieties of the paradox of inquiry already manifest themselves in earlier dialogues. At the start of the \emph{Protagoras,} we find Socrates cautioning Hippocrates, who wishes to learn wisdom from Protagoras. As it happens, Hippocrates can't say just what it is that Protagoras teaches---he suggests lamely that Protagoras will teach him ``about what [Protagoras] knows'' (312e). Socrates warns Hippocrates that a sophist is ``a sort of merchant or pedlar of goods for the nourishment of the soul,'' but that sophists don't know whether their merchandise is of any value---and this is a danger because you can't even take the sophist's goods to an expert for inspection before you consume them. Hippocrates is in a predicament: he has a vague notion that wisdom is desirable thing, but no way to pursue it. He can't safely go to a teacher unless he already possess the expertise he would learn.\footnote{cf. \citet[10--12]{nehamas1999mps}.}

%So, not knowing wisdom, how will he search for or learn it?

The \emph{Charmides} generalizes the point: only a craftsman will be able to tell whether another claimant to the same \emph{techne} is or is not an impostor (170d--c). This is perhaps not a very serious problem, if a problem at all, when it comes to, say, weaving, but it might be a problem with something like medicine (to which Socrates appeals in the \emph{Charmides}), at least in the ancient world, with its rival schools of medicine. If we add the point from the \emph{Gorgias} that only a doctor can reliably distinguish genuine health from merely apparent health (464a), then we could again pose Meno's question: supposing you think that health is a good thing, but don't \emph{know} health after the fashion of a doctor, how will you learn what it is?\footnote{This sort of question about expertise is reasonable. It's no surprise that similar questions show up in contemporary discussions as well. See for instance \citet{goldman2001experts}.} 

But the really pressing case is \emph{virtue}. Besides the general problem about inquiry, there is here the problem that there seem to be no authentic teachers at all---a point emphasized in the \emph{Meno}.\footnote{Unless we are to understand that Socrates himself is an exceptional case.} Socrates was perhaps prepared to accept that no moral \emph{techne} was attainable, and to satisfy himself with going about on his divine mission, showing that anyone who claimed such a \emph{techne} was a fraud (\emph{Apology} 23b). But Plato, as the \emph{Republic} shows, evidently had higher hopes. The difficulties gave him a reason to figure out a new view, and \emph{not} a reason to stop being interested in expertise. This meant, among other things, an emphasis on the \emph{a priori}, a corresponding shift in how the objects of \emph{technai} are understood metaphysically (i.e. as Forms), and thus also a greater emphasis on \emph{technai} like geometry which are more readily interpretable in those terms. But that's the start of another story.


%However, setting aside the passage from the \emph{Charmides}, the general assumption in the early dialogues is that you learn crafts from teachers. If, in a few unusual cases, identifying teachers is a problem, as with virtue, then you can bring along someone like Socrates to test the offerings. So back to Woodruff's point: why recollection, and what happened to teaching? Well, it seems there \emph{are} no teachers of virtue, a point emphasized in the \emph{Meno}\footnote{Unless we are to understand that Socrates himself is an exceptional case.} Now Socrates was perhaps prepared to accept that a moral craft was unattainable, and to satisfy himself by going about on his divine mission, showing that anyone who said otherwise was a fraud (\emph{Apology} 23b). But Plato, as the \emph{Republic} shows, evidently had higher hopes. So some story must be told about how moral learning happens.


%At the same time, we see a growing interest in the metaphysical nature of the objects of knowledge, or the qualities into which Socrates inquired. Thus the \emph{Phaedo} emphasizes that the objects of knowledge---the beautiful, health, equality---are sought by reasoning and not through the bodily senses (eg. 65a--e). Such a line of thought need not perhaps \emph{eliminate} the role of teachers, but it might very well lead to the sort of reinterpretation of the role of the teacher that we see in the \emph{Meno}. But none of this entails a reduced role for \emph{techne}.




%NOTE FROM THE HARDWIG VIEW THAT GOLDMAN CITES, AND FROM G HIMSELF, THAT THIS KIND OF WORRY CAN BE TAKEN QUITE SERIOUSLY EVEN NOW: H SAYS WE'RE BLIND IN RELAINCE ON EXPERTS ETC - THE PARADOX IS VERY MUCH A LIVE ISSUE IN THIS CONTEXT, I THINK...

%AG IS LESS SCEPTICAL :``A case might be made that children are in a position to get good inductive evidence that people usually make claims about things they are in a position to know about.'' - BUT SOCRATES WORRY IS APPROPRIATE - PEOPLE TAKE THEIR EXPERTISE TO RANGE MUCH FURTHER THAN IT DOES - AG DOESN'TTRY TO TKE IT VERY FAR THOUGH...\\

%(and even today the effectiveness of placebos makes the distinction between real and sham treatments problematic)

%SECOND PART OF THE PARADOX:
%The other half of Meno's question---how will you recognize what you didn't know if you come across it?---is perhaps trickier. Could a genuine doctor somehow fail to realize that he knows health, the thing he had set out to learn about? But perhaps one might worry: after all, if Meno could think, wrongly, that he had learned what virtue was from Gorgias (\emph{Meno} 71b-d), how would you know if you'd had the \emph{right} teacher? Or even if the craftsman can be confident in his knowledge, we could imagine that someone from the wrong medical school might have any number of interactions with doctors of the right medical school, and with their patients, and not realize that he had come in contact with what he had wanted to learn.

%grad school









%Finally I think that the question with regards to the presence of craft knowledge in the \emph{Meno} has to be: what becomes of craft knowledge in later dialogues, and how prominent is it? I have tried to show that many of the features of craft knowledge show up in the \emph{Meno}, sometimes very prominently, and that even the central problems and shifts---the paradox of inquiry, recollection, the problematic place of teaching---are quite intelligible within the context of a focus on specifically craft knowledge. On the other hand I would not claim that the the \emph{Meno} is a dialogue in which the constant presence of craft knowledge just leaps out at you---as I would say of the \emph{Charmides} or even the \emph{Theaetetus.} So all I'm concerned to maintain is that, while the new notion of recollection and the sharper, or more explicit, contrast between belief and knowledge constitute shifts in emphasis, nothing in the dialogue shows that craft knowledge is playing any less dominant a role than in earlier, more Socratic dialogues. For all we learn from the \emph{Meno,} the craft model may continue to be major force in Plato's epistemological thought, and the only way to go into other dialogues such as the \emph{Republic} is with an open mind.\\


















%BELIEF IN A CERTAIN SENSE IS NOT FIT TO BE K









%SOME POSSIBLE FURTHER ITEMS FOR CONCLUSION:\\

%Aristotle rejects the idea of a universal science - not too clear on how that wrks, though. Of course P wouldn't mean that one kind of knowedge could deliver actual capcity in everything. But there is certainly in some sense a universal science.\\

%Which way people read things - this may be sen as an experiment in how far we can get reading the Republic in light of the early epistemology.\\

%On my side we do have at least eg Kube and Parry I guess.\\

%Scott says that Vlastos' "Socrates: Ironist and Moral Philosopher" epitomizes the approach of looking at the first part of Meno for early elements and later part of Meno with the epistemology etc for later elements, so maybe use that.\\

%[[or intro]] For us the question ``does he know whether X is unjust?'' doesn't distinguish the specialist from the dilettante; after all the dilettante might know that X is unjust because he asked the expert. Socrates himself sometimes speaks of knowing that way, but the thrust of his thinking is that knowing that X is unjust is a mark of an expert: you don't count for dabbling, no matter how---and however soundly---you may have come by your information.\\










\begin{comment}


\section*{Appendix: Two Examples in the \emph{Meno}}


Again the \emph{Meno} will make it clear that stability is a mark of knowledge in general, as opposed to just moral knowledge (98a).\\


The earlier dialogues generally have the moral life particularly in view, but the \emph{Meno} makes it clear that belief in general is changeable, by contrast with knowledge: on its own, whether true or false, belief is unstable and flighty, and Socrates likens beliefs to runaway slaves, or the unsecured statues of Daedelus (97c--98b).\footnote{cf. also \emph{Hippias Minor} 372d--e, \emph{Euthyphro} 11b--c, \emph{Meno} 97c--98b.}\\

Twice above I appealed to a passage in the \emph{Meno} which I take to indicate that features of beliefs and [\emph{technai}] which I identified as holding in at least some specific cases are indeed features of belief and knowledge generally. In that passage, at 98a, it is said that beliefs become knowledge when tied down with ``an account of the reason why.''\footnote{in Grube's translation.} This statement will naturally seem to treat primarily of atomic knowledge, to suggest that it is analyzable partly in terms of belief, and to concern itself with what is added to belief to get knowledge. And it is not obvious that there is anything to do with [techne] lurking here. So one could fairly wonder whether knowledge here in this passage has anything to do with the form of knowledge I've been discussing, and whether this passage doesn't offer (or at least suggest) a different account of atomic knowledge than the one I just proposed.

%OCCURS TO ME THAT A BUNCH STUFF THAT SUSANNA SAID ABOUT BELIEFS AND CAUSATION ETC APPLIES HERE.

If at 98a Plato is primarily concerned with an analysis of knowledge in terms of beliefs, that would mark a departure from the earlier centrality of [techne]. The passage does certainly describe a process whereby beliefs become knowledge. However, this process is rather opaque, since the phrase ``an account of the reason why'' (or ``the cause''), is not terribly perspicuous. But, like most scholars, I take it to imply that knowledge requires some kind of understanding, or a capacity to explain why things are as they are. That seems to be borne out by Socrates identification of account giving with recollection---which is to say that he identifies giving an account with the sort of thing Meno's slave had embarked on earlier in the dialogue.\footnote{At 82b or 84a. Or perhaps properly speaking recollection is what the slave might go on to do, as Nehamas says---\citet{nehamas1999mps}, p. 18.} In that earlier conversation, Socrates and the slave together go through a demonstration to the effect that, taking one square, a second square of twice the area can be constructed by taking the diagonal of the first as a side of the second. But Socrates does \emph{not} say that the slave then knows that this can be done. Rather he says that ``these opinions have now just been stirred up like a dream, but if he were repeatedly asked these same questions in various ways, you know that in the end his knowledge about these things would be as accurate as anyone's'' (85c-d). Now this hardly tells us \emph{exactly} what or how much more the slave would have to understand before he could be said to know any geometry, but it certainly suggests that Socrates is not interested just in knowledge of particular, isolated facts, but in facility with a \emph{bodies} of knowledge, as in the earlier dialogues. A continued interest in [techne] would account for that.

That very tentative conclusion may be supported by the observation that although ``account of the reason why'' ([\emph{aitias logismos}]) is not very perspicuous on its own, it is suggestive of the discussions of [techne] in the \emph{Gorgias.} At 465a we have Socrates saying that ``cookery''\begin{quote}is not a craft, but a knack, because it has no rational account (\emph{logos}) by which it applies the things it applies, to say what they are by nature, so that it cannot say what is the explanation (\emph{aitia}) of each thing . . . (Irwin trans.)\end{quote}The point is reiterated at 501a with the example of medicine:\begin{quote}I said that medicine has considered the nature of what it cares for and the explanation (\emph{aitia}) of what it does, and can give a rational account (\emph{logos}) of each of these things.\end{quote}It could be that, in the \emph{Meno}, Plato is not committed to the [techne] model in particular, but that he remains committed to the place of explanatory facility in knowledge generally. Still, this is at least a point of continuity, and at best it offers us (the readers, if not Meno) a way of interpreting an otherwise fairly opaque remark about knowledge. The kind of explanatory ability required here would then be the general capacity to explain the principles and applications of a body of knowledge defining a [techne], at least so far as relevant to the particular beliefs in question.

Having said that, we could certainly also say---and I do not believe Plato would be averse to saying---that once the slave becomes expert in geometry, he will know things which previously he only believed, by tying the old beliefs down with the explanations he now has at hand. So the beliefs at 98a might themselves become instances of atomic knowledge when their possessor can give the right sort of account. But if I am right, it would remain the case that, for beliefs to become knowledge, you would effectively have to add \emph{knowledge} to them---that is, [techne] or some degree thereof.\footnote{The notion of `explanation' in Gail Fine's proposal that knowledge in the \emph{Meno} is justified (roughly: explained) true belief seems to me so expansive as to encourage just this conclusion---that (atomic) knowledge is true belief plus knowledge. [CITE?]} It would then remain the case that there is knowledge of the atomic sort only in the presence of [techne], in the way described above.

It is certainly true that, between the idea of recollection and the idea of somehow developing knowledge by way of belief, there is something new in the \emph{Meno} in terms epistemology. And that demands explanation. I do not believe that the correct explanation reveals the abandonment of the [techne] model. But I defer this and related issues for my conclusion.\\







%NOT SURE WHETHER I'LL STILL WANT TO DISCUSS THIS AS MUCH.... ALMOST WONDER WHETHER I COULD JUST MAKE IT AN APPENDIX OR SOMETHING... ACTUALLY NOW I THINK PARTS OF THIS SHOULD GO CONCLUSION AS GENERAL COMMENTS ABOUT TRANSITION TO MIDDLE PERIOD

Two prominent epistemological examples in the \emph{Meno} don't fit easily with the \emph{techne}-model, so I wish to address them briefly. The first occurs at 71b, where Socrates says that he couldn't know what qualities Meno possesses (say looks or riches) if he didn't know who Meno was. Socrates uses the example to illustrate the point that he can't say whether virtue can be taught since he doesn't know what virtue is. The other is the `road to Larissa' example [CITE]. This example is mean to illustrate the equal practical efficacy of knowledge and true belief.

Evidently neither Meno nor the road to Larissa is suitable material for a \emph{techne}, just as there is no \emph{techne} dealing with the poems of Homer in particular. Moreover the latter example has been cited to show that knowledge is ``a state improved with regard to the very same things that the unimproved state [i.e. belief] is concerned with.''\footnote{Julia \citet{annas1981ips}, p. 192. Annas also mentions the flute example in \emph{Republic} X and the the jury example in the \emph{Theaetetus}. See also Gail \citet{fine2004ktb}, p. 43.} Whereas I have resisted (except in relation to atomic knowledge) the sort of `analysis of knowledge in terms of belief' interpretation which this might suggest.

Now although neither example fits the \emph{techne}-model, they don't very well fit the epistemology of the \emph{Meno} as a whole, either, as Dominic \citet{scott2006psm} points out regarding the road to Larissa case. This is because it's not clear what knowing Meno or knowing the road to Larissa has to do with either recollection or explanation. And since it seems clear that the examples don't really suit the epistemological outlook of \emph{Meno}, it seems unnecessary to worry too much about them not suiting the \emph{techne}-model.\footnote{Interestingly, Plato also uses the road example to illustrate the difference between knowledge and belief at \emph{Republic} 506c, and there the example is deployed in an even more off-hand fashion, and is equally out of keeping with the general discussion.} The fact that Plato is quite happy to use these particular examples speaks more to the fact that Plato does not distinguish forms of knowledge the way we might than to the direction of his epistemological thought.

One the other hand, even if knowledge of Meno or the road to Larissa is not systematic or explanatory in a way befitting a \emph{techne}, there are affinities. Part of the point of the examples is presumably that knowing Meno or the road entails a degree of familiarity, and knowledge of the thing will entail, or be the source of, a great deal of \emph{further} knowledge. As a doctor recognizes ailments by their symptoms, someone who knows Meno will be able to recognize him, too. Knowing medicine means knowing various specific techniques; knowing a road means knowing its twists and turns, and knowing Meno means knowing things \emph{about} him. And we could expect knowledge of the road to Larissa to be reliable, in something like the way \emph{techne} is. (If you've travelled the road to Larissa often, it's unlikely someone could talk you into changing your view as to whether it goes through Chalcis.) So the examples share enough features (at least arguably) with the sort of systematic and explanatory knowledge Plato \emph{is} primarily interested in that they can serve to make the points Plato wants to make.\footnote{In fact the Meno example illustrates a point already familiar from earlier dialogues.}

What about the point that the road to Larissa example shows that knowledge and belief have a common object. It is less tempting to say here again that the examples are only imperfect illustrations of the sort of knowledge Plato is really concerned with, because other examples could be adduced, such as the case where Socrates says (at 85c--d) that the slave could come to know what he now only believes.\footnote{Fine does indeed mention this example too.} But experience with the earlier dialogues offers us a fine response.\footnote{Annas and Fine are looking ahead to disputes about the nature of Plato's middle-period metaphysics and epistemology. They want to resist too firm a distinction between the intelligible objects of knowledge on the one hand and the sensible objects of belief on the other. } Plato just isn't fastidious about, if cognizant at all of, the difference between knowledge of the \emph{techne} as a whole and knowledge about aspects of the \emph{techne} or about applications of the \emph{techne}. In the latter cases, knowledge may very well be constituted by belief, and that's enough to account for the fact that the same objects are sometimes spoken of as objects of both belief and knowledge.

%On the one hand a doctor is concerned with eg. sight, that by which an eye sees. This is not a visible thing. On the other hand he is concerned with eyes, which are visible, and with making them see.

\end{comment}


\bibliographystyle{apalike}
\bibliography{bibliography}

\end{document}