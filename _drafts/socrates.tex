% Dropbox/Dissertation/Socrates.tex

% !TEX TS-program = xelatex
% !TEX encoding = UTF-8

%COMPILATION COMMANDS:
% xelatex -interaction=nonstopmode -file-line-error-style -synctex=1  -output-directory="Dropbox/Dissertation" "Dropbox/Dissertation/Socrates.tex"
% bibtex "Dropbox/Dissertation/Socrates"

%GIT REPOSITORY:
%git clone git@git.scribtex.com:britchie/dissertation.git
% git pull
    
% cp Dropbox/Dissertation/* dissertation/
% git add file1 file2 file3             %git diff --cached              %git status             %git commit
% git commit -a (this will automatically notice any modified (but not new) files without running git add beforehand, add them to the index, and commit, all in one step.)
% git push git@git.scribtex.com:britchie/dissertation.git master

%TURN ON LAYOUT SWITCHING
%setxkbmap -option grp:switch,grp:alt_shift_toggle,grp_led:scroll us,el

\documentclass[11pt,letterpaper,oneside]{amsart} %FOR PRINTING
%\documentclass[12pt,letterpaper,oneside]{amsart} %FOR SCREEN READING

%FOR SCREEN READING
\usepackage{geometry}
\geometry{paperheight=8in,paperwidth=6in,top=0.5in,bottom=0.5in,left=0.5in,right=0.5in}


%\usepackage{custom}
\usepackage{customxetex}


\author{Brendan Ritchie}
\title{Socrates' Conception of Knowledge}

\begin{document}

\maketitle

%Rep II 359e--360a Glaucon makes the observation about a craftsman knowing what he can and can't do - obviously takes this to mean actually succeeding. - he's explicit about failure entailing ineptitude - ``complete injustice''

%THINGS TO GET:

%GISELA'S PAPER
%WOODRUFF'S EVERSON PAPER
%Rod Jenks, Plato on Moral Expertise (2008)
%SPRAGUE IS ALSO ALL ABOUT THIS
%PARRY BOOK
%Wisdom in practice: Socrates' conception of techne by Roberts, Clifford Masood, M.A., QUEEN'S UNIVERSITY , 2007, 70 pages; MR37341


%SUBHEDINGS ARE WIERD - EG II.0.1 MAYBE JUST ELIMINATE THE NUMBERING?

\section{Introduction: Socrates as an Epistemologist}

When the layman is asked, ``give me an example of something you know,'' he will tend to produce examples apt for rendering in propositional form: ``two plus two equals four,'' or ``Addis Ababa is the capital of Ethiopia.'' He could, with equal propriety, mention some of his skills---``carpentry'' or ``how to play the piano''---but he doesn't. Contemporary epistemologists also address the latter sorts of cases only exceptionally or secondarily. The emphasis is rather on knowledge of particular facts. Thus a recent introductory text in epistemology can open by saying flatly that ``knowledge is constituted by belief.''\footnote{Robert Audi, in his introduction to \citet[p.\ 1]{huemer2002ecr}.} One would not merely assume such a thing of the carpenter's knowledge, or of the pianist's.

%``knowledge is constituted by belief (of a certain kind).''

%stanley2001knowing

%If you ask our contemporaries for an example of something they know, they'll offer you something in (or suitable to) propositional form:

Socrates, by contrast, gravitated in his philosophizing toward the examples provided by the skilled, typically professional practices known to the Greeks as `τέχναι.' His contemporaries noticed this fondness. Xenophon records the Athenian tyrants, concerned about the bad light being shed on their ability as rulers, issuing Socrates a warning:\begin{squote}ὁ δὲ Κριτίας, ἀλλὰ τῶνδέ τοί σε ἀπέχεσθαι, ἔφη, δεήσει, ὦ Σώκρατες, τῶν σκυτέων καὶ τῶν τεκτόνων καὶ τῶν χαλκέων: καὶ γὰρ οἶμαι αὐτοὺς ἤδη κατατετρῖφθαι διαθρυλουμένους ὑπὸ σοῦ.

\vspace{0.05in}

\noindent ``You see, Socrates,'' explained Critias, ``you will have to avoid your favourite topic,---the cobblers, builders and metal workers; for it is already worn to rags by you in my opinion.''\footnote{\emph{Memorabilia} I.i.37 [CORRECT? OR I.ii.37?], trans.\ \citetalias{marchant1923xenophon}.}\end{squote} That fondness is also manifest in Plato's earlier dialogues,\footnote{By ``earlier dialogues,'' I mean the \emph{Apology, Charmides, Crito, Euthydemus, Euthyphro, Gorgias, Hippias Major, Hippias Minor, Ion, Laches, Lysis,} and \emph{Protagoras}; I also draw on the \emph{Meno} and \emph{Republic} I.\label{dialogues} (I do not mean to suggest that similar examples are not also prominent in later dialogues, but those are not my focus.)} and is likewise noted there (again with exasperation):\begin{squote}νὴ τοὺς θεούς, ἀτεχνῶς γε ἀεὶ σκυτέας τε καὶ κναφέας καὶ μαγείρους λέγων καὶ ἰατροὺς οὐδὲν παύῃ, ὡς περὶ τούτων ἡμῖν ὄντα τὸν λόγον.

\vspace{0.05in}

\noindent Callicles: By the gods! You simply don't let up on your continual talk of shoemakers and cleaners, cooks and doctors, as if our discussion were about them!\footnote{\emph{Gorgias} 491a, trans.\ \citetalias{zeyl1987plato}.}\end{squote} We, too, are apt to be surprised at some of Socrates' invocations of τέχναι for examples in connection with all manner of epistemological matters. So, for example, when it is suggested by Critias in the \e{Charmides} that temperance (σωφροσύνη) is knowing oneself (τὸ γιγνώσκειν αὐτόν), Socrates promptly suggests that temperance will then be analogous to medicine.\footnote{165b--c.}




%Plato on moral expertise By Rod Jenks

%His examples of knowledge are examples of τέχναι---crafts or forms of expertise, though no English word is very suitable. 

%[[want to observe how Vlastos---a great and I think sympathetic interpreter---thought that Socrates was no epistemologist - apparently because S's epistemology s just in some sense so foreign! - and expalin a bit how this could be missed - eg fact that often there are just these free propositional uses.]]

It would be hard, then, to altogether fail to notice the presence of \techne\ in Plato's early dialogues. Nevertheless, scholars have largely failed to appreciate the epistemological significance it has there. In particular, it has not been appreciated that Socrates' reflections on τέχνη---which make up a significant part of the dialogues---constitute a distinctively Socratic set of reflections on the nature of knowledge. My aim in this paper is to rectify this: I will argue that there is a distinctively Socratic conception of τέχνη which constitutes a distinctively Socratic (or early Platonic) conception of knowledge \e{simpliciter}.\footnote{When I speak of a `Socratic conception of knowledge,' I mean a conception of knowledge distinctive of the Socrates in the Platonic dialogues mentioned in n.\ \ref{dialogues} above. I say `conception' rather than `theory,' because, first, we see Socrates deploying an understanding of knowledge rather than setting out an account of it, and, second, because Socrates expresses doubts (or serious puzzlement) about how to understand knowledge. Nor do I insist that these dialogues are perfectly consistent. On the other hand, I do mean to suggest more than merely that these dialogues manifest a conception of knowledge in the same way that everybody manifests some conception of knowledge. What Socrates says reflects deliberate consideration of the topic---of the formal features of \techne, for example.}

What this thesis means is, first, that Socrates has a distinctive conception of \techne\ considered in its own right. Scholarly discussion of τέχνη in the early dialogues has largely centred on the so-called ``craft analogy''---the supposed Socratic analogy between virtue and τέχνη---and this emphasis on the relationship between τέχνη and virtue has generally been at the expense of attention (or disinterested attention) to the nature of Socratic τέχνη proper. It is not often noticed that there even is a specifically Socratic conception of τέχνη. Terence Irwin, author of perhaps the most important recent book on  Plato's moral theory, tells us that ``Socrates does not say exactly what  he takes to be implied by saying that something is a craft [i.e.\ a τέχνη] or is similar to a craft.'' That is not altogether unfair, but Irwin then goes too far:\begin{squote}Aristotle, however, has a fairly clear and explicit view of the character of a  craft. It is useful, then, to replace the rather imprecise question `Does Socrates treat virtue as a craft?' with the more precise question `Does Socrates treat virtue as the sort of thing that Aristotle regards as a craft?'\footnote{\citet[70]{irwin1995pse}.}\end{squote} Socrates certainly says enough about τέχνη to  distinguish his conception thereof from Aristotle's, as well as from  later Platonic understandings and from understandings mooted in other literature and by other characters in Plato's dialogues.\footnote{I won't claim that Socrates' conception of τέχνη (or of knowledge generally) is totally unique in the history of philosophy---his τέχνη-oriented approach to epistemology had a good deal of influence in the ancient world, not least in Plato's case. But his conception of τέχνη is different enough from that of Plato, Aristotle, and various of his contemporaries, and in sufficiently significant ways (philosophically and interpretatively), that we need to consider it in its own right if we are to understand him correctly.} %[GISELA ON HEIRS?]


% [WOULD NEED TO DO MORE TO TO SHOW THIS.]


%---besides obscuring the epistemological significance of Socratic τέχνη---

%Socrates' interest in τέχνη cannot be overlooked the way Socratic epistemological theorizing can be be. 

Second, my thesis means that Socrates is interested in, and has something distinctive to say about, epistemology. This is not always recognized: so great and sympathetic an interpreter as Gregory Vlastos was able to say that Socrates was ``exclusively a moral philosopher.''\footnote{\citet[p.\ 41]{vlastos1991socrates}.} But this is an untenable position if we keep firmly in mind the fact that τέχνη is at the very least a \e{type} of knowledge---indeed in these dialogues `τέχνη' is frequently synonymous with `ἐπιστήμη' (itself commonly translated as `knowledge'). Socratic theorizing about τέχνη is \e{eo ipso} epistemological theorizing, and Socrates does a great deal of it: he considers, for example, the nature of the objects of knowledge (\ie the subject matter of a true \techne---\e{Laches}, \e{Ion}, \e{Gorgias}), the characteristics of, and how to identify, the knower (\ie the expert or \technites---\e{Ion}, \e{Euthyphro}, \e{Gorgias}, \e{Protagoras}), the value and efficacy of knowledge (\e{Euthydemus}, \e{Hippias Major}, \e{Gorgias}, \e{Protagoras}), and the psychological role of knowledge (\e{Hippias Minor}, \e{Protagoras}). True, Socrates does not discuss our preferred cases---but neither do we discuss his. It is also true that Socrates is first of all concerned with virtue. The situation in Plato's early dialogues is nicely captured in Socrates' own response to the tyrants, as reported by Xenophon: if he cannot discuss ``cobblers, builders and metal workers,''\begin{squote}οὐκοῦν, ἔφη ὁ Σωκράτης, καὶ τῶν ἑπομένων τούτοις, τοῦ τε δικαίου καὶ τοῦ ὁσίου καὶ τῶν ἄλλων τῶν τοιούτων;

\vspace{0.05in}

\noindent Then must I keep off the subjects of which these supply illustrations, Justice, Holiness, and so forth?\end{squote}But this is a motive for, not a bar to, the pursuit of epistemology.\footnote{A few scholars have recently taken more of an interest in the Socratic conception of τέχνη in its own right, and as a matter of \emph{epistemology} rather than just ethics---most notably Paul \citet{woodruff1990pse} and Angela \citet{asmith1998}. A useful discussion with more attention to the contemporary views about τέχνη may be found in \citet[37--45]{reeve1989sita}, though Reeve's discussion generally gives the (in my view mistaken) impression that there was a single going view of \techne, albeit one which underwent some development.}
%\footnote{[AN ANALOGY FROM KUMARILA'S/THE INDIAN CASE]}

%But this only means that Socrates has strong motivation to get these cases right, not that  

Thirdly, it is my thesis that, for Socrates, knowledge is \techne. Even on those occasions when the topic of Socratic epistemology \emph{is} taken up in its own right, \techne\ is not recognized to have this epistemological significance. A common view is that Socratic knowledge is something like \emph{understanding}. But that positive suggestion misleads. Consider Hugh Benson's ``paradigmatic examples of the kind of understanding [Benson has] in mind'':\begin{squote}We say things like Einstein understands gravity, Richard Feynmann understands quantum mechanics, R. G. Collingwood understands history, or James Boswell understands Samuel Johnson.\footnote{\citet[pp.\ 211--212]{benson2000swm}.}\end{squote}But these are not like Socrates' own examples. Callicles does not complain that Socrates never leaves off talking about cosmological theories; the tyrants do not warn Socrates that he should stop talking about historical figures. It is doubtful whether Socrates would even accept that there is such a thing as knowledge of history or people.\footnote{Cf.\ \citet[p.\ 326]{allen1989dpe}.} He rejects Ion's representation of himself as an expert on Homer and \e{only} on Homer, and I see no reason why he should take a different view of ``understanding Johnson.''\footnote{\e{Ion} 531a--35a.} Again, Socrates is hilariously dismissive of Hippias' historical `knowledge' in the \e{Hippias Major}.\footnote{285d--286a: ``I forgot you had the art of memory. So I understand: the Spartans enjoy you, predictably, because you know a lot of things, and they use you the way children use old ladies, to tell stories for pleasure.'' If the bit about children and old ladies doesn't give it away, we will see in \S\ref{distinct} that doing things from memory with the aim of pleasing is for Socrates the opposite of acting out of knowledge.} The Einstein and Feynman examples are better insofar as they are meant to suggest the mastery of some general, atemporal domain. On the other hand, insofar as the examples are meant to suggest a special, perhaps even unique, insight, they are best avoided. For example, it has been said of Ramanujan that ``if [he] had any peers in the formal manipulation of infinite series, they were only Euler and Jacobi.''\footnote{\citet[p.\ 6]{berndt1985ramanujan}.} But if it is also true that Ramanujan ``never comprehended the idea of proof,''\footnote{Thus \citet{chaitin2007less}.} then he would be a poor example of a Socratic expert: the Socratic expert is supposed to be able to explain himself, and to teach what he knows.\footnote{See \S\ref{distinct} below.} Indeed Ramanujan seems to have said about himself what Socrates says about Ion: his inspiration was divine. The point is not that Socrates is uninterested in understanding, or that he would deny that Benson's people know (or understand) what Benson says they do. The point is rather that Socrates is only interested in understanding---or any other kind of knowledge---insofar as it is an aspect of \techne.


%Fairness demands that we note Brendt's assessment: ``Ramanujan basically thought like most mathematicians. In other words, Ramanujan \e{proved} theorems like any other serious mathematician'' (p.\ 8, his emphasis).

%The kind of understanding of interest to Socrates is the understanding one posses in virtue of possessing a \techne. For Socrates, knowledge is, first of all, τέχνη. What else one knows one knows only secondarily.

%Fourthly and finally, it is my thesis that, for Socrates, knowledge in \emph{only} \techne. 

%So, for example, David Woodruff, having claimed that definitional knowledge is prior to expertise (quoted above) proceeds thus:\begin{squote}Socrates' inquiry depends on knowledge about the nature of Courage. Unless his procedure is sharply out of tune with itself, the knowledge he uses must differ from the knowledge he cannot claim until after successfully defining Courage. The \e{Apology} uses a distinction between expert and non-expert knowledge that supports a defense of Socrates' inquiry. His arguments in the \e{Laches} make use only of non-expert knowledge.\footnote{Ibid.\ p\ 80}\end{squote} And that is the idea he goes on to explore.

%And contrast with the fairly clear and demanding standards for craft knowledge, Socrates would apparently have \emph{no} particular criteria for any sort of non-expert knowledge.\footnote{As \citet[p.\ 77]{woodruff1990pse} points out.}

%EXAMPLES OF THE VIEW THAT SOCRATIC KNOWLEGDE IS UNDERSTANDING: \footnote{See for example \citet[p.\ 11]{benson2000swm}, [MORE CITES].}

I proceed as follows. In \S\ref{distinct}, I explain what \technai\ are, and describe some of the contentious issues surrounding the concept. This will allow us to better understand Socrates' own conception of \techne, and how it differs from other conceptions.

In \S\ref{propositional}, I show that knowledge attributed to others by Socrates should generally be understood as manifesting \techne-possesion---for example, a man might know that so-and-so has such-and-such an ailment, and how to treat it, because he is a doctor. I also argue that, for Socrates, \techne\ is not rooted in any other form of knowledge---it is not partially or wholly constituted by prior factual knowledge, for example.

In \S\ref{varieties}, I acknowledge that there are Socratic attributions and avowals of knowledge that do not fit within the \techne-scheme. But I argue that these instances are trivial reflections of the fact that Plato wrote in a colloquial style. Only in the case of \techne\ do we find any serious epistemological theorizing.

In \S\ref{knowledge}, I conclude that, for Socrates, knowledge is \techne. In suport of this conclusion, I also provide some examples of Socrates' tendency to treat any kind of question about knowledge as a question about \techne.


% !TEX root = Socrates.tex

\section{Texnh and Socratic Texnh}
\label{distinct}

%[NEED TO SHOW IN THIS SECTION THAT THE DIFFERENCE MAKES A DIFFERENCE?]
 
\subsubsection*{\techne\ in general}

\technai, as noted, were skilled, often professional, practices. Building or carpentry (\tektonia) was a pardigmatic case; the cobbling, metal-working, shoemaking, and medicine we encountered earlier are likewise good examples, and others would include mathematics and musical arts. For the possessor of a \techne\ and the possession of a \techne\ respectively, `expert' and `expertise' are good fits. For the body of knowledge itself, as opposed to the mastery thereof, there is no very good English equivalent. In the context of Plato's dialogues, `\techne' has been translated variously as `craft,' `skill,' `expertise,' `art,' `science,' and `profession,' none of which is ideal. `Profession' entails earnings, which is too strong. `Skill' is too broad, as is `know-how': Harold Shipman was an skillful killer, but his \techne\ was medicine, which \e{imparted} that skill; again, one knows how to walk up stairs, but that is neither an expertise nor a \techne, nor in this case even a manifestation of one. `Science,' `art,' and 'craft' are the nearest English equivalents, especially on older (obsolete?) understandings; they are still sometimes good fits in particular cases (e.g.\ `science' for medicine and `craft' for carpentry), though each has nowadays too narrow a range. With that qualification understood, I will sometimes employ `craft' or `art.'


%WHAT ABOUT JUST KNOWLEDGE AS A TRANSLATION FOR TECHNE?

%except that the one nowadays suggests painting and the other Sunday-school projects and paper-mache; I will employ `craft' or else simply `\techne.'


%have these days the wrong connotations, suggesting as they do painting and sunday school projects respectrively.

%I will employ `craft,' which is the nearest English equiva lent, despite the unfortunate suggestion of Sunday school projects and paper-mache.


%---examples include medicine, carpentry, and mathematical or musical arts. On the Socratic view, possession of a τέχνη is not like know-how in Ryle's sense.\footnote{\citet{ryle1945kak}.} Ryle's know-how is not essentially articulate, but Socrates' τέχνη \emph{is}. On other views, according to which a τέχνη might be more simply a product of experience and practice, we \emph{could} think of a τέχναι as a species of know-how. But knowing how to walk up a flight of stairs, say, or how to open a door, would not be a τέχνη ---walking up stairs does not make you an expert or a specialist. But as a rule a τέχνη \emph{would} make you an expert, and it would mean that you had an ability of use to your community.

%[[ALSO WOULD WANT ELEMENTS OF BOTH K-HOW AND K-THAT IN OUR CASE---TECHNE JUST IS THE BASIC CASE, I THINK]]



Since people need and rely upon experts, `τέχνη' was a term of commendation. As such, there was dispute about whether given practices deserved the title, and various considerations were adduced in support on one or the other positon.\footnote{See for instance Isocrates' \emph{Against the Sophists} on oratory and the Hippocratic \emph{De Arte} on medicine.} A simple example is provided in the \e{Laches}, which opens with two generals in disagreement about whether ``fighting in armour'' is a τέχνη. Nicias identifies various likely advantages to studying this art, but Laches wonders whether there is really anything here to study at all:\begin{squote}ἀλλ᾽ ἔστι μέν, ὦ Νικία, χαλεπὸν λέγειν περὶ ὁτουοῦν μαθήματος ὡς οὐ χρὴ μανθάνειν: πάντα γὰρ ἐπίστασθαι ἀγαθὸν δοκεῖ εἶναι. καὶ δὴ καὶ τὸ ὁπλιτικὸν τοῦτο, [182ε] εἰ μέν ἐστιν μάθημα, ὅπερ φασὶν οἱ διδάσκοντες, καὶ οἷον Νικίας λέγει, χρὴ αὐτὸ μανθάνειν: εἰ δ᾽ ἔστιν μὲν μὴ μάθημα, ἀλλ᾽ ἐξαπατῶσιν οἱ ὑπισχνούμενοι, ἢ μάθημα μὲν τυγχάνει ὄν, μὴ μέντοι πάνυ σπουδαῖον, τί καὶ δέοι ἂν αὐτὸ μανθάνειν;

\vspace{0.05in}

\noindent It's hard to say of any study whatever that it shouldn't be learned, because it seems good to know everything, and in particular this skill in arms, if it's really the subject of study its teachers claim and Nicias says; but if it's not a subject of study and those who profess it are practicing a deception, or if it is a subject of study but not a very serious one, why learn it?\footnote{182d--e. Neither Laches nor Nicias actually uses the word `τέχνη' when speaking of fighting in armour; rather they refer to it as a (possible) \mathema\ (as here) or ἐπιστήμη. But `ἐπιστήμη,' as noted, is a frequent synonym for `τέχνη' in the early dialogues, and Socrates is treating `\mathema' as implying a τέχνη at 185e, and when he begins speaking of `τέχνη' back at 185a, Nicias assumes that the τέχνη in question would be fighting in armour (185c).}\end{squote}Laches inclines to the more sceptical view: he says that the practicioners of this supposed art have not shown themselves of any value in battle---this is his own experience, and confirmed by the fact that the Spartans (who ought to know) do not employ any of these fighters in armour. A basic expectation of a \techne, then, is that it can accomplish something; if the practicioners of a purported \techne\ cannot demostrate such accomplishments, then it is doubtful whether there is any such \techne.%\footnote{Helpful discussions of \techne\ and conceptions thereof include \citet{heinimann1961vpt}, [ALLEN, SCHIEFSKY, ETC].}



%(elsewhere pedigree is invoked, or success in training others.\footnote{\emph{Euthyphro} 16a, \emph{Laches} 185b--187b, \emph{Gorgias} 514a--515a.})


In particular, a \techne\ was expected to provide \e{reliable} success. Reliability was in turn commonly supposed to hinge on the precision (\akribeia) attainable by a \techne. In the \e{Philebus}, for example, Socrates distinguishes two varities of (``so-called'') \technai: on the one hand there are `\technai' practiced with a low degree of precision (\akribeia) which would thus be less reliable (they possess much unclarity, ``σμικρὸν δὲ τὲ βέβαιον'');\footnote{55e--56c.} on the other hand there are those practices---\technai\ properly speaking---which employ numbers and weights and measures:\begin{squote}οἷον πασῶν που τεχνῶν ἄν τις ἀριθμητικὴν χωρίζῃ καὶ μετρητικὴν καὶ στατικήν, ὡς ἔπος εἰπεῖν φαῦλον τὸ καταλειπόμενον ἑκάστης ἂν γίγνοιτο.\ldots\ τεκτονικὴν δέ γε οἶμαι πλείστοις μέτροις τε καὶ ὀργάνοις χρωμένην τὰ πολλὴν ἀκρίβειαν αὐτῇ πορίζοντα τεχνικωτέραν τῶν πολλῶν ἐπιστημῶν παρέχεται.

\vspace{0.05in}

\noindent Socrates: If someone were to take away all counting (ἀριθμητικός), measuring (μετρητικός), and weighing (στατικός) from the arts and crafts, the rest might be said to be worthless.\ldots\ As to building (\tektonikh), I believe that it owes its superior level of craftsmanship over other disciplines to its frequent use of measures (\metra) and instruments, which give it high accuracy (\akribeia).\footnote{55e--56b, trans.\ \citetalias{frede1993plato}.}\end{squote} As Mark Schiefsky observes, the presentation of this point suggests that it was not original, and Felix Heinimann, who has explained that number, weight, and measure were a standard triad of features defining \techne, likewise says that ``Platon hat die Scheidung zwischen den unexakten und den mit Zahl und Mass arbeitenden K\"{u}nsten offenbar \"{u}bernommen.''\footnote{\citet[p.\ 15]{schiefsky2005}; \citet[p.\ 194]{heinimann1975mass}.}

Just how much reliability or success should be expected of a \techne\ was a matter of dispute.\footnote{\citet{allen1994} provides an excellent survey of this topic.} Any given `expert' might of course fail at a task, if only from inexperience or lack of aptitude. The difficulty arose when even the most skilled practitioners of the so-called τέχνη were liable to failure.

%Thus consider a supposed contrast between such paradigms of \techne\ as carpentry and the writing of letters.


There was a trope to the effect that when you call the builder you will get a house---and you won't get a house otherwise.\footnote{Cf.\ Isocrates (\emph{Against the Sophists}) on writing out letters.} But other putative craftsman aren't always successful. Doctors don't always restore people to health, and, what's more, people sometimes regain health without the aid of doctors. On such grounds some denied that medicine was a τέχνη at all.\footnote{No doubt the contrast is spurious---a builder needs good materials, and on the other hand, nature supplies dwellings in the forms of caves.} The orator or sophist, too, was susceptible to such charges, especially given the boast that he could teach nothing less than virtue. Again it was observed that these teachers were not always successful (how interesting that they insist on securing payment up front!), and that their students were sometimes surpassed by the untrained.\footnote{For this sort of objection see e.g.\ \emph{Protagoras} 319a--320b, \emph{Laches} 185d--e, Isocrates \emph{Against the Sophists} 14, \emph{Dissoi Logoi} 6.5--6, \emph{De Arte} 4; cf.~\citet[PP.\ 123--6]{heinimann1961vpt}.}

It seems that some doctors accepted the terms of this challenge. One Hippocratic text, \e{On Regimen in Acute Diseases}, claims of the study of regimen that\begin{squote}καὶ γὰρ τοῖσι νοσέουσι πᾶσιν ἐς ὑγείην μέγα τι δύνασθαι, καὶ τοῖσιν ὑγιαίνουσιν ἐς ἀσφαλείην, καὶ τοῖσιν ἀσκέουσιν ἐς εὐεξίην, καὶ ἐς ὅ τι ἂν ἕκαστος ἐθέλῃ.

\vspace{0.05in}

\noindent To the sick it is a powerful aid to recovery, to the healthy a means of preserving health, to athletes a means of reaching their best form and, in short, the means by which every man may realize his desire.''\footnote{\emph{Regimen in Acute Diseases} 9, trans.\ \citetalias{chadwick1950medical} [ACTUALLY \S3? ALSO QUOTE ACCURATE?].}\end{squote}Because we'll run into it again, it's worth noting that the ``all or nothing'' attitude underlying this challenge and defence is surprisingly common in the history of medicine. We also find it in Chinese and Indian medical texts, for example.\footnote{The Song emperor Huizong claimed of the doctrines contained in his medical encyclopedia that ``a physician who uses or applies them can end [all] diseases.'' (Quoted in \citet[p.\ 181]{goldschmidt2008evolution}; cf.\ \citet[pp.\ 67--70]{needham2000science}.) And according to the \emph{Caraka Sa\d mhit\=a}, a classic text of Indian medicine (ca.\ 3rd c.\ BC) the ``physician fit for a king knows: cause, symptom, cure and prevention of all diseases'' (\emph{S\=utrasthana} IX.19, trans.\ \citetalias{sharma2004charaka}; cf.\ Halbfass, \citeyear{halbfass1991therapeutic}).} Even a widely-employed contemporary version of the Hippocratic Oath calls upon doctors to avoid the trap of ``those twin traps of over-treatment and therapeutic nihilism [i.e.\ the view that medical treatment is useless or harmful, either in general or in a particular sphere];''\footnote{The lines are from the ``\href{http://www.pbs.org/wgbh/nova/doctors/oath_modern.html}{Lasagna Oath},'' penned in 1964. On therapeutic nihilism see e.g.\ Starr, \citeyear{starr1976politics}.} that such a warning should have seemed necessary is testament to the enduring force of the view that medicine should be able to do anything if it is not worthless. (As an indication of how general this temptation is, we may note the absence of a convenient English word for `a lack of success' which doesn't also carry a tone of censure.)



%``those twin traps of overtreatment and therapeutic nihilism''


% It has been reported, for example, that ``in the nineteenth century [in America], many people, even some leading physicians, believed that most medical practice was completely ineffective, or even harmful,''\footnote{\citet[p.\ 24]{starr1976politics}.}


% building or carpentry (\tektonia), which seems to have been an early paradigm of τέχνη , and medicine.\footnote{\citet[19--20]{roochnik1986}. Cf.\ Isocrates (\emph{Against the Sophists}) on writing out letters.} 
 
%A τέχνη was expected to provide success within its proper domain, and being prone to failure could call the τέχνη-status of a given practice into question. Nevertheless some forms of failure could be thought compatible with τέχνη. 





%But so far that says nothing about potential limits on τέχνη proper.



%One possibility is that, whatever might ideally be accomplished through τέχνη , any given craftsman might make mistakes, as a result, perhaps,  


%might check Schiefsky p10, Diseases and VM

%\footnote{[CITES for the trope about building- could add the the letters example]} 


%fix the refs in forgoing trope was from Allen perhaps? Or Roochnik?s



%Schiefsky 116 (more of this there
%The author of VM himself notes that the difference between good and bad practitioners is often not clear to lay people (9.4–5); the author of Acut. goes so far as to claim that differences in practice among doctors often led lay people to conclude that medicine was not a τνη at all (Acut. 8, 39.10– 20 Joly, L. 2.240–244; cf. ch. 6, 38.6–18 Joly, L. 2.234–238).



%ISOCRATES MAYBE AN EXAMPLE OF THE TENSION BETWEEN TWO OPTIONS


%[THE SECRET ETC?]


%These are not `principles' in the proper sense, as they obviously commit the doctor to no more than he would have been committed to anyway, since overdoing things and not doing enough are always errors by definition. The lines are included not because these `requirements,' or rather qualifications, wouldn't otherwise apply, but rather that, to a rather astonishing extent, both doctors and laymen have tended throughout history to suppose that medicine should be able to do everything if it can do anything at all. ( But in this case strength of will does seem to be called for.





%The all or nothing attitude the challeneg and the answer have in common seems to be a recurring theme---see Starr and the Oath





%For discussion and citations see \citet[pp.~84--5]{allen1994}.


A somewhat more modest and more sophisticated response to the challenge is offered in \emph{De Arte}, another Hippocratic text. We are told that nature (φύσις) sets limits on what is possible for the medical art. Thus fire is the ultimate cauterizer available to medicine, and if this cauterizer is sometimes insufficient, that is no fault of medicine. The doctor still knows what ailments are amenable to treatment, and what to do in those cases. It is significant that φύσις is said to set the limits here. Τέχνη was often viewed as a way of mastering or overcoming τύχη---i.e. chance or luck.\footnote{\emph{Gorgias} 448c, \emph{Euthydemus} 279ff., \emph{De Arte} 4--7, \emph{VM} 1, 12, \emph{Places in Man} 41, 44, 46, Xenophon \emph{Memorabilia} III.ix.14--15; cf.\ \citet[54--6]{vlastos1946eap}, \citet[108, 123--4]{heinimann1961vpt}, \citet{allen1994}.} The author of \emph{De Arte} can say that medicine, like any good τέχνη, is indeed insurance against the vicissitudes of fortune, but that while the doctor will understand the inexorable workings of nature, he could not be expected to resist them.\footnote{As for the fact that some people get better without a doctor, that may be credited to their doing what the doctor \emph{would} have recommended (\emph{De Arte} 5--6).} Indeed, medicine has been singled out quite unfairly; what has been said about medicine's reliance on available resources goes just as well for other \technai:\begin{squote}[GREEK]

\vspace{0.05in}

\noindent Other crafts are exercised on materials on which mistakes can easily be rectified, as is the case with those which employ wood or hides\sdots but the craft cannot be practised at all if one of the materials be missing.\footnote{\e{DA} 11.}\end{squote} But \emph{De Arte} still offers us an impressively demanding standard for τέχνη: James Allen observes that the author ``seems to envisage\ldots\ a physician who will at no point undertake cures with less than complete confidence of success,''\footnote{\citet[p.\ 85]{allen1994}.} and indeed that is explicitly the point: we are told that the aims of medicine, which it ``does accomplish\ldots\ and is ever capable of,'' are \begin{squote}[QUOTE]

\noindent the complete removal of the distress of the sick, the alleviation of the more violent diseases and the refusal to undertake to cure cases in which the disease has already won the mastery, knowing that everything is not possible to medicine.\footnote{\e{DA} III, trans.\ \citetalias{chadwick1950medical}.}\end{squote} \e{De Arte} acknowledges that doctors do sometimes make incorrect diagnoses and prescribe ineffective therapies, but these failures are due to patient error of one kind or another.\footnote{\e{DA} 7, 11.} The doctor can then claim, not absolute power over disease, but still perfect precision and reliability---which is all any τέχνη could claim.






%NEED TO GET: WITTERN \& PELLEGRIN, Hippokratische Medizin und antike Philosophie

The author of \e{De Arte} again has company. Thus Glaucon in the \e{Republic}:\begin{squote}A first-rate captain or doctor\ldots\ knows the difference between what his craft can and can't do. He attempts the first but lets the second go by, and if he happens to slip, he can put things right.\footnote{\emph{Republic} II, 360e--361a, trans.\ \citetalias{grube1992plato}. Cf.\ also \emph{Prognosis} 1.}\end{squote} And such a view has again been a recurring one in the history of ideas. For example, Albert Schweitzer reported in the 1920s that \begin{squote}in the opinion of the Black Africans the efficacy of medicine is demonstrated primarily in that the physician knows whether the patient will die or not, and in that the physician does not apply his art to someone who is really already dead.\footnote{\emph{Briefe aus Lambarene 1924--1927}, quoted by \citet[p.~181]{horstmanshoff1990ancient}.}\end{squote} It appears to remain influential in modern societies: according to a 1977 discussion of the attitudes and expectations of doctors, ``patients who defy diagnosis, or who obstinately refuse to get better tend to be regarded as `awkward', `attention-seeking', or `neurotic'.''\footnote{\citet{holden1977needs}.} 

%the aforementioned contemporary medical \e{Oath} also has the doctor swear that he ``will not be ashamed to say `I know not','' evidently precisely because that is still hard for doctors to do: a 1977 discussion of the respective expectations of doctors and patients said that patients believe ``the doctor will \emph{know} what is wrong'' (emphasis in original), and suggests that doctors themselves incline somewhat to this view, as betrayed by the fact that ``patients who defy diagnosis, or who obstinately refuse to get better tend to be regarded as `awkward', `attention-seeking', or `neurotic'.''\footnote{\citet{holden1977needs}.} (\e{DA} likewise suggests that the patients are to blame when treatments don't work out.\footnote{\e{DA} 7.}.)

A third approach is suggested by the author of a third Hippocratic text, \emph{On Ancient Medicine}. He is seemingly willing to countenance principled limits on the precision available to doctors:\begin{squote}δεῖ γὰρ μέτρου τινὸς στοχάσασθαι. μέτρον δὲ οὔτε ἀριθμὸν οὔτε σταθμὸν ἄλλον, πρὸς ὃ ἀναφέρων εἴσῃ τὸ ἀκριβές, οὐκ ἂν εὕροις ἀλλ᾽ ἢ τοῦ σώματος τὴν αἴσθησιν. διὸ ἔργον οὕτω καταμαθεῖν ἀκριβέως, ὥστε σμικρὰ ἁμαρτάνειν ἔνθα [20] ἢ ἔνθα. κἂν ἐγὼ τοῦτον τὸν ἰητρὸν ἰσχυρῶς ἐπαινέοιμι τὸν σμικρὰ ἁμαρτάνοντα. τὸ δὲ ἀτρεκὲς ὀλιγάκις ἔστι κατιδεῖν.

\vspace{0.05in}

\noindent One must aim at a measure (μέτρον); but you will find no measure (μέτρον)---nor number (ἀριθμόσ) nor weight (σταθμόσ) besides---by referring to which you will know with precision (\akribeia), except the feeling of the body. Hence it is difficult to acquire knowledge so precise (ἀκριβῶς) that one errs only slightly in one direction or the other. And I would strongly praise this doctor, the one who makes only small errors; perfect accuracy (τὸ ἀτρεκές) is rarely to be seen.\footnote{\e{VM} 9.3--4, trans.\ \citetalias{schiefsky2005}; On precision in the \e{VM} see \citet[app.\ 2]{schiefsky2005}.}\end{squote} The author of the \e{VM} here indicates that medicine, like any proper \techne, is possessed of a μέτρον, while also observing that not every μέτρον allows for equally easy attainment of precision. In that case, a doctor could be wrong or off sometimes or to some degree and still be a genuinely good doctor.

It is Aristotle who first plainly states what I take to be the proper view. He points out early in the \e{Nochomachean Ethics} that ``precision (τὸ ἀκριβές) is not to be sought for alike in all discussions, any more than in all the products of the crafts (ἐν τοῖς δημιουργουμένοις),'' and says that ``it is the mark of an educated man to look for precision (τὸ ἀκριβές) in each class of things just so far as the nature (φύσις) of the subject admits.''\footnote{\e{NE} I.3.} He elaborates on the thought later:\begin{squote}[QUOTE]

\vspace{0.05in}

\noindent But this must be agreed upon beforehand, that the whole account of matters of conduct must be given in outline and not precisely (\akribos), as we said at the very beginning that the accounts we demand must be in accordance with the subject-matter; matters concerned with conduct and questions of what is good for us have no fixity (οὐδὲν ἑστηκὸς ἔχει), any more than matters of health. The general account being of this nature, the account of particular cases is yet more lacking in exactness (ἀκριβής); for they do not fall under any art (\techne) or precept but the agents themselves must in each case consider what is appropriate to the occasion, as happens also in the art of medicine (ἡ ἰατρική [\techne]) or of navigation.\footnote{\e{NE} II.2.}\end{squote} Note that medicine is for Aristotle a genuine \techne, despite its admitted lack of precision. The idea is that an expert works with materials not themselves provided by his knowledge, so knowledge alone cannot ensure success. We ought then to distinguish \emph{competent application of methods} from \emph{successful outcome}, or the \emph{exercise} of an art from the \e{aim} thereof.\footnote{Cf.\ \emph{Top}.\ 1.3 (101b5--10), \emph{Rhet.} 1355b10--11, 25--6. On the notion of a \techne\ στοχαστική see \citet[pp.\ 88 ff.]{allen1994}.} Such distinctions may seem obvious, but centuries later we still find Alexander of Aphrodisias (2nd--3rd c.\ AD) urging these Aristotelian points against the the Stoics,\footnote{\emph{Supp.\ De Anima} 160.1 ff.} and we've also seen that doctors needed reminding of them as recently as 1977.

    

%the aforementioned contemporary medical \e{Oath} also has the doctor swear that he ``will not be ashamed to say `I know not','' evidently precisely because that is still hard for doctors to do: a 1977 discussion of the respective expectations of doctors and patients said that patients believe ``the doctor will \emph{know} what is wrong'' (emphasis in original), and suggests that doctors themselves incline somewhat to this view, as betrayed by the fact that ``patients who defy diagnosis, or who obstinately refuse to get better tend to be regarded as `awkward', `attention-seeking', or `neurotic'.''\footnote{\citet{holden1977needs}.} (\e{DA} likewise suggests that the patients are to blame when treatments don't work out.\footnote{\e{DA} 7.}.)




 

%cf. \citet[88 ff.]{allen1994}.} To those of us who do not hope for a salvific knowledge, these points may seem obvious, but 

%But again these are later developments, and they separate the Aristotelian and the later Platonic conceptions of τέχνη from the earlier Socratic conception.




%ARISTOTLE TOPICS 1.3: We shall have a complete grasp of our method when we are in the same condition as in the case of rhetoric, medicine, and other such abilities. [And that is: to do what we choose with what is available.] For the rhetorician will not convince under all circumstances, nor the physician heal; however, if he leaves out nothing that is possible, then we shall say that he has a sufficient grasp of his craft.





%In later dialogues Plato will take a view more generous than either of these.---TRUE? While in the \emph{Gorgias} Socrates insists that mere experience and geussing are the marks of something \emph{less} than τέχνη (462b--c, 465a, 501a) [[SUGGESTING....]], 

%\g{smikr'on d`e t`o b'ebaion}'

%[EXAMPLES, VERY MUCH SECONDARY]

%VM 9 (TRANS SCHIEFSKY P.85)
%9. 1 Now if it were as simple as has been suggested, and stronger foods harmed while weaker ones benefited and nourished both the sick and the healthy, then things would be easy: for it would simply be necessary to lead a patient towards the weakest diet, and one could do so with a good deal of security. 2 But in fact the error is no less, nor does it harm the human being less, if one administers food deficient in quantity and quality to what is needed: for the might of hunger penetrates forcefully into the human constitution to lame and weaken and kill. And many other ills, different from those arising from repletion but no less serious, also arise from depletion. 3 For this reason the doctor’s tasks are much more varied and require more precision. For one must aim at a measure; but you will find no measure—nor number nor weight besides—by referring to which you will know with precision, except the feeling of the body. Hence it is difficult to acquire knowledge so precise that one errs only slightly in one direction or the other. 4 And I would strongly praise this doctor, the one who makes only small errors; perfect accuracy is rarely to be seen. For I think that most doctors are in the same situation as bad helmsmen. These people, when they err while steering in a calm sea, are not revealed; but when a great storm and a driving wind takes hold of them, it is manifest to all that they have lost their ship through ignorance and error. 5 The same holds for bad doctors, who make up the great majority: when they treat patients suffering from a condition that is not serious, patients who would not be seriously harmed even if one were to make the greatest errors—there are many such diseases, and they come upon people much more often than serious ones—in such cases their errors are not evident to laymen. But whenever they meet with a great, powerful, and dangerous disease, then their errors and incompetence are evident to all. In both cases retribution is not far off, but swiftly at hand.












%Less ambitious attitudes towards τέχνη and \emph{tuche} may be found in other Hippocratic writings.

%[[KEEP THIS COMPARISON TO BUILDERS??]]And to that ext ent, the doctor compares favourably with the builder. Perhaps you cannot summon a doctor and expect to wind up healthy the way you can summon a builder and expect to wind up with a house, but nature places limits even on the builder---he cannot pull houses out of the air, after all.


Another point of occasional disagreement concerned the proper or necessary theoretical basis of the requisite precision and reliability. For example, the author of \emph{VM} observes that\begin{squote}[GREEK]

\vspace{0.05in}

\noindent some doctors and sophists say that it is impossible for anyone to know medicine who does not know what the human being is; anyone who is going to treat patients correctly must, they say, learn this. Their account tends towards philosophy, just like Empedocles or others who have written about nature from the beginning, what the human being is and how it originally came to be and from what things it was compounded.\footnote{\e{VM} 20.1.}\end{squote}He replies that medicine is in its own right a source of knowledge about nature, and indeed prior to the kind of knowledge envisioned by these other thinkers.\footnote{\e{VM} 20.2.} The author's own view is that\begin{squote}[GREEK]

\vspace{0.05in}

\noindent This I think is what it is necessary for a doctor to know about nature (\phusis) and to make every effort to know, if he is going to do any of the things that he must: what the human being is in relation to foods and drinks, and what it is in relation to other practices, and what will be the effect of each thing on each individual---not simply that `cheese is harmful food, for it causes trouble to one who has eaten too much of it', but rather what trouble, and why, and which of the things in the human being it is inimical to.\footnote{\e{VM} 20.3.}\end{squote}Mark Schiefsky sums up the view thus:\begin{squote}The author’s position is clearly quite different from that of the [later] Empiricists. Like them, he rejects certain kinds of theorizing as speculative and irrelevant to medicine. But unlike them he insists that medicine must be based on an explanatory theory of human \phusis\ that makes reference to factors that cannot be observed directly.\footnote{\citet[p.\ 357]{schiefsky2005}. Schiefsky argues for a late 5th-century dating of the \emph{VM} (pp.\ 63--4).}\end{squote}











%PLATO? ARISTOTLE? OTHER VIEWS? ROLE OF EG MATHEMATICS?

%The empiricists claimed for their part that the kind of experience envisioned was impossible, incomplete, or not rising to the level of techne (ἄτεχνος).\footnote{Ibid.}




There is another example of a dispute on this point in \e{Gorgias}, where Socrates is unsatisfied with Polus' statement that ``it is experience that causes our times to march along the way of craft (\techne), whereas inexperience causes them to march along the way of chance (\tuche).''\footnote{448c; cf. 462a.} He says that when some practice ``has no rational account by which it applies the things it applies, to say what they are by nature, so that it cannot say what is the explanation of each thing,'' then that practice is merely a product of experience (\empeiria), and not a τέχνη. Socrates goes on: ``I don't call anything a craft which is unreasoning (\alogos)'' (465a). He derides Polus' own `τέχνη' of rhetoric, comparing it to ``cookery,'' which ``by habit and experience\ldots keeps only memory of what usually happens.''\footnote{501a.} \begin{squote}Socrates: I was saying, wasn't I, that I didn't think that pastry baking is a craft (\techne), but a knack (\empeiria), whereas medicine is a craft (\techne). I said that the one, medicine, has investigated both the nature (\phusis) of the object it serves and the cause (\aitia) of the things it does, and is able to give an account (\logos) of each of these. The other, the one concerned with pleasure, to which the whole of its service is entirely devoted, proceeds toward its object in a quite uncraftlike way (ἀτεχνῶς), without having at all considered either the nature of pleasure or its cause. It does so completely irrationally (ἀλόγως), with virtually no discrimination. Through routine and knack it merely preserves the memory of what customarily happens, and that's how it also supplies its pleasures.\footnote{500e--501a.}\end{squote}

%FIRST LINE OF ABIVE QUOTE: Come then, and agree further with me about what I was saying to them too, if you think that what I said then was true. 


We can distinguish in this passage three Socratic objections to the notion of a \techne\ based strictly on experience. The first is that mere experience is insufficient for attaining the reliability characteristic of the expert: hence Socrates' claim that pastry making ``merely preserves the memory of what customarily happens,'' and proceeds ``with virtually no discrimination,'' whereas the doctor knows why he does the things he does---which I take to mean that the doctor does \e{each} of the things he does with attention to any relevant peculiarities of given circumstances.\footnote{cf. \emph{VM} 20: doctors know ``what the human being is in relation to foods and drinks, and what it is in relation to other practices, and what will be the effect of each thing on each individual.'' \citet{dodds1959pg} and \citet{irwin1979pg} think the use of `προσφέρειν' at 465a suggests Plato had medicine in mind there too, which is perhaps borne out by 501a.}


%``which I take to mean that he does \e{each} of the things he does with attention to any relevant peculiarities of given circumstances. '' [THIS IS SUPPORTED ELSEWHERE.]

%[ALSO NOTE CONTRAST BETWEEN CUTOMARY AND HAVING AN ACCOUNT OF EACH THING IT DOES (COMPARE VM)] [SAY THAT ALSO IN PHILEBUS]

The second objection is that even insofar as the merely experienced practitioner may be reliable, his efforts will be misdirected. The contrast between \techne\ and \empeiria\ is treated in this passage as a contrast not only between understanding and guess-work, but as a contrast between aiming at the good and aiming at pleasure. The pastry chef is a doctor without real understanding: like the doctor, he is concerned with the body, but his guide is not what is good for the body but only what feels good for it. That is all he can aim at, because he doesn't really know what health is; medicine, by contrast, ``has investigated both the nature of the object it serves and the cause of the things it does.'' For this point it is not necessary to suppose that the pastry chef is no good at what he does---indeed we are well-advised to be wary of just how convincing he can be. Earlier on, Socrates had acknowledged that the orator may be more persuasive concerning medicine even than a doctor, at least among the ignorant, though of course he ``isn't knowledgeable in the thing in which a doctor is knowledgeable'' (459b). That the orator may achieve that kind of success does not count in favour of his claim to possession of a \techne: Socrates asks whether\begin{squote}the orator is in the same position with respect to what's just and unjust, what's shameful and admirable, what's good and bad, as he is about what's healthy and about the subjects of the other crafts? (459d)\end{squote} The doctor, by distinction, \e{does} know know health and whatever else his τέχνη concerns, and a rhetorical τέχνη ought to be similar.\footnote{Cf.\ also 459c--e, 464a--465d, and 500a--b.}


%A doctor, for example, would know ``what sight itself is'' (\emph{Laches} 189e--190a). The same expectation is strongly intimated at \emph{Gorgias} 459c--e. It has been observed that the orator


%[AGAIN TECHNE VS PLEASURE IN PHILEBUS; THE ESSAY ON THIS]

The third objection we can distinguish here is that a practice founded only on experience is undeserving of the title of `\techne.' We can see that this would be an objection even granting that the object of the practice were legitimate and that it reliably achieved success from the fact that pastry baking ``proceeds toward its object in a quite uncraftlike way, without having at all considered either the nature of pleasure or its cause''---in other words, pastry baking doesn't investigate even its own already degenerate object in the manner proper to \techne.% [ESP OBVIOUS IN THE ION]

To sum things up, I take Socrates' point to be that if you don't understand the nature of your aims and of your subject matter, then you're in no position to know what would be the ``good condition'' of the things you treat.\footnote{464a, cf.\ \citet{irwin1979pg} 500e--501a \emph{ad loc}, and \emph{Laches} 189e--190a.} Thus you cannot aim at that good condition but only, and only by guesswork, at what will please some person at some time. We might contrast an idealized cobbler, who knows the support a foot needs and how to provide it, with a shoe designer, who is concerned with aesthetic pleasure rather than comfort or health. The cobbler's principles are always the same; the trends are changing and unpredictable.\footnote{Cf. \citet{moss2006pai} on the relationship between pleasure and illusion in Plato.}

It is doubtful whether we ought to attribute a peculiar and disputable view of \techne\ to the \e{Gorgias}' Polus;\footnote{As \citet[pp.\ 346--8]{schiefsky2005} points out.} Aristotle mentions his statement as if it were a truism.\footnote{\emph{Metaphysics} 981a.} But something like the disagreement between Polus and Socrates created, in later centuries, a schism between medical empiricists and medical rationalists; Galen says that ``medicine is the science (\episteme) of what is healthy and what is unhealthy,'' but that\begin{squote}whence one may come by the knowledge of these is no longer universally agreed upon. Some say that experience (\empeiria) alone suffice for the art ([ἡ ἰατρικὴ] \techne), whereas others think that reason (\logos), too, has an important contribution to make.\footnote{\e{De Sectis} 1, trans.\ \citetalias{frede1985galen}.}\end{squote}For our purposes we may forgo the details of this disagreement, noting only that the empiricists claimed that the kind of theoretical understanding insisted upon by the rationalists was, if not impossible, then useless, or at any rate superfluous to the practice of medicine; the possibility or possession of such understanding was irrelevant to the possibility of successful medical practice.\footnote{\e{De Sectis} 5.} And the rationalists for their part developed Socratic objections: \begin{squote}Some have said that this kind of experience is unrealizable (ἀσύστατος), and others, that it is incomplete (ἀτελής), while a third group has claimed that it is not technical (ἄτεχνος).\footnote{\e{De Sectis} 5.}\end{squote} 


A related Socratic point that emerges in several dialogues is the idea that  that every genuine τέχνη must have a domain or subject matter (or \pragma) with a certain natural integrity.\footnote{On `systematicity' see also \citet[71--2]{woodruff1990pse} and \citet[135--6]{asmith1998}.} In the \emph{Laches} the proposal that courage might be knowledge of future goods and evils is rejected on the grounds that there is no such thing as knowledge concerned with farming \emph{in the future} in particular, or with health or strategy \emph{in the future.} The topic will just be health \emph{period}, or strategy, or farming, or goods and evils generally (198d--199d). Similarly Ion is taken to task for the suggestion that he knows and can interpret only the works of Homer, but that as far as other poets go he ``has no power to contribute anything worthwhile'' (\emph{Ion} 532b--c). One learns, or should learn, ``a subject as a whole'' (532e); i.e. poetry, or sculpture, not specific poets or sculptors. Moreover, knowing a subject entails being able to explain the principles of the τέχνη, to teach the τέχνη, and to act as an arbiter on questions about the domain governed by the τέχνη (eg.\ \emph{Gorgias} 450). And these capacities come as a package, it seems: even to claim be able to ``speak well'' on a given topic is \emph{thereby,} in Socrates' view, to claim craft knowledge in that domain. Thus Ion fails not only on generality but also, like Homer, gets his fingers into too many pies. He claims to speak authoritatively on sailing and racing and military affairs and so on (all the subjects Homer addresses), but of course nobody could really be expert in all of these things at once. So here we find that Socrates has a rather more demanding understanding of \techne\ than Ion would have (not that Ion will have considered the matter carefully before), and hence disagrees about whether certain facilities ought to be considered τέχναι. Instead, Socrates says that Ion is possessed by a sort of divine madness.





%[GOLDMAN ETC?]


%Alvin Goldman writes that\begin{squote}The difference between\ldots expert and novice bird-watchers evidently resides in the differences between the cognitive processes they respectively use in arriving at their bird identification beliefs. The expert presumably connects selected features of his current visual experience to things stored in memory about pink-spotted flycatchers, securing an appropriate ``match'' between features in the experience and features in the memory store. The novice does no such thing; he just guesses. Thus, the expert's method of identification is reliable, the novice's is unreliable.\footnote{\citet{sep-reliabilism}.}\end{squote}Perhaps Polus assumes something of this sort about \techne; certainly one could object to Goldman's understanding of expertise for the same reasons that Socrates objects to Polus's understanding of \techne. 

%[[first whether possible---then completeness---then understanding]]

%[second], one might suppose that the expert ought not merely to be reliable, , but also on account of some understanding or intellectual mastery of his domain. That is a plausible thought even if we suppose no loss of reliability. Consider again Ramanujan, who, as we saw, may have had more facility with infinite series than anyone else ever. Suppose, though, that his conception of his topic was as described here:\begin{quote}One idea Ramanujan bruited about dealt with the quantity 2\e{n}-1. That, a friend remembered him explaining, stood for ``the primordial God and several divinities. When \e{n} is zero the expression denotes zero, there is nothing; when \e{n} is 1 the expression denotes unity, the Infinite God. When \e{n} is 2, the expression denotes Trinity; when \e{n} is 3, the expression denotes 7, the Saptha Rishis [the seven Vedic sages], and so on.''\footnote{\citet[p.\ 66]{kanigel1991ramanujan}.}\end{quote} and that his `method' was as described here:\begin{squote}T. K. Rajagopolan, a former accountant general of Madras, would tell of Ramanujan's insistence that after seeing in dreams the drops of blood that, according to tradition, heralded the presence of the god Narasimha, the male consort of the goddess Namagiri, ``scrolls containing the most complicated mathematics used to unfold before his eyes.''\footnote{Ibid.\ p.\ 281.}\end{squote}[COMPLETE THIS; ANALOGY WITH GOLDMAN-BONJOUR DISAGREEMENT]---astrology



%Medicine has, as a \techne\ should, its own μέτρον, but that 



%οἷον πασῶν που τεχνῶν ἄν τις ἀριθμητικὴν χωρίζῃ καὶ μετρητικὴν καὶ στατικήν, ὡς ἔπος εἰπεῖν φαῦλον τὸ καταλειπόμενον ἑκάστης ἂν γίγνοιτο.

%




%[EXAMPLES LIKE MEDICINE ETC; LETTERS ON THE OTHER HAND]










%UNDERSTANDING, AND REAL NOT APPARENT ACCOMPLISHMENTS



%Disagreements about which practices constituted τέχναι were complicated, and engendered, by the fact that different people had different ideas about what a τέχνη should be or do. For example, when you hire a craftsman, should you be able to expect that he deliver what you hire him for? That would be a natural expectation of a carpenter, but what about a doctor? And should a genuine τέχνη be theoretical, and in what way theoretical? Or might a mere body of experience also constitute a τέχνη?\footnote{\citet[63--4]{schiefsky2005} argues for dating \emph{VM} to the late fifth century.} 





% [JUST AS RELIABLE IN PRATICE---WITH ARISTOTLE I GUESS?]



%R. Radhakrishna Iyer, a classmate at Pachaiyappa's College, recalled



%Socrates, too, takes positions on such questions: especially notable is his claim in the \emph{Gorgias} that τέχνη is explanatory.

%, contrasting his view with that of Polus, who had said that τέχνη was a product of experience.


%[MS on pushing out tuche---de arte ect].---COuld say that S is actually more demanding...

%A brief summary may be found in \citet{cuomo2007tac}, pp. 18-22. - [eg Might exclude eg Epidemics, Breaths - Cuomo p15][and also: see \citet{cuomo2007tac}, pp. 16-17 on harmony of nature and techne.]



%Now the argument here is clearly fallacious, and this is a largely comic dialogue. And the argument is actually capped at 280a-b with Socrates' slightly peculiar remark that ``we finally agreed (I don't quite know how) that, in sum, the situation was this: if a man had wisdom, he had no need of good fortune in addition.'' The ``I don't quite know how'' might suggest that Socrates is more invested in the intended conclusion about the sufficiency of virtue for happiness than in the argument from crafts which he has appealed to. 






%On the limits being set by Phusis see also Heinimann 122





%Here we would distinguish between competent application of the methods associated with a τέχνη and actual successful outcome, or between the exercise of a τέχνη and the aim of that τέχνη . This is clearly a more appropriate model for medicine: we would no longer expect the doctor to know in every case whether or not the treatment will succeed; medicine would provide no \emph{absolute} guarantee against luck.


%




%[[FIX: Aristotle had seen the problem with thinking about τέχνη in this way. 

%A competent doctor may not, then, always accomplish his goal, and lack of success may have to be blamed on complications he could not have foreseen.





%In fact Socrates and Plato had themselves recognized a number of difficulties with their use of τέχνη ---only to look for ways around those problems.}]]


\subsubsection*{The distinctiveness of Socratic \techne}

In the course of the forgoing discussion we have seen that Socrates articulates and defends a number of criteria for \techne. [BRIEF RECAP?] As such it is clear that Socrates has a considered (though not systematically articulated) conception of \techne, and one which is, moreover, clearly at odds with various other conceptions of \techne. So, for example, Socrates himself expressly contrasts his own conception of \techne\ with Polus' supposedly deficient conception; likewise he would have had no truck with medical empiricism. Again, Socrates' conception of \techne\ leads him to various specific judgements about which practices are and are not \techne---he views medicine as a genuine τέχνη, but raises doubts about the status of poetry, `cookery,' and rhetoric---sometimes putting him at odds with his interlocutors. 

I will conclude this section by demonstrating briefly that, in particular, Socrates' conception of \techne\ differs Aristotle's and from what we find in Plato's other, later dialogues.



%; Plato, for his part agrees with Aristotle that medicine and navigation [?] don't satisfy such standards, but on that account gives them a lower status [LOOKS FORWRD TOO MUCH?].


In their different expectations for the precision and reliability of \techne, we have a clear contrast between Socrates and Aristotle. Aristotle's insistence that a high degree of precision and hence of reliability could not be demanded of every legitimate \techne\  was (more or less) original to him. Where Aristotle is at pains to argue that moral inquiry is legitimate although imprecise, Socrates is happy to invoke the marks of mathematical precision in his investigation of the virtues, for example at \e{Protagoras} 356d--e, where he invokes a hypothetical ``craft of measurement'' as our ``salvation'' (see further \S\ref{propositional} below). Socrates' expectations for the reliability of \techne\ are comparable to those in \e{De Arte}, or even to the view in \e{On Regimen}.\footnote{This conclusion accords with recent assessments of \citet[p.\ 356]{schiefsky2005} and \citet{allen1994}.} Socrates allows, of course, that any given person might fail in the employment of their \techne---for example at \emph{Euthydemus} 279e--280a, where he contrasts wise and foolish doctors or generals. It is not clear that Socrates regognizes any further limitations on \techne. In \emph{Republic} I, Thrasymachus says that experts, insofar as they are experts, never make mistakes (340e), and Socrates accepts this assumption (342b). The context is the question whether a ruler will ever mistakenly issue laws which in fact turn out to be detrimental to himself, so the point in saying that experts never make mistakes seems to be that, if that the expert properly employs his \techne, he will never fail to attain his intended goal (within the domain of that \techne; 340e). Here, as in \e{De Arte} and as with Glaucon's view in \e{Republic} II, the possibility of sheer misfortune is rejected.\footnote{cf. Allen pp.\ 85--6.} In the \emph{Euthydemus}, Socrates, adducing a variety of τέχναι, deploys the notion that τέχνη is a guard against bad τύχη (279d ff.), finally going so far as to say that \sophia\ (here synonymous with τέχνη) \emph{is} (good) τύχη (279d), or at any rate provides it (282a). Here, then, Socrates himself deliberately pushes luck out of the picture.\footnote{At \emph{Protagoras} 344c--d, Socrates says misfortune (\sumphora) might incapacitate a craftsman but not an ordinary person (who has no τέχνη to start with). Here misfortune interferes with τέχνη, but by destroying it, not by preventing it from attaining its goals.} At \emph{Euthydemus} 280a, `being lucky' is glossed as `never making mistakes,'\footnote{``Wisdom makes men fortunate in every case [of τέχνη], since I don't suppose that she would ever make any sort of mistake but must necessarily do right and be lucky---otherwise she would no longer be wisdom.'' (trans.\ \citetalias{sprague1993euthydemus}.)} and this latter characterization of τέχνη is also found at \emph{Protagoras} 357d--e, and the notion that τέχνη is infallible at least within the realm of τύχη certainly accords with the salvific conception of a τέχνη for living found in that passage (356d--e).\footnote{Cf.\ \e{DA} XIV: ``that medicine has plentiful reasoning in itself to justify its treatment, and that it would rightly refuse to undertake obstinate cases, or under-taking them would do so without making a mistake [i.e.\ it succeeds in what it undertakes], is shown both by the present essay and by the expositions of those versed in the art.''}


%SCHIEFSKY 365: ``In general, then, Plato conceives of doctors as able to achieve complete \akribeia in their practice, at least in principle.''








% if a τέχνη is being practiced properly \emph{then the expert should always be successful} (within the realm of possibility, presumably). 
 
 
 


%In \e{Republic} II, as we saw, Glaucon had allowed that [], [actually thrasymachus could perhaps be understood this way....]

%Where in this is Socrates?


%Socrates' understanding of the reliability of τέχνη seems to accord with that found in \emph{De Arte}. On the one hand, Socrates treats medicine as a paradigmatic τέχνη, and there's no reason to suppose that he or anyone else would have thought that a doctor could heal anyone no matter what their injuries or age or health.





%So Socrates could hardly have a \emph{more} stringent view than the one found in \emph{De Arte}. 


%On the other hand, his view seems to be no \emph{less} stringent either.

%Thrasymachus assumes the same view of τέχναι in general 




%Compare DA IV: ``Now I, too, do not rob luck of any of its prerogatives, but I am nevertheless of opinion that when diseases are badly treated ill-luck generally follows, and good luck when they are treated well.''


%, expositions set forth in acts, not by attention to words, under the conviction that the multitude find it more natural to believe what they have seen than what they have heard.




% This view, and the thesis that virtue suffices for happiness which it is meant to support, is open to the sorts of objections raised later against the Stoics by Alexander (\emph{Supp.\ De Anima} 160.1 ff.). The argument in the \emph{Euthydemus,} is, moreover, clearly fallacious. But the conclusion that virtue is sufficient for happiness is after all Socratic, and the case of \emph{De Arte} shows that this position was a contemporary possibility.





%\footnote{Though the expectations are not so uncommon. Albert Schweitzer reported that ``in the opinion of the Black Africans the efficacy of medicine is demonstrated primarily in that the physician knows whether the patient will die or not, and in that the physician does not apply his art to someone who is really already dead.'' (\emph{Briefe aus Lambarene 1924--1927}, cited by \citet[p.~181]{horstmanshoff1990ancient}.) Many of our own contemporaries seek `alternative' medical help out of frustrations based on very similar expectations; the sitting President tends to take the blame for whatever happens in the economy.}

%I said that much of the scholarly interest in τέχνη has circled around







% Plato also puts more emphasis on specifically mathematical precision. Angela Smith has suggested that for Plato ``mathematical knowledge---geometry, in particular---becomes the paradigm of expert knowledge,'' probably (following Vlastos) because ``Plato himself has become absorbed by mathematical studies by the time he writes the \e{Meno}.''\footnote{\citet[p.\ 157]{asmith1998}; \citet[pp.\ 117--24]{vlastos1991socrates}.}\begin{squote}When this occurs, the prominent features of mathematical reasoning---deduction, proof, and infallibility---come to be seen as definitive of expertise, and the relatively unstructured model of expert knowledge displayed in the early Platonic dialogues begins to evolve into the highly systematic model of deductive knowledge displayed in the philosophy of Aristotle. If this is correct, then the model of deductive knowledge does not replace the model of expertise, but becomes wedded to it; we should expect, therefore, to see features of both models in the developing epistemology. And this, I submit, is exactly what we do see.\footnote{Smith Ibid.}\end{squote}That mathematical \technai\ become paradigms of \techne\ is clearly right; this is indeed explicit in passage such as the one in the \e{Philebus}. And whereas medicine was a for Socrates a paradigmatic \techne, in the \e{Philebus} it is relegated to a second-class status.

%The image of two models for knowledge being fused is, however, backwards: in fact \techne\ and \episteme\ become increasingly distinct over the course of the dialouges. In the early dialogues there is only \techne-\episteme. [ALSO EXPLICIT]. To see this, we can begin with Smith's brief elaboration of her suggestion as regards Plato:\begin{squote}Plato is impressed with mathematics as a τέχνη (because of its elegant method, its consistent and reliable results, and its successful application to other domains such as harmonics and astronomy), and it is his appreciation of mathematics \e{as a model of expertise} that leads him to emphasize mathematical training in his later philosophical work.\footnote{Ibid.\ p.\ 158 (her emphasis).}\end{squote} There are two ways of reading this. On the first reading---a reading which fits naturally with the notion of an evolving fusion of two models of knowledge---Plato would think of mathematics as a \techne, but mathematics would also bring with it a new conception of knowledge (\ie the `deductive conception'); treating mathematics as a model \techne\ would then fuse the two models. On the second reading, Plato is impressed by mathematics as a \techne\ on terms already set by Socrates---in that case mathematics would become a paradigm of \techne\ not by introducing a new conception of knowledge but by satisfying existing criteria for \technai\ in an exceptional fashion.

%That mathematics influences Plato's conception of knowledge in some fashion is unquestionable, but this second way of describing the effect is the superior one. The line of thought is in fact explicit in the \e{Philebus} passage. There we saw that carpentry was said to owe ``its superior level of craftsmanship over other disciplines to its frequent use of measures and instruments, which give it high accuracy.'' In particular, ``it employs straightedge and compass, as well as a mason's rule, a line, and an ingenious gadget called a carpenter's square.''\footnote{56b--c.} The use of such measures give carpentry its ``high accuracy.'' But this observation lends a two-fold support to the view that mathematical \technai\ in particular are paradigms of \techne: firstly because, considered as \technai\ in their own right, they are the \e{most} accurate of all \technai\ (especially when pure rather than applied)\footnote{56c--e.}; secondly because not only the carpenter but all experts depend on these subjects for their measures and for the possibility of their accuracy:\begin{squote}If someone were to take away all counting, measuring, and weighing from the arts and crafts, the rest might be said to be worthless\sdots All we would have left would be conjecture and the training of our senses through experience and routine. We would have to rely on our ability to make the lucky guesses that many people call art, once it has acquired some proficiency through practice and hard work.\footnote{55d--56a.}\end{squote}  As noted earlier, Plato's reference to ``counting, measuring, and weighing'' is an invocation of standard precision- and \techne-indicating triad:\begin{squote}In griechischer Dichtung und Philosophie begegnet seit dem 5.\ Jahrhundert in scheinbar recht verschiedenen Zusammenhängen die Gruppe der drei Begriffe Mass, Gewicht und Zahl.\footnote{\citet[p.\ 183]{heinimann1975mass}; cf.\ \citet[p.\ 194]{heinimann1975mass}.}\end{squote} And such thoughts are by no means foreign to the early dialogues.\footnote{Cf.\ \citet[p.\ 193]{heinimann1975mass}.} 

%Thus Socrates invokes the standard number-measure-weight triad in the \e{Euthyphro}:\begin{squote}\textsc{Socrates}: What are the subjects of difference that cause hatred and anger? Let us look at it this way. If you and I were to differ about numbers as to which is the greater, would this difference make us enemies and angry with each other, or would we proceed to count and soon resolve our difference about this?

%\noindent\textsc{Euthyphro}: We would certainly do so.

%\noindent\textsc{Socrates}: Again, if we differed about the larger and the smaller, we would turn to measurement and soon cease to differ.

%\noindent\textsc{Euthyphro}: That is so.

%\noindent\textsc{Socrates}: And about the heavier and the lighter, we would resort to weighing and be reconciled.

%\noindent\textsc{Euthyphro}: Of course.\fotnote{7b--c. Cf.\ also \e{Protagoras} 356--357b on `ἡ \metrike \techne.'}\end{squote}

%It is remarkable, then, that in the \Philebus} medicine should be placed in the category of merely the merely so-called \technai\ characterized by much imprecision and little reliability, whereas in the \e{Gorgias} Socrates had presented it as a model of a true \techne.  But the difference is [But it is in the verdicts about given \technai\ rather than in the basic contrasts that the position expressed here differs from that found in the early dialogues.] [SO NOT A FUSION---MOTIVE IS THE ONE ALREADY PRESENT NOT ONLY IS S BUT GENERALLY---RATHER PRECISION ETC---PLATO HAS FOUND SOMETHING BETTER THAN SOCRATES' DOCTOR][So mathematics doesn't bring with it a new conception of knowledge; instead it is an essential aspect of \techne\ in the first place.]


%[SCHIEFSKY'S POINT, p 15---has various other cites around here]
 
%Thus \citet[p.\ 194]{heinimann1975mass}: ``Platon hat die Scheidung zwischen den unexakten und den mit Zahl und Mass arbeitenden K\"{u}nsten offenbar \"{u}bernommen, er differenziert aber weiter und stuft die Künste nach ihrem zunehmenden Gehalt an Mathematik und der dementsprechend höheren Eignung zur Seinserkenntnis.''

%Plutarch quote in Schiefsky is good.

%[FURTHER EXAMPLES---PROTAGORAS AND ELSEWHERE WITH MATHEMATICAL STUFF AS AN ELEMENT OF OTHER TECHNAI]


%Socrates: Bear in mind then that I did not bid you tell me one or two of the many pious actions but that form itself that makes all pious actions pious, for you agreed that all impious actions are impious and all pious actions pious through one form, or don't you remember?

%\begin{squote}Tell me then what this form (ἰδέα) itself is, so that I may look upon it, and using it as a model (\paradeigma), say that any action of yours or another's that is of that kind is pious, and if it is not that it is not.\end{squote} Euthyphro's answer is that ``what is dear to the gods is pious, what is not is impious.'' Socrates says they will have to investigate this answer, and sets about it thus:\begin{squote}If you and I were to differ about numbers (ἀριθμός) as to which is the greater (πλείων), would this difference make us enemies and angry with each other, or would we proceed to count (ἐπὶ λογισμὸν ἐλθεῖν) and soon resolve our difference about this? \ldots Again, if we differed about the larger and the smaller (μείζων καὶ ἐλάττων), we would turn to measurement (ἐπὶ τὸ μετρεῖν ἐλθεῖν) and soon cease to differ.\ldots And about the heavier and the lighter (βαρύτερος καὶ κουφότερος), we would resort to weighing (ἐπὶ τὸ ἱστάναι ἐλθεῖν) and be reconciled.\end{squote}The point turns out to be that since different things are dear to different gods (as also to different people), looking to see what is dear to the gods is not a reliable means of settling questions about what kind of behaviour is pious. That was what Socrates wanted: some \paradeigma\ by  which Socrates could whether any given action is pious. Here, then, Socrates appeals to the mathematical arts as providing an ideal of the accurate and and hence reliable knowledge that we may aspire to even in moral life.

% The suggestion that mathematical \technai\ are in some way components of, or relied upon by, other \technai---\ie that \technai\ more generally might rely on mathemetical tools---can also be found in the earlier dialogues. At \prot 357a--c, Socrates concludes his account of the strength of knowledge thus:\begin{squote}εἶεν, ὦ ἄνθρωποι: ἐπεὶ δὲ δὴ ἡδονῆς τε καὶ λύπης ἐν ὀρθῇ τῇ αἱρέσει ἐφάνη ἡμῖν ἡ σωτηρία τοῦ βίου οὖσα, τοῦ τε πλέονος καὶ ἐλάττονος καὶ μείζονος καὶ σμικροτέρου καὶ πορρωτέρω καὶ ἐγγυτέρω, ἆρα πρῶτον μὲν οὐ μετρητικὴ φαίνεται, ὑπερβολῆς τε καὶ ἐνδείας οὖσα καὶ ἰσότητος πρὸς ἀλλήλας σκέψις;

%ἀλλ᾽ ἀνάγκη.

%ἐπεὶ δὲ μετρητική, ἀνάγκῃ δήπου τέχνη καὶ ἐπιστήμη.

%ἥτις μὲν τοίνυν τέχνη καὶ ἐπιστήμη ἐστὶν αὕτη, εἰς αὖθις σκεψόμεθα: ὅτι δὲ ἐπιστήμη ἐστίν, τοσοῦτον ἐξαρκεῖ πρὸς τὴν ἀπόδειξιν ἣν ἐμὲ δεῖ καὶ Πρωταγόραν ἀποδεῖξαι περὶ ὧν ἤρεσθ᾽ ἡμᾶς.

%Very well, gentlemen: since then the salvation of life has appeared to us to consist in the right choice of pleasure and pain---more and fewer, greater and less, nearer and farther-doesn't it in the first place appear to be an art of measurement, an inquiry into excess and deficiency and equality relative one to another?"

%Why, necessarily.

%"But since of measurement, by necessity it is surely art and knowledge."

%They'll agree.

%"Now, just what art and knowledge this is we'll consider later; but that it's knowledge by so much suffices for the proof Protagoras and I must provide about what you asked us.\footnote{CITE; TRANS ALLEN.}\end{squote} It is tempting to translate `\metrike' here by `an art [\ie a \techne] of measurement, as Allen does here, since we have, after all, just been told (at 356d--e) that our salvation will indeed be ``ἡ μετρητικὴ τέχνη.'' But here Socrates is recapitulating the argument: that our salvation is τέχνη καὶ ἐπιστήμη is obsevered to \e{follow} from the fact that our salvation is \e{something} concerned with measurement. [Donald?] Zeyl properly takes `\metrike' substantively: ``doesn't our salvation seem, first of all, to be measurement\sdots And since it is measurement, it must definitely be an art, and knowledge."\footnote{[CITE ZEYL]. This may mislead if we suppose that Socrates means to assert the identity of his salvific craft and measurement, or of this this salvific craft with \e{the} \techne\ of measurement (if there is such a thing); that we are not meant to suppose this can be seen from the last line, in which it is indicated that we have not yet identified the \techne\ in question. (We are never told what it is because Socrates doesn't know what it is, that being the premise of the Socratic dialogues).}

%The point of the passage, then, is this. The operative assumption has been that deliberation concerns pleasure and pain. Since pleasures and pains are, as we would say, quantifiable. As such, any hope of reliable success in our deliberation will require possession of some `metrical' or `mensural' facility which allows us to choose well---this mensural facility will thus be our salvation. Next, since this salvific facility is mensural, it must be some τέχνη καὶ ἐπιστήμη. This inference, notice, is intended to be trivial: in saying that deliberation concerns the choice ``τοῦ τε πλέονος καὶ ἐλάττονος καὶ μείζονος καὶ σμικροτέρου καὶ πορρωτέρω καὶ ἐγγυτέρω,'' Socrates has invoked the same standard triad he invokes in the \e{Euthyphro}: [VERIFY DETAILS]. 


%What is quantifiable in some such manner is the very model of 

%, since it will admits of precise and reliable handling. Correspondingly, \episteme\ and \techne\ paradgmatically have [SUCH A AN OBJECT]


%This helps to account for the fact that Socrates does not suppose himself to have actually identified the \techne\ in question. The point in saying that the salvific knowledge in question is `ἡ μετρητικὴ τέχνη' is not that it is \e{the} \techne of measurement, or simply measurement, but that because it involves measurement it must unquestionably be, in fact, a \techne. But there will be many μετρητικαὶ \technai---geometry, carpentry, etc.---because that is, paradigmatically, what \technai\ are. 


%[SECOND: also that it incorporates measurement]

%[BIT OF A QUESTION WHY NOT JUST THE TECHNE OF PLEASURE---BUT OBVIOUS REASONS HERE---S OF COURSE DOENS'T KNOW OR THE QUEST WOULD BE OVER---AGAIN CLEAR ELSEWHERE THAT JUST AS MEASURENT IS A TROPE FOR TECHNE, PLEASURE AS AN OBJECT IS A TROPE FOR NOT HAVING A TECHNE AT ALL]

%one, and very from the other.

%\begin{squote}Visitor: Yes, Socrates; and what many sophisticated people sometimes say, supposing themselves to be expressing something clever, to the effect that there is in fact an art of measurement relating to everything that comes into being (μετρητικὴ περὶ πάντ᾽ ἐστὶ τὰ γιγνόμενα)---that's actually the very thing we have just said. For it is indeed the case, in a certain way, that all the products of the various sorts of expertise share in measurement (μετρήσεως μὲν γὰρ δή τινα τρόπον πάνθ᾽ ὁπόσα ἔντεχνα μετείληφεν).\footnote{284e--85a, trans.\ \citetalias{rowe1999statesman}.}\end{squote}

%The Visitor goes out of his way here to point out that every \techne\ is in sense a \techne\ of measurement, that the point is far from being an original one, and that to say that something is a \techne\ of measurement isn't yet to say much.


%Lafrance, Y., 1995, “Métrétique, mathématiques et dialectique en Politique 283c–285c”, in Reading the Statesman. Proceedings of the Third Symposium Platonicum, C. J. Rowe (ed.), Sankt Augustin: Academia Verlag. 89–101

%[GOES OUT OF HIS WAY TO POINT OUT THAT THAT IT'S NOT ORIGINAL]


%ὃ γὰρ ἐνίοτε, ὦ Σώκρατες, οἰόμενοι δή τι σοφὸν φράζειν πολλοὶ τῶν κομψῶν λέγουσιν, ὡς ἄρα μετρητικὴ περὶ πάντ᾽ ἐστὶ τὰ γιγνόμενα, τοῦτ᾽ αὐτὸ τὸ νῦν λεχθὲν ὂν τυγχάνει. μετρήσεως μὲν γὰρ δή τινα τρόπον πάνθ᾽ ὁπόσα ἔντεχνα μετείληφεν



%; practical deliberation, being about pleasure, is thus \ips also deliberation about things which are, in Socrates language, 



%text Alc. 1, section 126d: ... περὶ σπιθαμῆς καὶ πήχεος ὁπότερον μεῖζον; οὐ διὰ τὴν μετρητικήν; Ἀλκιβιάδης τί μήν; Σωκράτης


%\begin{squote}``Very well, gentlemen: since then the salvation of life has appeared to us to consist in the right choice of pleasure and pain---more and fewer, greater and less, nearer and farther---doesn't it in the first place appear to be [concerned with measurement (μετρητικός)], an inquiry into excess and deficiency and equality relative one to another?''

%Why, necessarily.

%``But since of measurement, by necessity it is surely art and knowledge.''

%They'll agree.

%``Now, just what art and knowledge this is we'll consider later; but that it's knowledge by so much suffices for the proof Protagoras and I must provide about what you asked us. You asked it if you recall, when we agreed with each other that nothing is stronger than knowledge and that it always governs pleasure and all the rest wherever it is present.''\end{squote}


%NEED TO READ GISELA ON THIS AGAIN




%I have argued that, in the early dialogues, knowledge just is \emph{techne}. Among other things, this means that there is no significant difference between \emph{techne} and \emph{episteme}---the medical \emph{techne}, for instance, is simply the \emph{episteme} of health.




There is little or no space between Socrates and Plato on the question of precision and reliability, except that Plato at least more expressly acknowledges the kind of principled limits on \techne-success identified by the author of \e{De Arte}. There are other significant differences, though. Chief among these is the fact that Plato has a more developed view of the structure of \techne-possesion. We can see this especially in the evolving relationship between `\techne' and `\episteme.'

In later dialogues, as in the earlier, `\episteme' and `\techne' are often interchangeable. So, for example, early in the \e{Theaetetus}, Socrates says that the \techne\ of cobbling just is the \episteme\ of making shoes (146c--d). In other cases, though, \episteme\ begins to take on a life of its own---not necessarily as something altogether independent of \techne, but as something with a distinct role. In the \emph{Phaedrus}, for example, `\episteme' occurs nine times but is never equivalent to `\techne,' though the two are related. When Phaedrus asks Socrates how someone might acquire the \techne\ of the true and persuasive rhetorician, Socrates answers thus:\begin{quote}Well, Phaedrus, becoming good enough to be an accomplished competitor is probably---perhaps necessarily---like everything else. If you have a natural ability for rhetoric (\phusis\ ῥητορική), you will become a famous rhetorician, provided you supplement your ability with knowledge (\episteme) and practice (μελέτη). To the extent that you lack any one of them, to that extent you will be less than perfect.\footnote{269d, trans.\ \citetalias{nehamas1995plato}.}\end{quote}The following pages make it clear that the \episteme\ in question is an understanding of the soul, its varieties, and the effects different sorts of speeches have on different souls (which the orator will need to augment with experience so as to discern reliably how to apply this knowledge; 271--272). Here, then, \episteme\ is an aspect of \techne, or the intellectual aspect of \techne-possession. 

In the \emph{Republic}, too, `\episteme\ and \techne\ are sometimes equivalent\footnote{As is the case throughout book I, but also elsewhere: for example at 428b--29a and 438c--e.}, but, as in the \e{Phaedrus}, `\episteme' may also have the more restricted sense of the knowledge which is one component of the complete or ideal possession of a craft. So at 374d Socrates observes that ``no \ldots\ tool makes anyone who picks it up a craftsman (δημιουργός) or champion unless he had acquired the requisite knowledge (\episteme) and has had sufficient experience (μελέτη).'' Indeed this passage (374b--e) draws attention to the three aspects of expertise identified in the \emph{Phaedrus}---a suitable nature (\phusis), knowledge (\episteme), and practice (μελέτη)---though not so sharply or emphatically as in the other work.

Elsewhere in the \emph{Republic}, \episteme\ also seems to have its own independent existence. Thus `\episteme' is not always used just as a word for an aspect of \techne\ or as offering an alternative way of characterizing a \techne\ (i.e.\ as the ``\episteme of \ldots''), but as a name for more abstract forms of knowledge or expertise. Thus `\episteme' is apt particularly for arithmetical, geometrical, and astronomical knowledge, the last being understood not primarily as dealing with the actual movements of heavenly bodies but with motion proper (529d--e). These are forms of knowledge which deal with what really is, and which lead the mind away from what is visible and tangible (524e ff.). And [dialectic], or the knowledge it provides, is even more properly called `\episteme\' (533d). We might then suppose that in one sense of `\episteme,' `\episteme' is a name for a type of body of knowledge distinct from the type of body of knowledge constituted by \technai. On the other hand, if we can here draw a distinction between \techne and \episteme, we could equally well call it a distinction between different sorts of \technai: on the one hand more directly practical \technai, and on the other hand more theoretical \technai. For although arithmetic, geometry, astronomy, and dialectic are largely treated under the label of `\episteme,' they can equally well be called \technai\ as well (532b, 533d). And it's possible as well to see a connection between the role of \episteme as an aspect of \techne and the fact that more cognitive \technai\ most properly receive the title of `\episteme.' \technai\ When Socrates is introducing ``number and calculation'' (ἀριθμός καὶ λογισμός; 522c) as subjects which will lead the soul ``from the realm of becoming to the realm of what is'' (521d), he introduces them also as subjects that are common to ``every craft (\techne), every type of thought (διάνοια), and every science (\episteme)'' (522b--c). The general, for instance, needs to be able to count his troops and ships and to arrange them (522c--e). But of course arithmetic is also a type of study in its own right. Arithmetic could therefore be said to be \episteme\ both in the sense that it forms (a part of) the theoretical aspect of various other \technai, and also in the sense that it is itself \episteme, which is to say a relatively abstract \techne. Geometry, astronomy, and dialectic, introduced in the following pages, are perhaps increasingly less implicated in other crafts, and increasingly constitute strictly abstract studies in their own right. At least in the case of arithmetic, it's hardly necessary to distinguish the two respects in which it is \episteme. But in the other cases as well, it seems natural enough that where \episteme\ has come to stand for the more cognitive aspects of \techne, more cognitive \technai\ should also come to be called \epistemai.

In the earlier dialogues, Socrates occasionally acknowledged, in casual remarks, the importance of aptitude and experience, but there was little or no reflective attention to these aspects of \techne. Plato winds up paying much more attention to them, and his views about \techne\ and knowledge evolve accordingly; eventually, even if Plato does not draw quite the sort of distinctions between form of knowledge that Aristotle does, he certainly winds up with some fairly sharp gradations---more theoretical varieties of knowledge, now especially called `\epistemai,' are up top, and more practical forms, still well described as `\technai.'




%But in other cases \techne\ and \episteme\ are distinct, though not necessarily independent.



% It is hard to see, for example, how the Protagorean proposal in the \emph{Theaetetus} to the effect that \emph{episteme} is \emph{aisthesis} could be taken seriously if `\emph{episteme}' were simply understood as synonymous with `\emph{techne}.' [NOT SURE ABOUT THAT.] 




%\footnote{Unless the \episteme\ that is ``close to change'' and ``becomes different'' (247d) includes \technai.} In most cases it has to do with acquaintance with what really is, the ``place beyond heaven,'' the forms (247c-e). \emph{Episteme} in the \emph{Phaedrus} is, in that respect, like \emph{episteme} in the \emph{Phaedo}, where \emph{episteme} is again mostly not equivalent to \emph{techne}. [MAYBE NEVER?] 

%That \emph{episteme} in these cases cannot simply be equivalent to \emph{techne} does not necessarily mean that the two do not have an intimate connection. In the \emph{Theaetetus}, the principle objection---the objection that comes both first and last---to the Protagorean identification of knowledge and perception is that it makes nonsense of expertise. \emph{Techne} is still in the \emph{Theaetetus} a paradigmatic variety of, or locus of, \emph{episteme}. 



%near the start of the \emph{Theaetetus}, when Socrates first asks what \emph{episteme} is, Theaetetus says that cobbling, as any other \emph{techne}, is \emph{episteme}. This is the wrong \emph{kind} of answer, since it offers an example of an \emph{episteme} rather than an account of \emph{episteme}, but there is certainly nothing wrong with cobbling as an illustration of \emph{episteme}---


% Similar examples crop up in other dialogues as well, including the \emph{Republic}.

%\footnote{eg. [Sophist (258-9?), REPUBLIC 428? - might leave this out since I'm comingback to it...]}


%484e is actuall arguably a statement of the tripartite thing as well.... if virtue means basically the suitable character/nature....





%[GOES WHERE?] In some cases, then, the independent role of \emph{episteme} reflects in part a broadening of Plato's conception of \emph{techne}, or a greater emphasis on the different aspects thereof. 




%[summation]






%BUT AVOID CONTROVERSY, SO APPEAL TO ALEXANDER




%which rely on guesswork (\stochasmos) rooted in experience, 

%\

%IMPORTANCE RE PLATO


%MY DRAFTS FROM LATER CHAPTERS WILL BE USEFUL HERE

%\begin{squote}The view I have been espousing here---that in the early Platonic dialogues ``expertise'' constitutes a model of knowledge altogether distinct from the model of axiomatic deduction---has one remaining difficulty to confront: If Socrates was not working with a deductive model of knowledge, how can we account for the evident interest in deductive knowledge displayed in later Platonic dialogues and in the works of Aristotle?\end{squote}

%Smith: there is an issues about
%``the evident interest in deductive knowledge displayed in later Platonic dialogues and in the works of Aristotle.''

%Angela Smith has also insisted on the continuing significance of \techne\ as a model for knowledge well after the \e{Meno}. She notes for example the case of the \e{Theaetetus}:\begin{squote}When Socrates asks Theaetetus for a definition of knowledge, moreover, the first answer he gives is a list of τέχναι, including both theoretical disciplines and productive crafts (146c8--d3). Socrates does not correct Theaetetus by telling him that he actually intends to investigate the nature of \e{factual} knowledge, not of systematic knowledge. Instead, he corrects Theaetetus by telling him that he wants to be told what all of these ``knowledges of'' have in common: ``Now I want you [\ldots] to give a single account of the many branches of knowledge.'' (148d6 f.)\footnote{p.\ 158 (her emphasis).}\end{squote}


%Smith's view, then, is that Socrates ``is still thinking of knowledge in terms of expertise.''\footnote{Ibid.}  Accordingly, she suggests that a deductive model doesn't \e{replace} the \techne-model; rather 

%PAUL WOODRUFF that whereas learning is in the \emph{Meno} said to be a matter of recollection, and thus involves no teaching, teachability was a mark of τέχνη.\footnote{\citet[p.\ 81]{woodruff1990pse}.} He thinks that the τέχνη-model for knowledge has therefore been abandoned by the time of the \emph{Meno}.

%But this cannot be right. Geometry is itself a \techne, as well as a crucial tool for other \technai [EG PHIL 56E AND ELSEWHERE]

%[in the republic there are teachers]

%Now that seems hasty. Is geometry no longer a τέχνη? Does Socrates imagine a future for geometry without teachers?  Or if teachers must now be known as `people who inspire recollection,' have we abandoned the teachability criterion or only come to a new understanding of `teaching'?\footnote{Woodruff himself allows that the \emph{Meno} is not consistent on the teachability question, and says that the suggestion that there might be knowledge without teachers ``is resisted in the balance of the dialogue'' (p.\ 83). The teachability criterion is important as part of the \emph{Meno'}s guiding hypothesis that if virtue is knowledge then it will be teachable.} Or couldn't Plato be toying with new ideas about how τέχναι are acquired?\footnote{In fact a number of familiar features of τέχνη reappear in the \emph{Meno}. The completeness requirement seems to show up in Socrates' insistence that the slave would have to go through many proofs repeatedly before he would have knowledge. The explanatory condition shows up at 98a, as does reliability.}

%[GEO IS OF COURSE A PARADIGM TECHNE IN PHILEBUS AS WE SAW; THIS PERHAPS ALL BELONGS IN DISTINCTIVENESS SECTION]

%However exactly the connections between the various meanings and uses of `\emph{techne}' and `\emph{episteme}' might best be spelled out, the point I want to make is this: in the \emph{Republic}, \emph{episteme} never loses its connection with \emph{techne}. Either `\emph{episteme}' is employed in a casual fashion which seems to have no particular implications for an understanding of Plato's conception of knowledge, or \emph{episteme} is equivalent to \emph{techne}, or \emph{episteme} is something distinct from \emph{techne}, but as a component of \emph{techne}. (In a given case a single one of these options need not unambiguously be the only possibility.) This is true even of \emph{episteme} in the strictest sense: the knowledge of the Good. For this knowledge is precisely the theoretical aspect of the Philosopher's craft.\footnote{Similarly for other deployments epistemological vocabulary: either they have no great significance for an understanding of Plato's epistemology, or else can be connected to the \emph{techne}-conception of knowledge. I assume that some varieties of \emph{episteme} constitute the difficult case, so that having argued that \emph{episteme} should still be understood within the context of \emph{techne}, the epistemology of the \emph{Republic} in general should be understood in terms of \emph{techne}.}


%\subsubsection*{The distinction makes a difference.}

%IMPORTANCE RE ARISTOTLE


%The Socratic understanding of the reliability of τέχνη would be easy to overlook, if only because it would not occur to most of us (at least to most of us philosophers) to hold expertise to such rigorous standards. But it's important to realize that Socrates expects more by way of the \technites\ than does (for example) Aristotle. We can see this importance by way of two puzzles about how Socratic virtue could be a τέχνη.


%The difference between the Socratic and the Aristotelian conceptions of \techne\ are, I believe, very signifcant for more general interpretative purposes. 


%[BRIEFLY ILLUSTRATE IMPORTANCE --- RE PLATO ONE POINT IS THAT THERE REALLY IS TO SOME EXTENT A DISTINCTION BETWEEN DIFFERENT FORMS OF KNOWLEDGE WHICH WE SHOULD AVOID READING BACK INTO SOCRATES.]

%Socrates frequently compares virtue with various τέχναι. An obvious explanation for this (in my view the correct one) would be that Socrates makes these comparisons because he thinks that virtue is knowledge, and that knowledge is τέχνη.\footnote{The notion of a ``craft analogy'' is misleading, as it invites the question ``how far does the analogy with craft hold?'' Socrates treats \emph{specific} τέχναι as \emph{analogues} of virtues, because he thinks that medicine, arithmetic, justice, etc., are \emph{all equally} τέχναι. Contrast Aristotle, who appeals to τέχνη for a (limited) analogy with virtue, e.g. at \emph{NE} 1103a32--b2.} But that is a matter of controversy, because many scholars think that τέχναι have features unsuitable to virtue.\footnote{Some scholars, such as Terence Irwin, take the analogy to be exact. Others, including Gregory Vlastos and R. E. Allen, think that the analogy has definite limits. David \citet[6]{roochnik1996aaw} says that ``Plato rejects techne as a model of moral knowledge,'' and that Plato juxtaposes τέχνη and virtue to show how \emph{poor} a model τέχνη provides.} But, at least sometimes, that's because the subtleties of the specifically Socratic conception of τέχνη are not appreciated.


%One concern is that if virtue is a τέχνη, then virtue will be of only instrumental value, and Socrates surely does not think \emph{that.}\footnote{\citet[198--9]{irwin1995pse} thinks that, for Socrates, τέχναι are of instrumental value, virtue is τέχνη, and so virtue is of instrumental value. Gregory Vlastos and George Klosko, for example, want to resist that implication, and rightly so.} Now I don't think it's clear that Socrates thinks τέχναι are of only instrumental value.\footnote{What about the case of draughts, say? (\emph{Petteutike}---\emph{Gorgias} 450d; see Dodds \emph{ad loc.}: Plato is ``probably thinking of a game of pure skill.'') \citet[100]{gosling1978pmt} draws attention to flute- and lyre-playing as well. But \citet[pp.\ 187--8]{roochnik1986} goes too far in claiming that Socrates draws a distinction between productive and theoretical crafts at \emph{Charmides} 165, so that ``the value of theoretical knowledge is not instrumental: its worth derives solely from itself.'' (cf.~Smith, 137.)} But even if we suppose that he does think that about the typical τέχνη, it's not clear that Socrates would have to think that about a \emph{moral} τέχνη, given what we have seen above about the reliability of τέχνη. If a τέχνη overcomes luck within a given domain, then one might aspire to a master-τέχνη which overcomes luck altogether---the sort of thing Socrates describes as being or superseding luck in the \emph{Euthydemus} (279d ff.). And once you have the notion of such a τέχνη, it's a short leap to the notion of a τέχνη which is intrinsically rather than instrumentally good, since good luck (or fortune) may itself be thought of as a or indeed \emph{the} intrinsic good: Socrates says everyone considers good fortune the greatest good (\emph{Euthydemus} 279c), and Aristotle says that good fortune (\eutuchia) is or is near to well-being (\eudaimonia---\emph{Physics} 197b3). Hence the view of the \emph{Euthydemus}: \sophia\ is \eutuchia, precisely because τέχνη is infallible within the realm of τύχη.

%Even in the early dialogues this is not really seen as an ultimately workable position; if it makes wisdom a techne of instrinsic value, it does so only by conflating the expertise and that which the expertise provides. It identifies the good with knowledge, but to the question what it's knowledge \emph{of}, Socrates seems to want to answer that it's knowledge of the good, which is unilluminating and violates the explicit stricture that a τέχνη have some object independent of itself.\footnote{\emph{Charmides} 165c--166c, \emph{Clitophon} 409a--d, \emph{Euthydemus} 292a--e, \emph{Republic} 505b--c.} Socrates recognizes the difficulties his model of knowledge presents for an understanding of virtue. But that does not mean he rejects the use of the τέχνη-model of knowledge for virtue---he has, as I'll be arguing, no other model of knowledge. It is only later, in the \emph{Republic}, that we see Plato offering a different sort of approach, and giving that expertise possessed by his rulers a separate object in the Form of the Good.


%There's another worry about the potentially instrumental nature of τέχνη: it seems that in general τέχναι can be misused, put to work thwarting their natural ends as well as advancing them, as Socrates points out in the \emph{Hippias Minor}. Aristotle gives this as a reason for saying that virtue is not a τέχνη; some scholars think Socrates is trying to make that point too.\footnote{\emph{NE} II.4; e.g.\ \citet[p.\ 29]{allen1996pih}.} But in fact we find here another indication of a difference in approach between Socrates, Plato, and Aristotle. Aristotle, it seems, \emph{does} simply allow that a τέχνη does not dictate its own use. In the \emph{Republic}, Plato, who I take still to conceive of that wisdom possessed by his rulers as a τέχνη, spends a good deal of time emphasizing the role of practice and experience in τέχνη; in particular, a substantial portion of the \emph{Republic} discusses cultivation of the character that is a prerequisite for moral knowledge. Thus for Plato virtue is still a τέχνη, but it is necessary for the possession of this τέχνη that one has cultivated such a character that one would never abuse that τέχνη. Socrates approach is different again, and seems again to be guided by the thought that τέχναι are infallible. As I'll explain in \S2, Socrates' rejection of the possibility of \g{>akras'ia} ensures that virtue will not \emph{actually} be abused. If wisdom is to be the valuable and salvific τέχνη that Socrates imagines it, it could not be possible to fail to employ through weakness; that would be for an accident of character---an accident of fortune---to impede the work of wisdom.


%ἢ ἐπὶ λογισμὸν ἐλθόντες περί γε τῶν τοιούτων ταχὺ ἂν [7ξ] ἀπαλλαγεῖμεν;


%ROUGHLY SAMNE AS REGARD ARISTOTLE:

%FINALLY, IN ARISTOTLE:\begin{squote}In Aristotle, this fusion of expertise with the deductive model seems to be complete. In the \e{Posterior Analytics}; Aristotle indicates that absolute, unqualified scientific knowledge requires that we know why something is the case, and that we can only know why something is the case by locating its necessary antecedents within a deductive system.\footnote{p.\ 160; Smith quotes \e{An. Post.}\ A2, 71b9--16.}\end{squote} INCLUDE THE AN PO QUOTE? She concludes that\begin{squote}the two models are still evident here, but they are now indissolubly linked. For Aristotle, having expert knowledge just is having a grasp of the axiomatic structure of a deductive system.\footnote{p.\ 161.}\end{squote}


%WELL, BASICALLY THE SAME STUFF S CARED ABOUT: PRECISION AND RATIONALITY---IN THIS RESPECT NOT A SHIFT IN THE CONCEPTION, JUST IN THE ARADIGM CASE

%SHIFT IN THE CONCEPTION IS IN THE CLEARER DISTINCTION BETWEEN TECHNE AND EPISTEME AS AN ASPECT THEREOF

%A HAS MORE CONCEPTIONS OF KNOWLEGDE; ACTUALLY REQUIRES LESS RELIABILITY AND PRECISION FROM TECHNE



%It may also be thought that the identification of knowledge and τέχνη is at any rate peculiar to Plato's earlier dialogues. The \emph{Meno} in particular is (rightly) seen as a ``transitional'' dialogue, in which more distinctively Platonic, as opposed to Socratic, epistemological themes are emerging. But in fact the τέχνη-model is no less important in the \emph{Republic} (for example) than in earlier dialogues. Socrates and Plato do grapple with difficulties presented by the model, or by the uses they put it to. New epistemological themes do emerge. But this is only to say that Plato eventually saw the need for a \emph{modified} understanding of τέχνη. 

%And this must be seen if the epistemology of the middle dialogues is to be understood. The famously cryptic account of knowledge and belief at the end of \emph{Republic} V, for example, will be unintelligible if it is not noticed that the strength and reliability attributed to knowledge by Plato there is to be interpreted in the light of what Socrates says about the strength and reliability of τέχνη in the \emph{Protagoras}. 

%To vindicate that claim I would need to defend a detailed interpretation of the epistemology of the \emph{Republic} and other dialogues. Here I will settle for illustrating one way in which Plato seems to have been motivated to rework the Socratic conception of τέχνη.\footnote{Cf. \citet[pp.\ 157--161]{asmith1998}.}

%We can start with the observation of Paul Woodruff (whose view of Socratic epistemology is in many ways similar to my own) that whereas learning is in the \emph{Meno} said to be a matter of recollection, and thus involves no teaching, teachability was a mark of τέχνη.\footnote{\citet[p.\ 81]{woodruff1990pse}.} He thinks that the τέχνη-model for knowledge has therefore been abandoned by the time of the \emph{Meno}.

%We can start with the observation of Paul Woodruff (whose view of Socratic epistemology is in many ways similar to my own) that whereas learning is in the \emph{Meno} said to be a matter of recollection, and thus involves no teaching, teachability was a mark of τέχνη.\footnote{\citet[p.\ 81]{woodruff1990pse}.} He thinks that the τέχνη-model for knowledge has therefore been abandoned by the time of the \emph{Meno}.

%Now that seems hasty. Is geometry no longer a τέχνη? Does Socrates imagine a future for geometry without teachers?  Or if teachers must now be known as `people who inspire recollection,' have we abandoned the teachability criterion or only come to a new understanding of `teaching'?\footnote{Woodruff himself allows that the \emph{Meno} is not consistent on the teachability question, and says that the suggestion that there might be knowledge without teachers ``is resisted in the balance of the dialogue'' (p.\ 83). The teachability criterion is important as part of the \emph{Meno'}s guiding hypothesis that if virtue is knowledge then it will be teachable.} Or couldn't Plato be toying with new ideas about how τέχναι are acquired?\footnote{In fact a number of familiar features of τέχνη reappear in the \emph{Meno}. The completeness requirement seems to show up in Socrates' insistence that the slave would have to go through many proofs repeatedly before he would have knowledge. The explanatory condition shows up at 98a, as does reliability.}

%[GEO IS OF COURSE A PARADIGM TECHNE IN PHILEBUS AS WE SAW; THIS PERHAPS ALL BELONGS IN DISTINCTIVENESS SECTION]

%may be acquired rather than saying that he is introducing a new conception of knowledge

%and we might also observe that the status of the teachability criterion is in doubt even in the \emph{Protagoras}.

%But there is a more substantial point. The proposal that learning is actually recollection is a response to the paradox of inquiry, posed by Meno in two parts at 80d: how will you search for something if you don't know what it is? And how will you recognize it even if you find it? This paradox is a sophistical trick, as Socrates points out. And yet Plato makes use of the paradox to introduce recollection. And one reason for this is that the paradox is a genuine problem given the Socratic conception of knowledge.

%, and it makes sense that Plato would attempt to answer it seriously. That leaves open the possibility that Plato's response amounts to a rejection of the Socratic conception. But it at least means that Plato takes that conception seriously.

%So, not knowing wisdom, how will he search for or learn it?




\section{Texnh and Propositional Knowledge}
\label{propositional}




%ITEMS I MIGHT WANT TO INTEGRATE:\\

%a few items re craft analogy: parrry 90ffm bambrough, tiles, o'brien socratic paradoxes, penner socrates on virtue and motivation
  %parry thinks you get out of the instrumentality issue by knowing good more generally - a craft directed at that - I still don't see how this helps (89-90
  %but I think arguably it's exactly the opposite - you have to have basically the right (good) dispositions to be able to see the Good in the first place - certainly doesn't look like you make contact with the good and then all the character falls into place - maybe it'll improve things but you have to already be going in the right direction etc - I mean you start by loving the good etc - suppose you had to first love health, all health and in every way etc before you could become a doctor - then why would we worry that you might abuse your skills by killing people?
  %maybe an analogy - perhaps something like counselling. Some of the people I've known who have gone into counselling are really generous people with a high degree of empathy and sympathy for people. It seems to me quite likely that 1) it would be hard to be a counsellor if you weren't like this already, although of course you also have to go through all this counselling training at the local college or whatever. But then I'm also not afraid that these people would want to use their counselling training to harm people. Obviously I don't know how far this is true, but seems at least plausible that there could be differences in some things of this sort - eg no reason someone who will excell as a tax lawer will have to be the kind of person who wouldn't use their knowledge to use their knowledge to defraud or whatever.
  %the spirited person maybe another case... why would they use their military knowledge to hide etc?
    %anways would use eg 518 to make a related point... and 514 - the *effect* of education is the getting up to the sun etc
      %was thinking of this reading Reeve - not sure how much I ight be stealing from there...\\

%EXPAND AKRASIA MATERIAL FROM M and P presentation and conference submission\\

%There are two obvious ways in which one might know \e{that} something is the case without 


%concerning particular facts may beurelated to the possession of some expertise.


\subsubsection*{\techne\ as a source of knowledge} Οἱ πολλοί, according to Socrates, believe that one can know what one should do, and yet not do it: they believe that knowledge (\ie \episteme) is weak, and that we are often governed by other things instead---``sometimes anger, sometimes pleasure, sometimes pain, on occasion love, often fear.''\footnote{\prot\ 352b.} Socrates and Protagoras disagree with the many; they think that knowledge is a  ``strong'' (ἰσχυρός) and governing (ἡγεμονικός, ἀρχικός) thing. The explicit and important discussion of knowledge which follows is therefore prompted by a worry which seems to be, at least in the first place, about the significance of knowing what one ought to do on a given occasion---i.e.\ it seems to concern something like propositional knowledge rather than \techne. But, on examination, the discussion reflects the idea that what a person knows about some particular situation may reflect the employment of some relevant expertise---for example, I may not know whether the discolouration on my arm is melanoma, but the doctor will, because he is a doctor. The discussion will therefore be useful as an illustration of the often significant interest that Socrates takes in knowledge which does not constitute a \techne, but nevertheless depends upon and manifests a \techne.

%, namely that one ought to do something or other. But that we find such discussions is not at odds with my thesis that Socratic knowledge is \e{first of all} \techne. To understand Socrates' understanding of the relationship between \techne\ and knowledge of particular facts, we will do well to start here with his defense of the strngth of knowledge in the \prot.


%propositional knowledge. So how does the existence of such discussions fit with my claim that, for Socrates, knowledge just is τέχνη? To begin answering that question, it's best to begin with this very treatment of \g{>akras'ia} in the \emph{Protagoras}.

%A Socratic τέχνη is a variety of knowledge, but, being a kind of practice, it is not a variety of belief. The relationship between knowledge and belief is important for Socrates nevertheless. To understand how Socrates envisions it, 

%[FIRST Socrates argues that \g{>akras'ia} is impossible.] The argument is framed as a refutation of the many, who \emph{do} believe in \g{>akras'ia}: they believe that, thinking you should do one thing, you can be ``overwhelmed'' by fear or the prospect of some pleasure, and act contrary to your better judgement.\footnote{As observed just above, they say that \emph{knowledge} can be overwhelmed, but I leave knowledge aside for now.} 

Socrates begins by attributing to his opponents the view that the best action is the most pleasant action, or the action most productive of pleasure over time---i.e.\ hedonism. From this view it follows that for pleasure to overwhelm you is for some \emph{good} to overwhelm you. And this good which ``overwhelms'' you must not be the greater good, or you would not be doing anything wrong by pursuing it. So being overwhelmed must entail choosing lesser goods over greater. But this is unintelligible, since there is no longer a question of being deceived by pleasures and pains distinct from the goods and evils in question (255d--e).\footnote{On pleasure as a trope for practical illusion in Plato, see \citet{moss2006pai}.}

%[IE STANDS IN FOR THE OTHER CASES] [CITED THIS EARLIER?]

Looked at from the other direction, being overwhelmed would, again by hedonism, have to mean choosing lesser pleasures over greater. But, says Socrates, there is nothing more to comparing the values of two sets of pleasures and pains than weighing them up against each other, and weighing is treated as a purely quantitative matter (256a). So now again, Socrates thinks, there is nothing here to be overwhelmed by: it is not intelligible that a person would choose a set of pleasures which they see to be the smaller.%  [STRAIGHT TO TECHNE THING?]

The language of `outweighing' is, Socrates thinks, quite apt:\begin{squote}Weighing is a good analogy; you put the pleasures together and the pains together, both the near and the remote, on the balance scale, and then say which of the two is more. For if you weigh pleasant things against pleasant, the greater and the more must always be taken; if painful things against painful, the fewer and the smaller. (356b)\end{squote}For reasons discussed in \S\ref{distinct}, this language also indicates to Socrates that our ``salvation'' will be a ``craft of measurement,'' which would \begin{squote} render the appearance ineffective: by making clear the truth, it would cause the soul to be at peace by abiding in the truth, and so save our life. (356d--e)\end{squote} And this completes the argument: to know what to do is, in fact, to possess a suitable \techne\, which \techne\ will ensure a clear view of the situation.%; we won't waver in our judgement because we will always 

%Socrates says that this moral master-τέχνη is a mathematical one (since it involves measurement), and indeed he identifies it with arithmetic (357a).\footnote{Although Socrates promises a later consideration of just what this ``craft and knowledge'' is, he does not bring up the topic again---but perhaps he would intend to qualify the identification with arithmetic, or to say how pleasures and pains are measured?} But the general point of importance for us is the role of knowledge---and specifically τέχνη---in ensuring correct beliefs in particular cases.


For our purposes, it is useful to think of the argument as having two stages (not actually distinct in the text). First, Socrates rejects the possibility of `akrasia' on the typical contemporary understanding thereof, \ie as acting otherwise than as you believe best; second, Socrates argues that \technai\ are decisive as regards belief formation.\footnote{In a sense Socrates refutes the idea that knowledge is weak \emph{twice}. If you think of knowledge in terms of knowing the right thing to do on a given occasion, then the refutation does consist in showing that akrasia is impossible. But if we think of knowledge in terms of τέχνη, then the refutation is completed only later with the τέχνη of measurement.} The first stage alone would not be of great interest to Socrates because, in his view, merely believing that some course of action is best is largely worthless from the perspective of producing that action. The ``power of appearances,'' we are told, causes us ``to wander and to change back and forth, to accept and reject the same things in actions and in choices of large and small'' (i.e.\ large and small pleasures).\footnote{\emph{Protagoras} 356d, trans.\ \citetalias{allen1996pih}.} So we do what we believe best---but we are easily misled, and we easily change our minds.

%\footnote{In the \emph{Gorgias,} too, Socrates emphasizes the vulnerability of the ignorant to the manipulation of charlatans who play on their appetites. Cf.\ also \emph{Hippias Minor} 372d--e, \emph{Euthyphro} 11b--c, \emph{Meno} 97c--98b.}



%(You have resolved to avoid unhealthy foods, but the offer of a tempting dessert prompts you to make an exception, since after all it would be rude to refuse---but immediately the thing is consumed, you know that your host would have been perfectly understanding.)




%\emph{Akrasia,} then, is impossible. Whatever you believe it best to do you will in fact do.\footnote{Like most scholars, I don't believe Socrates really accepts hedonism. But the assumption of hedonism lets Socrates deploy the above argument against \g{>akras'ia} and allows for the development of a sort of toy-model for displaying Socrates' views.}%[HURLEY AND RAZ?]



%\emph{Akrasia,} then, is impossible---assuming hedonism, anyway, which Socrates here accepts or pretends to accept, at least for the purposes of this refutation.\footnote{Like most scholars, I don't think Socrates really accepts hedonism. The assumption of hedonism lets Socrates deploy the above argument against \emph{akrasia} and allows for the development of a sort of toy-model for displaying Socrates' views.} Whatever you believe it best to do you will in fact do.%[HURLEY AND RAZ?]


%[[(And other misleading psychological forces are now presumably being understood in terms of the pursuit of pleasures and avoidance of pains.)]]

%Socrates does briefly entertain the imagined suggestion that there is some difference between pleasures and pains which are near and those which are far which accounts for being overwhelmed. Perhaps, though aware that you ought to opt for some course of action which will result in great pleasures which will however be long delayed, lesser but temporally more proximate pleasures exercise an undue influence on your actions? Socrates still insists otherwise: it is unintelligible that one should voluntarily choose less over more, regardless of how far off the various pleasures and pains might be. But there \emph{is} a difference which distance makes, by analogy with physical distance. What is further off appears smaller than what is near. And such appearances may mislead. Here then seems to be the nearest thing to akrasia that Socrates is prepared to recognize, namely ignorance about what is good caused by the relative nearness or remoteness of various pleasures and pains. Pleasures and pains cannot exercise an influence which causes us to act wrong \emph{by our own lights,} but they can throw off our perception.[INCORPORATE ANY OF THIS?]


%do I need to say that this is not just value hedonism but psychological hedonism?


 
%Emotions, personal attachments and desires can have this kind of influence, as well.\footnote{Which gets away from the influence simply of proximate pleasures and pains a bit, but Socrates clearly has other emotional states in mind and again I do not believe we should take the hedonism of the \emph{Protagoras} too seriously (see previous note).}
 
%The prospect of a donut \emph{now,} for example, may tend to \emph{produce} in us the thought that it would after all be best to eat said donut: certainly it would be enjoyable; a dollar is a trivial amount of money; there will be ample future opportunities for exercise; no one violation of my resolution to give up donuts can matter much---and so on. So whatever my prior view about donuts, that view may be at risk on entering a donut-occupied room, or if I spot a billboard advertising donuts, \emph{if} that view was \emph{merely} belief. Of course immediately I put the thing in my mouth I may promptly think of my health and curse my weakness, but then it is too late. 


Beliefs are only of value insofar as they reflect wisdom or \techne. In the \emph{Crito} Socrates tells the eponymous character that we should be attentive to the beliefs of wise people rather than to those of foolish people (46d--47b); it is also a point of the \e{Gorgias} that it is to the knowledgeable that things will appear as they are. The point is not that wisdom consists in having the right views but that some people may be expected to have the right views because they are wise. This is just the point Socrates goes on to make in the \emph{Protagoras}. Beliefs have many sources besides one's own expertise: wishful thinking, sadness or anger, nearby pleasures or faraway pains, optical illusions, the advice of experts or of charlatans. These forces sometimes compete, and some are stronger than others. So, in the speeches of Phaedrus and Agathon in the \emph{Symposium,} love dominates other emotions: fear would not drive you from battle while you fight beside your beloved. Likewise, in the \emph{Gorgias}, Socrates concedes that the testimony of the orator, who knows how to play on the emotions, will, among the ignorant, be more compelling than the testimony of a genuine expert.\footnote{459a--b. Cf.\ also \emph{Hippias Minor} 372d--e, \emph{Euthyphro} 11b--c, \emph{Meno} 97c--98b.} But, according to Socrates, knowledge (if you possess it yourself) dominates all other sources of belief. That is what he means by saying that knowledge is \emph{strong,} as he puts it in the \emph{Protagoras} (352b): without a τέχνη, your beliefs are at the mercy of other forces; if you possess a τέχνη, you will think the right way about the appropriate cases.\footnote{Thus Terence \citet{penner1997ssk} is right to insist that the ``knowledge is strong'' thesis is not implied by the denial of \akrasia. The denial of \akrasia\ is rather required to ensure we do as knowledge directs. Knowledge is strong because it ensures that you \emph{think the right way}; \emph{then} the denial of \akrasia\ takes over. Cf.\ also \citet{allen1960sp}, pp.\ 257--8 and n.\ 6.}


%NOTICE IF AKRASIA IS POSSIBLE THAT IT'S NOT SO IMPORTNAT WHETHER OUR VIEWS ARE RIGHT BECAUSE AKRAISA OFTEN GETS US TO DO THE BETTER THING ANYWAYS.







%We have now introduced a place for propositional or atomic knowledge within the Socratic scheme. 

%For Socrates, this is first of all a challenge to the value of knowledge, and, a

%Now the original dispute between Socrates and the many was actually about whether it is possible to \emph{know} what to do \emph{in a particular case} and yet not to do it. Can you know, say, that you ought to get out of bed right now, but not do it because the blankets are so warm? As we have now seen, Socrates' response is two-fold: first he argues against the possibility of \g{>akras'ia}, and then he explains how a hypothetical τέχνη of measurement would ensure that the comfort of the warm blanket won't fool you into thinking you should stay in bed. The result is that your craft knowledge delivers you (and allows you to maintain) the correct belief that you ought to get up, or---to fall in line with the way the many originally made their claim---the (propositional) knowledge that you ought to get up.

In the \emph{Crito} Socrates calls the views that stem from expertise ``good beliefs,'' but sometimes he simply calls such views \emph{knowledge}, just as the many talked of knowledge of particular facts in the \emph{Protagoras.} Thus we have \e{Ion} 537c--e, where Socrates is indeed explicit about this relationship between \techne\ and knowlede about particular facts:\begin{squote}Take these fingers: I know (\gignoskein) there are five of them, and you know (\gignoskein) the same thing about them that I do. Now suppose I ask you whether it's the same craft (\techne)---that of arithmetic---that teaches you and me (\ie by which we know: \gignoskein) the same things, or whether it's two different ones. Of course you'd say it's the same one.\footnote{trans.\ \citetalias{woodruff1983two}, with `craft' substituted for his `profession.' Cf.\ \emph{Hippias Minor} 265--7.}\end{squote} And likewise the charioteer will be the one to judge advice on racing, and so also the doctor and the pilot in their respective spheres (537). We therefore find that in different senses τέχνη \emph{is} knowledge and \emph{is a source of} knowledge.\footnote{A similar observation is made by \citet[pp.\ 205--11]{benson2000swm}, who distinguishes between Scoratic knowledge as a \dunamis\ and as a state brought about by such a \dunamis; cf.\ also \citet[p.\ 37]{brickhouse1994pss} on the early dialogues and \citet{smith2000power} on the \e{Republic}.} This must not, of course, be allowed to obscure the priority of \techne\ over other forms of knowledge (a priority indicated in the case of the \prot\ by the very existence of the second stage of the argument).\footnote{Smith [CITE] also observes the importance of distinguishing a putative Socratic interest in propositional knowledge from a Socratic interest in knowledge expressible in propositions.}



%\footnote{Strictly speaking there is a three-way distinction between 1) the body of knowledge, 2) the possession of the τέχνη, and 3) the beliefs resulting from τέχνη. EXPLAIN? DO SOMETHING WITH THIS?}



% It's not surprising that τέχνη comes up here, because that's the topic, but the suggestion that it's in virtue of a τέχνη that we know something so simple as that we have five fingers is interesting---it certainly isn't the case that anyone who can count to five is a mathematician. So Socrates seems quite prepared to subsume very mundane sorts of knowledge under τέχνη. 


%εἰ γάρ που τῶν αὐτῶν πραγμάτων ἐπιστήμη εἴη τις, τί ἂν τὴν μὲν ἑτέραν φαῖμεν εἶναι, τὴν δ᾽ ἑτέραν, ὁπότε γε ταὐτὰ εἴη εἰδέναι ἀπ᾽ ἀμφοτέρων; ὥσπερ ἐγώ τε γιγνώσκω ὅτι πέντε εἰσὶν οὗτοι οἱ δάκτυλοι, καὶ σύ, ὥσπερ ἐγώ, περὶ τούτων ταὐτὰ γιγνώσκεις: καὶ εἴ σε ἐγὼ ἐροίμην εἰ τῇ αὐτῇ τέχνῃ γιγνώσκομεν τῇ ἀριθμητικῇ τὰ αὐτὰ ἐγώ τε καὶ σὺ ἢ ἄλλῃ, φαίης ἂν δήπου τῇ αὐτῇ.




%Thus we see Socrates asking Ion whether each of them don't know that Socrates has five fingers on his hand ``by the same craft, that of arithmetic''? (537e). That is, because you know (the craft of) arithmetic, you can know (the fact that) there are five fingers on a hand.

\subsubsection*{Knowledge as a source of \techne?}

It is natural to suppose that while propositional knowledge will sometimes be an expression of expertise, propositional knowledge may also play a constitutive role---perhaps a foundational role---in expertise. For example, while the doctor's knowledge that my skin discolouration is melanoma is not properly a part of his medical knowledge (if my case is routine he will not have learned anything new, medically speaking), presumably his knowledge that melanoma may often be distinguished from moles by coloration \e{is} a constituent of his medical knowledge, and probably something he had to know long before he had actually earned the title of `doctor.' On some views, expertise is to be understood (at least partly) in terms of the accumulation of such information. According to Alvin Goldman,\begin{squote}an expert \ldots\ is someone who possesses an extensive fund of knowledge (true belief) and a set of skills or methods for apt and successful deployment of this knowledge to new questions in the domain.\footnote{\citet[p.\ 92]{goldman2001experts}.}\end{squote} On this view, expertise is something that can indeed be deployed to gain new (belief-constituted) knowledge, but is also itself to be understood in terms of knowledge constituted by true beliefs, coupled with ``skills or techniques'' or ``know-how.''\footnote{Ibid.\ pp.\ 91--2.} Should we attribute some such view to Socrates?

A difficulty in trying to identify a determinate Socratic answer to the question whether or how propositional knowledge might play a constituent role within \techne\ is that (as is one of the theses of this paper) Socrates does not---at least not explicitly---draw any distinctions whatever between different forms of knowledge, let alone our distinctions.\footnote{I ignore [EG HUMAN KNOWLEDGE STUFF; NOT OF ANY OBVIOUS PHIL SIGNIFICANCE].} But here is what Angela Smith says:\begin{squote}[Socrates] is investigating the conditions under which one can be said to know a subject, the conditions under which one can be said to be an expert in a field. Knowledge of this sort will turn out to involve a complex combination of knowledge of propositions, knowledge of skills, and knowledge of things, though the differences among these three kinds of knowledge are never emphasized.\footnote{\citet[p.\ XX]{asmith1998}.}\end{squote} I would suppose that Smith has in mind such passage as this one from the \e{Euthyphro}, where `telling' might suggest propositional knowledge to us, `looking' acquaintance, and `using' knowledge-how, and the three are all bundled together:\begin{squote}Socrates: Tell me then what this form [of piety] itself is, so that I may look upon it, and using it as a model, say that any action of yours or another's that is of that kind is pious, and if it is not that it is not.\footnote{[CITE; TRANS]}\end{squote} We should be careful to observe, though, that such passages don't tell us anything interesting about Socrates' own view of the philosophical terrain. (We would not attribute a to someone a three-way distinction between forms of knowledge simply on account of having heard them say something like: ``You know this John fellow I've heard about? Tell me what he looks like so I'll recognize him if I see him.'') They certainly don't constitute explicit statements of epistemological doctrine. That Socratic \techne\ a involves ``a complex combination of knowledge of propositions, knowledge of skills, and knowledge of things'' must be our thought, and our analysis, not his.\footnote{I do not claim that Smith herself means to deny any of this.}


%; certainly it would be misleading to say (and I do not believe Smith means to suggest) that Socrates thinks \techne\ is literally a combination of \e{other} kinds of knowledge. 





%[XX] is changeable [XX], by contrast with knowledge [XX]:

Nevertheless, there is some reason to say that, insofar as Socratic expertise might be, in some manner or to some degree, constituted by beliefs, those beliefs do not constitute instances of actual knowledge in their own right unless and until they are held by someone who possess the \techne\ properly  speaking. To see this, let us first look forward to the \e{Meno}, where something very near our present question is explicitly addressed. Socrates there indicates that belief on its own, whether true or false, is unstable and flighty (97c--98b). But he also says that beliefs become knowledge when tied down with ``an account of the reason why'' (αἰτίας λογισμός).\footnote{98a, trans.\ \citealias{grube2002five}.} I take it that the earlier conversation between Socrates and the slave is mean to illustrate some part of this `tying down' process: there the two had gone through a demonstration to the effect that, taking one square, a second square of twice the area can be constructed by taking the diagonal of the first as a side of the second.\footnote{82b ff. [CITE NEHAMAS?]} Afterwards, Socrates had said, not that the slave now knows that this can be done, but rather that ``these opinions have now just been stirred up like a dream, but if he were repeatedly asked these same questions in various ways, you know that in the end his knowledge about these things would be as accurate as anyone's'' (85c-d). All this suggests that transition from belief to knowledge essentially requires the kind of understanding or facility that attends an increasing mastery of some domain.

I think that the earlier dialogues manifest the analogous view of the relationship between belief and \techne. In the \prot\ passage considered just above, there was an explicit contrast between the security and stability of \techne\ and the instability of belief, and these are common themes in the earlier dialogues.\footnote{\cf \eg \emph{Hippias Minor} 372d--e, \emph{Euthyphro} 11b--c, CHECK THESE; ADD MORE; EG GORGIAS WITH PEOPLE GETTING CONVINCED.} And in the \e{Gorgias} passage discussed in \S\ref{distinct} above, we say Socrates contrasted real \technh\ with the kind of facility that ``merely preserves the memory of what customarily happens'' (501a). Now we might suppose that memorizing a lot of information about how things tend to work is at least laying the groundwork for mastery of the field, but Socrates himself is quite dismissive. Thus while there is no explicit theory in the early dialogues about what, if any, constitutive role beliefs or propositional knowledge might play for \techne, we do find a consistent and sometimes explicit contrast between, on the one hand, possessing beliefs, which we are liable to change because we don't have any real understanding, and, on the other hand, possession of a \techne. It would seem, then, that we can attribute propositional knowledge to experts, but only to experts.\footnote{Cf.\ \citet[p.\ 132]{asmith1998}: ``Socrates\ldots\ expects an expert to be able to give an account of his field of expertise, its methods and its aims. But\ldots\ it would be a mistake to assume from this that Socrates conceives of expertise as merely the accumulated knowledge of a vast number of propositions in a given domain.''}

%[TRANS; PULLED FROM DIFF PASSAGES, EG 501A AND EARLIER?] [A GENERAL LACK OF K ASCRIPTIONS OTHERWISE...WAIT, ALSO THE CASES APPEALED TO IN PRIORITY DISCUSSIONS] [FURTHER CITES]

To make the proposed view a bit more intelligible, notice that we need not suppose that Socrates imagines some sharp and sudden transition from mere guessing to full-blown expertise---Socrates need only think that knowledge even of particular facts can be attributed only insofar as we can also attribute some more  general understanding. In fact, although Socrates is dismissive of people to whom we would attribute a great deal of knowledge, something closely approximating this Socratic way of think about knowledge is perfectly familiar. When I first took a course in logic, there was a period during which myself and many of my fellow students could employ conditional and inductive proofs, but didn't understand why they worked. We could often have claimed certainty about our results: often we could, in one sense, have claimed to know that our results were correct, and to a logical certainty. But at the same time the process was mysterious, and we were sometimes bewildered by the results. As such, one of us would sometimes say, even as we were rigorously establishing a conclusion, ``I have no idea what I'm doing.'' Or again one of us might say ``so that's the answer?'' and receive the reply ``I dunno, I guess so.'' Our unwillingness to claim knowledge was, moreover, connected with the fact that our lack of understanding entailed a kind of insecurity and instability in our procedures. For example, some trivial novelty of formulation or notation would sometimes cause us uncertainty as to whether induction could still be employed. It is also easy to imagine even our deductive certainty about particular results withering at a frown from our professor. If we had really understood induction and conditional proof, such things could not easily have happened. Often, then, we thought about our relationship to logic in terms of rather Socratic contrast between mere guessing or rote application on the one hand, and genuine understanding and a broader facility on the other.


%Socrates saying that ``cookery''\begin{squote}is not a craft, but a knack, because it has no rational account (\g{l'ogos}) by which it applies the things it applies, to say what they are by nature, so that it \ldots (Irwin trans.)\end{squote}The point is reiterated at 501a with the example of medicine:\begin{squote}I said that medicine has considered the nature of what it cares for and the explanation (\g{a>it'ia}) of what it does, and can give a rational account (\g{l'ogos}) of each of these things.\end{squote} 

\subsubsection*{The case of `Socratic definition'} Paul Woodruff, one of relatively few scholars to have emphasized and tried to explicate the role of \techne\ in the epistemology of the early dialogues, also gives the knowledge of `definitions' a central role \e{within} \techne. For example:\begin{squote}If you do not know what Virtue is, Socrates asks in the \e{Laches}, in what way could you advise someone else as to how that is best acquired? His questions expects Laches' answer: ``none.'' The ensuing inquiry presupposes the principle of the \e{priority of definition}: knowing the definition of Courage is necessary to expert knowledge on that subject.\footnote{\citet[p.\ 79]{woodruff1987eka}, his emphasis. Cf.\ \citet{woodruff1990pse, sep-plato-ethics-shorter}. [CHECK THESE]}\end{squote}If, as Woodruff says, these definitions ``state the essence (οὐσία) of the definiendum,'' then we would seem to have prominent Socratic candidates for propositional knowledge which is prior to and constituent of Socratic \techne.\footnote{\citet[n.\ 4]{woodruff1987eka}.}

Whatever kind of answer Socrates expects to his familiar ``what is X'' question, we mistake the purpose of the question if we suppose the desired answer to represent or reflect knowledge that might be prior to anything else. Rather Socrates' question reflects his belief that the expert can articulate his understanding, and that, as such, he may be interrogated. Angela Smith has pointed out the crucial importance of attending to the context of Socrates' questions:\begin{squote}It is a striking fact that in every dialogue in which Socrates asks a ``What is X?" question, he asks for a definition only \e{after} his interlocutor has (a) explicitly claimed to be an expert (b), explicitly offered or agreed to teach Socrates or someone else, or (c) accepted (or not denied) an explicit claim to expertise made by another on his behalf.\end{squote} It is worth noticing, too, that the claims to expertise, and the invocations of \techne\ in particular, are often less obvious to us than they would have the the Greek reader. In the \e{Euthyphro}, for example, it is clear enough to the modern reader that Euthyphro purports to possess \e{some kind} of special insight into moral matters, but it is easy to overlook the symbolic importance of the invocation of precision (4e--5a) and of the measure-weight-number triad (7b--c); we are still less likely to pick up on the significance of the employment of `\epistasthai' (see \S\ref{knowledge} below); in other words, it's relatively easy not to see that, as far as Socrates is concerned, Euthyphro has claimed to possess a \techne\ concerned with piety. Socrates' familiar question in fact functions first of all as a test of putative experts, or, alternatively, as an expression of his expectation that whatever knowledge is in question ought to constitute a genuine area of expertise.\footnote{Thus Smith p.\ 149. Though some dialogues, e.g.\ \emph{Charmides}, have the character more of a shared investigation, so that it is an idea rather than a purported expert that is being tested.}\begin{squote}A definition is just a symbol, a particularly clear manifestation, of the kind of knowledge possessed by an expert.\footnote{Smith p.\ 148.}\end{squote} 




In that case, Socrates' ``what is X'' question expresses his interest in specifically expert knowledge, and his expectation that experts be able to express their knowledge in a certain way. It isn't evidence that he considers there to be some form of knowledge that might be prior to or constitutive of \techne. The answer to his question wouldn't even constitute the content of the expert's knowledge---it would simple express it. And that is so whatever exactly a Socratic definition is (Smith, for example, is willing to accept [SANTAS; CITES]). As Smith says, ``too much emphasis has been placed upon the formal features of the Socratic elenchus, and not enough on the purpose and spirit of it.''\footnote{Cf. \citet[153]{asmith1998}.}


%Gorgias 510A: SOCRATES: What, then, is the craft by which we make sure that we don’t suffer anything unjust, or as little as possible? Consider whether you think it’s the one I do. This is what I think it is: that one ought either to be a ruler himself in his city or even be a tyrant, or else to be a partisan of the regime in power.



%UPSHOT BEING THAT WHAT A DEF IS, DOESN'T GIVE US EVIDENCE OF SOMETHING PRIOR TO tECHNE OR OF SOME CONTENT OF IT---EVEN ALLOWING WITH SMITH THAT DEF ARE ESSENCE STATING AND WHATEVER ELSE.

%PROB SHOULD ADMIT NOT REALLY ADDING ANYTHING.



%Indeed there are a whole cluster of Socratic (and indeed generaly accepted) tests for expertise, of which this is only one.\footnote{As noted earlier, one way of telling a craftsman is by his work; yet another is by his pedigree, or his students (\emph{Euthyphro} 16a, \emph{Laches} 185b--187b, \emph{Gorgias} 514a--515a). Again, we sometimes see Socrates trying to elicit a suitable delineation of the scope of some supposed τέχνη (in the \emph{Ion}, the first third of \emph{Gorgias}, and later on in the \emph{Euthyphro} and \emph{Laches}.)}




%OTHER POSSIBILE POINTS:

%POINT ABOUT HOW WE WOULD DO THIS ILLUSTRATED WITH sCHIEFSKY?

%maybe draw this out thus: DO NON-EXPERTS KNOW WHEN THEY CONSULT AN EXPERT? (GOLDMAN AND HARDWIG)---NOT CLEAR SO EVEN ALLOWS THIS

%HAVE SAID WHAT THERE IS TO SAY ABOUT PRIORITY OF DEF?

%When Socrates asks his "expert" interlocutors for definitions, then, he is simply testing them: Do they really have the expert knowledge they claim to possess? Can they really give a definition of virtue which shows that they have grasped the essential nature of their subject matter? If they cannot, then they have no business lecturing on virtue and accepting money for their services.



%\footnote{This sort of test is prominent in \emph{Euthyphro,} \emph{Laches,} \emph{Charmides} and \emph{Hippias Major}.}

%Because the expert can articulate his reasoning, putative experts may be interrogated. It is this that motivates the familiar type of Socratic question: ``what is courage?'' or ``what is piety?''\footnote{On this point see Smith pp.\ 141--151.}











%Twice above I appealed to a passage in the \emph{Meno} which I take to indicate that features of beliefs and τέχναι which I identified as holding in at least some specific cases are indeed features of belief and knowledge generally. In that passage, at 98a, it is said that beliefs become knowledge when tied down with ``an account of the reason why.''\footnote{in Grube's translation.} This statement will naturally seem to treat primarily of atomic knowledge, to suggest that it is analyzable partly in terms of belief, and to concern itself with what is added to belief to get knowledge. And it is not obvious that there is anything to do with τέχνη lurking here. So one could fairly wonder whether knowledge here in this passage has anything to do with the form of knowledge I've been discussing, and whether this passage doesn't offer (or at least suggest) a different account of atomic knowledge than the one I just proposed.

%OCCURS TO ME THAT A BUNCH STUFF THAT SUSANNA SAID ABOUT BELIEFS AND CAUSATION ETC APPLIES HERE.



%If at 98a Plato is primarily concerned with an analysis of knowledge in terms of beliefs, that would mark a departure from the earlier centrality of τέχνη. The passage does certainly describe a process whereby beliefs become knowledge. However, this process is rather opaque, since the phrase ``an account of the reason why'' (or ``the cause''), is not terribly perspicuous. But, like most scholars, I take it to imply that knowledge requires some kind of understanding, or a capacity to explain why things are as they are. That seems to be borne out by Socrates identification of account giving with recollection---which is to say that he identifies giving an account with the sort of thing Meno's slave had embarked on earlier in the dialogue.\footnote{At 82b or 84a. Or perhaps properly speaking recollection is what the slave might go on to do, as Nehamas says---\citet{nehamas1999mps}, p. 18.} In that earlier conversation, Socrates and the slave together go through a demonstration to the effect that, taking one square, a second square of twice the area can be constructed by taking the diagonal of the first as a side of the second. But Socrates does \emph{not} say that the slave then knows that this can be done. Rather he says that ``these opinions have now just been stirred up like a dream, but if he were repeatedly asked these same questions in various ways, you know that in the end his knowledge about these things would be as accurate as anyone's'' (85c-d). Now this hardly tells us \emph{exactly} what or how much more the slave would have to understand before he could be said to know any geometry, but it certainly suggests that Socrates is not interested just in knowledge of particular, isolated facts, but in facility with a \emph{bodies} of knowledge, as in the earlier dialogues. A continued interest in τέχνη would account for that.

%That very tentative conclusion may be supported by the observation that although ``account of the reason why'' (\g{a>it'ias logism'os}) is not very perspicuous on its own, it is suggestive of the discussions of τέχνη in the \emph{Gorgias.} At 465a we have 




%Having said that, we could certainly also say---and I do not believe Plato would be averse to saying---that once the slave becomes expert in geometry, he will know things which previously he only believed, by tying the old beliefs down with the explanations he now has at hand. So the beliefs at 98a might themselves become instances of atomic knowledge when their possessor can give the right sort of account. But if I am right, it would remain the case that, for beliefs to become knowledge, you would effectively have to add \emph{knowledge} to them---that is, τέχνη or some degree thereof.\footnote{The notion of `explanation' in Gail Fine's proposal that knowledge in the \emph{Meno} is justified (roughly: explained) true belief seems to me so expansive as to encourage just this conclusion---that (atomic) knowledge is true belief plus knowledge. [CITE?]} It would then remain the case that there is knowledge of the atomic sort only in the presence of τέχνη, in the way described above.

%It is certainly true that, between the idea of recollection and the idea of somehow developing knowledge by way of belief, there is something new in the \emph{Meno} in terms epistemology. And that demands explanation. I do not believe that the correct explanation reveals the abandonment of the τέχνη-model. But I defer this and related issues for my conclusion.\\

%practical stuff is why techne matters of course



\section{Other Varieties of Knowledge?}
\label{varieties}


%That we may indeed find in Socrates a distinction between different forms of knowledge is a point accepted even by Scholars who have otherwise given a central role to \techne. 

%so I don't want to say that there's any definite regimentation in terms ofprop-K produced by power-K as I guess Benson would say - just that K secures doxastic outlook generally as per Crito


In the previous section I argued that \techne\ is for Socrates a foundational form of knowledge, first in that it cannot be understood in terms of any other kind of knowledge, and second in that we can attribute further knowledge to a person in virtue of their possession of a \techne, whether as a manifestation, exercise, or aspect of it. But that leaves open the possibility that Socrates acknowledges varieties of knowledge which are neither prior nor posterior to \techne, but not equivalent to it either.

This much is undeniable: Socrates avows knowledge, and attributes knowledge to others, in cases where there is no relevant \techne, or where the knower is at any rate not possessed of any relevant \techne. So, for example, Charmides knows Greek.\footnote{\emph{Charmides} 159a; Socrates here employs `\epistasthai,' but with Lyons I believe languages are not thought of as \technai.} \e{Euthydemus} 293b has Socrates knowing ``many things but slight things.'' ``To do injustice and disobey my superior,'' Socrates insists in the \e{Apology}, ``I know to be evil and base'' (29b). In the \emph{Gorgias} he tells Polus: ``I know well that if you will agree with me on those things which my soul believes, those things will be the very truth'' (486e). And many other such cases might be adduced.\footnote{BENSON CATALOGUE; WOLFSDORF CATALOGUE} 


Socrates' claims to moral knowledge are often and understandably considered particularly striking, since Socrates also disavows moral knowledge in sweeping terms. Vlastos, for example, was impressed by Socrates claim in the \e{Gorgias} to have \e{proved} that  ``one who does what's unjust is always more miserable than the one who suffers it, and the one who avoids paying what's due always more miserable than the one who does pay it.''\footnote{479d [TRANS?]; VLASTOS CITE.} Socrates' confidence cannot be based on any moral \techne, since he is quite explicit that he possesses no such \techne, much as he would like to: in the \e{Apology}, for example, Socrates observes that training children is far more important than training horses, and laments the fact that neither he nor anybody else seems to know how to do the former. [A GUIDING ASSUMPTION] \footnote{\emph{Apology} 19e--20c, [COMPLETE]; cf.\ [ MORE CITES? \emph{Laches} 186a--c; \e{Meno} 71d]} 


[IDEA THAT CAN'T KNOW WHERE YOU DON'T HAVE A TECHNE]

On my view, Socrates remarks are properly construed as expressing the view that one possess no knowledge in or about a domain unless one possesses the appropriate \techne. But in either case the result is that we have Socrates espousing knowledge in precisely that domain which he says most clearly that does not and cannot be based on assess any knowledge. [MUST WE THEN...?] 


---and many other\footnote{WOODRUFF CATALOGUE}---passages as expressing  Hence even if other avowals or knowledge could somehow be subsumed under \techne, that cannot be so for any moral claims Socrates makes.








FOOTNOTE: The apparent tension is often understood in terms of the putative Socratic commtment to `the priority of definition,' which is often construed as the view that one cannot know anything about e.g.\ virtue without knowing what virtue is, and again one cannot know whether something (i.e.\ some act or object) is e.g.\ admirable (\emph{kalon}) without knowing what the admirable \emph{is}.\footnote{\emph{Meno} 71a--b, \emph{Hippias Minor} 304d--e [CITES]. FIX THIS: Whether Socrates in fact thinks that knowledge of what a thing is is necessary and/or sufficient for knowledge of its qualities or bearers, and how these theses should be understood, has attracted much controversy, mostly under the heading of ``the priority of definition.'' A review of construals of, and of evidence for and against, these theses may be found in  \citet[chs.\ 6 \& 7]{benson2000swm}.} 



%Such avowals of knowledge are thought notable because Socrates also disavows moral knowledge in sweeping terms. [EXAMPLES] 







(Many of the purported Socratic avowals of moral knowledge look suspiciously like truisms or tautologies.\footnote{\citet[p.\ 84]{wolfsdorf2004socrates} considers thirty-two passages in which Socrates has been said to avow knowledge, and counts ``six sincere avowals of ethical knowledge'' (Euthydemus 296e--297a; Gorgias 521c--d; Protagoras 310d; Apology 22c9--d3 and 29a4--b9; Laches 190b--c).) Among those he excludes are five of the six passages that \citet[73]{vlastos1985sdk} had offered as illustrating that ``Socrates is himself convinced that he has found what he has been looking for: knowledge of moral truth.'' The one he admits is \e{Apology} 29b, which itself is more exhortation than substantive claim.} For the sake of argument, I ignore this fact, but it is not an unimportant point, and it fits well with the general response I offer in this section.)








%Vlastos was expecially struck by \e{Gorgias} 479d--e:\footnote{CITE VLASTOS}\begin{squote}Socrates: Now wasn't this the point in dispute between us, my friend? You considered Archelaus happy, a man who committed the gravest crimes without paying what was due, whereas I took the opposite view, that whoever avoids paying his due for his wrongdoing, whether he's Archelaus or any other man, is and deserves to be miserable beyond all other men, and that one who does what's unjust is always more miserable than the one who suffers it, and the one who avoids paying what's due always more miserable than the one who does pay it. Weren't these the things I said?

%Polus: Yes.

%Socrates: Hasn't it been proved that what was said is true?\end{squote}






%So if knowedge is τέχνη, what's happening in passages such as these?






%After examining one politician, Socrates reflects as follows:\begin{squote}I am indeed wiser than he. It is unlikely that either of us knows anything noble or good, but he thinks that he knows something which he does not know, whereas, as I don't actually know, neither do I think I do.




%\end{squote} [SHOULD GIVE MORE EXAMPLES AND DROP THE MENO CASE] In the \emph{Meno} we find him saying flatly, ``I confess to my shame that I have no knowledge at all about virtue.''\footnote{71d [??], trans.\ [TRANS].} 






%The claims to moral knowledge have been thought especially notable on account of the [PRIORITY OF DEF] which is sometimes 









%at least Socrates' claims to moral knowledge cannot be rooted in τέχνη. 








%Benson 2000 223-8 catalogues thirty-two cases.

%I do know, however, that it is bad and shameful to do wrong, to disobey one’s superior, be he god or man. I shall never fear or avoid things of which I do not know, whether they may not be good rather than things that I know to be bad. [Apology 29b6-9; based on Grube trans.]





%Re Definition from Benson: See (Dancy 2004:23–24) who points out that only one of the six occurrences of ‘ορος’ (Republic I 331d2-3), and only two of the 15 occurrences of ‘οριζειν’ (Charmides 173a9 and Laches 194c8) in the Socratic dialogues are best' translated as ‘definition’. In (Benson forthcoming) I mistakenly claimed that Socrates does sometimes use ‘ορισμος’ in the Socratric dialogues. To the best of my knowledge he does not. [Dancy, R. M. (2004). Plato’s Introduction of Forms. Cambridge: Cambridge University Press.]







The question, then, is whether we ought to say that there are for Socrates any further forms of knowledge---not more or less basic than \techne, but independent of it? And if so, how can this be, given [FINISH].



On some views, taking all of Socrates' avowals and disavowals of knowledge at face value would quite simply require us to recognize inconsistencies in the early dialogues. Terence Irwin assumes this in suggesting that the positive knowledge claims are anomalous, and that what Socrates really avows (or means to avow) in these cases is not knowledge but merely warranted true belief.\footnote{\citet[p.\ XX]{irwin1977pmt}; cf.\ \citet[pp.\ 28--9]{irwin1995pse}.} Richard Kraut assumes the same thing, but thinks the inconsistencies are real, and takes their existence to support a developmentalist reading of the early dialogues: so, for example, Kraut thinks that Socrates' epistemological standards are lower in the \e{Apology} and higher in the (later) \e{Hippias Major}.\footnote{\citet[p.\ XX]{kraut1984sas}.}In principle, some such strategy, insofar as possessed of any merit, might also be invoked in order to handle putative examples of \techne-independent knowledge. But they seem to me unmotivated.

%It does not follow from Socrates' invocation of heterogeneous varieties of knowledge that there is some incoherence in epistemological outlook, 

Socrates' reflective epistemological outlook does not embrace all of the varieties of knowledge he invokes in speech, but it does not follow from this that there is some incoherence in his epistemological outlook. To see why this is, consider the various ways we use knowledge-talk ourselves. Take the following exercise from Jech's \e{Elements of Set Theory}:\begin{squote}\textsc{exercise}. Assume that \textsf{〈}\textsc{a, r}\textsf{〉} is a linearly ordered structure with \textsc{a} countable and \textsc{r} dense. Show that \textsf{〈}\textsc{a, r}\textsf{〉} is isomorphic to \textsf{〈}\textsc{b,~<}{\kern-2pt}$\begin{smallmatrix}_{\circ}\ \\\text{\textsc{q}}\end{smallmatrix}${\kern-3pt}\textsf{〉} for some subset \textsc{b} of \footnotesize\textsf{ℚ}\normalsize.\end{squote}And consider now whether I know that, given a linearly ordered structure \textsf{〈}\textsc{a, r}\textsf{〉} where \textsc{a} is countable and \textsc{r} is dense, it follows that \textsf{〈}\textsc{a, r}\textsf{〉} is isomorphic to \textsf{〈}\textsc{b,~<}{\kern-2pt}$\begin{smallmatrix}_{\circ}\ \\\text{\textsc{q}}\end{smallmatrix}${\kern-3pt}\textsf{〉} for some subset \textsc{b} of \footnotesize\textsf{ℚ}\normalsize. (For convenience, let us give this conditional proposition the label `\e{A}.') Well, on the one hand, Jech wouldn't have set this excercise if it weren't the case that \e{A}. In some circumstances, that alone would lead me to claim to know that \e{A}, even if I had only the vaguest understanding of what \e{A} actually means. Suppose for example that a friend calls me up and says he is trying to remember whether \e{A}---in that case I would tell him that indeed I know that \e{A}, because I just came across it as an assumed result in Jech. On the other hand, if that were all there were to it, then probably I would more typically disavow knowledge. Suppose, for instance, that another friend were to ask whether I know anything about isomorphisms with linearly ordered structures, and that although I've just come across \e{A} assumed as a result in Jech, I'm foggy about what exactly `\textsf{〈}\textsc{b,~<}{\kern-2pt}$\begin{smallmatrix}_{\circ}\ \\\text{\textsc{q}}\end{smallmatrix}${\kern-3pt}\textsf{〉}' is supposed to represent---in that case, I certainly would not say ``Yes, I know one thing about that, namely that \e{A}''; Instead, I would say ``No, I don't know anything about that.'' Now only a pedant would refrain from these sorts of natural and expected ways of talking, since we could only confuse by doing so. In that case, the mere fact that I have, in some sense, said or implied both that I do and that I do not know whether \e{A}, it would be unfair to attribute some incoherence to \e{me}---after all, I am constrained by the normal use of the English language---and it hardly matters for this point what I think about epistemology as a theoretical matter.


% Plainly the `incompatibilities' in such assertions would not warrant attributing any incoherence or inconsistency to me; the pattern [WHAT PATTERN?] reflects my competence with English, not some peculiar psychological or conceptual incoherence on my part.

That example involves two first-order knowledge claims. With Socrates, however, the conflict is understood to a rise between first-order claims and a second-order claim (or thesis): Socrates avows knowledge about various matters falling within some domain, and also says that he cannot have any knowledge within that domain. But this kind of `incompatibility' needn't have any significance either. Suppose a friend asks me me: ``so, did you know that \e{A}?'' and that I answer ``Nope, I didn't. I don't even know what it means for something to be a `linearly ordered structure'.'' Here I rely on the principle: ``I can't know what I don't understand.'' Nevertheless, I might later avow knowledge that \e{A} as in the earlier case where a friend was just trying to check a result. So it turns out that I endorse a thesis about knowledge which entails that I cannot know whether \e{A}, and yet go on to claim to know that \e{A}. But we are still clearly within the realm of truisms and colloquial English usage, and as such it would confused to suppose that I have betrayed some incoherence by my willingness to say all of the things that I do.

Finally, it has sometimes been thought that near juxtapositions of `incompatible' knowledge claims in the early dialogues are more problematic than other more scattered claims, because they are more noticeable. But this again makes no difference on its own. For example, considering the question just more abstractly, do I in fact know that \e{A}? The natural, and indeed correct, answer is: Yes and No. Yes, in that if Jech set it then it can be proved, and, moreover, I once proved \e{A} myself some years ago. No, in that I don't remember the exact significance of all of these symbols and concepts anymore, and, even if I did, I would need to spend some time brushing up on set theory generally to have a clear understanding of \e{A}. This is, again, just the natural way of describing the situation; that someone says such things shows nothing about what if any epistemological views they hold; nor can it sit poorly with any epistemological views, because whatever happens with one's epistemological theorizing, these will still be natural ways to talk.% Refusing to talk in natural ways shows greater pedantry, not greater consistency.



%It is easy to see that it would again make a difference to my avowals and disavowls of knowledge that [X] i, though foggy on it's meaning now, I once proved it myself; if I have at least some general underastanding of the material, the it might make a difference whether the interlocutor seems to want help from me or only to assess my background so as to expalin some interesting fact to me. And so on: all sorts of factors can make a difference to the [PATTERN OF AVOWALS].  [MAKE STRONGER; LIKE IF i HAD SAID: CAN'T KNOW WHAT YOU DON'T UNDERSTAND.] 

% Suppose someone asked me what I knew about category theory.

%, in such a way as to make me suspect he had an interest in learning about it. Probably I would say that I knew nothing about it; maybe that ``I don't know the first thing about it.'' But if some nerdy party game required me to list the things I know about category theory, well, then: its name starts with a `c'; it's a highly abstract logical-mathematical system; its connectives are very powerful. And it is perfectly ordinary and intelligible that I should in this way be able to say both that I know nothing about category theory and that I know lots of things about category theory. %I might very well even say things like ``well, I know the connectives in category theory have unusual expressive power (my friend told me), but look, I don't know anything about this, you should ask him.''

%\citet{gentzler1995dbe}, p. 242, follows Kraut in this)



Now as philosophers we may be expected to bring some order to our knowledge-talk, and as such we will want to find some way of understanding how various truisms about knowledge fit together, if indeed they do. And there are different ways we might proceed when confronted with examples of the sort I've presented. We might, for example, invoke some form of contextualist semantics for knowledge attribution, whereby different standards are established by different conversational settings. Alternatively, we might distinguish different senses of `know,' so that in one sense of `know' I might know that P so long as I have it on good testimony, while in another sense I can know that \e{P} only if I understand or can explain why it is that \e{P}. Or, again, we might say that there are different kinds of things I am claiming to know in each case: in the first case I am claiming to know a particular fact, and in the second case denying that I'm familiar with a subject matter. It might even be that we are driven to the conclusion that ordinary knowledge-talk is itself ultimately confused or incoherent. But whatever our view of such matters, we cannot read any such understandings into anyone's use of the relevant language, \e{even if they are philosophers}.

This is, I think, a point that we do not need to be reminded of when reading contemporary philosophy in English. So, for example, I earlier quoted Robert Audi as saying that ``knowledge is constituted by belief (of a certain kind).''\footnote{Robert Audi, in his introduction to \citet[p.\ 1]{huemer2002ecr}.} Now in a discussion of testimonial knowledge in a companion volume, Audi sketches an example in which he meets a woman on a plane, and ``she tells me about a conference in which a speaker I know lost his temper.''\footnote{\citet[pp.\ 134--5]{audi2003eci}, his emphasis.} Speaking for myself, I feel no temptation to read anything at all into the fact that Audi here speaks of knowing \e{a person}, and I doubt whether any reader of this passage has ever felt that it should be of any particular significance for an understanding of Audi's epistemological views. And yet although we know that Plato's style is colloquial, it is routinely assumed that every single Socratic avowal or attribution of knowledge in the early dialogues is relevant to an understanding of the epistemology of those dialogues. On reflection it is clear that the painstaking catalogues of such avowals could not possibly have more than a linguistic interest.

%\footnote{[NAGEL]}


[]

%



%(i.e. without having the τέχνη which takes the admirable as its object)

%What's more,  (which I take to mean that without possessing the virtue-τέχνη you cannot know anything about virtue), 

% I prefer to ask whether or not it is possible to know anything if you don't possess a relevant τέχνη.} So not only does Socrates claim knowledge where there seems to be no τέχνη, he claims knowledge where he \emph{disavows} the appropriate τέχνη, and even says that, lacking such expertise, he \emph{can't} have any knowledge.









%[[Time for another admission. Socrates also talks about knowledge where no (actually possessed) τέχνη seems to be in question. For example, Charmides knows Greek; in the \emph{Apology} Socrates says: ``to do injustice and disobey my superior, this I know to be evil and base''; in the \emph{Gorgias} he say s ``I know well that if you will agree with me on those things which my soul believes, those things will be the very truth.'' So what sorts of knowledge are appealed to in such cases?]]

%Or could there be atomic knowledge---or any other sort of knowledge, for that matter---which is independent of τέχνη-possession? 

%[ACK - TWO Q'S HAE TO SORT OUT WHAT EXACTLY I'M TRYING TO ANSWER. this is not really what I wind up answering - I more just say look there's no distinction drawn.]

%The question is complicated by the fact that Socrates does not himself distinguish different sorts of knowledge or senses of `knowledge.' But 

%LYONS DISCUSSES THIS CASE ON 184

%BUT FOR VLASTOS AND OTHERS THERE IS OF COURSE ALSO A TENSION, AND THAT'S ALSO WHAT THEY'RE RESPONDING TO






%I agree, therefore, with Woodruff, Nehamas, and Reeve, that 

%FOOTNOTE NMORE SUITED TO NEXT SECTION?\footnote{Smith's claim is somewhat more modest:  (p.\ [XX]). Other scholars have reached similar conclusions; }






%\footnote{cf. \citet[210--11]{reeve2000rot}---though Reeve still puts an emphasis on definition and deduction that I would not.}
  %but also Smith, presumably? What about Woodruff?





%Thus Socrates tells Meno:\begin{squote}You must think I am singularly fortunate, to know whether virtue can be taught or how it is acquired. The fact is that far from knowing whether it can be taught, I have no idea what virtue itself is. . . . And how can I know a property of something when I don't even know what it is? (71a--b)\end{squote} And in the \emph{Hippias Minor} Socrates laments his inability to answer an imagined questioner who asks him:\begin{squote}How do you know whether someone has spoken admirably or not, or done anything admirable whatsoever, when you do not know the admirable? (304d--e)\end{squote} 


%So how can Socrates claim knowledge in those spheres where he has no expertise? And if such knowledge is not, and does not derive from, craft knowledge, what sort of knowledge \emph{is} it?


%But in this section I will sometimes speak of the priority of definition in deference to the usage of the scholars whom I discuss. [[WILL I? IN NOTES MAYBE?]]





%In my view the attention to propositional knowledge of what a thing is is undesirable (cf. \citet{penner1992sae}, p. [n. ]), but I also hope that my approach circumvents these questions to some extent.

%But besides wanting to interpret the importance of knowledge of what a thing is in terms of craft knowledge, I do not see the kind of conflict between a commitment to the priority of definition (or for me: the priority of craft knowledge) and the existence of knowledge in the absence of definitions (for me: craft knowledge) which other scholars see, as I will explain below. I think this means these debates can be avoided to some extent.











%For example, knowledge might require being able to rule out alternative possibilities, with context determining \emph{which} possibilities it's necessary to rule out. 


%(and choosing the best approach might require a more careful specification of the cases)

%Similarly one of, or some combination of, these strategies might be deployed to show that Socrates is not contradicting himself when he claims, on the one hand, that he has no moral knowledge (because he has no moral craft), and also, on the other hand, that he has moral knowledge. And 

Bearing in mind plausible strategies of these sorts, I do not see any need to be very concerned about supposed inconsistencies in Socrates' various attributions of, and views about, knowledge. And indeed the issue for most scholars has been not so much to worry that Socrates' various knowledge claims and ascriptions are contradictory, but to determine \emph{what sort of knowledge} Socrates is avowing when he avows knowledge, and \emph{what sort of knowledge} he is disavowing when he disavows knowledge.\footnote{For instance \citet[esp.\ p.\ 36]{brickhouse1994pss} say that Socrates disavows systematically reliable knowledge of whole domains, but that this is compatible with having knowledge of specific facts, to any degree of certainty or stability, within those domains. (I would say: Socrates has no craft---at least not of a moral sort---but this does not mean that he doesn't know particular facts which would fall within the purview of some craft.) But this approach leads them to deny that, on the Socratic view, knowledge of definitions is necessary for knowledge of specific facts (p.\ 45). But you could perfectly well still say exactly that, once you have distinguished different forms of knowledge.}


But it would be a mistake to attribute \emph{to Socrates himself} a \emph{reflective conception} of any form of knowledge besides τέχνη. What there are in the early dialogues are attributions of knowledge which do not fit the τέχνη-conception. But to find in such knowledge-attributions a \emph{conception of a form of knowledge} would be to find a Socratic form of knowledge which would have to be ``characterized almost entirely in negative terms.''\footnote{Benson p.\ 237. Though he speaks of understanding rather than τέχνη.} And contrast with the fairly clear and demanding standards for craft knowledge, Socrates would apparently have \emph{no} particular criteria for any sort of non-expert knowledge.\footnote{As \citet[p.\ 77]{woodruff1990pse} points out.} At best we might hope to identify some ways Socrates thinks you could come by such knowledge---by divine revelation, perhaps, or as derived from a life's experience interrogating others.\footnote{See Brickhouse and Smith pp.~39--41 for some possibilities (including these two).}

To say ``I know X,'' and thereby to claim knowledge of some form---with however much confidence or care or sincerity---is not thereby to deploy a conception of that form of knowledge. It does not amount to having a view about that form of knowledge or even to distinguishing it from others. Socrates talks about knowing τέχναι, people, inanimate objects, qualities, and languages, but that is not cause for supposing either that he thinks these all involve the same sort of knowledge or that he has conceptions of five different types of knowledge. Perhaps Socrates \emph{could} have distinguished, say, the kind of knowledge a craftsman has from the (propositional, factual) knowledge which the (merely) experienced person has. Later Aristotle did just that (e.g.\ \emph{Metaphysics} 981a28--30). But Socrates only spoke a language for which such distinctions \emph{could} be drawn.\footnote{For the same reason, Benson is wrong to suggest that expressions of knowledge which fall outside his expertise sense are ``loose'' or ``in the manner of the vulgar'' (pp.\ 236, 238). He worries about cases where disavowals of definitional knowledge are juxtaposed closely with avowals of non-expert knowledge. (He appears---from p.~236 with n.\ 34---to be thinking of \emph{Apology} 29a--b and 37b, and to a lesser extent of \emph{Euthydemus} 296e--297a and \emph{Hippias Minor} 304e.) He concludes that ``we can either take these passages as evidence for another sense of knowledge that Socrates periodically makes use of---although not nearly so frequently as we have been led to suppose---or we can take them as misstatements made in the heat of the moment or in the manner of the vulgar. In either case the account of Socratic knowledge offered in this examination escapes unscathed'' (p.\ 238).} 




%Neither is it necessary to suppose that attributions of knowledge which do not fit the τέχνη-framework are in any way loose or careless.




%Later, Aristotle will explicitly distinguish the kind of knowledge a craftsman has from the (atomic, factual) knowledge which the (merely) experienced person has (eg. \emph{Metaphysics} 981a28--30). And Socrates already speaks (or Plato writes) a language for which someone could draw up such distinctions. But this means neither that the distinctions have already been made nor that the locutions which prompt Aristotle to make them are in any sense `loose.'

%What licenses attributing the [techne]-conception to Socrates are the identifiable, explicit, and often debatable criteria he has for it. And he has no criteria for any other form of knowledge.

We may decide that the Socratic conception of knowledge is defective in some way, or that it covers only a limited range of the epistemological phenomena, and we may even think that some of his own claims to knowledge show this. We could say that his attributions of knowledge are actually incoherent, or we could defend the coherence of his attributions by invoking a contextualist semantics. But my topic is Socrates' \emph{own} conception or conceptions of knowledge. And as far as I can see, the only variety of knowledge of which he has any conception is τέχνη.


\




[HAVEN'T ADRESSED EVERY POSSIBLE CASE, OF COURSE Can't of course go through everything, but take for example:
Euthydemus 293b has socrates knowing "many things but slight things" so I don't want to say that there's any definite regimentation in terms ofprop-K produced by power-K as I guess Benson would say - just that K secures doxastic outlook generally as per Crito
]

[BUT LET'S GIVE A COUPLE OF EXAMPLES OF TENDENCY TO READ TO MUCH INTO THE TEXT]

FIRST: ARTIFICIAL DISTINCTIONS


Nevertheless he think that despite Socrates' lack of both expertise and definitional knowledge, Socrates mode of inquiry ``depends on knowledge;'' knowledge of, for example, courage in the case of the \e{Laches}:

So, for example, David Woodruff, having claimed that definitional knowledge is prior to expertise (quoted above) proceeds thus:\begin{squote}Socrates' inquiry depends on knowledge about the nature of Courage. Unless his procedure is sharply out of tune with itself, the knowledge he uses must differ from the knowledge he cannot claim until after successfully defining Courage. The \e{Apology} uses a distinction between expert and non-expert knowledge that supports a defense of Socrates' inquiry. His arguments in the \e{Laches} make use only of non-expert knowledge.\footnote{Ibid.\ p\ 80}\end{squote} And that is the idea he goes on to explore.



%EG WOODRUFF AND SCHIEFSKY AND SMITH....



%\footnote{cf. \citet[210--11]{reeve2000rot}---though Reeve still puts an emphasis on definition and deduction that I would not.}
%but also Smith, presumably? What about Woodruff?


Smith seems on the whole to think that Socrates' conception of expertise constitutes his conception of knowledge. So for example she says that\begin{squote} It would be a mistake\ldots\ to think that the Socratic conception of knowledge will look anything like modern epistemological theories: Socrates is talking about the conditions of expertise, not about the conditions of factual knowledge
\e{per se}.\footnote{Smith p.\ 131.}\end{squote}And later she observes that ``Socrates never shows much interest in the knowledge of single propositions.''\footnote{Smith p.\ 152.} But her intended conclusion seems to be the more modest one that Socrates is \e{mainly} or \e{largely} concerned with \techne\, which he would consider to be of \e{primary} significance: so for example she she writes that ``Socrates would not have considered it a great epistemic achievement to possess a pocketful of truths unconnected to systematic knowledge.''\footnote{Smith p.\ [XX]; for a similar conclusion without the focus on \techne, see \citet[pp.\ XX]{penner1992sae}: QUOTE.} And Smith tells us she agrees with Woodruff (and others) regarding the presuppositions of Socratic inquiry: ``the relevant distinction is not between certain knowledge and elenctic knowledge [as on Vlastos' view], but between expert knowledge and non-expert knowledge.''\footnote{Ibid.\ n.\ 56; cf.\ NEHAMAS, REEVE.} [AND WHAT ABOUT PARRY?]




FIRST: TOO MUCH ABSTRACTION IN GENERAL

We see a lot of slippery language.... eg Woodruff


Recall for instance Woodruff's motivation for []: ``Socrates' inquiry depends on knowledge about the nature of Courage,'' and so, to avoid incoherence, ``the knowledge he uses must differ from the knowledge he cannot claim until after successfully defining Courage.''



If `uses' means `draws' or `reflectively employs' two different conceptions of knowledge, then there seems to be little basis for the claim. In pariticular, what does the second (\ie non-expert) variety of knowledge look like? Woodruff himself observes that Socrates seems to have no criteria whatever here.\footnote{\citet[p.\ 77]{woodruff1990pse}. Benson, though treating Socratic knowledge as understanding rather than \techne, also points out that any further conception of knowledge attributed to Socrates' on the basis of apparently exception avowals of knowledge would have to be ``characterized almost entirely in negative terms.'' He rightly sees this a reason for discounting the epistemological significance of those avowals, [BUT STILL TOO WORRIED; p.\ 237].} [ALSO NOTE ARBITRARINESS---WHY ONLY ONE, EG?] If what is meant by `uses a distinction' is rather that Socrates relies upon, or invokes, or deploys, some conception or conceptions of knowledge (or some sense or sense of the relevant Greek words) that doesn't entail or involve the \techne\ model, then the claim is true, but of no significance for understanding the epistemology of the early dialogues: it is likely enough that every language user and every work of literature of any bulk and employing colloquial language will `use' distinctions between different forms of knowledge (including of expert and non-expert varietes) in this fashion. The colloquial [etc]; nor can the employment of such language [obviously matter; we can decide---if W wants to say *that*, then fine.]




[DISTINGUISH: S HAS CONCEPTIONS OF TWO VARIETIES OF K, AND S HAS TECHNE PLUS NOT TECHNE---ADRESS THAT second question IN NEXT SECTION] 








%suppose that either,.. or that a determinate answer...

%Now what are we to make of that bit about ``not \emph{knowing}'' this woman? Is knowing someone a matter of taking that person as an object of belief somehow? Perhaps this knowledge consists of some complex of beliefs? Or have we identified a second form of knowledge recognized by Audi? Could there even be a contradiction here? Perhaps `know' is ambiguous? Maybe the expression is careless? And yet surely the manuscript was carefully edited?

%I think it's fairly clear that, if we had only the evidence I've provided here, there would be \emph{nothing} to say. We just can't know from this what if anything Audi might say if we were to ask him about this, and we would not be inclined to think there is a problem if he doesn't indicate his answer somewhere. He might have some analysis of this kind of knowledge in terms of belief. He might claim `know' has a different sense here. He might admit a gap in his epistemological theory, or even an incoherence. He may already have one of these approaches in mind, but he might also adopt one only once asked, in which case there's just not yet any fact of the matter at all as to what he thinks about the case. The case is exactly parallel with our interpretative situation with Socrates. Socrates is interested in one sort of thing which goes by the name `knowledge,' he has nothing to say about whatever else may also do so; but as Audi speaks English, so Socrates speaks Greek, and language is not so tidy as philosophical views.



CONSTRAINTS AND SUCH







Consider for example Socrates occasional instruction to his interlocutors to tell him what they really think [CITES]. Vlastos has described this as a [VLASTOS]. Hugh Benson argues that in fact Socrates accepts the `doxastic constraint':\begin{squote}[DC] Being believed by the interlocutor is a necessary and sufficient condition for being a premise of a Socratic elenchos.\footnote{[BENSON 38]}\end{squote} The difference between this and Vlastos' say-what-you-believe `constraint' is the addition of `sufficient.'





Now imagine the following conversation with a doctor:\begin{squote}Me: So, doctor, melonoma is life-threatening, right?

Doctor: Sure, I'll accept that.

Me: \ldots I wasn't looking for agreement---I wasn't basing that on anything. I wanted \e{your} view.

Doctor: Yeah, I guess melonoma seems like it might be life-threatening.

Me: It \e{seems} like it? Look, is it or isn't?\end{squote}Moving to a higher level of abstraction, we could say that, in at lenversational settings, I impose the following `doxastic requirement': the interlocutors sincere assertion of some thesis is both necessary and sufficient in order for us to proceed. This is, I suppose, true, but entirely trivial; it reflects nothing peculiar to my conception of knowledge (anyone would have acted as I did), and indicates nothing about what if any considered views I might possess about knowledge---my response would have been equally natural from an epistemologist (regardless of his epistemological theories) or from someone who had never heard of the subject. To say that I accept the `doxastic requirement' for some conversational settings would, moreover, entirely misconstrue the matter: to attribute it to me is to show that you don't u [[]] [he only proper way to express ]




[point being not innocent---people are assuming that because abstraction is safe, it's innocent, and it's not.]






DEFINITION STUFF ESPECIALLY


Santas, for example, says that\begin{squote}the definiendum of a Socratic definition of F-ness or the F is probably an attribute, which (a) is one and the same in all things that are F, (b) is that by reason of which all F things are F, (c) is that by which all F things do not differ but are all the same, and (d) is that which in all F things we call `F-ness' or `the F'.\footnote{SANTAS CITE}\end{squote}Terence Penner rejects the view that Socrates is interested in definitional knowledge, or indeed any kind of propositional knowledge at all:\begin{squote}One needs to know what the reference of courage is. Frege says about reference: "Comprehensive knowledge of the reference would require us to be able to say immediately whether any given sense belongs to it. To such knowledge we never attain." 80 That is the kind of knowledge of virtue, knowledge, power, desire, good, and so forth that I see Socrates striving for.\footnote{Penner p.\ 147}\end{squote} Angela Smith, while insisting on the preeminence of \techne, thinks that ``{`}telling us what X is'{''} turns out to be a rather complicated affair, much more complicated than merely defining the meaning of a word or expression.'' She adduces Santas' criteria as an artciulation of this complexity, and observes that\begin{squote}These are very demanding criteria---so demanding, in fact, that one might begin to wonder how knowledge can ever be gained, if a definition of this sort is a necessary first step to its achievement. As we look at these criteria more carefully, it becomes quite clear that only someone with a \e{very} sophisticated understanding of X could even begin to try to tackle this Socratic request. Indeed, I think it would be fair to say that only someone who had already achieved expertise could have the breadth of knowledge and experience necessary to produce a definition of this sort. This is a demand, in short, that only an expert, reflecting on the nature of his knowledge, could ever hope to fulfill. A Socratic definition is not the starting point of knowledge; it is, rather, the culmination of it.\footnote{Smith p.\ 147.}\end{squote} As Smith herself points out, [SEEMS NOT FITTING FOR OTHER CASE]. Considered in that light, it is unsurprising that WOodruff simply carves Socrates' desired moral \techne\ apart from other \techne: [QUOTE, CITE]


It is worth dwelling on the resulting interpretative situation. There is widespread agreement amongst that Socrates accepts a set of criteria (which are not explicitly formulated) for definition (a notion for which there is no Socratic term---still less for `reference');\footnote{So \citet[pp.\ 113-14]{robinson1950definition}: ``If we describe [the What-is-X? question], as I have so far refrained from doing, by means of such words as 'definition' and 'example,' if we extract from it explicit rules and principles of definition, we pass to a stage of abstraction higher than the dialogues themselves display. We can, indeed, pick out an occasional word to be appropriately translated by `example' or `definition,' and we can easily formulate, from Socrates instructions to his hearers, rules resembling those in a modern textbook; but that is only to say that each level of abstraction is near to the next! The actual picture in the dialogues is not more but less abstract than the picture here given; for Socrates does not use the letter X; he never gives the function but one of it's arguments.''} criteria which are hard even to make intelligible when applied to his own preferred examples of standard \technai\ (the very \technai\ Socrates invoking precisely to motivate these criteria), if they don't simply fall short of it (despite their constant employment as paradigms of knowledge). This

[ought to give pause]






%\begin{squote}Now, ``telling us what X is{''} turns out to be a rather complicated affair, much more complicated than merely defining the meaning of a word or expression. It is generally acknowledged that Socrates never does show any interest in nominal definitions of this sort. What Socrates is looking for, apparently, is a description of the explanatory properties of X, an account of X's essential nature that actually explains why X is as it is. Santas comes to the conclusion that ``the finiendum of a Socratic definition of F-ness or the F is probably an attribute, which (a) is one and the same in all things that are F, (b) is that by reason of which all F things are F, (c) is that by which all F things do not differ but are all the same, and (d) is that which in all F things we call `F-ness' or `the F'.''


[REMINDER ABOUT SANTAS AND SMITH, OR SHIFT HERE]

But this cannot be right. Consider Socrates' instructions to Euthyphro:




\begin{squote}Socrates: Bear in mind then that I did not bid you tell me one or two of the many pious actions but that form itself that makes all pious actions pious, for you agreed that all impious actions are impious and all pious actions pious through one form, or don't you remember?

Euthyphro: I do.

Socrates: Tell me then what this form itself is, so that I may look upon it, and using it as a model, say that any action of yours or another's that is of that kind is pious, and if it is not that it is not.\end{squote}This passage, as much as any other, is responsible for the impression that Socrates' demand is a very stringent one. But that is not a tenable view when we consider the lines which immediately follow:\begin{squote}Euthyphro: If that is how you want it, Socrates, that is how I will tell you.

Socrates: That is what I want.

Euthyphro: Well then, what is dear to the gods is pious, what is not is impious.

Socrates: Splendid, Euthyphro! You have now answered in the way I wanted.\end{squote}As it turns, out the answer is not acceptable. But Socrates thinks it is at least superficially a candidate for being the right answer. But it is clear 




But if it were true that Socrates were after an answer that only a true expert could hope to give, he could not possibly think that Euthyphro may just have given such a thing. [] To all appearances, Socrates says what he does about looking upon the form of piety because Euthyphro has been giving him a list of piosu actions, Socrates wants something more general, and the suggestion that 



In fact Socrates is pleased because ....



MAYBE ADDRESS DEFS AGAIN HERE BECAUSE CAUSING PROBLEMS HERE: SENSE THAT IT'S THIS BIG AMBITIOUS THING IS THROWING OFF WOODRUFF, BUT ALSO SMITH...





[IN NEXT SECTION? OR TO BUTRESS PRESENT POINT?]The interest in definitions should not be construed as an interest in knowledge which may be captured by a verbal formula.\footnote{Socrates is as happy with expressions of the form ``know X'' as ``know what X is''---he thinks of knowledge as much in terms of acquaintance or perception as in terms of anything linguistic. See e.g. \emph{Charmides} 159a, \emph{Euthyphro} 6e, \emph{Gorgias} 503e.} Knowledge of a given entity or quality, in the sense Socrates is concerned with, already indicates mastery of the whole rational practice that constitutes a τέχνη.\footnote{See, for example, \emph{Laches} 190a--c and \emph{Gorgias} 503d--e. Indeed a phrase like ``knowing X'' can simply be used synecdochally to characterize a τέχνη as a whole by reference to its proper object---thus medicine is the ``knowledge of health,'' or the ``knowledge of health and disease'' (e.g. \emph{Charmides} 165c, \emph{Gorgias} []).}


Socrates' interest in \emph{what things are} has been described as an interest in real as opposed to nominal definitions. This is fair in the sense that Socrates expects experts to be be able to articulate their knowledge. Doctors and laymen alike \emph{talk} about health, and can point to examples of healthy people, or give general characterizations of health---as ``good condition of the body,'' for example. But the layman relies on more superficial signs, and is liable to error, while the doctor can tell healthy people apart from those who merely appear healthy (\emph{Gorgias} 464a--b).\footnote{Cf. \emph{Euthyphro} 6e, where piety is a \paradeigma\ or model one may look to.} The doctor's ability to give a ``real definition'' of health is a reflection of this ability to distinguish appearance from reality. So to know what the qualities in question are is, in part, to be able to employ \emph{and explain} reliable standards of assessment. %An answer to the question ``what is X'' that did not reflect such an ability would not be an adequate answer to the question as Socrates understands it.




[puzzlement about defs ka as nec and sufficient; puzzle largely disipates on understanding the role]






[]

Vlastos, Socrates disavowal of Knowledge, in Fine p. 84: Socrates will never be contradicting himself by saying, or implying, that he both has and hasn't knowledge, for he will not be saying or implying that he does and doesn't have knowledgeE, or that he does and doesn't have knowledgeE, but only that he does have knowledgeE and does not have knowledgeC. Thus his avowal of ignorance will never generate practical inconsistency or doctrinal incoherence. When he tells the interlocutor that he has no knowledge he will not be violating the "say what you believe" rule of elenctic debate, for he will not be feigning ignorance: he does believe with full conviction, with utter sincerity, that he has no knowledgeC.51 Nor will he be endangering his doctrine that "Virtue is knowledge" when this is read, as it should be, "Virtue is
knowledgeE".






Benson has the 'doxastic contsraint' at 38 and distinguishes it from Vlastos' say what you beleive constraint at 39

Gorgias 479e8 cited by Vlastos in 1983b: has it not been proved that what I said before was true? [think of my logic example]

``different working conceptions of knowledge'' - only if we mean by this \emph{only} in the sense that knowledge in his mouth can mean different things - and in one important sense there's only oneworking conception - this is etrayed by tendency to think of all K in terms of craft

`two' is arbitrary - again just craft and other - W also on eg 66 taking the other side too seriously - on the other hand 77 plays it down - no criteria etc - a good



PROBLEMS ARE ALL ENTIRELY DISSOLVED BY SAYING THAT sOCRATES DOESN'T DISTINGUISH CASE.

This suggests that whatever Charmides or Euthyphro could \emph{say} about temperance or piety would represent only a reflection or aspect of their knowledge.

CHarm 165: arithmetic is knowledge of the even and the odd etc

asUSEFUL EXAMPLES FROM PARRY SEP: In some dialogues, craft (technê) and knowledge (epistêmê) seem interchangeable in much the same way as in Xenophon's Socratic dialogues. In the Charmides (165c) Socrates says that medicine, i.e., the physician's craft (iatrikê technê), is the knowledge (epistêmê) of health. In Euthydemus (281a) Socrates says that what guides right use of materials in carpentry is the knowledge of carpentry (techtonikê epistêmê). In Ion (532c) Socrates tells the rhapsode Ion that he is not able to talk about Homer with craft and knowledge. In Protagoras (356d-e) Socrates refers to measuring as both a craft and a kind of knowledge




[]

%David Woodruff [remember he allows...]


%NOTE WORRY ABOUT ELENCHUS: THAT IT PRESUPPOSES OR SOMEHOW PRODUCES KNOWLEDGE.






%Thus at \emph{Euthyphro} 6e Socrates speaks of piety as a \emph{paradeigma} or model to which one may look, so that knowledge of piety means being able to tell whether or not a given action is pious. 




%Socrates is interested in one sort of thing which goes by the name `knowledge,' he has nothing to say about whatever else may also do so; but as Audi speaks English, so Socrates speaks Greek, and language is not so tidy as philosophical views.

%I could go through Woodruff p64 a bit:

%a distinction - in some sense sure but it's ours, and all it'll have is craft on one side and everything else on the other.


%also wrong to say S disavows all expert K - if that's what Woodruff is doing on 65 - yeah, seems he does on 66

%Having said all that, there is one case where Socrates does actually explicitly distinguish different sorts of knowledge---it is an interesting case and a case of an exception that proves the rule. %the human wisdom - which may basically be thought of as his conception of what a craft requires

%Look at Benson's citations here too - eg Menn 1994?

%There are sources of belief besides K - are there sources of K besides craft K? Perhaps at least say finding out from an expert? Perhaps not - In the Gorgias we havethe sopst convining the patient, not the doctor.

%So we may indeed always do what we think it best to do, but that is of little significance if any passing prospect of some good may reshape our beliefs at any moment: the appearance of some new good or the sudden thought of some old good may at any moment recast our conception of what is worth trading for what. There is no such thing, except perhaps by chance, as a temporally stable (mere) belief that a donut is worth roughly so much, fitness so much more or less, and so on.\footnote{Compare Terence Penner, [Apeiron 1996, AGP 1997]; although I follow Penner in insisting that we take the appeal to knowledge more seriously than is typically done, my exposition of the argument for the strength of knowledge differs substantially from Penner's (and has here been presented much less formally). For one thing he gives no special role to the apparent hedonism which is invoked, perhaps because he takes the `hedonism' here to be simply Socratic eudaimonism. I find that suggestion dubious. Protagoras clearly is put off by hedonism; but why should he be so put off if this `hedonism' is compatible with the view that the good life is a life of wisdom and righteousness? Or why couldn't Socrates have clarified his meaning? And the hedonistic thesis is also ascribed to the many---are we to suppose that Socrates thinks that the many accept the moral views he himself espouses in other dialogues? (Compare also Aristotle's attribution of hedonism to the \emph{hoi polloi} in \emph{Nicomachean Ethics} 1.5.) Most importantly, it seems to me, as reflected in my exposition, that hedonism secures the idea that goods are subject to the simple form of measurement which renders akrasia unintelligible. This simplicity seems to me lost if `hedonism' is simply eudaimonism. But no doubt that helps explain why Penner's exposition differs from my own.}

%If that is our situation, then our good depends on being able to resist these fluxuations in our evaluative views---on being able to attain and hold to a correct picture of the importance of the various possible objects of our pursuit. And so, says Socrates, our ``salvation'' will be an ``art of measurement,'' which would \begin{squote} render the appearance ineffective: by making clear the truth, it would cause the soul to be at peace by abiding in the truth, and so save our life. (356d-e)\end{squote} Socrates clearly thinks that this is a basically mathematical art, and indeed he goes on to identify it with (the art of) arithmetic (357a). By the very fact that it is a kind of measurement that is the answer to fluctuating beliefs about the good, it is ``art and knowledge'' that is the solution. (Although Socrates promises a later consideration of just what this ``art and knowledge'' is---perhaps he would intend to qualify the identification with arithmetic, or perhaps to say how pleasures and pains are measured?---he takes up the topic again neither here nor in any other dialogue.) And that is the refutation of the \emph{hoi polloi:} we have now seen that knowledge is strong, because knowledge is precisely the answer to the difficulty of unstable evaluative beliefs. 

%******On the otehr hand in chap 2 I think we're going to see that S has moved his subtlety to another place: the dobtfulness of belief - so I don't think we should \emph{simply} say that it's this totally crude intellectualist picture

%So given the craft model for virtue, at least as found in the early dialogues, the denial of akrasia would seem to be required if the implication that virtue may be abused is to be resisted. However there are also aspects of the craft model that either do, or may have at least seemed to, make the rejection of akrasia rather natural, so that we need not see this rejection as in any way an ad hoc  effort to save the craft model. One is the notion of professionalism bound up with [TECHNE]. [EG YOU EXPECT THEM TO PRODUE], ALSO FROM TEACHING TAKE THIS AS A WAY OF ASKING ABOUT CRAFT]

%I wonder if one thing that could have made the idea that a knowledgeable person will do what's good is the professionalism aspect of techne - iie you actually do the thing. Ie Dads's not a lwayer although he has the LLB he doesn't practice law. And the virtuos person's product would f course be a good life.

%[of course the no akraisa thesis is much broader - applies to belief as well - and there is a snese here in which the view is intellectuaist. However important to see that S viewed beliefs as highly maliable/unstable, so that not clear this is really as objectionable given his way of thinking about B. THis should be the transition to chapter 2.]

%- actually one problem for me might be the bit t Prot 331c-d where S says he doesn't want to deal with "ei soi dokei"s. that's odd from my perspective, though may be something tirvial having to do with the semantic range of this expression - and see eg citations at Benson 54 where this expression sems not to be a problem and indeed B takes it as a way of pointing to fact people really do believe what they're saying which I think is probably fine - so really we just have a somewhate messier situation here than I might have liked.

%GottA read Gorgias Encomium to Helena and I think there's some defense of rhetoric too - see Roochnik chapter 1 - basically he has a techne controlling doxa, and what's more all with out truth!
  %Note that it woudbe characteristically Socratic to say that ever techne controls doxa or should

%Benson's puzzle on 211 def one I want to pick up on

%Benson 206-7 gets close in some ways to the picture I have in mind of state poducing powers - ad rather off inother ways - the ambiguity he identifies at 207-8 between K as the state and K as the dunamis is one I would probably accept as well - excpet that thinking of the state as something diff from a doxa state is here pointless (if we take the Crito expressions seriously0, if that's what he's doing, which I think it is - bascically he justdoesn't have this stuff tied together in a perspicuous way - he has refs here that should be helpful as well - also cites Menn 1994?

%also up to 209 got I think a roughly right: dunamis produces the right belief for a particular case" sort f idea

%Protagoras at Prot 327 contrasts experts with laymen in Loeb trans - should check greek on that

%Euthydemus 281 has orthos chresthai (right use) being ruled by knowledge of eg. carpentry. I want to say that 

%Gorg 454e-455a is a case where we have another belief produing dunamis, perhaps - if S would count it as a dunamis? probably so 

%Nice line at Wieland 224, point is (I think) that we think of the propositional K as fixed and don't consider whetehr it represents some knowledge which deals with it - and n my mind that's just what we should mostly be doing....
  %though of course many K claims that aren't this

%Wieland 225 suggests tom me the point: Knowledge in Plato is not the kind of mental state that could have no object - the way for us the mental state in K might just be belief, which could fail to have an object. (Would be denied by say WIlliamson - also what do I mean by object here? Cause if any belief has to have in sme sense a propositional object then ore complicated.)

%Wieland 229: nonprop K can only be had or not, possibility of error not on,  - has no opposite which is it's denial

\section{Knowledge as Texnh}
\label{knowledge}

%re the bit about having only the craft or nothing options might connect the question whetehr it's teachable in eg Meno (87)

%Euthydemus 294 is an example of going right to craft K, I think - interestingly Ctessipus uses an example with  "how many teeth" at 294c - interesting not SOcrates but s still  of course Plato... - knowing how to dance at e - this one is Socrates'

%A lot of this might conceivably be better n the intro....

		
We have seen that Socrates has a considered and distinctive conception τέχνη. We've also seen that a lot of Socrates' epistemological reflections should be understood within the τέχνη-framework: knowledge of particular facts, which contemporary epistemology treats as the basic case, is often for Socrates the manifestation of expertise. And while there are in the early dialogues references to, or invocations of, knowledge that cannot be understood in terms of τέχνη, there is no sign of any organized or reflective Socratic view about whatever form or forms of knowledge might be at play there. But I wish now to emphasize the positive tendency on Socrates' part to understand knowledge generally in terms of \techne\ specifically---I wish, that is, to urge the point that that the role of \techne\ in Socrates' epistemological thought is such that we may say simply that, for Socrates, knowledge is \techne. Whatever else knowledge is for Socrates, it is only secondarily or marginally or colloquially.

First, then, observe that seemingly more general epistemological questions are frequently interpreted or addressed in terms of \techne, or by appeal to the case of \techne. So, for example,  Socrates' attempt, in the \prot, to demonstrate that \sophia\ and ἐπιστήμη are `strong'\footnote{At 352b--d Socrates and Protagoras employ `ἐπιστήμη,' `\phronesis,' `\sophia,' and `\gignoskein.'} culminates, as we've seen, in the point that knowledge, being a τέχνη of measurement, will ensure that you get things right; Socrates' discussion simply assumes that the resources of \techne\ may be deployed in resolving this question. The popular view had seemed initially to concern no more than the possibility of doing some particular thing that you know not to be the best thing to do. But Socrates treats the view of the many as a challenge to the value of knowledge taken as \techne, or as a manifestation thereof. The obvious reason why he would so construe the popular view is that he thinks that knowing what is to be done in a particular instance must be a manifestation of the possession of some relevant \techne.

%[KNOWING THE GOOD ETC]



% Now although `ἐπιστήμη' is indeed frequently synonymous with `τέχνη' in the Socratic dialogues, but if Socrates is thinking of ἐπιστήμη in terms of τέχνη right from the start at 352b, that is itself noteworthy.
 
 
  





%\footnote{[This point is apt to be overlooked if is not realized that the impossibility of AKRAISA and the strength of knowledge are two totally different theses and that neither implies the other.]}



%Here are a few illustrations of that tendency to think of knowledge as such in terms of τέχνη specifically.


%\footnote{OBVIOUS SOCRATIC MOTIVE HERE IS VIEW TAHT CAN'T KNOW THE CASES IF YOU DON'T KNOW THE THING ITSELF.}


Again, consider Socrates reply to Critias' proposal in the \charm\ that temperance (\sophrosune) is knowing oneself (τὸ γιγνώσκειν ἑαυτόν; 165a). Socrates says that if temperance is knowing (\gignoskein), then it must be some kind of ἐπιστήμη, and it must be \emph{of} something; his examples are medicine and housebuilding, which are `of' health and building houses respectively---and these are, of course, paradigmatic \technai. Some debate follows about what, if anything, knowledge must in fact be `of,' but Socrates' assumption that it will be some or other \episteme\ and \techne\ remains in effect. It is notable how peculiar this transition would be except for the powerful force of \techne\ in Socrates' epistemological thinking: Socrates supposes that self-knowledge could only be interpreted as some kind of τέχνη, despite the fact that there is obviously no τέχνη in the neighbourhood---as the rest of the dialogue makes clear enough---nor any evident reason to suppose right at the start that there should be.



%Though he then grants (at 166a) that knowledge need not \emph{always} have a product (as evidenced by mathematical knowledge), Socrates continues to assume that knowledge must be useful. So here again we have an immediate transition from knowledge to τέχνη (`ἐπιστήμη' is here equivalent to `τέχνη,' which also appears several times in this passage). 
 
 
That these transitions to \techne\ in the \prot\ and \charm\ are noteworthy is confirmed by an examination of the language employed. According to John Lyons' analysis of Plato's epistemological vocabulary, the verbs `\eidenai,' `\epistasthai,' and `\gignoskein' are grouped in the following way: `\eidenai' is largely interchangeably with either `\epistasthai' or `\gignoskein,' but `\epistasthai' and `\gignoskein' play somewhat different roles and are not interchangeable with each other.\footnote{\citet{lyons1972ssa}, chapter 7. Lyons' discussion concerns the Platonic corpus as a whole.} For example, `\gignoskein,' unlike `\epistasthai,' is particularly associated with personal nouns.\footnote{Lyons pp.\ 179, 199 ff.} `\epistasthai' is in a ``generator'' for `τέχνη,' in that when someone knows in the sense indicated by `\epistasthai,' then you can invoke a corresponding τέχνη.\footnote{Lyons pp.\ 160 ff. \emph{Charmides} 159a may be an exception---Socrates observes that Charmides knows Greek, employing `\epistasthai,' but, on the evidence of the \emph{Protagoras}, I don't not think Socrates would speak of a τέχνη of Greek; Lyons' conclusion is the same (pp.\ 184--5, 221). Also, it is natural---but noteworthy---that, given the close connection between \epistasthai\ and τέχνη, Socrates never claims to know things in a propositional sense with \epistasthai, although he makes such avowals with \eidenai.}

In the \charm, Socrates proceeds thus:\begin{squote} εἰ γὰρ δὴ γιγνώσκειν γέ τί ἐστιν ἡ σωφροσύνη, δῆλον ὅτι ἐπιστήμη τις ἂν εἴη καὶ τινός: ἢ οὔ;---ἔστιν, ἔφη, ἑαυτοῦ γε.---οὐκοῦν καὶ ἰατρική, ἔφην, ἐπιστήμη ἐστὶν τοῦ ὑγιεινοῦ;---πάνυ γε.---εἰ τοίνυν με, ἔφην, ἔροιο σύ: ``ἰατρικὴ ὑγιεινοῦ ἐπιστήμη οὖσα τί ἡμῖν χρησίμη ἐστὶν καὶ τί ἀπεργάζεται,'' εἴποιμ᾽ ἂν ὅτι οὐ σμικρὰν ὠφελίαν: τὴν γὰρ ὑγίειαν καλὸν ἡμῖν ἔργον ἀπεργάζεται, εἰ ἀποδέχῃ τοῦτο.---ἀποδέχομαι.---καὶ εἰ τοίνυν με ἔροιο τὴν οἰκοδομικήν, ἐπιστήμην οὖσαν τοῦ οἰκοδομεῖν, τί φημι ἔργον ἀπεργάζεσθαι, εἴποιμ᾽ ἂν ὅτι οἰκήσεις: ὡσαύτως δὲ καὶ τῶν ἄλλων τεχνῶν.\footnote{356c--d.}\end{squote} Here  we move from `\gignoskein' to `τέχνη,' but the example is not at odds with Lyons' analysis. What we see here is an explicit \emph{inference} on Socrates' part: we have a case of knowing something (γιγνώσκειν τί) and \emph{therefore} of ἐπιστήμη, and thus also of τέχνη. This case then shows that Socrates is prepared to infer a τέχνη not only where the sort of knowledge in question (knowing oneself) seems to have no evident connection with τέχνη, but also where the specific epistemic vocabulary has no particular connection with τέχνη in general Platonic usage.

In the \prot\ we also have a transition from \gignoskein\ to \techne\ by way of \episteme:\begin{squote}καὶ δοκεῖ, ἔφη, ὥσπερ σὺ λέγεις, ὦ Σώκρατες, καὶ ἅμα, εἴπερ τῳ ἄλλῳ, αἰσχρόν ἐστι καὶ ἐμοὶ σοφίαν καὶ ἐπιστήμην μὴ οὐχὶ πάντων κράτιστον φάναι εἶναι τῶν ἀνθρωπείων πραγμάτων.

οἶσθα οὖν ὅτι οἱ πολλοὶ τῶν ἀνθρώπων ἐμοί τε καὶ σοὶ οὐ πείθονται, ἀλλὰ πολλούς φασι γιγνώσκοντας τὰ βέλτιστα οὐκ ἐθέλειν πράττειν, ἐξὸν αὐτοῖς, ἀλλὰ ἄλλα πράττειν.\footnote{352c--d.}\end{squote}The popular view is assumed from the outset to be a rejection of the view of \sophia\ and \episteme\ shared by Protagoras and Socrates, and \episteme\ turns out later to entail \techne. There are in fact a number of instances of logical rather than liguistic transitions from \gignoskein\ to \techne; Lyons himself discusses an case in the \e{Ion}, where `\gignoskein' is used in connection with `τέχνη' not for linguistic reasons but as part of an argument connecting good judgment in a particular field with an appropriate τέχνη.\footnote{540e; Lyons, pp.\ 188--198.} 


%Socrates only speaks of τέχνη after eliciting from Critias some examples of what are in fact conventional τέχναι, but under the heading of `ἐπιστήμη,' which may very well reflect the fact that `\gignoskein' is not so naturally associated with `τέχνη.' [IS THAT TRUE? WHAT IS THE SITUATION WITH EPISTEME?]

A second, and related, observation is that Socrates tends either to subsume whatever one might know under some \techne, or to disregard or dismiss such `knowledge' altogether. Recall for instance that Socrates and Ion were said each to know (\gignoskein) that Socrates has five fingers on a hand by the same \techne, namely arithmetic (537e). Now one might think that knowing one has five fingers on a hand is pretty elementary---not the kind of thing one would need any \techne\ for at all. Socrates and Ion aren't mathematicians; a child might know he has five fingers on a hand because he proudly counts them up every day, five being the highest number he can get to. Nevertheless, Socrates subsumes his and Ion's knowledge under the arithmetical \techne. The thought, as I understand it, is that to the extent you \e{know} that there are five of something, that will be because to some degree or in form you possess the arithmetical \techne\ (recall that one may benefit from mathematical \technai\ without actually being a mathematician, as is the case with carpenters, and perhaps with all experts). 

Where there is no such \techne\ or no possession of such---where `knowledge' could \e{only} come in bits and pieces, or where a putative body of knowledge is in some other way incomplete or unsystematic---then Socrates isn't interested. He doesn't for example, concern himself with historical knowledge.\footnote{Cf. \citetalias{allen1989dpe} (\citeyear[p.\ 326]{allen1989dpe}).} This can't be attributed just to Socrates' pursuit of wisdom, since that pursuit doesn't prevent him from invoking other more `theoretical' varieties of knowledge---like arithmetic---so long as these constitute \technai.\footnote{Even Aristotle makes somewhat disparaging remarks about history, on account of its dealing only with particular facts: \emph{Poetics} 1451a36--b11.} That Socrates doesn't discuss a given topic doesn't show much on its own, of course, but we can look at what he \e{does} say about poetry and oratory. Not only does Socrates not consider these to be τέχναι, he never considers the possibility that they might still involve impressive bodies of psychological or dramatic knowledge. Socrates is quite willing to grant that poets and orators may be, in some sense, extremely capable and successful, but so far as knowledge goes, the choices seem to be: τέχνη or nothing. We can see this at \emph{Gorgias} 464c--465a, where it's clear that saying rhetoric (or here ``flattery,'' of which rhetoric is a sub-type) is not a τέχνη is equivalent to saying that it just guesses. And then there is the poor Ion, who for all his apparent gifts is reduced by Socrates to a conduit for the divine. Socrates' view, it seems, is that if you have some kind of success, then it's because of τέχνη, or else merely a result of guessing or divine inspiration or madness. In this respect Aristotle, who tabulates various forms of knowledge,  is an obvious contrast.\footnote{For example at the start of \emph{NE} VI.3: ``ἔστω δὴ οἷς ἀληθεύει ἡ ψυχὴ τῷ καταφάναι ἢ ἀποφάναι, πέντε τὸν ἀριθμόν: ταῦτα δ᾽ ἐστὶ τέχνη ἐπιστήμη φρόνησις σοφία νοῦς.''} Another interesting contrast is with Isocrates's position in \e{Against the Sophists}. Isocrates also argues that the ῥήτορες do not possess a \techne, despite their invocation of such paradigmatic cases as knowing one's letters:\begin{squote} I am amazed when I see these men claiming students for them­selves; they fail to notice that they are using an ordered art (τεταγμένη \techne) as a model for a creative activity (ποιητικόν πρᾶγμα).\footnote{\e{Against the Sophists} 12, trans.\ \citetalias{mirhady2000isocrates}.}\end{squote}He maintains nevertheless that there is a role for teaching in public speaking, though a much more modest one than would be indicated by claiming a \techne, and his goal in the following passages is then to show that there is something to be taught here even though it doesn't rise to level of \techne---a task which he evidently doesn't consider to be a totally trivial one. We do not find anything like this kind of proposal being offered by Socrates---quite the opposite.\footnote{It is sometimes claimed that Socrates (or Plato) was gesturing in some way towards a different model of knowledge. So for example \citet[p.\ 6]{roochnik1996aaw} says that ``Plato rejects techne as a model of moral knowledge,'' and that he juxtaposes τέχνη and virtue precisely in order to show how poor a model τέχνη provides APPLIES TO SOCRATES?. BUT THAT'S WIERD.}


%He doesn't for example, concern himself with historical knowledge.\footnote{Cf. \citetalias{allen1989dpe} (\citeyear[p.\ 326]{allen1989dpe}), [NOTE PENNER AND OTHERS NOT GOING FAR ENOUGH].}

%[IF HAVEN'T ADDRESSED BEFORE, NEED TO NOTE THE OUR PERSPECTIVE VS THEIR THING; ADRESSED IN PREV SECTION PERHAPS?]



% wants to deny that oratory is a τέχνη but \emph{also} to claim that it's worth teaching. In this he is more tolerant than Socrates, and yet it is clear that he is \emph{struggling} to characterize this knowledge. Whereas contrast the various forms of knowledge Aristotle tabulates at the start of 


%That suggests another general point: Socrates doesn't concern himself with knowledge that can come \emph{only} in bits and pieces, such as historical knowledge. Now the lack of discussion of such knowledge doesn't prove much on its own, since Socrates' emphasis on moral knowledge will naturally make practical varieties of knowledge more prominent.\footnote{Though such τέχναι as arithmetic and measurement show up plenty as well, and \emph{take on} practical forms. }



%The \emph{Ion} furnishes another example we've touched on. At 537c--e Socrates is discussing the way different τέχναι have different subject matters.\begin{squote}Take these fingers: I know there are five of them, and you know the same thing about them that I do. Now suppose I ask you whether it's the same craft---that of arithmetic---that teaches you and me the same things, or whether it's two different ones. Of course you'd say it's the same one.\footnote{trans.\ \citetalias{woodruff1983two}, with `craft' substituted for his `profession.'}\end{squote} It's not surprising that τέχνη comes up here, because that's the topic, but the suggestion that it's in virtue of a τέχνη that we know something so simple as that we have five fingers is interesting---it certainly isn't the case that anyone who can count to five is a mathematician. So Socrates seems quite prepared to subsume very mundane sorts of knowledge under τέχνη.

Finally, all of this fits with and accounts for the striking fact about the dialogues with which we began: as Callicles says, Socrates simply doesn't let up on the \techne-talk. In prompting Euthyphro to tell him what piety is, Socrates invokes the measure-weight-number triad which constitutes the definitive characteristic of technical precision. He thinks that the notion of \sophrosune\ as self-knowledge may best be elucidated by comparison with carpentry and arithmetic. He thinks that the baker deserves scorn because he isn't a doctor. He finds it astonishing that there is no equivalent of animal-husbandry for humans. If the standard to which Socrates holds claim to knowledge is the standard set by his conception of \techne, then the simple and obvious explanation for this is the one I have proposed: \techne\ is what knowledge is for Socrates.


%STILL SO EVEN IF HE THINKS THERE ARE PROBLEMS

%euthyphro and prot invoking the triad for moral case---and lots of other techne invocations besides for example Apology on training for children---there is an obvious explanation for this


%PERHAPS HERE IS THE PLACE TO ADDRESS THE PROBLEMS?


%But the really pressing case is \emph{virtue}. Besides the general problem about inquiry, there is here the problem that there seem to be no authentic teachers at all---a point emphasized in the \emph{Meno}.\footnote{Unless we are to understand that Socrates himself is an exceptional case.} Socrates was perhaps prepared to accept that no moral τέχνη was attainable, and to satisfy himself with going about on his divine mission, showing that anyone who claimed such a τέχνη was a fraud (\emph{Apology} 23b). But Plato, as the \emph{Republic} shows, evidently had higher hopes. The difficulties gave him a reason to figure out a new view, and \emph{not} a reason to stop being interested in expertise. This meant, among other things, an emphasis on the \emph{a priori}, a corresponding shift in how the objects of τέχναι are understood metaphysically (i.e.\ as Forms), and thus also a greater emphasis on τέχναι like geometry which are more readily interpretable in those terms. But that's the start of another story.




% [[LACHES 198D--199D AN EXAMPLE; ALSO MAYBE ION'S FINGERS IN PIES, NOW THAT I CUT IT ABOVE]]

%I don't know that such examples could be considered decisive. If poets and orators were claiming a special ability as educators, then perhaps it was appropriate to hold them to high standards. And perhaps the \emph{Protagoras} and \emph{Charmides} examples could be put down to Socrates' penchant for appealing to τέχναι whenever virtue is in question. But on the whole I think that there is a good case here for saying that, for Socrates, knowledge \emph{is} τέχνη. To be sure, Socrates does claim knowledge, or attribute it to others, in other contexts. And Socrates cannot be said to have distinguished craft knowledge from any other sort of knowledge. \emph{Euthyphro} 6e, for example, seems to have propositional knowledge, acquaintance, and τέχνη all jumbled together. But Socrates' thought tends strongly in the direction of craft knowledge, and this is really no surprise, since it's the only sort of knowledge of which he has any definite and developed conception.







%The conception of knowledge I have described is easy enough to overlook. For us it's an unfamiliar conception, developed in the context of unfamiliar debates. It's a problematic conception of knowledge as well, since the treatment of virtue as τέχνη creates difficulties made plain even in the dialogues themselves. The role of τέχνη is further obscured by the room it makes for propositional knowledge, by the fact that Socrates' conception of knowledge doesn't entirely constrain his epistemological language, and by the fact that Socratic method is more prominent than Socratic theory.



%This fact has sometimes led scholars to underplay the importance of τέχνη even in Socrates' conception of knowledge, since Socrates plainly thinks that virtue is \emph{some} variety of knowledge.%---hence one might suppose that [techne] is not the only variety of knowledge for Socrates, nor even the most important variety.





%ABBREVIATIO NS
\begin{comment}
\sectio{Abbreviations}
\e{DA}		\e{De Arte}
\e{VM}		On Ancient Medicine
\e{Acute}		Regimen in Acute Diseases
\e{Prog.}		Prognosis
\end{comment}

\bibliographystyle{apalike}
\bibliography{bibliography}

\end{document}