% !TEX TS-program = xelatex
% !TEX encoding = UTF-8

\documentclass[11pt]{amsart}

\usepackage{ge_custom}

 
\author{Brendan Ritchie\\
bhritchie@me.com}
\title{`Good' and Evil} 

\begin{document}

\maketitle

\begin{comment}

\singlespacing

\begin{flushright}
Brendan Ritchie\\
britchie@fas.harvard.edu
\end{flushright}

\begin{center}
\emph{`Good' and Evil---draft of \today}


\end{center}

\doublespacing

\end{comment}



%For presentation: I should make the point about two senses of God's plans - this secures the sense in which all is accoording to plan and the sense in which things are not quite nicely.


%people who have given me comments on this: Tristram McPherson, Jon Litland, Michael Kenneally, Michael Jordan, Dan Turello, Japa Pallikkathayil

%Many comments and notes at the end of the document.


\section{From value theory to the problem of evil}

%\textbf{I. Introduction: from value theory to the problem of evil}

%have to decide if I want that third footnote...

Peter Geach argued that `good' is a ``logically attributive adjective,'' by which he meant that anything good is good \emph{as} something: a thing cannot be \emph{simply} good.\footnote{Peter Geach, ``Good and Evil,'' \emph{Analysis} 17 (1956)} Geach's idea has needed refinement.\footnote{See, for example, J. J. Thomson, ``On Some Ways in Which a Thing Can Be Good.'' \emph{Social Philosophy and Policy} 9 (1992).} My own way of articulating the idea is that goodness occurs in \emph{functional contexts:} evaluative language operates against a conception of how things are supposed to be work or what they are for. I will try to explain what I mean by this shortly. But my principal interest here is in how this idea sheds light on the problem of evil. Taking this approach to value, we might interpret evil in terms of human dysfunction. Then we should observe that there is a form of dysfunction which stems from incompletion. Such incompletion, I will argue, does no discredit to a designer. So if evil is a matter of incompletion---if we ourselves are imperfect \emph{because} incomplete---then we cannot with justice suppose our Designer to lack power or any interest in us.


%But Geach is right that it does not make sense to speak of goodness in just any context, or in none whatever. And what we say of something in calling it good varies from case to case. The principal implication is that calling something good is a way of describing it---I need imply no practical interest in what I say to be good, and that something is good may entail or be entailed by perfectly ordinary facts.

%It is doubtful, indeed, whether there is any useful \emph{grammatical} characterization of what Geach wanted to say.

%Remember (I was confusing myself about this after Doug said something) that the last sentence is meant to the a gloss on the second-last sentence.

%\footnote{An wonderful recent paper on the problem of evil by Cristopher Coope also starts with Geach but goes in a quite a different direction than mine. This is partly because Coope takes the grammatical idea more seriously than I do. ``Good-Bye to the Problem of Evil, Hello to the Problem of Veracity,'' \emph{Religious Studies} 37 (2001)}

\section{Function and goodness}

%\textbf{II. Function and goodness}

In central cases, what we say of something in saying that it is good depends on what we may refer to as standards internal to it. Take an artifact such as a knife. A knife is made to cut. A good knife (a knife that is good \emph{as} a knife) is therefore one well suited to cutting. A knife can be merely instrumentally good as well---a kitchen knife could do fine duty as a weapon. Human purposes or desires are not required for the existence of internal standards. When we talk about the good of plants and animals we draw on a conception of the proper functioning of these organisms or their parts.

There are also derivative forms of goodness. There is such a thing as good complexion, for example, and complexions may or may not themselves be functional, but good complexion, in the basic case, betrays good health, and that brings us to a functional context.

Where no functional context exists (or where we see no functional context), it is unintelligible to speak of goodness. I would not know what to make of the question whether the number three is good; neither could I say whether Saturn is good, unless it were specified that we were after something to colonize, or mine, or investigate scientifically.

Some cases are difficult, particularly as we turn to moral philosophy. What makes something a good thing to do, morally speaking? Could there be standards internal to action? Or is the goodness of a person conceptually prior to the goodness of action? Or should what is good \emph{for} people be of primary interest? It is no part of my view that answers to such questions must always be obvious, though if my view is assumed, then I believe we ought to approach moral theory along Aristotelian lines---I would hold, in particular, that a conception of what it is to be a good person is parasitic on a conception of what a person \emph{is} or \emph{is for}.\footnote{But that is not inevitable. The Geachean view about value might also be amenable to the Kantian view that action has, just as such, aims and therefore standards internal to it. Action aspires to be action that satisfies those aims; good action is therefore action that achieves them. See, for example, Christine Korsgaard, ``Self-Constitution in the Ethics of Plato and Kant,'' \emph{The Journal of Ethics,} vol. 3 no. 1 (March 1999).}

%And what if anything humans are for I take to be an historical question.


%The important standards in the case of moral philosophy would then be the standards internal to the human kind. 

Start with a suggestion from Michael Thompson's ``The Representation of Life.''\footnote{in Hursthouse, Lawrence, and Quinn eds., \emph{Virtues and Reasons,} Oxford University Press (1995)} This is that the possibility of describing a given organism as having arms rather than, say, deformed wings depends on a conception of the form of life of which that organism partakes. Nothing else could settle what that stuff at its sides are. Nothing could settle the question, for example, in the case of a creature which, resembling a man, is only a freak amalgamation of unlikely quantum happenings.

Next, our conception of a form of life finds canonical expression in irreducible, non-statistical generalizations which Thompson calls ``Aristotelian categoricals.'' An example is ``cats have four legs.'' This is true although not all cats have four legs, and could be true even if by a series of traumatic events most cats had some other number of legs. For Thompson, these generalities constitute the primitive form of our understanding of life. And they further constitute the standards of evaluation for organisms: if cats have four legs and this cat has only three, then it is defective.

This will not quite do---Thompson's account makes no room, for example, for the possibility that in failing to satisfy some generality, an individual might be not defective but exceptional. Neither will attention to the grammatical form of the Aristotelian categorical distinguish characterizations of human life as such from characterizations of a given form of social life. Philippa Foot offers ``humans establish rules of conduct and recognize rights'' as an Aristotelian categorical, but why shouldn't the following serve just as well: ``when rations are meagre, the Karamojong give the available food to their girls, who can be traded for cattle [thereby leaving the boys to starve]''?\footnote{\emph{Natural Goodness,} p. 51; the quote about the Karamojong is borrowed from David Lamb's \emph{The Africans,} Vintage Books (1987), p. 81. It is inessential to the example that the second description mentions only the Karamojong---it is an historical accident that they and not others behave this way; by a similar accident they might have been the only humans to respect rights. And yet respecting rights, and not starving boys, might be proper to human life.} I think we should agree that what constitutes an organism's thriving depends on what it \emph{is,} but we should locate the relevant descriptions within an historical context; we should draw them from a conception of how an organism is \emph{designed} to operate. We could try to understand this in evolutionary terms, but in the present context we may appeal to a conception of what it is that God made us for---of how he intended for us to live. What it is to \emph{be} a human, then, is a question about what a human is \emph{for,} and that determines what it is to succeed as a human. A good life, in turn, is a life in accord with the virtues which allow for such success, and supplied with the materials for their exercise.\footnote{This follows Aristotle. A good life is not merely \emph{success,} since certain functions might be as easily carried out in a coma, or even in death, as in action, and our concern is a practical one. Attention to the conceptual structure here also provides resources for responding to certain objections such as ``what if we were designed to suffer?'' But this is complicated and I will pursue the matter no further here.}

%why a practical matter? well, look, when our question is what it is to lie well, we are asking something that arises in our case - not in the case of a bicycle, say. (Nice example might be one of those sentimental stories about some toy or tool that doesn't get used... trying to think of one...

This has been a very brief summary of the value theory I favour, without any serious attempt at argument.\footnote{I have passed over, or at least not addressed at all directly, many important questions like ``why think we have a function?'' ``why should we care whether we have a function?'' and ``what if it turned out we had some really weird function?'' (But see n. 6.) I flag these questions only to avoid pretending that I've answered them.} Its distance from familiar metaethical theories will be plain; it holds, for example, that the infamous open-question argument merely trades on a confusion about what sort of property goodness is. I do hope the approach might be reasonably amenable to Christians in particular. Perhaps there is no great consensus on metaethical matters among Christian philosophers, but many of them have found it natural to work within an Aristotelian framework, and these ideas should not seem strange to Christians: that humans are artifacts; that what we are is therefore importantly a matter of our history, which has been partially revealed to us; that we cannot live well apart from God's plans.

\section{The nature of evil}

%\textbf{III. The nature of evil}

Rocks make bad cushions, my father takes a perverse delight in awful puns, Budweiser is a bad beer, un-oiled motorcycles run poorly. The world is full of these and other sorts of badness, but they are not problems \emph{with} the world, and no one could think to wonder how God could allow them. But the presence of \emph{evil} might be a problem with the world that we could lay at God's feet. How is evil to be understood in terms of the present theory, and will evil so understood pose a challenge to belief in a benevolent and almighty God?

I suppose that evil is badness in certain spheres. In particular, what is bad for humans, or what manifests harm already done to humans, is evil. Natural evils, then, are natural events that damage or destroy lives. Evil thoughts are the thoughts of a damaged person. Evil actions are actions doing harm or reflecting evil thoughts.

On the understanding of the human good which I elucidated in the previous section, what I have just said is that evils are things that prevent us from living out our proper aim, or that reflect our so failing to live it out. Natural evils obstruct our proper realization. Evil thoughts are the thoughts of imperfect creatures, insofar as they are imperfect or corrupt. Evil actions reflect such thoughts, or make perfection harder or impossible for ourselves or others.

If that is reasonable, then I suggest that the question ``why is there evil'' is equivalent to the question ``why aren't we perfect?''\footnote{I shall have nothing to say specifically about animals in this paper. What I say about pain below is relevant, though, and will probably reveal the fact that I do not think the animal case particularly interesting.}

\section{Imperfection and incompletion}

%\textbf{IV. Imperfection and incompletion}


%``We say that each thing's nature�for example, that of a human being, a horse, or a household�is the character it has when its coming-into-being has been completed.'' - Aristotle, Politics I.2

More than one interesting answer might be offered in response to the question just posed. The one I want to suggest is the following: we aren't perfect because we aren't finished yet.

What do I mean by saying that we are not finished? Well, John tells us that what exactly we shall be has not yet been revealed.\footnote{1 John 3:2} If John doesn't know what a completed human looks like then I certainly don't. But it seems we're not it (though we've been given instruction for getting there). Paul also tells us as much: he compares us to buildings, laid on the foundation of Jesus Christ, and expresses confidence that the One who began that work will bring it to completion; even as our outer nature wastes away we do not lose heart, because we know our inner nature is being renewed---that we are transformed into the Lord's image by stages.\footnote{1 Corinthians 3:9-15, Philippians 1:6, 2 Corinthians 4:16-17.} The Christian view, then, is that we are are extraordinary artifacts, capable of contributing to (or even thwarting) our own formation---and this formation is not yet complete.%\footnote{I do not mean to say that evil is always just a matter of this imperfection of ours. Murder is evil, but we are not, just in virtue of being imperfect, murderers. Not all of us murder, after all, and Jesus could call us to a perfection suitable to our current state (Matthew 5:48). Such evils are, I suggest, nevertheless secondary to---explained by---our incompleteness. As with children, some of our weaknesses are precisely a matter of our being less than fully developed.}

%Possible objection: maybe the imperfection Paul is talking about is just precisely our sinfulness, so that after all has to be the thing explained first. But I think I can reject that - Jesus says there is no marriage in heaven etc, and lots of these passages clearly suggest that we will be something quite other...

Now the idea I want to exploit is that it shows no lack of skill or care in a designer that his creation is not finished all at once; no incompetence on the part of the sculptor is revealed by the fact that he must chip away at a stone before it is a polished statue. God neither needs nor lacks time, but the \emph{time} involved in the creative process is not the point: the thing to see is that there are two different ways in which an artifact can be imperfect, holding the imperfection constant.

Suppose I have a tree fort out back. It is not in completely satisfactory shape, because when it rains water runs through the slats in the roof. The roof requires a tarpaulin. Now it might be that the fort once had such a tarpaulin and now lacks it, because I am not very good at affixing tarpaulins and the wind took it, or because I've been neglecting the fort altogether for some time. However, it might be that everything is on schedule---I'm just not finished yet. It's right there on the calendar. Day five (today): roof slats. Day six: tarpaulin. Day seven: relax in new tree fort.

There might be any number of reasons I planned to do things over six days---I'm busy with other things, or anyways I can only work so fast. I might just be savouring the creative process. What there cannot be here is any presumption of incompetence or neglect. No one can fairly ask of me, ``what went wrong?'' So long as there are no missteps in the process of construction, an artifact at a given stage just \emph{is} as it's supposed to be---that is, it is as it's supposed to be for now. So some imperfection indicates incompetence or negligence, in which case imperfection constitutes shoddiness or decay. And then there is imperfection which indicates no such thing. And it will not always be possible to tell which sort we are looking at.

Of course a craftsman might fail to complete a project in the first place out of incompetence or lack of interest. Likewise, defect could be introduced quite willfully post-completion, and then need not represent any kind of authorial \emph{failing.} So, in one sense, the distinction is ahistorical, and is just a distinction between the defects which find their origin in sloth, incompetence, neglect, and decay, and the lackings which reflect deliberate policy. It seems to me also important however whether or not the artifact in question has ever been complete, so that the concern is also an historical one. If I keep pulling my fort apart and reworking it, that looks like tinkering: like I either don't know how to finish my project off or don't much care how it winds up. Similarly, lack of care will look like the right explanation if I allow the once-finished fort to disintegrate. This is going to be trouble in the case of human imperfection. If we were once perfect and are so no longer, how is this? If the destructive agency was our own, why should we, having been perfect, have engaged in such self-destruction?\footnote{There are, certainly, interesting stories about how this might have happened: my own favorite can be found in the first chapter of Tolkien's \emph{Silmarillion.}} If the destructive agent is God, he begins to look somewhat capricious. That, of course, is why I have proposed that we are imperfect because incomplete.

That is my whole solution. Let me run through it once more. We begin with a sense that something is wrong with the world. But if things are according to plan, then nothing is wrong with the world. Now it could be that nothing is \emph{wrong,} but that the plan still reveals something unflattering about God. If the so-called plan is just to put things together and pull them apart again then we are dealing only with a divine tinkerer. If the plan was just to set everything going and then let it wind down, we have only the deists' god. But that is not how things in fact are: God has manifested concern for us and has shown no inconstancy of purpose.\footnote{This is just an appeal to Christian dogma, of course.}

There might seem to be a third worry, though. My tree fort could come out just as intended, and yet be a pretty cruddy tree fort. I do not set the standards here: for all that my fort matches my blueprints, I will have made a poor one if it is carried away by the slightest breeze. So couldn't we allow that we are proceeding according to God's plans, and still wonder at the wisdom of those plans? Perhaps. Keeping in mind the present theory of value, there is the difficulty of knowing what criteria we should be appealing to. We may not simply help ourselves to such. I suppose that a conception of the goodness of a world must draw principally on a conception of certain kinds of evil or damage \emph{in} that world, and this has already been our topic.

And yet there must be \emph{some} criteria for evaluating a world as a whole. God made the world and saw that it was good, and I at least want to agree with this. So what does it mean? Mainly, I think, that our world shows exquisite workmanship.\footnote{A sunset can be beautiful but not good; a painting of a sunset can be both. The difference is the technical aspect that is involved in painting. To see the craft our world displays, compare it to the one described by C. S. Lewis in ``The Shoddy Lands.''} But perhaps not only that the world displays great craft but that it is in some further sense \emph{beautiful}---worth the Artisan's time. I confess I have no account of aesthetic value. But I wonder if anyone has a pretty definite feeling that an omnipotent being should have made something \emph{nicer} than our world? And even if our complaint were that \emph{we} are shoddily done, this will be an unfair charge if my hypothesis is correct: I know of no serious complaint about what we \emph{will} be.


%I could conceivable start a new section here... not sure if it would be the most natural place... that does seem to be the end of the basic expostion...

Still it will be natural to worry that the problem of evil has been pushed back. Why does God create this way? Why not make humans whole and complete from the start, and skip over the period of imperfection and suffering? Do I think that the good of self-creation somehow outweighs the evil that is suffering? That would after all be a familiar sort of approach to the problem. But I am not saying that. ``Why would God create this way?'' may be an interesting question, but it calls for no \emph{excuse,} because the imperfections in question here are only of the work-in-progress type. Compare children. God could have created---perhaps has elsewhere created---intelligent creatures which reproduce by division. Such creatures would avoid some of our trials and errors. And yet all is in order when a human being is a child.

%That might seem to beg the question - if childhood were more painful, say? well, I dunno, some childhoods are extremely painful, also precisely because of the weakness of their position - think of Robert Harris. But that doesn't show that childhood is a problem.

Basically the same response can be made to another worry: why would God allow us to come to completion along paths like the prodigal son's, or indeed to refuse or fail to complete ourselves at all? Couldn't he orchestrate matters such that we complete ourselves without the errors and rebellion and regression? Must I here appeal to free will, so falling back on another standard response to the problem of evil? To be sure, if we are not free, then perhaps the fact that some of us are never completed would have to be attributed to God: it would be He who had not finished the work. But again I need not claim, what would in any case be doubtful, that granting us freedom was a \emph{risk worth taking} or that the evils attendant thereupon are \emph{costs worth bearing,} because I do not believe that there is any matter of weighing up goods here. Hirut, being a child, makes a child's errors: that is not a price she pays for childhood but a fact of childhood.\footnote{In particular my view is not like that of John Hick or Irenaeus, though we share the idea that evils exist because humans are not yet what they are meant to be. For Hick, soul-making is \emph{worth} the suffering, and there is much abstract talk of ``realizing value.'' Part of Irenaeus' line of thought is similar; he suggests what Tolkien has put so well, that without the present darkness ``we should not know, or so much love, what we do love.'' I make no appeal to a balance of good over evil, or to any good that evil brings. Irenaeus also defends the gradual creation of man by appeal to the claim that humans are ``too new'' to be perfect. But I do not think the creative process that has been employed needs to be defended by appeal to creative constraints. It is the Creator's to do as He will.}

%I sthere enough room on my view to allow a creator to make creatures which he is then cruel to? Or will that amount to having made them for that, especially if that creator is omnipotent? Yes. Partly (I think this works, but not totally sure) because function and living well not the same thing. Second, because the issue is kinds. All bear the same standards/have the same good, barring specific intevention - or does specific intervention idea just collapse given omnipotence? Just nothing to say about what counts as going well for a single thing with no kind. (might depend on details.) Still maybe leaves question: could He be cruel to the whole group? ANyway, think I needn't include this, but may need to ponder a bit more.

But all this may seem an overly abstract approach to the problem of evil: the problem of evil is a problem about---among other things---\emph{suffering.} Suffering is no laughing matter, and it is likewise a serious thing that we are imperfect and that we refuse to be obedient to God's will for us. But, to make a rather blunt appeal to my value theory here, nothing is \emph{just plain} good or bad. In particular, pain is not just plain bad. Pain, like pleasure, plays an important biological role, and its sheer visceral awfulness is required by that role---the \emph{inability} to feel pain is after all a serious medical problem.\footnote{CNN once reported on a girl with a congenital incapacity to feel pain: ``Some people would say that's a good thing. But no, it's not,'' says Tara Blocker, Ashlyn's mother. ``Pain's there for a reason. It lets your body know something's wrong and it needs to be fixed. I'd give anything for her to feel pain.'' 
I found this remarkable statement on September 16th, 2006, at http://www.general-anaesthesia.com/congenital-insensitivity/nopain.html.} Suffering is appropriate to damaged and imperfect creatures, and to the extent that we are damaged and suffer thereby, things are as they should be. It is where pain does not respond properly to damage that it is bad, just as pleasure is bad when it is taken in what is perverse. The special connection of pain to badness is not that it is somehow a primitive evil, but that, in the typical case, it \emph{registers} evils.

%make sure above footnote isn't getting split badly.


\section{Conclusion: some implications}

%\textbf{V. Conclusion: some implications}

Some final observations. The first regards natural evils, about which I may seem to have said nothing. On the Christian view, immortality goes with perfection. There must then be ways for imperfect man to die. That means natural evils. Not necessarily, maybe. Perhaps God could arrange the world so that, like Tolkien's elves, we could only die unnatural deaths at each other's hands. But then we might agree with Tolkien that natural death is a generous gift for a stunted creature.\footnote{Natural evils result in both more and less than death, of course. It is indeed a feature of our present life that we are part of a natural world which contains, besides death, destruction and pain and decay. But then this is just a more general feature of what it is for a human to be incomplete.}

The second observation is that my proposal fits happily with our evolutionary origins. This greatly pleases me, since evolution has been thought, also by my earlier self, to make the problem of evil \emph{more} difficult. It fits rather less well with the view that there was any such historical event as the fall. That, I believe, is satisfactory---true both to our scientific understanding of the world and to fair biblical interpretation. I believe the Genesis account is true, though not that it is true \emph{history.} It tells us that God created us and everything around us, that God cares about us and requires our obedience, and that things are in their imperfect state because of our failure to be obedient. To all of this my view is amenable.

%That last thing might be too strong....well I could say perhaps that death wasn't inevitable? nah, tha'll just run afaul of evo again...

Finally, I want to go back to Geach, who could be a tough-minded man.\begin{quote}\small{Resurrection is in any case a merely gratuitous gift of God. A race of rational creatures to whom this gratuitous gift had not been given might well be mortal and perish at death; God would not thereby default on any implied promise to them or in any way be acting unjustly.}\footnote{\emph{Providence and Evil,} Cambridge University Press (1977), p. 127}\end{quote}Paul says something similar in a difficult passage which however bothers me less than it once did:\begin{quote}\small{For the Scripture says to Pharaoh, ``For this very purpose I have raised you up, that I might show my power in you, and that my name might be proclaimed in all the earth.''  So then he has mercy on whomever he wills, and he hardens whomever he wills. You will say to me then, ``Why does he still find fault? For who can resist his will?'' But who are you, O man, to answer back to God? Will what is molded say to its molder, ``Why have you made me like this?'' Has the potter no right over the clay, to make out of the same lump one vessel for honored use and another for dishonorable use?\footnote{Romans 9: 17-21}}\end{quote} If God had made us to be no more than we now are, we should have no cause for complaint. Paul and Geach are reminding us that it is a mistake about value to suppose we ought to live, or could live better by living, some life other than the one God has for us, or to think that God could err in making us. So maybe we should turn our problem on its head. If we feel ourselves to have grounds for complaint, this suggests we think that we were made to live in some other way than we do. On what grounds should we think such a thing? \emph{My} grounds are that God promised it. So perhaps Christians have the resources not only to turn the complaint but to make it intelligible.

%perhaps not great interpretation  of Paul there - he definitely seems to be saying something stronger which I'm still not sure I like...



%\theendnotes




%This last thought goes like this: you want not to die? Why shouldn't you die? You just want to always have enough to eat or not to grow ill? How do you intend to die? Only silently and painlessly and at an expected time? Should there be no grieving? Come on...




%There's that song "Men of Constant Sorrow" with that great question "why must we die?" If some remarks about entropy aren't a satisfactory answer here, that is already a suggestion that you think we are for something more...



%yet another possible topic - plans. There's a sense in which things are not according to God's plans - seems clear he is angry and grieve's etc - tells em not to do stuff and I do it. But another sense in which God cannot be thwarted. My view seems to me to capture this idea well. All is according to plan in that we are presently in a an imperfect state - we can use it as we will and that does not show that things are not as they should be in my sense, but of course they are not as they should be in the other sense - we should be getting somewhere and indeed God has instructed us to come along.



%I've got a short footnote about Coope at the start now.

%Since writing this paper, I have become aware of a very interesting paper on the problem of evil by Cristopher Coope which also sets out from Geach's observations about the garmmar of `good' (Cristopher Miles Coope, ``Good-Bye to the Problem of Evil, Hello to the Problem of Veracity,'' \emph{Religious Studies} 37 (2001)). Despite the similar starting point, our papers go in quite different directions, but it may be worth saying something briefly about why and how they do so. Coope says that what it is for God to be good might be quite other than what it is for us to be good, and urges that there is no reason to think that God's goodness requires Him to go about preventing so-called "evil states of affairs." This is intended to resolve the problem of evil. The problem of veracity is constituted by a worry that the kind of goodness God has might also not involve truth-telling, a difficulty in particular for a revealed religion. Coope's approach to the language of goodness draws very heavily on Geach's, so that the idea is that we must always be casting about for a particular substantive or noun phrase to which 'good' attaches (Coope suggests `intelligent agent'). I think it is a mistake to take the grammatical idea so seriously, and I have eschewed employing it here. While it is a complicated question what it means to speak of God's goodness, I think the bulk of what we mean by it is that God is \emph{good to us.} (Compare Moloch or the bloodthirsty mesoamerican gods.) So I pursue a different approach, for which I do not believe the problem of veracity arises.}

%notice that Coope thinks benevolence doesn't help with the veracity stuff. that might be so. In general I'll have to carefulin any case with what I say here since I don't want to attribute goodness in some sense that's going to look like everything demanded of us is somehow demanded of God.






 
 
 
 
%One problem I'm obviously going to have to work out some kind of answer to (knew it anyway, raised pointedly by Jon and Tris: what about some creator who just makes creatures to suffer. Indeed we might be those creatures. Tris adds: and doesn't in some sense God's goodness just fall out his power? Or what's supposed to be meant by "God is good"?

%one thing Tris is thinking is isn't whatever a powerful creatures makes going to be good. I want to say there might be like aesthetic constraints here of some kind. (God was thinking something non-vacuous when he looked at the world and saw that it was good. But this is shaky ground for me because I really don't know what to say about aesthetics. But at least I have something to wave at here...

%So my take on God's goodness: he is good to us, basically. Does the thing Tris was worried about still arise? Well, might think he get sit pretty easily: what's good for us falls out of what we were made for. So does God get to be good on my view just for making us whatever he want and not obstructing us in it? In general I want to say that suffering is a constraint here: suffering means something is going wrong. If our creator doesn't care about this or how much stuff goes wrong etc he isn't good to us. But this brings us back to: what is he just made us to live like this? (basically what I said about animals!) Or what if he made us to positively suffer in the first place?

%Possible responses:

%Bite the bullet?? This is the good life??

%Suffering just is the mark of something going wrong. Somehow incoherent to suppose that our good could consist in it. This could be worth something: how would I make it fly?

%Constraints: could get something out of evolution, perhaps. Suffering plays this role such that it will rewrite our programming, as it were, if the point was that... but then I might have to worry about nat selection swamping all design anyways...

%constraints from reasons for action

%or maybe I just need to stick closer to Aristotle's pattern. There's a goal. There are virtues. There are goods. Then there is the excellent functioning. That is what A calls the good in the case of humans. Could suffering be the goal in the right sense? How would we have to alter it, and would it still be threatening? Still, what about some other creature meant to suffer? Perhaps anything made to suffer would have to be an active sort of thing, so same thing would work if indeed it works at all. ***THIS WOULD BE THE MOST ELEGANT SOLUTION AND MIGHT ALSO BE THE MOST PROMISING***



%What if our function is something very peculiar? If we were created as a deforestation agent my intelligent aliens, is the good life for all of us then the life of the lumberjack? Or suppose some demonic god creates a being meant only to suffer---is suffering then the good of those beings? We could even be those beings.

%To answer this objection we have to characterize the conceptual relationship between the human function and the human good more carefully.

%Aristotle gives us a four part structure for thinking about the good life. First, there is the function of man. Then there are the excellences of a man, the traits or virtues which allow him to carry out his function. Third, there are goods for man, which sustain him in this pursuit. Finally, there is the excellent functioning of the man. This is what Aristotle identifies with the good of man: the actual living out of a life in accord with his function.

%Our inquiry is about what it is to live well as a human being. Suppose our function were simply to provide rich fertilizer for plants on death. Since we all die, there is really no way to fail in this. And there may be nothing to say in this case about what constitutes a good life. In such a case we are left with no practical questions. A comatose man will do fine here.

%Living well means living with the virtues, or some approximation to them. These virtues are qualities that make non-accidental that you manage to live in accord with your function. There might be more than one set of qualities that would do. (Suppose you have none of these virtues, but are incredibly lucky, in the sense that your lack of virtue happens not to cost you the realization of your function. Are you living a good life? We should say again that you are living a successful life, but that this cannot be an example of a life well lived---again a comatose man might be successful in this sense.) SImilarly you need the goods---otherwise not going to be able to live this stuff out very well. But even having carrying out the function, having the virtues and the goods is not enough to live well---all these things can again be satisfied by the comatose man.

%Aside: isn't it a crazy result to think that there might be no answer as to what the good life for a person is? And that we then have no responsibilities, either? (I do recognize this as an implication of my view.) GOT TO FINISH THIS OFF

%Suppose then that humans are made to suffer a great deal. Success qua human would mean suffering a great deal. But what would a 'good' life mean here? One awkwardness is while it is perhaps possibkle to undertsand what it would be to conduct a life rationally�especially if we read: to conduct a life of rational inquiry�it is hard to understand what it would be to lea



%Try this again more concisely:

%Suppose some creator makes a creature intended only to suffer pain. Such creatures might even look just like us. Would suffering then be the good for these creatures? I confess I find this a difficult case to think about, since the point of pain seems to be to signal dysfunction, but even leaving that aside, the purported conclusion does not follow. Suppose we were made to provide rich fertilizer upon death. Since every mortal creature will eventually accomplish this end, it establishes no criteria of goodness: on my view there would be nothing to say in this case about what it is to live a good life. We need, then, a more precise characterization the conceptual relationship between the human function and the human good: the relationship is not simply identity. So it is an implication of my view, which I accept, that if we were made to suffer we will be successful qua humans if we suffer. But that does not yet settle that the good life for a human is a life of suffering. But it is my view that criteria for the good life fall out of criteria for success. Aristotle gives us a four part structure for thinking about how this works. First, there is the function of man. Then there are the excellences of a man, the traits or virtues which allow him to carry out his function. Third, there are goods for man, which sustain him in this pursuit. Finally, there is the excellent functioning of the man. This is what Aristotle identifies with the good of man: the actual living out of a life in accord with the virtues. But if the good life is something active, then it becomes more difficult to see how the present objection is supposed to work; pain and suffering, after all, drive one away from action. What qualities would aid one in the pursuit of a life of suffering? A simple answer would be: a capacity to enjoy pain (some people are said to have this). But of course then it's less clear why this should look like a bad life. (But how about instead a kind of irrational compulsion to get into situations in which one will suffer? I'm not sure what to say about this one. Generally I'm thinking: what would the virtues of a sufferer be? Will it look totally unacceptable to say that a good  life could be a life in accord with those virtues? Like I said above I'm having trouble thinking about this clearly, so any suggestions for 'virtues' or further responses I might make would be very welcome.)

%Another illustration might be helpful. Suppose the point of human life is the propagation of genetic material (this is Richard Dawkins' view). That does not mean that the good life is one in which genes are propagated---that again, is only success, and a man might achieve it even comatose, if someone went to the trouble of extracting his sperm. The \emph{good life} might be pretty much what we would have expected even before the dawn of evolutionary theory. A good strategy, for humans, for propagating genes probably involves establishing stable communities, falling in love and getting married, caring for children, etc.

%But here's another difficult case, which is almost a case of beings who pursue pain. Tolkien's orcs love death and destruction, they have no respect even for each other, are full of bile and hatred, and so forth. I am prepared to say that the good life for orcs really is rapine and plunder and such (I will ignore the complication that the orcs are supposed to be twisted elves---here I imagine them as an original creation). So the good of orcs is not the good of men. Neither have orcs and men any common ground---they kill each other as they can, and I would be happy to suppose that they have no reason not to do so. Still, we say that they are evil. Does this make sense on my view? Well, when we call a dog a good dog, we don't mean that it is good qua dog, but that it is obedient and so forth. We might mean by calling orcs evil that they are willful creatures whose wills are opposed to our own. But perhaps that could not be satisfactory. In some deeper way orcs seem to be on the side of evil---they delight in death and destruction and tears. 

%I don't see my way clearly here. I am satisfied with the conceptual connection I have outlined between function and the good life. And even if orcs are in some sense bad despite being accurate realizations of their designer's intentions, that is compatible with my view about value as long as there are criteria for goodness of creation which I have so far not discussed. These might be something like aesthetic criteria, for example, which I want to acknowledge (see f. 12) but confess I do not understand. (Notice that this is not a concession to the idea that there might be goodness outside functional contexts---a sunset can be beautiful but is not good. A painting of a sunset can be both beautiful and good. It is the technical facets of the second case that makes the difference here. The point is just that a good painting may have to both manifest technical skill and have a subject matter that is in some sense worthwhile. It is this `worthwhileness' that I have no account of.) But if criteria of goodness are fixed by more than merely the creator's aims, then that may seem a threat to my response to the problem of evil, since we might wonder whether our world meets these criteria. I do not believe this is a genuine threat. First of all, I do not think that the problem of evil is ordinarily thought of as being really a matter of the aesthetic or quasi-aesthetic considerations now in question. But in any case the concern is not that our completed selves will be somehow unsatisfactory, and on my view that would have to be the complaint. There is no cause for criticism of the painter even on aesthetic grounds if the painting just isn't finished.





%little exchange with Japa. I should develop, think through the child case a bit.

%On Dec 5, 2006, at 2:48 PM, pallikk@fas.harvard.edu wrote:

%Hi Brendan,

%I'll have to think some more about the philosophy in the paper, but I have one theological question.  You claim that your view can accomodate the truth of the fall, although not the fall understood as a historical incident.  One might of thought that part of the point of the story of the fall is that we are responsible for our own imperfection - having been made good, we turned away from God.  On your story, since we are not completed, I wonder how our sin is really attributable to us.

%Best, 
%Japa

%Hey Japa,

%Yeah, this is one of a number of things I leave quite obscure in the paper (There's a cryptic remark in f. 17). I was thinking something like this. Children are imperfectly formed in mind and body and will, and that is no tragedy---they are just developing. On the other hand, their imperfection in that sense (the sense I was trying to exploit in the essay) leads them to make mistakes an adult wouldn't. But of course children play a role in their own development through their choices, and while we hope they make good ones and encourage them to do so, we often let them suffer the consequences of bad choices even if we could prevent those consequences. I'm thinking we're like that. Imperfect in the innocent sense, but because of that we tend to be imperfect in the not so innocent sense.

%Is that any good?

%Brendan

 
 
 
 %More Comments follow (Tris' email, Mike Kenneally's email, Mike Jordan's email)
 
 \begin{comment}
 
 Dear Brendan: 

I read and very much enjoyed your paper.  A few thoughts: 

I had a nebulous feeling that the organization could be improved.  I will read it again and try to be more specific tomorrow.

The main line of argument seemed to me to be quite compelling.  

My main worry, in the 'one man's modus ponens' spirit, is that this discussion suggests a severe worry for the semantics that you favor.  Consider:

There might be an all-powerful and knowing deity.  

That deity decides to make some stuff.  One of the things it decides to make is some sufferers.  Now, it turns out, that just as Mill suggested that there are higher and lower pleasures, so in order to *really* suffer, one has to have some rational capacities, attachments, plans, etc.  And the best kind of sufferer might thus not look structurally terribly unlike a human being.  So the deity constructs a world of humanesque sufferers, and subjects them to exquisite torment.

Now, these things have been designed to suffer.  They function best according to plan when their suffering is *really* exquisite.

So, these creatures are achieving the good of their kind when suffering exquisitely.  For all we know, we might be these creatures.  But a deity who created only sufferers and a world that enabled them to suffer perfectly seems terribly awful.  This seems to reveal that there has to be someting more to evaluation than this functional business most broadly conceived (as functionally, everything is perfect here).   

A further worry.  Suppose that one bites the bullet on the above (!).  What is the sense in which we are evaluating God?  The idea of the problem of evil is that God is supposed to be perfectly good.  But a good what?  It's not clear what makes one a good member of the kind deity, and we might think that deities cannot be artifacts.  Perhaps 'good qua creator'?  But this looks dangerously just like it falls out of being powerful (i.e. good qua creator would be: I want to create a thing that is F, and what I create is /perfectly/ F: if I couldn't do that, I wouldn't be all powerful, would I?).  

I thus wonder if the stronger thing that falls out of the functional account of goodness is that the problem of evil doesn't make any sense, because the disputed thesis that God is good turns out simply to be unstatable.

I'm genuinely curious if the first thought experiment gives you pause about the functional account, and if not, what your take is on it.

Again, though, a lovely paper, a novel idea, and a good thing to run at a conference, due to its slightly shocking feel.

All the best,

Tristram



These are just some quick notes upon my first reading.  

First, I noticed two typographical errors.  On page 8: I believe a "were" should be "we're".  On page 9: the second sentence of the last paragraph seems to be lacking a verb. 

Second, I think more time is spent than is needed explaining your value theory.  I definitely think you should do so a bit where you do, but as it stands, that section seems to constitute the majority of the paper.  And since you admittedly are interested only in clarifying your position, and not defending it (which I think is perfectly fine), there's no need to devote so much space to it.  

Third, I think if you cut that down any, you'll have extra room to say more interesting things about imperfection/incompletion.  This section does seem a bit too compressed.  I think many of your readers will find it difficult to get their minds around the idea of an omnipotent creator whose creation is "incomplete."  After all, the central case of incomplete creation seems to be that of a finite creator.  Of course, your point about savouring the process of creation is a helpful one.  But it raises an at least prima facie worry in what one might call The God Case, viz. God's existence outside of time; how can he in any way be seen as savouring an intermediate stage in a time-extended process?  Now I take it this is the sort of question you don't want to have to answer.  But I still get the feeling it might help your case more--and make it seem less that you're simply pushing the problem of evil back (some of which pushing seems legitimate to me)--if you speculate on this aspect a bit more.  And it seems you could tell some sort of plausible theological story. 

To me, the word "incomplete" seems to imply imperfection.  Maybe that's just me.  It seems almost as though in some ways it would be more accurate to say about humanity at its current stage that we're merely in an intermediate stage in a process of metaphysical development, similar to the way in which a caterpillar is in an intermediate stage in its metamorphotic development.  It's possible that this line of thought is heretical, in which case I would not recommend adopting it.  I really don't know. 

Maybe none of that's very helpful.  I'll think about this more and see if I can make my own thoughts any more lucid.  

I don't know all that much about the philosophy of religion, but I don't know any other approach to the problem of evil along the lines of your value theory.  And it definitely seems to me a promising line of thought. 

Best,
Michael






	From: 	  mkjordan@uvic.ca
	Subject: 	RE: the problem of evil
	Date: 	November 19, 2006 11:59:54 PM EST
	To: 	  britchie@fas.harvard.edu

hi brendan,

how is everything going? what is this conference you were thinking of 
attending? anyways,
looks like an interesting paper. i will go through it in more detail as soon 
as i get some time. 
my intitial questions are as follows though,

1) have you modified your conception of God since we last talked? you only 
refer to an almightly and 
benevolent being. but this can be taken in a variety of senses from classical 
conceptions to less 
stringent or "grand" conceptions. obviously the various versions of the 
problem of evil will gain weight 
depending on which conception you are appealing to

2)it looks like you are jsut reponding to one particular version of the 
argument. primarily that of evil 
acts humans perform. (of course proponents will just push on you difffernt 
versions of the p of E)
while you do mention natural evil you dont have much to say on it. i gather 
this is because you take it 
that it is the entiretly of creation, human and non-human that is incomplete. 
but again the force of your 
argument  (esp in the latter half) is focused almost exclusively on human 
conduct. of course this 
intuitively appealing, however without really specifying in what way nature 
can be improved such that 
natural evil can be seen as being in some sense incomplete (excluding human 
conduct) will grant the 
propoent of the PofE a lot of playing room to respond

3) you mention that your conception is compatible with evolutionary theory. i 
wonder about the 
inclusion of this for the proponent of the p of e can say that the evidential 
role that evil acts play can be 
traced back to an evolutionary history such that it in no way indicates or can 
be passed of as 
incompleteness but rather propper function cashed out naturalistically.

4)going back to point 2. do you have anything to say about newer versions of 
the p of e that primariily 
discuss excessive evil and or divine hiddeness (checkout Swinburnes Existence 
of God ch 11. there 
seem to be some similarities with your paper, actually check it out anyways, 
he came up with asecond 
edition this year? and i wanted to know what you think about it.)













\end{comment}
 
 
 
 
  \end{document}
 
 
 
 
 
 
 
 