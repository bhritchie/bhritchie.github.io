\documentclass[11pt]{amsart}

\usepackage[polutonikogreek,english]{babel} \usepackage{verbatim}

\author{Brendan Ritchie} \title{Knowledge and Akrasia in the \emph{Protagoras}}

\begin{document}

\maketitle

%\marginpar{\begin{scriptsize}Delivered this version to M\&P workshop Nov. 05 2007. Notes from Q\&A in margins.\end{scriptsize}}

\section{Introduction}

Socrates is famous for denying that there is such a thing as weakness of the will, also known as akrasia: doing what you judge you ought not. Socrates also says that knowledge is strong, in what is obviously a practical sense. These views are related (but do not amount to the same thing), and the present essay is about how they are related. It is easy to read Socrates wrong on this matter because his conception of knowledge differs from ours. So I will wind up talking a bit about that conception in its own right. My goal will be to use that, the denial of akrasia, and the strength of knowledge thesis to illuminate each other.

\section{The Socratic denial of akrasia}

The \emph{Protagoras} contains the principal Socratic discussion of akrasia. Here we first find Socrates engaged in a discussion about the unity of virtue with the dialogue's eponymous sophist. That discussion is broken off in a rather abrupt fashion in order to consider hedonism, and then there is another shift and we are told that the \emph{hoi polloi} think that knowledge is not ``strong,'' that it does not ``govern'' action: that you can know what to do and not do it. It happens sometimes that something else ``overwhelms us'' and governs our actions---pleasure, pain, anger, love or fear, as it may be---and that knowledge is ``dragged about like a slave.'' Socrates and Protagoras disagree with the many. Knowledge \emph{is} authoritative. You cannot know how you ought to act and fail to so act.


%\marginpar{\begin{scriptsize}TS: this argument was laid out clearly - wonders at the end: is any of it left? I guess the answer is not really.\end{scriptsize}}


Socrates begins his refutation of the many by foisting hedonism upon them---the best action is the most pleasant action, or the action most productive of pleasure over the long haul. But this has implications for the popular conception of akrasia. The popular view was that, knowing what to do, you can be overwhelmed by some pleasure (presumably---we are not told explicitly---the other forces mentioned earlier are to be understood in terms of anticipation of pleasures and pains). But given hedonism, for some pleasure to overwhelm you is for some good to overwhelm you. This new good must not of course be greater than the good which you ought to pursue, or else ``being overwhelmed'' would not after all be an error. So being overwhelmed by good must mean choosing fewer goods over greater goods (255d-e). Again by hedonism, this means choosing lesser pleasures over greater pleasures. But---at least according to the rather crude hedonism Socrates obviously has in mind---there is nothing more to comparing the values of two sets of pleasures than weighing those pleasures against each other, and weighing is treated as a purely quantitative matter (256a). There is nothing more, then, to choosing between pleasures and pains---that is, between goods and evils---than choosing either a larger quantity or a smaller. Now, thinks Socrates, there is no longer anything here to be overwhelmed by: it is not, it seems, intelligible that a person would make a mistake here.\footnote{The problem here may not seem very sharp; exactly what the incoherence is supposed to be---and indeed exactly where the argument is completed---much exercises scholars. My feeling is that this is the argument, but that since Socrates' basic commitment is in any case to the rejection of \emph{akrasia} and to the strength of knowledge in any case, rather than to hedonism (which in fact he does not accept, I say: see below), Socrates (or rather Plato) just doesn't see the need to draw out the incoherence really sharply. I acknowledge that this is unsatisfying since it perhaps leaves us with less to sink our teeth into than we might have hoped.}

Socrates does briefly entertain the imagined suggestion that there is some difference between pleasures and pains which are near and those which are far which accounts for being overwhelmed. Perhaps, though aware that you ought to opt for some course of action which will result in great pleasures which will however be long delayed, lesser but temporally more proximate pleasures exercise an undue influence on your actions? Socrates still insists otherwise: it is unintelligible that one should voluntarily choose less over more, regardless of how far off the various pleasures and pains might be. But there \emph{is} a difference which distance makes, by analogy with physical distance. What is further off appears smaller than what is near. And such appearances may mislead. Here then seems to be the nearest thing to akrasia that Socrates is prepared to recognize, namely ignorance about what is good caused by the relative nearness or remoteness of various pleasures and pains. Pleasures and pains cannot exercise an influence which causes us to act wrong \emph{by our own lights,} but they can throw off our perception.



Now so far it is not clear why this has anything specifically to do with the strength or authority of \emph{knowledge,} which the many had questioned. What it is that makes \emph{knowledge,} as opposed to simply opinion or belief generally, strong? To say that ignorance misleads you is presumably to say that belief leads you; and indeed Socrates tells us that \begin{quote}no one voluntarily goes towards evils or what he thinks are evils, nor is it human nature, it seems, to be willing to go towards what one thinks are evils instead of goods.\footnote{358c-d [trans. Allen]}\end{quote} This indeed looks like the standard denial of akrasia: no one does anything but what he thinks best; the ``strength of knowledge'' would be a trivial result of that denial. In fact that is effectively how the \emph{Protagoras} is generally read.\footnote{For discussion see Terence Penner [cite]. Vlastos apparently recognizes that there is something important about knowledge in particular here but does not see what it is. [cite]} And that is indeed for us a natural way of thinking about things, since we (students of philosophy, anyways) are accustomed to thinking of knowing as a way of believing. Socrates does \emph{not,} as we shall later see, think of knowing in such a way.


But let us start with belief, since even Socrates' understanding of that may not be quite our own. For Socrates, it turns out, to believe that some course of action is best is largely worthless from the perspective of producing that action. That is because mere belief is unstable. The ``power of appearances,'' we are told, causes us ``to wander and to change back and forth, to accept and reject the same things in actions and in choices of large and small'' (ie. large and small pleasures; 356d). We do, as established, what we believe best---but we are easily mislead. The prospect of a donut \emph{now,} for example, may tend to \emph{produce} in us the thought that it would after all be best to eat said donut: certainly it would be enjoyable; a dollar is a trivial amount of money; there will be ample future opportunities for exercise; no one violation of my resolution to give up donuts can matter much---and so on. So donuts can cause positive beliefs about the pursuit of donuts: whatever my prior view about donuts, that view may be at risk on entering a donut-occupied room, or if I spot a billboard advertising donuts, \emph{if} that view was \emph{merely} belief. Of course immediately I put the thing in my mouth I may promptly think of my health and curse my weakness, but then it is too late. Generally, emotional states and needs can have this kind of influence: depression will make you think you are good at nothing, and whether people agree with the umpire generally depends on which team they are rooting for. (These examples get away from the influence simply of proximate pleasures and pains, but Socrates clearly has other emotional states in mind and I will shortly argue that we should not in any case take the hedonism here in the \emph{Protagoras} too seriously.)

So we may indeed always do what we think it best to do, but that is of little significance if any passing prospect of some good may reshape our beliefs at any moment: the appearance of some new good or the sudden thought of some old good may at any moment recast our conception of what is worth trading for what. There is no such thing, except perhaps by chance, as a temporally stable (mere) belief that a donut is worth roughly so much, fitness so much more or less, and so on.\footnote{Compare Terence Penner, [Apeiron 1996, AGP 1997]; although I follow Penner in insisting that we take the appeal to knowledge more seriously than is typically done, my exposition of the argument for the strength of knowledge differs substantially from Penner's (and has here been presented much less formally). For one thing he gives no special role to the apparent hedonism which is invoked, perhaps because he takes the `hedonism' here to be simply Socratic eudaimonism. I find that suggestion dubious. Protagoras clearly is put off by hedonism; but why should he be so put off if this `hedonism' is compatible with the view that the good life is a life of wisdom and righteousness? Or why couldn't Socrates have clarified his meaning? And the hedonistic thesis is also ascribed to the many---are we to suppose that Socrates thinks that the many accept the moral views he himself espouses in other dialogues? (Compare also Aristotle's attribution of hedonism to the \emph{hoi polloi} in \emph{Nicomachean Ethics} 1.5.) Most importantly, it seems to me, as reflected in my exposition, that hedonism secures the idea that goods are subject to the simple form of measurement which renders akrasia unintelligible. This simplicity seems to me lost if `hedonism' is simply eudaimonism. But no doubt that helps explain why Penner's exposition differs from my own.}

If that is our situation, then our good depends on being able to resist these fluxuations in our evaluative views---on being able to attain and hold to a correct picture of the importance of the various possible objects of our pursuit. And so, says Socrates, our ``salvation'' will be an ``art of measurement,'' which would \begin{quote} render the appearance ineffective: by making clear the truth, it would cause the soul to be at peace by abiding in the truth, and so save our life. (356d-e)\end{quote} Socrates clearly thinks that this is a basically mathematical art, and indeed he goes on to identify it with (the art of) arithmetic (357a). By the very fact that it is a kind of measurement that is the answer to fluctuating beliefs about the good, it is ``art and knowledge'' that is the solution. (Although Socrates promises a later consideration of just what this ``art and knowledge'' is---perhaps he would intend to qualify the identification with arithmetic, or perhaps to say how pleasures and pains are measured?---he takes up the topic again neither here nor in any other dialogue.) And that is the refutation of the \emph{hoi polloi:} we have now seen that knowledge is strong, because knowledge is precisely the answer to the difficulty of unstable evaluative beliefs.

\section{Interlude: the role of hedonism}

We will further elucidate this conception of knowledge, but in order to put knowledge in a proper light I must first ensure that hedonism, which played a very important part in the argument I described, does not steal too much of that light. Hedonism was certainly important in the argument as reconstructed above both for showing that one never does one what believes one ought not do, and also for showing that knowledge is strong. The fact that, hedonism being assumed, the options presented us in choice are so simple---just: a better balance of pleasure over pain or a worse---was employed by Socrates to make it seem absurd that anyone could ever choose what they thought they shouldn't. And then the putative ready-measurability of pleasures and pains was invoked to show that, knowing how to perform that measurement, we could reliably stay away from what is worse---the strength of knowledge. And that may make it seem as though the basic motivation for these classic Socratic positions is the supposed truth of hedonism. I think that this is not right and I think that the hedonism here obscures the actual motivation.

%\marginpar{\begin{scriptsize}GS: Hardly anyone besides Penner and Irwin thinks Socrates is a hedonist - fits too poorly with too much else.\end{scriptsize}}

It would be simplest to undermine the role of hedonism here if we could show that Socrates was not in fact a hedonist. So is he? On the one hand, Socrates \emph{does} put the doctrine forward himself; while he presumes it to be the view of the \emph{hoi polloi,} he also simply foists it upon a very reluctant Protagoras.\footnote{Interpreters who consider Socrates to be genuinely endorsing hedonism include T. H. Irwin, Terence Penner, and C. C. W. Taylor. [check, cite.]} However, his apparent hedonism here in the \emph{Protagoras} sits poorly with, for example, his invitation elsewhere to Crito to \begin{quote}consider very carefully whether you unite with me in agreeing that it can never be right to do injustice or return it, or to ward off the suffering of evil by doing it in return...\end{quote} And we might suppose the \emph{Crito}, given that dialogue's setting, to accurately represent the historical Socrates' views accurately if any do.\footnote{So for example R. E. Allen, p. 129.} And then in the \emph{Gorgias} Socrates agrees with one of his interlocutors, Polus, that all goodness is \emph{either} pleasure \emph{or} benefit, where the latter is, in context, clearly neither an empty category nor purely instrumental.\footnote{That is, \emph{hedone} and \emph{agathon} (the latter also here glossed as \emph{ophelimon}) are collected under \emph{to kalon}. [cite.]} And later in that same dialogue, in the course of his discussion with Callicles, Socrates appears to deny that pleasure is a form of goodness at all.\footnote{By showing that goodness and pleasure have incompatible features. [cite.]}

In contrast to the adoption of hedonism, which by all appearances is a matter of opportunism on Socrates' part, the view that no one errs willingly occurs in a variety of places, and without appeals to hedonism. The most telling example is provided by the discussion between Socrates and Polus to which I referred a moment ago. Socrates convinces Polus (or at any rate forces Polus to accept) that doing wrong is worse than suffering it, and that being punished for the wrong you have done is better than escaping punishment. The point is conceded in a such a way as also to deny akrasia:\begin{quote}Socrates: Would you then welcome what's more evil and what's more shameful over what is less so? \ldots

\noindent Polus: No, I wouldn't, Socrates.

\noindent Socrates: And would any other person?

\noindent Polus: No, I don't think so, not on this reasoning anyhow.

\noindent Socrates: I was right, then, when I said that neither you nor I nor any other person would take doing what's unjust over suffering it, for it really is more evil.\footnote{trans. Zeyl, 475d-e}\end{quote} Of course someone \emph{may} commit injustice rather than suffer it, and \emph{may in fact} try to escape punishment, but, like a child who avoids surgery because it's painful, not knowing ``what health and bodily excellence are like,'' people who avoid paying the penalty \begin{quote}focus on its painfulness, but are blind to its benefit and are ignorant of how much more miserable it is to live with an unhealthy soul than with an unhealthy body. (479b)\end{quote} As in the \emph{Protagoras}, one goes wrong through ignorance---but here pain (and pleasure, presumably?), just as such, cause ignorance. Observe that their role here in producing ignorance has nothing to do with their temporal proximity---it is not that our choices are between various assortments of pleasures and pains, as is the case in the \emph{Protagoras,} given the hedonism adopted there. You avoid punishment because it is painful, but it is not that getting punished will result in more, perhaps longer-term, pleasure overall. That possibility is explicitly ruled out (at 475c). Being punished surpasses escaping punishment not in pleasure but in ``benefit'' (these were, remember, the two subspecies of goodness in this part of the dialogue).

I conclude, therefore, that while the denial of akrasia---the view that no one errs willingly---is a central and fundamental Socratic commitment, the hedonism by which it is defended in the \emph{Protagoras} is not. The elaborate discussion of akrasia in the \emph{Protagoras} does, I believe, shed light on the Socratic motivation for the rejection of akrasia, however that motivation has nothing to do with hedonism but rather with the peculiar Socratic conceptions of belief and knowledge. We return then to epistemological matters.


\section{The Socratic conception of knowledge}

\subsection{Knowledge is craft, craft is knowledge.} When we cast about for things we know, we are apt to identify items in propositional form---for example I know that the former Nyasaland is now Malawi, and that the Norse visited Newfoundland. But for Socrates, and for Plato after him, the examples are nearly always \emph{crafts}: cobbling and medicine and carpentry, etc. In the \emph{Apology} Socrates says that the craftsmen have genuine knowledge (even if they exaggerate the extent of it), but reports that he has found knowledge nowhere else. A testament to Socrates' obsession is an almost charming passage in Xenophon according to which Critias, a former-companion-of-Socrates-turned-Athenian-tyrant, along with his fellow tyrant Charicles, having got wind of some implicit criticism of their management of the affairs of state, warn Socrates that he had better leave off taking about carpenters and herdsmen and blacksmiths and the like.\footnote{\emph{Memorabilia} 1.2} Illustrations involving expertise continue to be found all over the place in middle- and late-period dialogues; even the method of division in some later dialogues seems to have its principle application in the categorization of crafts. Now certainly there are throughout the dialogues examples of objects of knowledge which are not crafts---that one knows, for example, nothing, or the road to Larissa, or Theaetetus (the person), or the facts about a robbery. But even in such cases it is hard to be sure that these points of knowledge are not appealed to only as proxies for some craft-constituting form of knowledge. For example, in the \emph{Laches} we are told that we know that we have five fingers on a hand by the ``skill of numbers.'' That is an unobjectionable observation, perhaps, but it is not natural for us to sort out facts by the crafts that govern them; in the dialogues, it seems, that is the place to start. Even when the whole tenor of a discussion seems to show that craft is a poor model for some virtue or form of knowledge (remembering that virtue itself is, for Socrates, generally understood to be a form of knowledge), as for example in the preliminary discussion of justice between Socrates and Polemarchus in the first book of the \emph{Republic}, it is at any rate the first thing that is tried and a model for what follows.


I do in fact suspect that we will best appreciate Plato's understanding of knowledge by seeing that the case of craft is indeed the central case, or at least the paradigm case. For Plato, knowing something is first of all a matter of possessing some form of expertise, or must be in some way as reliable or as practical as craft knowledge. But our topic here is only the earlier dialogues, and here, where we are concerned with the \emph{Socratic} understanding of knowledge, it is tempting to say something stronger than just that craft is a \emph{model} for thinking about knowledge, namely that crafts are simply to be identified with the objects of knowledge; to know something \emph{is} to have a craft. The identification is already suggested by one of the passages from the \emph{Protagoras} which we already cited, the one in which, from the fact that our salvation from wandering beliefs was a form of measurement, it was inferred that our salvation was a craft, and thence that it was a form of knowledge. If, strictly speaking, that only shows one direction of the equivalence, we can find the other direction elsewhere, for example in the \emph{Charmides,} where Socrates infers from the fact that something is known that it is a craft (and even that it must have a product!).\footnote{[cite] Socrates here infers from the fact that something is known (\emph{gignoskein}) that it is an \emph{episteme,} where \emph{episteme} is here used as a synonym for \emph{techne,} as shown, for example, by the fact that, as mentioned, an \emph{episteme} is taken to have a product. [Smith AGP 134, 134n.12., Lyons 1963?]}\\

\subsection{Craftsmen get it right.} To have a craft is to get things right with respect to the subject matter governed by that craft. That is not an unnatural thought even for us---as you gain experience with many similar cases you gain facility in dealing with new cases---but Socrates perhaps takes the thought further than we would. There is an illuminating passage from the first book of the \emph{Republic} on this point.\footnote{The \emph{Republic} is a later dialogue, of course, but the first book is recognized to be very Socratic in form, and is indeed often thought to have been written earlier than the rest of the \emph{Republic.}} Thrasymachus, the interlocutor of the moment, claims that justice is ``the advantage of the stronger.'' Socrates asks whether we are to follow the laws laid down by the powerful when they are in error---ie. when they mistake their own advantage. Thrasymachus replies that they are not after all the stronger insofar as they are mistaken, and continues:\begin{quote}Do you mean, for example, that he who is mistaken about the sick is a physician in that he is mistaken? or that he who errs in arithmetic or grammar is an arithmetician or grammarian at the time when he is making the mistake, in respect of the mistake? True, we say that the physician or arithmetician or grammarian has made a mistake, but this is only a way of speaking; for the fact is that neither the grammarian nor any other person of skill ever makes a mistake in so far as he is what his name implies; they none of them err unless their skill fails them, and then they cease to be skilled artists. No artist or sage or ruler errs at the time when he is what his name implies; though he is commonly said to err, and I adopted the common mode of speaking. [cite]\end{quote} Socrates acquiesces in this at least for the moment and appears to endorse it more directly a few pages later.\footnote{[cite]} We should notice that the passages in the \emph{Protagoras} and the \emph{Republic} both take the notion of ``strength'' the same way: in both places it is associated with craft and avoidance of error.\footnote{The same Greek word---\emph{kreitton}---is used in each case, though not exclusively.}\\

\subsection{A place for propositional knowledge.} Consider the famous passage in the \emph{Meno}---here we move beyond thoughts that can be attributed to Socrates himself, but the passage is still suggestive---in which Socrates helps the slave to ``recollect'' that one can generate, from one square, another square of twice the area by treating the diagonal of the first square as a side of the second. Socrates has to correct a few missteps on the slave's part at first, but eventually leads the slave through a proof of this point. Now though Socrates and the slave walk through proofs together, Socrates does not say that the slave has now come to know a few things about mathematics, he says rather that\begin{quote}at present these notions have just been stirred up in him, as in a dream; but if he were frequently asked the same questions, in different forms, he would know as well as any one at last. [cite.]\end{quote} Knowledge, it seems, comes in packages, and it comes with expertise and experience. But we need not say that the conception of knowledge manifested here and in the early dialogues has \emph{no} room for propositional knowledge, only that such knowledge plays a secondary role. If craft always goes right, surely the right answer will sometimes take the form of a propositional attitude, as in the earlier case of knowing that you have five fingers by the craft of counting. So we could say that you know \emph{that} you have five fingers, or \emph{that} such a square is constructed in such a way, when a propositional attitude to that effect is delivered by knowledge of the craft form. Otherwise you have mere belief.

But might expertise itself not consist in some sort of systematic understanding of propositions? Even if that were true it could be that you have to know some total body of propositions to know any of them---not an unappealing idea. But even apart from that, the idea that crafts are constituted by bodies of propositional knowledge is certainly less appealing in some cases than in others. Protagoras raises the example of knowledge of a language, which we might reasonably consider a kind of expertise if not exactly in the ordinary sense a craft (it's perhaps doubtful whether Socrates would employ the example himself). We know English, and as a result we can know many things of propositional form, as in the finger case. For example, because of my facility with English I know \emph{that ``Kublai Kahn never led a caravan from Xanadu to Abidjan'' is a grammatical English sentence,} even though I'd never previously come across that sentence. The truth of that proposition is not one of the things I know just in knowing English, and it's hardly clear that I could be said to have deduced it from some propositions the knowledge of which constitutes my knowledge of English---if there were such propositions they would presumably be something like descriptions of the rules of English grammar, but I can know English while failing to know these, at least in any remotely articulate way, and conversely I could memorize, say, Smyth's \emph{Greek Grammar} and still struggle to produce in speech Greek sentences of any complexity whatever. Similar observations would hold for many skills and crafts.


How these cognitive processes really work is of course not our concern; I wish only to make the idea that propositional knowledge might emerge from fundamentally non-propositional knowledge look at least faintly sensible. More pertinent at present is the point that Socrates himself often stresses the articulability of craft.\footnote{And so Socrates' model of knowledge has sometimes been taken to be, not a craft model, but a deductive model. [eg. Vlastos.]} It is a notorious at least apparent fact that Socrates expects moral people to be able to give accurate definitions of moral notions. Perhaps that fact is \emph{only} apparent; it sometimes seems that he would accept as evidence that a person possesses craft  (or moral) knowledge the fact that they have trained up others in that same knowledge [cite]. And the context of the demand for definitions is often one in which his interlocutors have claimed some quite special moral expertise. And if Socrates impugns orators and Sophists for going on without adequate grasp of the theoretical principles of their supposed crafts, that is perhaps not unreasonable where these ``teachers'' are presenting all manner of philosophical views with, in the eyes of many people, results that are manifestly disastrous. It is worth observing that in the \emph{Protagoras} we find a bevy of Sophists hosted by one Callias, a man infamous in the ancient world for his scandalous private life and for squandering vast wealth (much of it on Sophists).\footnote{On these matters see David Wolfsdorf, ``The Historical Reader of Plato's \emph{Protagoras,}'' \emph{Classical Quarterly,} 48 (1998), pp. 126-33.} Now Socrates' understanding of craft knowledge and his conditions thereupon have been discussed competently and thoroughly elsewhere.\footnote{I am thinking in particular of Angela Smith's [AGP].} For our purposes let us just be reminded that in the case at hand Socrates identifies the saving art which will preserve us from any practical misstep with \emph{arithmetic!}\footnote{\emph{arithmetike.}} Whether or not Socrates actually expects articulacy of all possessors of crafts, it can at least be said that he is no great respecter of the distinction between practical and theoretical knowledge, and more probably that he simply insists on collapsing the two.\footnote{Here again I would refer the reader to Smith. I would just emphasize here that if for Socrates knowledge is always articulable (which is uncertain), this need mean only that knowledge is articulable \emph{in} propositions; it need not be the case that knowledge just is propositional (Smith p. 132.). We should still insist that the basic model of knowledge here is craft knowledge, just that for Socrates even theoretical knowledge is craft knowledge, so that, again, practical and theoretical aspects of thought are never pulled apart.}

\subsection{Belief and the analysis of knowledge.} The \emph{Meno} contains another instructive passage for our purposes, one which, I think, will help sharpen the differences between Socrates' conception of knowledge and ours. This is the passage in which Socrates observes that so far as our beliefs are true we will go right just as when have knowledge, and then that merely true beliefs have a tendency to escape, but, when ``tied down'' with an ``account of the reason,'' become knowledge (97b-98a). It has been said that it is here Plato first distinguishes knowledge from true belief.\footnote{[cite]} And it is true that knowledge and true belief are deliberately contrasted here (and in later dialogues such as the \emph{Symposium} and the \emph{Republic}) as not before. But the distinction is clearly implicit already in the \emph{Protagoras} in the contrast between knowledge which is secure and belief which oscillates back and forth---at least unless we are to suppose that the power of appearances somehow \emph{never} puts one on the right path (and why wouldn't it?)---as well as in such other early dialogues as the \emph{Ion} and \emph{Apology}, in which Socrates distinguishes knowledge from the kind of capacity the poets have to produce wonderful thoughts in their own poems or about other poets without really knowing what they are up to.

In fact Plato's dialogues distinguish, already before the \emph{Meno,} between knowledge and belief that is not only true but is produced by a reliable mechanism (the gods, in the case of the \emph{Ion}) which mechanism we may presume not to have failed in a given case; we may moreover allow that you recognize some belief of yours to be true and produced in such a fashion (just as Ion accepts Socrates' assessment of his situation) and still you might not have knowledge. Or, if the reader thinks that the element of irony in the \emph{Ion} is too strong to allow any confidence that Socrates really thinks Ion to have reliably true beliefs, we can appeal instead to passages like that in the \emph{Gorgias} in which oratory and pastry making are treated as ``knacks'' but not as forms of knowledge. The orator obviously has some facility in directing the emotions of his audience, which entails reliable beliefs within some sphere, though a sphere too narrow for his outlook to constitute knowledge. There is Socrates' own case too: he does not claim real moral knowledge, but that does not mean that he has no true beliefs of reliable origin about what he ought to do---take for instance case where his \emph{daimon} directs him. We find the reliable true belief/knowledge distinction later, too, for example in book ten of the \emph{Republic,} where we are told that a flute player knows the flute while the maker of the flute has only true beliefs about it, true beliefs which he gains by his interaction with the player. The player here is presumably a reliable source of information. The point, then, again, is that if you want to get propositional knowledge from true beliefs, it seems you must simply add some form of expert knowledge in the first place.\footnote{This last paragraph reveals an exception to something I said earlier, that mere true beliefs are stable only by chance.}



It is natural for us to read the thought that ``true opinions when tied down with an account become knowledge'' as ``knowledge is true belief plus an account'' because \emph{we} are interested in accounts of (propositional) knowledge, but it is important to realize that Plato's suggestion here answers less to the question: ``what is knowledge'' than to the question ``how does one come to know what one does not know,'' or perhaps ``how is it that some people seem to be virtuous even though they have no knowledge?'' The answer would be something like: there is at least something to work with even in the absence of knowledge, namely true belief. The point, then, is that if Socrates does not distinguish true belief and knowledge carefully then that is because it would not occur to him to think that knowledge could be a kind of believing at all. Neither does the fact that a more explicit contrast does eventually get made mean that Plato was after all operating with a propositional conception of knowledge.


\section{Knowledge and akrasia}


We may observe one more point of resonance in the \emph{Meno} by way of transition back to the \emph{Protagoras.} In the \emph{Meno} knowledge makes beliefs stable---knowledge ties down unstable beliefs, beliefs which otherwise wander away thereby causing \emph{us} to lose \emph{our} way. Likewise in the \emph{Protagoras} we saw a contrast between the instability of belief and the stability of knowledge. In neither dialogue is it entirely clear whether we are meant to understand that what you know (so far as this might take propositional form) you also believe, or whether alternatively these states are meant to be incompatible. Perhaps nothing in the text would settle the matter, and it seems harmless, both philosophically and interpretively, to suppose that what you know you also believe.


In that case, the following picture seems to me attractive, applying the lessons of the previous section. Knowledge, being craft, ensures that we get things right---as in the passage from the \emph{Republic,} it would not be craft if it got things wrong. in the present context ``getting it right'' means correctly determining what is good and what is bad, and the measurement art---a kind of craft and knowledge---is introduced precisely as what would ensure that we do not mistake our good. We said that the ``right answer'' supplied by a given craft might perfectly well take propositional form, so in a given case that means that, possessed of our practical art, we will know, now in the propositional sense, that such-and-such is more important than that other thing, and so that we should do such-and-such. What we know propositionally we believe, so by the principle that we never do wrong unwillingly---by the general rejection of akrasia---we do what we know we should do. That is, the craft of knowledge produces correct beliefs and these are in turn acted upon.


Now knowledge is not the only thing that produces belief. In the \emph{Protagoras} we hear that the ``power of appearance'' produces unstable beliefs; in the context of the \emph{Protagoras} this power of appearance may be identified simply with the tendency of proximate pleasures or pains to have a distorting affect on our judgement. Setting aside hedonism, we could say, with the \emph{Gorgias,} that pleasure just as such can have such a distorting affect. Or also that generally whatever emotional forces are relevant are potential belief-producers.

%\marginpar{\begin{scriptsize}TS still found this SofK stuff a bit confusing - isn't it the belief that's strong? Need to bring out the competition aspect a bit more perhaps.\end{scriptsize}}

The strength of knowledge thus consists of two things (which perhaps really amount to the same thing)---first it consists in always getting things right, and second in that it always dominates other sources of beliefs. There is, as it were, between fears and angers and nearby pleasures and faraway pains, a competition to produce belief. We might rank some of these sources amongst themselves; so for example in the speeches of Phaedrus and Agathon in the \emph{Symposium,} love dominates the other emotions: fear would not drive you from battle while you fight beside your beloved, for example. But when knowledge is an entrant in the competition as well, it wins.

%\marginpar{\begin{scriptsize}Generally explain better, at greater length. Nico: any reason the explanation couldn't go the other way? First deny akrasia, then on that basis the craft conception - ie your knowledge must have failed you? Won't work for Plato case, but perhaps for S, and P could have got it there. GS: Why isn't this just appeal to guise of the good thesis now? GS: Aristotle: that's why V is not craft: can't be misused. But also thinks you can't be argued out of K - fits well with craft idea. AR interesting thought: if master V is K of good, maybe problems - eg circularity? And if not K of good then maybe can't appeal to guise of good? perhaps a fork here. Which would perhaps leave my conclusion in place? TS: condition that crafts be good? - sure but doesn't mean can't be misused. [cont next page]\end{scriptsize}}

It should be more than clear now why the strength of knowledge is not a trivial implication of the fact that we never willingly err. The strength of knowledge is rather an implication of the craft conception of knowledge, and its strength consists in its ability to produce practically effective beliefs in the face of other belief-producing forces. But if I have tried to explain what is meant by the ``strength of knowledge,'' I have not said anything about why Socrates thinks we never do other than we think best. Allowing that our knowledge of goods and evils ensures that, propositionally speaking, we know---and so believe---that such and such is the right thing to do in a given case, what ensures that that belief is effective, or that we indeed act in pursuit of the good rather than, quite voluntarily, in pursuit of the evil? Socrates is convinced, it seems, that a person will always do what they think best, and so will always pursue the good indicated by knowledge of the good. But it is unclear on what basis he can be so confident of this. However one early dialogue, the \emph{Hippias Minor,} appears to grapple with this question. There it is observed that someone with a given skill or power or virtue will be best able not only to put that power to good ends but to bad ends: someone who knows the truth is best able to ensure that they never speak it, since they can avoid hitting on it just by accident. But then if possession of a virtue just as such makes a person better, someone who misuses that virtue will be as `good' as someone who uses it well---a paradoxical result. Socrates declares himself uncertain what to say about this. This perplexity is perhaps inevitable on the present knowledge of virtue, which treats it as a kind of knowledge and craft. A given craft is directed by something above it: the general directs the cavalry and the politician directs the general. Once we get to the highest level crafts, meta-crafts like knowledge of the good, what directs those? Socrates is certain a virtuous man will indeed do good but cannot say why; the answer to this question would be the answer to the question why a person never does anything but what they think best. But that itself is illuminating. To realize that the strength of knowledge is not an implication of the rejection of akrasia is thereby to realize \emph{why} Socrates rejects the possibility of akrasia. Socrates \emph{has} to reject it precisely \emph{because} he has a craft model of knowledge, unless he wants to say that the best of men may be liars and crooks as easily as they may be honest and upright.

%this needs to be sharpened.

\section{Conclusion: Socratic intellectualism}

%\marginpar{\begin{scriptsize}[cont from previous page] DL: Does S have space for continence? Drew (rightly) takes issue with craft can be misused at will idea - eg music so on. Observes possible overemphasis on theoretical aspect is perhaps having a bad influence here? Def could be. GS: disagrees on interp of \emph{HM.}\end{scriptsize}}

By way of summary. Two things were distinguished which are sometimes collapsed. There is the denial of voluntary error, and there is the strength of knowledge. Knowledge is strong because knowledge is craft and it is definitive of craft that it is successful. But a craft, being a kind of power, is at the service of its possessor, for good or for ill. Therefore we must suppose that there is some fundamental drive towards the good if we are to think of moral virtues as powers. So we find in the craft model of knowledge the motivation for the rejection of what we would call akrasia, though not any further defence of that rejection.

And a few general comments, aimed at putting those views in a slightly more favourable light. Many of us, including myself, find the rejection of akrasia quite difficult to take seriously (if anything, personal experience leads me to doubt the possibility of non-akratic action). Perhaps it hardly makes us more amenable to Socrates' view to find that he is committed to it on the basis of a conception of knowledge arrived at by the observation of small-time fifth-century craftsmen. But it should at least be observed that Socrates' view is not simply that the intellect rules the body unflinchingly---he knows that the emotions, at least where untempered by knowledge, affect our thoughts and so lead us wrong. This at least is an astute psychological observation. It does not make room for ``clear-eyed'' akrasia, but it does focus attention on the loss of perspective, the semi-rationalizations and the excuses that are certainly part of the background of most cases of akrasia.

Also the view that virtue is knowledge, which is of course the upshot of the discussion of the \emph{Protagoras} (modulo the qualifications demanded by the considerations of the \emph{Hippias Minor}), is more intelligible when we remember that knowledge is craft. Because now to say that vice is ignorance is not just to say that vice is simply not knowing what to do, which seems an insane definition of vice; rather to say that vice is ignorance is to say that a person without moral knowledge is a person moved to moral error by his pleasures and his parochial concerns. And that is not an insane understanding of vice.

Certainly the standard of virtue set here is high; high enough that Socrates is unable to identify \emph{anyone} who \emph{is} virtuous. Our own sympathies will perhaps be with Protagoras, who had suggested earlier in the dialogue that virtue is taught by everyone to everyone else. Socrates outlook really is, I think, much more pessimistic. His suspicion of democracy is to be remembered (not everyone can be an expert, least of all in state-craft) and indeed also his apparent scepticism about of the capacities of great statesmen of his day (this was why Critias and Alcibiades wanted Socrates to stop talking about craftsmen: they thought they were being made out to measure up badly against them). Part of the issue is perhaps that Socrates' conception of virtue is more political than ours, but leaving that aside, radical pessimism about virtue would not be unheard of. ``There is no one does good, no not one,'' says the Psalmist. On the Biblical understanding, it is possible to do good only with divine aid. Socrates, in fact, says the same thing, in the \emph{Meno.} Although Socrates' own divine commission was only to prove that it was so and not to do anything about it.

\end{document}